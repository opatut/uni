\newcommand{\authorinfo}{Paul Bienkowski, Arne Struck}
\newcommand{\titleinfo}{Optimierung Blatt 01 zum 21.10.2013}

% PREAMBLE ===============================================================

\documentclass[a4paper,11pt]{article}
\usepackage[german,ngerman]{babel}
\usepackage[utf8]{inputenc}
\usepackage[T1]{fontenc}
\usepackage[top=1.3in, bottom=1in, left=1.0in, right=0.6in]{geometry}
\usepackage{lmodern}
\usepackage{amssymb}
\usepackage{mathtools}
\usepackage{amsmath}
\usepackage{enumerate}
\usepackage{pgfplots}
\usepackage{breqn}
\usepackage{tikz}
\usepackage{fancyhdr}
\usepackage{multicol}

\usetikzlibrary{calc}
\usetikzlibrary{patterns}

\author{\authorinfo}
\title{\titleinfo}
\date{\today}

\pagestyle{fancy}
\fancyhf{}
\fancyhead[L]{\authorinfo}
\fancyhead[R]{\titleinfo}
\fancyfoot[C]{\thepage}

\begin{document}
\maketitle
\begin {enumerate}
\item[\textbf{1.}]
    \begin{enumerate}
    \item[a)]
        \begin{enumerate}
        \item[(i)]
            maximiere \(-2x_1-x_2+x_3-2x_4\) \\
            unter den Nebenbedingungen \\
            \(\begin{array}{rcr}
                3x_1 + x_2 - x_3 & \leq & -2 \\
                - 7x_1 - x_2 + x_4 & \leq & 3 \\
                - x_2 - x_3 + x_4 & \leq & - 7 \\
                x_2 + x_3 - x_4 & \leq & 7 \\
                x_1,x_2,x_3,x_4 & \geq & 0 \\
            \end{array}\)

        \item[(ii)]
            Substitution \(x_1 = x_1'-x_1''\) mit \(x_1', x_1'' \in \mathbb{N} \) \\
            maximiere \(2x_1'-2x_1''+x_2-x_3+2x_4\) \\
            unter den Nebenbedingungen \\
            \(\begin{array}{rcr}
                3x_1'-3x_1'' + x_2 - x_3 & \leq & -2 \\
                -7x_1'+7x_1'' - x_2 + x_4 & \leq & 3 \\
                - x_2 - x_3 + x_4 & \leq & -7 \\
                x_2 + x_3 - x_4 & \leq & 7 \\
                x_4 & \leq & 9 \\
                x_1',x_1'',x_2,x_3,x_4 & \geq & 0
            \end{array}\)
        \end{enumerate}

    \item[b)]
        Skizze:

        \begin{tikzpicture}
            \path[fill = gray!50] (16/5,6-16/5)--(30/7,1/3 * 30/7+3)--(9/4,6-9/4)--cycle;
            \begin{axis}[
                ymin = 0,ymax = 8.5,
                xmin = 0,xmax = 8.5,
                x = 1cm, y = 1cm,
                axis x line=middle,
                axis y line=middle,
                axis line style=->,
                xlabel={$x_1$},
                ylabel={$x_2$},
                ]
                \addplot[very thick, no marks, black!50] {3*x/2 - 2}; \addlegendentry{NB 1};
                \addplot[very thick, no marks, blue!50] {-x + 6}; \addlegendentry{NB 2};
                \addplot[very thick, no marks, red!50] {(1/3) * x +3}; \addlegendentry{NB 3};
                \addplot[very thick, no marks, black, dotted] {2/5 * x};
                \addplot[very thick, no marks, black, dotted] {2/5 * x + 57/20};
                    \addlegendentry{$2x_1+5x_2+c$};
            \end{axis}
        \end{tikzpicture}

        Damit ist das Maximum bei ca \(2x_1 + 5x_2 + 2.8\) erreicht und liegt bei ca. \(P(2,25|3,75)\).
    \end{enumerate}

\newpage
\item[\textbf{2.}]
    \begin{enumerate}
    \item[a)]
        \textbf{Pauls Diätproblem}\\
            maximiere \( -3x_1 - 24x_2 - 13x_3 - 9x_4 - 20x_5 - 19x_6 \) \\
            unter den Nebenbedingungen \\
            \(\begin{array}{rcr}
                - 110 x_1 - 205 x_2 - 160 x_3 - 160 x_4 - 420 x_5 - 260 x_6 & \leq & -2000 \\
                - 4 x_1 - 32 x_2 - 13 x_3 - 8 x_4 - 4 x_5 - 14 x_6 & \leq & -55 \\
                - 2 x_1 - 12 x_2 - 54 x_3 - 285 x_4 - 22 x_5 - 88 x_6 & \leq & -800 \\
                x_1 & \leq & 4 \\
                x_2 & \leq & 3 \\
                x_3 & \leq & 2 \\
                x_4 & \leq & 8 \\
                x_5 & \leq & 2 \\
                x_6 & \leq & 2 \\
                x_1, x_2, x_3, x_4, x_5, x_6 & \geq & 0 \\
            \end{array}\)

        \textbf{Problem 1.2}\\
            maximiere \( -3x_1 + x_2 \)\\
            Substitution 1: \(x_1 = x_1'-x_2''\) mit \(x_1', x_1'' \in \mathbb{N} \) \\
            Substitution 2: \(x_4 = x_4'-x_4''\) mit \(x_4', x_4'' \in \mathbb{N} \) \\
            unter den Nebenbedingungen \\
            \(\begin{array}{rcr}
                x_1 - 6x_2 + x_3 - x_4 & \leq & 3 \\
                7x_2 + 2x_4  &\leq& 5 \\
                - 7x_2 - 2x_4  &\leq& -5 \\
                x_1 + x_2 + x_3 &\leq& 1 \\
                - x_1 - x_2 - x_3 &\leq& -1 \\
                x_3 + x_4 &\leq& 2 \\
                x_1', x_1'', x_2, x_3, x_4', x_4'' &\geq& 0
            \end{array}\)

    \item[b)]
        Der Index 1 bezieht sich auf einfache Einheiten, der Index 2 gibt die
        in regulärer Arbeitszeit veredelten Einheiten and, Index 3 in
        "Uberstunden veredelten Einheiten.

        maximiere \( 5a_1 + 13a_2 + 8a_3 + 9b_1 + 15b_2 + 12b_3 + 5c_1 + 14c_2 + 10c_3 \)\\
        unter den Nebenbedingungen \\
        \(\begin{array}{rcr}
            a_1 + a_2 + a_3 &\leq& 400 \\
            - a_1 - a_2 - a_3 &\leq& -400 \\
            b_1 + b_2 + b_3 &\leq& 480 \\
            - b_1 - b_2 - b_3 &\leq& -480 \\
            c_1 + c_2 + c_3 &\leq& 230 \\
            - c_1 - c_2 - c_3 &\leq& -230 \\
            a_2 + b_2 + c_2 &\leq& 420 \\
            a_3 + b_3 + c_3 &\leq& 250 \\
            a_1, a_2, a_3, b_1, b_2, b_3, c_1, c_2, c_3 &\geq& 0
        \end{array}\)

    \end{enumerate}
\end {enumerate}
\end{document}
