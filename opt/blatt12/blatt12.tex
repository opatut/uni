\newcommand{\authorinfo}{Paul Bienkowski, Nils Rokita, Arne Struck}
\newcommand{\titleinfo}{Optimierung Blatt 12 zum 20.01.2014}

% PREAMBLE ===============================================================

\documentclass[a4paper,11pt]{article}
\usepackage[german,ngerman]{babel}
\usepackage[utf8]{inputenc}
\usepackage[T1]{fontenc}
\usepackage[top=1.3in, bottom=1in, left=1.0in, right=0.6in]{geometry}
\usepackage{lmodern}
\usepackage{amssymb}
\usepackage{mathtools}
\usepackage{amsmath}
\usepackage{enumerate}
\usepackage{pgfplots}
\usepackage{breqn}
\usepackage{tikz}
\usepackage{fancyhdr}
\usepackage{multicol}

\usetikzlibrary{calc}
\usetikzlibrary{patterns}

\newcommand{\bra}[1]{\left(#1\right)}

\author{\authorinfo}
\title{\titleinfo}
\date{\today}

\pagestyle{fancy}
\fancyhf{}
\fancyhead[L]{\authorinfo}
\fancyhead[R]{\titleinfo}
\fancyfoot[C]{\thepage}

\renewcommand\arraystretch{1.2}

\tikzset{
    n/.style={
        circle,
        draw,
        minimum width=0.6cm,
        minimum height=0.6cm,
        inner sep=0
    }
}

\begin{document}
\maketitle
\begin{enumerate}
\item[\textbf{1.}]
    \begin{enumerate}
    \item[a)]
    \begin{multicols}{2}
        Dijkstra-Algorithmus:

        \begin{tabular}{c|cccccc}
            von & a & b & c & d & e & f \\ \hline
            s & 6s & 5s & 2s & $\infty$ & $\infty$ & $\infty$ \\
            c & 6s & 4c & 2s & $\infty$ & 3c & 5c \\
            e & 5e & 4c & 2s & $\infty$ & 3c & 4e \\
            b & 5e & 4c & 2s & 8b & 3c & 4e \\
            f & 5e & 4c & 2s & 8b & 3c & 4e \\
            a & 5e & 4c & 2s & 7a & 3c & 4e \\
            d & 5e & 4c & 2s & 7a & 3c & 4e \\
        \end{tabular}

        An der letzten Zeile dieser Tabelle kann man die kürzesten s,v-Pfade
        sowie deren Länge ablesen.

        \begin{tikzpicture}
            \node[n] (s) at (0, 0) {s};
            \node[n] (c) at ($(s) + ( 0, -1.5)$) {c};
            \node[n] (b) at ($(c) + (-1, -1.5)$) {b};
            \node[n] (e) at ($(c) + ( 1, -1.5)$) {e};
            \node[n] (a) at ($(e) + (-1, -1.5)$) {a};
            \node[n] (f) at ($(e) + ( 1, -1.5)$) {f};
            \node[n] (d) at ($(a) + ( 0, -1.5)$) {d};

            \path[draw]
                (s) -- node[midway,left]{2} (c)
                (c) -- node[midway,left]{2} (b)
                (c) -- node[midway,right]{1} (e)
                (e) -- node[midway,left]{2} (a)
                (e) -- node[midway,right]{1} (f)
                (a) -- node[midway,left]{2} (d);
        \end{tikzpicture}
    \end{multicols}

    \item[b)]
        In diesem Graph gilt $C=1000$, damit wäre der untere Pfad $2001$ lang,
        der obere nur $2000$. Dijkstra würde also den falschen Pfad finden.

        \vspace{5mm}\begin{center}
            \begin{tikzpicture}
                \node[n] (s) at (0, 0) {s};
                \node[n] (a) at (2, -1) {c};
                \node[n] (b) at (4, -1) {b};
                \node[n] (t) at (6, 0) {t};

                \path[draw] (s) edge[->,bend left=30] node[midway,above]{1000} (t);
                \path[draw] (s) edge[->,bend right=10] node[midway,below left]{-1000} (a);
                \path[draw] (a) edge[->,bend right=10] node[midway,below]{1} (b);
                \path[draw] (b) edge[->,bend right=10] node[midway,below right]{1000} (t);
            \end{tikzpicture}
        \end{center}
    \end{enumerate}

\item[\textbf{2.}]
    \begin{enumerate}
    \item[a)]
        \begin{itemize}
            \item $(a, e)$ oder $(e, f)$
            \item $(e, f)$ oder $(a, e)$ -- bedingt durch vorherige Kante
            \item $(e, b)$ oder $(f, b)$ -- die andere wird dadurch ausgeschlossen
            \item $(f, g)$
            \item $(g, h)$
            \item $(g, c)$
            \item $(g, d)$
        \end{itemize}

        Die Unterschiede in der Reihenfolge entstehen also durch Wahl der ersten
        Kante. Es können 2 unterschiedliche Bäume entstehen, einer enthält
        $(e, b)$, der andere $(f, b)$.

    \item[b)]
        \begin{enumerate}
        \item[(i)]
            $(a, b)$, $(b, c)$, $(c, g)$, $(g, d)$, $(g, f)$, $(g, h)$, $(a, e)$
        \item[(ii)]
            $(a, b)$, $(d, g)$, $(g, f)$, $(g, h)$, $(b, c)$, $(c, g)$, $(a, e)$
        \item[(iii)]
            $(a, f)$, $(f, b)$, $(b, g)$, $(e, f)$, $(d, h)$, $(c, d)$
        \end{enumerate}
    \end{enumerate}
\end{enumerate}
\end{document}
