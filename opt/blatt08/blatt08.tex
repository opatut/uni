\newcommand{\authorinfo}{Paul Bienkowski, Nils Rokita, Arne Struck}
\newcommand{\titleinfo}{Optimierung Blatt 08 zum 09.12.2013}

% PREAMBLE ===============================================================

\documentclass[a4paper,11pt]{article}
\usepackage[german,ngerman]{babel}
\usepackage[utf8]{inputenc}
\usepackage[T1]{fontenc}
\usepackage[top=1.3in, bottom=1in, left=1.0in, right=0.6in]{geometry}
\usepackage{lmodern}
\usepackage{amssymb}
\usepackage{mathtools}
\usepackage{amsmath}
\usepackage{enumerate}
\usepackage{pgfplots}
\usepackage{breqn}
\usepackage{tikz}
\usepackage{fancyhdr}
\usepackage{multicol}

\usetikzlibrary{calc}
\usetikzlibrary{patterns}

\newcommand{\bra}[1]{\left(#1\right)}

\author{\authorinfo}
\title{\titleinfo}
\date{\today}

\pagestyle{fancy}
\fancyhf{}
\fancyhead[L]{\authorinfo}
\fancyhead[R]{\titleinfo}
\fancyfoot[C]{\thepage}

\renewcommand\arraystretch{1.2}

\begin{document}
\maketitle
\begin{enumerate}

\item[\textbf{1.}]
    \begin{enumerate}
    \item[a)]
        minimiere \( y_1 + 2y_2 + 3y_3 \)\\
        unter den Nebenbedingungen \\
        \(\begin{array}{rcrcrcr}
            2y_1 & + & 3y_2 & + &  y_3 & \geq & 5 \\
            3y_1 & + &  y_2 & + &  y_3 & \geq & -7 \\
            -y_1 & + & 4y_2 & - & 2y_3 & \geq & 3 \\
             y_1 & - & 2y_2 & - & 1y_3 & =    & 1 \\
                 &   &      &   & y_{2,3} & \geq & 0
        \end{array}\)

    \item[b)]
        maximiere \( -y_1 + 16y_2 + 5y_3 + 8y_4 + y_5 - 4y_6 - 10y_7 + 9y_8 \)\\
        unter den Nebenbedingungen \\
        \(\begin{array}{rcrcrcrcrcrcrcrcr}
            2y_1 & + &  y_2 & + &  y_3 & + & 2y_4 & + &  y_5 & + & 4y_6 & - & 4y_7 & + &  y_8 & \leq & -2 \\
           -4y_1 & + & 5y_2 &   &      & + & 4y_4 & - & 3y_5 & - & 3y_6 & + & 3y_7 & + & 2y_8 & =    & 3 \\
             y_1 & + &  y_2 & + &  y_3 & - &  y_4 & + &  y_5 &   &      & - & 5y_7 & + &  y_8 & =    & 22 \\
                 &   &      &   &      &   &      &   &      &   &      &   &      &   & y_{4,5,6,7} & \geq & 0
        \end{array}\)
    \end{enumerate}

\item[\textbf{2.}]
    \begin{enumerate}
    \item[a)]
        \begin{enumerate}
        \item[(i)]
            \textbf{Starttableau}:

            \[\begin{array}{rcrcrcr}
                x_3 & = &  100 & - &   x_1 & - &   x_2 \\
                x_4 & = & 4000 & - & 10x_1 & - & 50x_2 \\ \hline
                  z & = &      &   & 40x_1 & + & 70x_2\\
            \end{array}\]

            Eingangsvariable: $x_2$, Ausgangsvariable: $x_4$

            \( x_2 = 80 - \frac{1}{5}x_1 - \frac{1}{50}x_4 \) \\
            \( x_3 = 100 - x_1 - \bra{80 - \frac{1}{5}x_1 - \frac{1}{50}x_4} = 20 - \frac{4}{5}x_1 + \frac{1}{50}x_4 \)\\
            \( z = 40x_1 + 70\bra{80 - \frac{1}{5}x_1 - \frac{1}{50}x_4} = 5600 + 26x_1 - \frac{7}{5}x_4 \)\\

            \textbf{Tableau nach 1. Iteration}:

            \[\begin{array}{rcrcrcr}
                x_2 & = &   80 & - & \frac{1}{5}x_1 & - & \frac{1}{50}x_4 \\
                x_3 & = &   20 & - & \frac{4}{5}x_1 & + & \frac{1}{50}x_4 \\ \hline
                  z & = & 5600 & + &          26x_1 & - & \frac{7}{5}x_4\\
            \end{array}\]

            Eingangsvariable: $x_1$, Ausgangsvariable: $x_3$

            \( x_1 = 25 + \frac{1}{40}x_4 - \frac{5}{4}x_3 \)\\
            \( x_2 = 80 - \frac{1}{5}\bra{25 + \frac{1}{40}x_4 - \frac{5}{4}x_3} - \frac{1}{50}x_4 = 75 - \frac{1}{40}x_4 + \frac{1}{4}x_3 \)\\
            \( z = 5600 + 26\bra{25 + \frac{1}{40}x_4 - \frac{5}{4}x_3} - \frac{7}{5}x_4 = 6250 - \frac{3}{4}x_4 - \frac{65}{2}x_3 \)\\

            \textbf{Tableau nach 2. Iteration}:

            \[\begin{array}{rcrcrcr}
                x_1 & = &   25 & + & \frac{1}{40}x_4 & - & \frac{5}{4}x_3 \\
                x_2 & = &   75 & - & \frac{1}{40}x_4 & + & \frac{1}{4}x_3 \\ \hline
                  z & = & 6250 & - &  \frac{3}{4}x_4 & - & \frac{65}{2}x_3\\
            \end{array}\]

            Aus diesem Tableau kann man $x^* = (25, 75)$ sowie $y^* = (0.75, 32.5)$
            ablesen. $\Box$

        \item[(ii)]
            \[\begin{array}{lclcll}
                x_1^* \neq 0 &:&  1 \cdot 32.5 + 10 \cdot 0.75 &=& 40 \\
                x_2^* \neq 0 &:&  1 \cdot 32.5 + 50 \cdot 0.75 &=& 70 \\
                y_1^* \neq 0 &:&  1 \cdot   25 +  1 \cdot   75 &=& 100 \\
                y_2^* \neq 0 &:& 10 \cdot   25 + 50 \cdot   75 &=& 4000 & \Box \\
            \end{array}\]

        \end{enumerate}

    \item[b)]
        \textbf{Starttableau}:

        \[\begin{array}{rcrcrcr}
            x_3 & = &  100 & - &   x_1 & - &   x_2 \\
            x_4 & = & 4000+t & - & 10x_1 & - & 50x_2 \\ \hline
              z & = &      &   & 40x_1 & + & 70x_2\\
        \end{array}\]

        Eingangsvariable: $x_2$, Ausgangsvariable: $x_4$

        Die Ausgangsvariable $x_4$ kann gewählt werden, da im ,,schlechtesten'' Fall
        von $t=1000$ diese Variable genauso schnell $0$ ist wie $x_3$, ansonsten schneller.

        \( x_2 = 80 + \frac{t}{50} - \frac{1}{5}x_1 - \frac{1}{50}x_4 \) \\
        \( x_3 = 100 - x_1 - \bra{80 + \frac{t}{50} - \frac{1}{5}x_1 - \frac{1}{50}x_4} = 20 - \frac{t}{50} - \frac{4}{5}x_1 + \frac{1}{50}x_4 \)\\
        \( z = 40x_1 + 70\bra{80 + \frac{t}{50} - \frac{1}{5}x_1 - \frac{1}{50}x_4} = 5600 + \frac{7}{5}t + 26x_1 - \frac{7}{5}x_4 \)\\

        \textbf{Tableau nach 1. Iteration}:

        \[\begin{array}{rcrcrcr}
            x_2 & = &   80+\frac{t}{50} & - & \frac{1}{5}x_1 & - & \frac{1}{50}x_4 \\
            x_3 & = &   20-\frac{t}{50} & - & \frac{4}{5}x_1 & + & \frac{1}{50}x_4 \\ \hline
              z & = & 5600+\frac{7}{5}t & + &          26x_1 & - & \frac{7}{5}x_4\\
        \end{array}\]

        Eingangsvariable: $x_1$, Ausgangsvariable: $x_3$

        Die Ausgangsvariable $x_3$ kann gewählt werden, da im ,,schlechtesten'' Fall
        von $t=0$ diese Variable schneller auf 0 ist.

        \( x_1 = 25 - \frac{t}{40} + \frac{1}{40}x_4 - \frac{5}{4}x_3 \)\\
        \( x_2 = 80 - \frac{1}{5}\bra{25 - \frac{t}{40} + \frac{1}{40}x_4 - \frac{5}{4}x_3} - \frac{1}{50}x_4 = 75 + \frac{t}{40} - \frac{1}{40}x_4 + \frac{1}{4}x_3 \)\\
        \( z = 5600 + 26\bra{25 - \frac{t}{40} + \frac{1}{40}x_4 - \frac{5}{4}x_3} - \frac{7}{5}x_4 = 6250 + \frac{3}{4}t - \frac{3}{4}x_4 - \frac{65}{2}x_3 \)\\

        \textbf{Tableau nach 2. Iteration}:

        \[\begin{array}{rcrcrcr}
            x_1 & = &   25-\frac{t}{40} & + & \frac{1}{40}x_4 & - & \frac{5}{4}x_3 \\
            x_2 & = &   75+\frac{t}{40} & - & \frac{1}{40}x_4 & + & \frac{1}{4}x_3 \\ \hline
              z & = & 6250+\frac{3}{4}t & - &  \frac{3}{4}x_4 & - & \frac{65}{2}x_3\\
        \end{array}\]

        Die optimale Lösung lässt sich ablesen als $x^* = (25 - \frac{t}{40}, 75 + \frac{t}{40})$.

        Man sieht außerdem, dass der Zielfunktionswert im optimalen Fall $6250 + \frac{3}{4}t$
        beträgt, also um genau $\frac{3}{4}t$ mehr Gewinn erzielt wird.

    \end{enumerate}
\end{enumerate}
\end{document}
