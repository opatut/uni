\newcommand{\authorinfo}{Paul Bienkowski, Nils Rokita, Arne Struck}
\newcommand{\titleinfo}{Optimierung Blatt 07 zum 02.12.2013}

% PREAMBLE ===============================================================

\documentclass[a4paper,11pt]{article}
\usepackage[german,ngerman]{babel}
\usepackage[utf8]{inputenc}
\usepackage[T1]{fontenc}
\usepackage[top=1.3in, bottom=1in, left=1.0in, right=0.6in]{geometry}
\usepackage{lmodern}
\usepackage{amssymb}
\usepackage{mathtools}
\usepackage{amsmath}
\usepackage{enumerate}
\usepackage{pgfplots}
\usepackage{breqn}
\usepackage{tikz}
\usepackage{fancyhdr}
\usepackage{multicol}

\usetikzlibrary{calc}
\usetikzlibrary{patterns}

\newcommand{\bra}[1]{\left(#1\right)}

\author{\authorinfo}
\title{\titleinfo}
\date{\today}

\pagestyle{fancy}
\fancyhf{}
\fancyhead[L]{\authorinfo}
\fancyhead[R]{\titleinfo}
\fancyfoot[C]{\thepage}

\renewcommand\arraystretch{1.2}

\begin{document}
\maketitle
\begin{enumerate}

\item[\textbf{1.}]
    \begin{enumerate}
    \item[a)]
        \[\begin{array}{lrcrcrcrcr}
        P_1:&         \frac{33}{4} &+&         0 &+& 3 \cdot \frac{3}{2} &=& 12\frac{3}{4} &<& 30 \\
        P_2:& 2 \cdot \frac{33}{4} &+& 2 \cdot 0 &+& 5 \cdot \frac{3}{2} & &               &=& 24 \\
        P_3:& 4 \cdot \frac{33}{4} &+&         0 &+& 2 \cdot \frac{3}{2} & &               &=& 36
        \end{array}\]

        Die Ungleichung $P_2$ und $P_3$ sind mit Gleichheit erfüllt, daher muss
        $y_1^* = 0$ gelten.

        Aus $x_1 \ne 0, x_3 \ne 0$ folgt, dass $D_1$ und $D_3$ mit Gleichheit
        erfüllt sein müssen:

        \[\begin{array}{rcrcrcr}
         y_1 &+& 2y_2 &+& 4y_3 &=& 3 \\
        3y_1 &+& 5y_2 &+& 2y_3 &=& 2
        \end{array}\]

        Durch Einsetzen von $y_1 = 0$ und Auflösen erhält man:

        \[y^* = (0, \frac{1}{8}, \frac{11}{16}) \]

        Dies muss eine gültige Lösung für $(D)$ sein, allerdings erhält man in
        $D_2$:

        \[ 0 + 2 \cdot \frac{1}{8} + \frac{11}{16} = \frac{15}{16} \geq 1 \]

        Damit ist dies keine gültige Lösung für $(D)$, die vorgeschlagene Lösung
        kann daher nicht optimal sein.

    \newpage
    \item[b)]
        \textbf{Duales Problem (D):}

        minimiere \( 7y_1 + 8y_2 + 12y_3 \)\\
        unter den Nebenbedingungen \\
        \[\begin{array}{rcrcrcrcr}
             y_1 &   &      & + & 2y_3 & \geq & 2\\
             y_1 & + &  y_2 & + &  y_3 & \geq & 3\\
                 &   &  y_2 &   &      & \geq & 2\\
            &&&& y_{1..3} & \geq & 0
        \end{array}\]

        Einsetzen von $x^* = (5, 2, 6)$ in $(P)$ erfüllt alle Ungleichungen mit
        Gleichheit, also können wir keine Aussage über $y^*_{1..3}$ treffen:

        \[\begin{array}{rcrcrcrcr}
                5 &+& 2 & &   &=&  7\\
                  & & 2 &+& 6 &=&  8\\
        2 \cdot 5 &+& 2 & &   &=& 12\\
        \end{array}\]

        Da $x_1, x_2, x_3 \ne 0$ sind, müssen $D_{1..3}$ mit Gleichheit erfüllt
        sein:

        \[\begin{array}{rcrcrcrcr}
            y1 & &      &+& 2y_3 &=& 2\\
            y1 &+&  y_2 &+&  y_3 &=& 3\\
               & &  y_2 & &      &=& 2\\
        \end{array}\]

        Die eindeutige Lösung für dieses Gleichungssystem ist $y^* = (0, 2, 1)$.
        Da alle Ungleichungen am Gleichungssystem beteiligt sind, ist dies ebenfalls
        eine zulässige Lösung für $(D)$. Damit ist die vorgeschlagene Lösung optimal.

    \end{enumerate}

\newpage
\item[\textbf{2.}]
        Der Index 1 bezieht sich auf einfache Einheiten, der Index 2 gibt die
        in regulärer Arbeitszeit veredelten Einheiten, Index 3 in
        "Uberstunden veredelten Einheiten.

        Die vorgeschlagene Lösung lautet: \( a^*=(0, 400, 0), b^*=(440, 10, 30), c^*=(0, 10, 220) \).

        \textbf{Primales Problem (P):}

        maximiere \( 5a_1 + 13a_2 + 8a_3 + 9b_1 + 15b_2 + 12b_3 + 5c_1 + 14c_2 + 10c_3 \)\\
        unter den Nebenbedingungen \\
        \[\begin{array}{rcr}
            a_1 + a_2 + a_3 &\leq& 400 \\
            - a_1 - a_2 - a_3 &\leq& -400 \\
            b_1 + b_2 + b_3 &\leq& 480 \\
            - b_1 - b_2 - b_3 &\leq& -480 \\
            c_1 + c_2 + c_3 &\leq& 230 \\
            - c_1 - c_2 - c_3 &\leq& -230 \\
            a_2 + b_2 + c_2 &\leq& 420 \\
            a_3 + b_3 + c_3 &\leq& 250 \\
            a_1, a_2, a_3, b_1, b_2, b_3, c_1, c_2, c_3 &\geq& 0
        \end{array}\]

        \textbf{Duales Problem (D):}

        minimiere $400 y_1 - 400 y_2 + 480 y_3 - 480 y_4 + 230 y_5 - 230 y_6 + 420 y_7 + 250 y_8$\\
        unter den Nebenbedingungen \\
        \[\begin{array}{rcrcrcr}
            y_1 & - & y_2 &   &     &\geq& 5 \\
            y_1 & - & y_2 & + & y_7 &\geq& 13 \\
            y_1 & - & y_2 & + & y_8 &\geq& 8 \\
            y_3 & - & y_4 &   &     &\geq& 9 \\
            y_3 & - & y_4 & + & y_7 &\geq& 15 \\
            y_3 & - & y_4 & + & y_8 &\geq& 12 \\
            y_5 & - & y_6 &   &     &\geq& 5 \\
            y_5 & - & y_6 & + & y_7 &\geq& 14 \\
            y_5 & - & y_6 & + & y_8 &\geq& 10 \\
            y_{1..8} &\geq& 0
        \end{array}\]

        Aus $a_2, b_{1..3}, c_{2..3} \ne 0$ folgt, dass $D_{2,4,5,6,8,9}$
        mit Gleichheit erfüllt sein müssen. Schaut man sich nur $D_{4, 5, 8, 9}$ an,
        erhält man:

        \[\begin{array}{rcrcrcr}
        y_3 & - & y_4 &   &     & = & 9  \\
        y_3 & - & y_4 & + & y_7 & = & 15 \\
        y_5 & - & y_6 &   &     & = & 5  \\
        y_5 & - & y_6 & + & y_7 & = & 14 \\
        \end{array}\]

        Die ersten 2 Gleichungen hiervon ergeben $y_7 = 6$, die letzten den
        im Widerspruch stehenden Wert $y_7 = 9$. Damit gibt es keine gültige
        Lösung $y^*$ für das duale Problem, die vorgeschlagene Lösung ist nicht
        optimal.


\end{enumerate}
\end{document}
