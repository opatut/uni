\newcommand{\authorinfo}{Paul Bienkowski, Nils Rokita, Arne Struck}
\newcommand{\titleinfo}{Optimierung Blatt 11 zum 13.01.2014}

% PREAMBLE ===============================================================

\documentclass[a4paper,11pt]{article}
\usepackage[german,ngerman]{babel}
\usepackage[utf8]{inputenc}
\usepackage[T1]{fontenc}
\usepackage[top=1.3in, bottom=1in, left=1.0in, right=0.6in]{geometry}
\usepackage{lmodern}
\usepackage{amssymb}
\usepackage{mathtools}
\usepackage{amsmath}
\usepackage{enumerate}
\usepackage{pgfplots}
\usepackage{breqn}
\usepackage{tikz}
\usepackage{fancyhdr}
\usepackage{multicol}

\usetikzlibrary{calc}
\usetikzlibrary{patterns}

\newcommand{\bra}[1]{\left(#1\right)}

\author{\authorinfo}
\title{\titleinfo}
\date{\today}

\pagestyle{fancy}
\fancyhf{}
\fancyhead[L]{\authorinfo}
\fancyhead[R]{\titleinfo}
\fancyfoot[C]{\thepage}

\renewcommand\arraystretch{1.2}

\begin{document}
\maketitle
\begin{enumerate}
\item[\textbf{1.}]
    \begin{enumerate}
    \item[a)]
        \( s(-, \infty) \)\\
        \( a(s, +, 38) \)\\
        \( b(s, +, 1) \)\\
        \( f(s, +, 2) \)\\
        \( c(a, +, 10) \)\\
        \( d(a, +, 38) \)\\
        \( g(f, +, 2) \)\\
        \( e(c, +, 10) \)\\
        \( t(c, +, 10) \)

        \begin{enumerate}
            \item[(i)]
                Die Markierungsreihenfolge ist also: \texttt{sabfcdget}. Kein Knoten
                bleibt unmarkiert.
            \item[(ii)]
                Der erste Flussvergrößernde Pfad lautet \texttt{s-a-c-t} und
                vergrößert den Pfad um $10$.
        \end{enumerate}

    \item[b)]
        Die Menge S beschreibt alle am Fluß beteiligten Knoten, V sind alle anderen.
        Erkennbar ist dies daran, welcher Knoten ein Label hat und welcher nicht.

    \item[c)]
        Kreuzchen 2: (i) ist wahr, (ii) ist falsch.

    \item[d)]
        n/a

    \end{enumerate}

\item[\textbf{2.}]
    \begin{enumerate}
    \item[a)]
        \( M_1 = \{ (x_1, y_1) \} \) \\
        \( M_2 = \{ (x_1, y_1), (x_2, y_2) \} \) \\
        \( M_3 = \{ (x_1, y_1), (x_2, y_2), (x_3, y_5) \} \) \\
        \( M_4 = \{ (x_1, y_1), (x_2, y_2), (x_3, y_5), (x_4, y_3) \} \) \\
        \( M_5 = \{ (x_1, y_4), (x_2, y_2), (x_3, y_5), (x_4, y_3), (x_5, y_1) \} \) \\
        \( M_6 = \{ (x_1, y_4), (x_2, y_2), (x_3, y_5), (x_4, y_3), (x_5, y_1), (x_6, y_6) \} \)

    \item[b)]
        \( M_1 = \{ (x_1, y_1) \} \) \\
        \( M_2 = \{ (x_1, y_1), (x_2, y_2) \} \) \\
        \( M_3 = \{ (x_1, y_3), (x_2, y_2), (x_3, y_1) \} \) \\
        \( M_4 = \{ (x_1, y_3), (x_2, y_2), (x_3, y_1), (x_4, y_4) \} \) \\
        \( M_5 = \{ (x_1, y_3), (x_2, y_2), (x_3, y_1), (x_4, y_4), (x_5, y_6) \} \) \\
        \( M_6 = \{ (x_1, y_3), (x_2, y_5), (x_3, y_2), (x_4, y_4), (x_5, y_6), (x_6, y_1) \} \) \\

    \end{enumerate}
\end{enumerate}
\end{document}
