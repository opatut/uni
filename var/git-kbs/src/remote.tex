\section{Entfernte Operationen}
\subsection{Remotes}
\begin{frame}{Clones}
    \begin{itemize}
        \item in einem Repository entstehen nur neue Objekte (Dateien)
        \item bestehende werden nicht verändert
        \item jedes Objekt hat einen eindeutigen Namen (SHA kollidiert
            \textit{praktisch} nicht)
        \item $\rightarrow$ Synchronisation ohne zentrale Verwaltung möglich
    \end{itemize}
\end{frame}

\begin{frame}{Remotes}
\iffinal
    \tikzset{
        repo/.style={
            rectangle,
            minimum width=1.7cm,
            minimum height=2cm,
            rounded corners=3pt,
            fill=black!10!white,
            draw=black!50!white,
        },
        commit/.style = {
            circle,
            fill,
            thick,
            draw=black!80!white,
            inner sep=1mm,
        },
        arrow/.style = {
            draw,
            thick
        },
        connection/.style = {
            very thick,
            draw=black!30!white,
            fill=black!30!white,
            -latex,
        },
        biconnection/.style = {
            connection,
            latex-latex,
        }
    }

    \colorlet{c1}{Plum!70!white}
    \colorlet{c2}{cyan}
    \colorlet{c3}{green}
    \colorlet{c4}{orange}
    \colorlet{c5}{yellow}
    \colorlet{c6}{Red}

    \makebox[\textwidth][c]{
        \begin{tikzpicture}[x=1cm,y=-1cm]
            \node[repo] (br) at ( 3,  2) {};
            \node[repo] (tr) at ( 3, -2) {};
            \node[repo] (tl) at (-3, -2) {};
            \node[repo] (bl) at (-3,  2) {};

            \node[font=\tiny] at ($(br)-(0,0.85cm)$) {Horst};
            \node[font=\tiny] at ($(tr)-(0,0.85cm)$) {Ronny};
            \node[font=\tiny] at ($(tl)-(0,0.85cm)$) {Heidi};
            \node[font=\tiny] at ($(bl)-(0,0.85cm)$) {Dimitri};

            % Top left
            \node[commit,fill=c1] (tl1) at ($(tl)+(-0.2,-0.5)$) {};
            \node[commit,fill=c2] (tl2) at ($(tl)+(-0.2, 0.0)$) {};
            \node[commit,fill=c4] (tl4) at ($(tl)+(+0.2, 0.0)$) {};
            \draw[arrow] (tl1) -- (tl2);
            \draw[arrow] (tl1) -- (tl4);
            \uncover<2-> {
                \node[commit,fill=c5] (tl5) at ($(tl)+(+0.2, 0.5)$) {};
                \draw[arrow] (tl4) -- (tl5);
            }

            % Top right
            \node[commit,fill=c1] (tr1) at ($(tr)+(-0.2,-0.5)$) {};
            \node[commit,fill=c2] (tr2) at ($(tr)+(-0.2, 0.0)$) {};
            \node[commit,fill=c4] (tr4) at ($(tr)+(+0.2, 0.0)$) {};
            \draw[arrow] (tr1) -- (tr2);
            \draw[arrow] (tr1) -- (tr4);
            \uncover<3->{
                \node[commit,fill=c6] (tr6) at ($(tr)+(+0.2, 0.5)$) {};
                \draw[arrow] (tr4) -- (tr6);
            }

            % Bottom left
            \node[commit,fill=c1] (bl1) at ($(bl)+(-0.2,-0.5)$) {};
            \node[commit,fill=c2] (bl2) at ($(bl)+(-0.2, 0.0)$) {};
            \node[commit,fill=c4] (bl4) at ($(bl)+(+0.2, 0.0)$) {};
            \draw[arrow] (bl1) -- (bl2);
            \draw[arrow] (bl1) -- (bl4);
            \uncover<4-> {
                \node[commit,fill=c3] (bl3) at ($(bl)+(-0.2,+0.5)$) {};
                \draw[arrow] (bl2) -- (bl3);
            }

            % Bottom right
            \node[commit,fill=c1] (br1) at ($(br)+(-0.2,-0.5)$) {};
            \node[commit,fill=c2] (br2) at ($(br)+(-0.2, 0.0)$) {};
            \node[commit,fill=c4] (br4) at ($(br)+(+0.2, 0.0)$) {};
            \draw[arrow] (br1) -- (br2);
            \draw[arrow] (br1) -- (br4);

            \uncover<5> {
                \draw[biconnection] (br) -- (bl);
                \draw[biconnection] (br) -- (tl);
                \draw[biconnection] (br) -- (tr);
                \draw[biconnection] (bl) -- (tr);
                \draw[biconnection] (bl) -- (tl);
                \draw[biconnection] (tr) -- (tl);
            }

            % Center
            \uncover<7-> {
                \node[repo] (cc) at ( 0,  0) {};
                \node[font=\tiny] at ($(cc)-(0,0.85cm)$) {origin};
            }
            \uncover<8-> {
                % push from br
                \node[commit,fill=c1] (cc1) at ($(cc)+(-0.4,-0.5)$) {};
                \node[commit,fill=c2] (cc2) at ($(cc)+(-0.4, 0.0)$) {};
                \node[commit,fill=c4] (cc4) at ($(cc)+( 0.0, 0.0)$) {};
                \draw[arrow] (cc1) -- (cc2);
                \draw[arrow] (cc1) -- (cc4);
                \draw[connection,onslide=<12>biconnection] (br) -- (cc);
            }
            \uncover<9-> {
                % push from bl
                \node[commit,fill=c3] (cc3) at ($(cc)+(-0.4,+0.5)$) {};
                \draw[arrow] (cc2) -- (cc3);
                \draw[connection,onslide=<12>biconnection] (bl) -- (cc);
            }
            \uncover<10-> {
                % push from tl
                \node[commit,fill=c5] (cc5) at ($(cc)+( 0.0, 0.5)$) {};
                \draw[arrow] (cc4) -- (cc5);
                \draw[connection,onslide=<12>biconnection] (tl) -- (cc);
            }
            \uncover<11-> {
                % push from tr
                \node[commit,fill=c6] (cc6) at ($(cc)+(+0.4, 0.5)$) {};
                \draw[arrow] (cc4) -- (cc6);
                \draw[connection,onslide=<12>biconnection] (tr) -- (cc);
            }
        \end{tikzpicture}
    }
\else
    GRAPH!
\fi
\end{frame}

\begin{frame}{Remotes}
    \begin{itemize}
        \item Referenz auf entferntes Repository (clone)
        \item Erstellen mit
            \bash{git remote add <name> <url>}
        \item ``Standard''-Remote \emph{origin} wird beim \emph{clone}n erstellt
        \item Operationen
            \begin{description}
                \item[fetch] Objekte herunterladen
                \item[pull] \emph{fetch} + \emph{merge}
                \item[push] Objekte hochladen + Referenzen aktualisieren (branches)
            \end{description}
    \end{itemize}
\end{frame}

\begin{frame}{Pull / Push}
    \textbf{Push}\\[0.5em]

    \bash{git push [options] <remote> <from-ref>:<to-ref>}
    \bash{git push -u origin master}
    \bash{git push -u origin :useless-branch}
    \bash{git push}

    \vspace{1em}

    \textbf{Pull}\\[0.5em]

    \bash{git pull <remote> <branch>}
    \bash{git pull origin master}
    \bash{git pull}
\end{frame}

\begin{frame}{Und wann mach ich das jetzt?}
    \pause

    \begin{itemize}
        \item vor dem Arbeiten: \textbf{pull} (am aktuellen Stand arbeiten)
        \item Änderungen durchführen
        \item nach jeder Einheit (z.B. ein Bugfix, ein kleines Feature): \textbf{commit}
        \item möglichst häufig: \textbf{push} (minimiert Merge-Konflikte)
    \end{itemize}

    \pause

    \vspace{1cm}

    \begin{quote}
        Commit often and early!
    \end{quote}
\end{frame}

