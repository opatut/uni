\section{Anhang}

\subsection{Tipps und Tricks}
\begin{frame}[fragile]{Tipps und Tricks}
    \begin{itemize}
        \item Log mit Änderungen (diff) zeigen:
            \bash{git log -p}
        \item Bunter Log mit Baum-Struktur:

{\scriptsize\begin{verbatim}
git log --graph --all --format=format:'%C(yellow)%h%C(reset) -
    %C(cyan)%ci%C(reset) %C(green)(%cr)%C(reset) %C(bold yellow)%d%C(reset)%n
    %C(white)%s%C(reset) %C(bold white)- %cn%C(reset)%n'
    --abbrev-commit --date=relative
\end{verbatim}}
        \item \texttt{$\sim$/.gitconfig} anpassen!
    \end{itemize}
\end{frame}

\begin{frame}[fragile]{$\sim$/.gitconfig}
\begin{verbatim}
[user]
    name = Firstname Lastname
    email = user@host.tld

[color]
    diff = auto
    status = auto
    branch = auto
    interactive = auto
    ui = true

[push]
    default = simple
\end{verbatim}
\end{frame}


\subsection{Übersicht}
\begin{frame}{Übersicht}
    \textbf{Begriffe}\\[0.5em]

    {\scriptsize
        \begin{tabular}{p{2.5cm}l}
            \textbf{commit}
                & Versions-Snapshot\\
            \textbf{working directory}
                & Arbeitskopie der aktuell gewählten Version\\
            \textbf{clone}
                & Kopie eines Repositories\\
            \textbf{remote}
                & Referenz im lokalen Repository auf (entfernten) clone\\
            \textbf{branch}
                & Zeiger auf einen Zweig der History, wird aktualisiert\\
        \end{tabular}
    }\\[1em]

    \textbf{Kommandos}\\[0.5em]

    {\scriptsize
        \begin{tabular}{p{2.5cm}l}
            \textbf{clone}
                & ein Repository von einer URL kopieren (init + remote add + pull)\\
            \textbf{status}
                & aktuellen Status des working directories anzeigen\\
            \textbf{add}
                & Dateien/Verzeichnisse stagen\\
            \textbf{commit}
                & Version in Kontrolle aufnehmen\\
            \textbf{pull}
                & commits auf remote übertragen und remote-branch updaten\\
            \textbf{push}
                & commits von remote empfangen und in aktuellen branch mergen\\
            \textbf{log}
                & Vorgänger-Versionen auflisten\\
            \textbf{checkout}
                & bestimmte Version für einzelne Dateien oder
                das WD auswählen\\
            \textbf{rm}
                & Dateien löschen\\
            \textbf{branch}
                & Branches verwalten (auswählen mit \texttt{checkout})\\
            \textbf{diff}
                & Versionen vergleichen\\
            \textbf{merge}
                & zwei Änderungen zusammenführen\\
        \end{tabular}
    }
\end{frame}
