\section{Lokale Operationen}

\begin{frame}{}
    \vspace{2cm}
    \center \Huge \texttt{git init}
\end{frame}

\subsection{Status}
\begin{frame}{Status / Diff}
    Aktuellen Status des Working Directories:\\[1em]

    \bash{git status}

    \vspace{1em}
    \pause

    Änderungen zwischen Versionen:\\[1em]

    \bash{git diff}
    \bash{git diff myfile.txt}
    \bash{git diff some-ref other-ref}
\end{frame}

\againframe{life}

\subsection{Staging \& Committing}
\begin{frame}{Staging \& Committing}
    Dateien zum Index (Stage) hinzufügen:\\[1em]

    \bash{git add <files/directories>}
    \bash{git add -i}

    \vspace{1em}

    Commit-Objekt erzeugen:\\[1em]

    \bash{git commit}
    \bash{git commit -m "Message"}
    \bash{git commit -a -m "Message"}
\end{frame}

\againframe{life}

\subsection{Checkout}
\begin{frame}{Checkout}
    \begin{itemize}
        \item Erinnerung: aktuelle Version liegt im Working Directory
        \item diese können wir mit per \textsc{checkout} wechseln
            \bash{git checkout 4b5c8e2f95a4407c4d0c596565b367eaca07af57}
            \bash{git checkout other-branch}
            $\rightarrow$ nicht möglich, wenn unversionierte Änderungen vorliegen
        \item einzelne Dateien können auch auf HEAD (oder speziellen commit)
            zurückgesetzt werden
            \bash{git checkout edited-file.txt}
            \bash{git checkout aa5bcd5a some-file.txt}
    \end{itemize}
\end{frame}

\againframe{life}

\subsection{Log}
\begin{frame}{Log}
    \begin{itemize}
        \item zeigt History an
        \item nur von HEAD rückwärts
        \item enthält Commit-Nachricht, Datum, Autor, ...
        \item Befehl:
            \bash{git log}
    \end{itemize}
\end{frame}

\subsection{Branching}
\begin{frame}{Branching}
    \begin{itemize}
        \item Commit-Objekte liegen ungeordnet vor
        \item $\rightarrow$ welche ist die neueste Version?
        \item ein \textbf{Branch} kann auf einen Commit zeigen (``Pointer'')
        \item Branches haben \textbf{Namen} (z.B. \textit{master} oder \textit{my-feature})
        \item man kann zwischen Branches wechseln wie zwischen Commits (\textsc{checkout})
        \item nach dem committen zeigt der aktuelle Branch auf den neuen Commit
    \end{itemize}
\end{frame}

\begin{frame}{Branching}
    \only<1>{\bash{git status}}
    \only<2>{\bash{git checkout -b foobar}}
    \only<3>{\bash{git commit -m \qq{D}}}
    \only<4>{\bash{git checkout master}}
    \only<5>{\bash{git checkout -b hotfix}}
    \only<6>{\bash{git commit -m \qq{E}}}
    \only<7>{\bash{git checkout master}}
    \only<8>{\bash{git merge hotfix}}
    \only<9>{\bash{git branch -d hotfix}}
    \only<10>{\bash{git checkout foobar}}
    \only<11-12>{\bash{git commit -m \qq{F}}}
    \only<13>{\bash{git checkout master}}
    \only<14>{\bash{git merge foobar}}
    \only<15>{\bash{git branch -d foobar}}

    \vspace{1em}

\iffinal
    \makebox[\textwidth][c]{
        \begin{tikzpicture}[x=3em,y=2em]
            \uncover<1-> {
                \node[commit] (a) at (0, 0) {A};
                \node[commit] (b) at (1, 0) {B};
                \node[commit] (c) at (2, 0) {C};

                \draw[child] (a) -> (b);
                \draw[child] (b) -> (c);
            }

            \uncover<1-7> {
                \node[branch,onslide=<1>current,onslide=<4>current,onslide=<7>current] (master) at (2, 1) {master};
                \draw[pointer] (master) -> (c);
            }

            \uncover<2> {
                \node[branch,current] (foobar) at (2, -1) {foobar};
                \draw[pointer] (foobar) -> (c);
            }

            \uncover<3-> {
                \node[commit] (d) at (3, -1) {D};
                \draw[child,onslide=<12-13>{red}] (c) -> (d);
            }

            \uncover<3-10> {
                \node[branch,onslide=<3>current,onslide=<10>current] (foobar) at (3, -2) {foobar};
                \draw[pointer] (foobar) -> (d);
            }

            \uncover<5> {
                \node[branch,current] (hotfix) at (2, 2) {hotfix};
                \draw[pointer'] (hotfix) -- (master);
            }

            \uncover<6-> {
                \node[commit] (e) at (3, 0) {E};
                \draw[child,onslide=<12-13>{green}] (c) -> (e);
            }

            \uncover<6-8> {
                \node[branch,onslide=<6>{current}] (hotfix) at (3, 1) {hotfix};
                \draw[pointer] (hotfix) -> (e);
            }

            \uncover<8> {
                \node[branch,current] (master) at (3, 2) {master};
                \draw[pointer'] (master) -> (hotfix);
            }

            \uncover<9-13> {
                \node[branch,onslide=<9>current,onslide=<13>current] (master) at (3, 1) {master};
                \draw[pointer] (master) -> (e);
            }


            \uncover<11-> {
                \node[commit] (f) at (4, -1) {F};
                \draw[child,onslide=<12-13>{red}] (d) -> (f);
            }

            \uncover<11-14> {
                \node[branch,onslide=<-12>current] (foobar) at (4, -2) {foobar};
                \draw[pointer] (foobar) -> (f);
            }

            \uncover<14-> {
                \node[commit] (g) at (5, 0) {G*};
                \draw[child] (e) -> (g);
                \draw[child] (f) -> (g);
            }

            \uncover<14-> {
                \node[branch,current] (master) at (5, 1) {master};
                \draw[pointer] (master) -> (g);
            }

            \uncover<15> {}
        \end{tikzpicture}
    }
\else
    -- tikz picture here, finaltrue to show --
\fi
\end{frame}

\subsection{Merging}
\begin{frame}{Merging}
    Was passiert?
    \begin{itemize}
        \item gemeinsamen Vorgänger finden
        \item Änderungen ermitteln
        \item beide Änderungssätze auf gemeinsamen Vorgänger anwenden
        \item neuen Commit erstellen (automatische \emph{message})
    \end{itemize}

    Konflikte
    \begin{itemize}
        \item bei Änderung gleicher Zeile kann git nicht entscheiden, welche
            Änderung übernommen werden soll
        \item
            {\scriptsize
            \texttt{Auto-merging <filename>}\\
            \texttt{CONFLICT (content): Merge conflict in <filename>}\\
            \texttt{Automatic merge failed; fix conflicts and then commit the result.}\\
            }
    \end{itemize}
\end{frame}

\begin{frame}[fragile]{Merging - Konflikte}

    \begin{verbatim}
vor dem Konflikt
<<<<<<< HEAD
meine Änderungen
=======
deine Änderungen
>>>>>>> fremder-branch
nach dem Konflikt
    \end{verbatim}
\end{frame}

\subsection{Rebasing}
\begin{frame}{Rebasing}
    \only<1>{\bash{git checkout foobar}}
    \only<2>{\bash{git rebase master}}

    \vspace{1em}

\iffinal
    \makebox[\textwidth][c]{
        \begin{tikzpicture}[x=3em,y=2em]
            \node[commit] (a) at (0, 0) {A};
            \node[commit] (b) at (1, 0) {B};
            \node[commit] (c) at (2, 0) {C};
            \node[commit] (e) at (3, 0) {E};
            \draw[child] (a) -> (b);
            \draw[child] (b) -> (c);

            \uncover<1> {
                \node[commit] (d) at (3, -1) {D};
                \node[commit] (f) at (4, -1) {F};
                \draw[child,green] (c) -> (e);
                \draw[child,red] (c) -> (d);
                \draw[child,red] (d) -> (f);

                \node[branch,current] (foobar) at (4, -2) {foobar};
                \draw[pointer] (foobar) -> (f);
            }
            \uncover<2-> {
                \node[commit] (d) at (4, 0) {D'};
                \node[commit] (f) at (5, 0) {F'};
                \draw[child,green] (c) -> (e);
                \draw[child,red] (e) -> (d);
                \draw[child,red] (d) -> (f);

                \node[branch,current] (foobar) at (5, 1) {foobar};
                \draw[pointer] (foobar) -> (f);
            }

            \node[branch] (master) at (3, 1) {master};
            \draw[pointer] (master) -> (e);
        \end{tikzpicture}
    }
\else
    -- tikz picture here, finaltrue to show --
\fi
\end{frame}
