\section{Glossar}

\begin{description}

\item[Agent] Natürliches oder künstliches System mit gewissen Eigenschaften (Oberbegriff).

\item[Chiastischer Informationsweg] Die Nervenbahnen überkreuzen sich, linke Gehirnhälfte
    verarbeitet Informationen rechter Sinnesorgane und vice versa. Vorhanden u.A. bei
    visueller und auditiver Kognition.

\item[Ergonomie] Wissenschaft der Leistungsmöglichkeit und -grenzen von Menschen. Anpassung
    der Arbeitsumgebung an menschliche Fähigkeiten.

\item[Kognition] Prozesse des Denkens (Folgern, Probleme lösen),
    Kommunizierens (Sprache, Gestik, Grafik), Gedächnis

\item[Kognitive Artefakte] Objekte zur Unterstützung der menschlichen Kognition,
    Werkzeuge zur Unterstützung \emph{geistiger Prozesse}.

\item[Large Grain Feature] Nicht-exklusive Einteilung der Gehirnareale nach kognitiver
    Funktion, vgl. \textsc{Phrenologie}.

\item[Motorik] Fähigkeit der Bewegung (Gehen, Greifen)

\item[Perzeption] Wahrnehmung (Sehen, Hören, Tasten, ...)

\item[präattentiv] ohne fokussierte Aufmerksamkeit

\item[Phrenologie] Zusammenhang zwischen kognitiver Funktion und Areal des Gehirns.

\end{description}
