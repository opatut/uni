\section{01 - Einleitung}

Der Mensch und der Computer sind gut erforscht, nur die Verbindung dazwischen nicht: ,,missing link''.

Schreiben: Schrift entsteht am Werkzeug. Tippen: Schrift entsteht woanders - man muss ,,blind tippen''.

Maus: Hand-Auge Koordination, Alignment!

\subsection{Computersysteme zur Problemlösung}

Ist die Interaktion zwischen Benutzer und Computer kognitiv und perzeptiv auf den
Benutzer abgestimmt? Versteht der Benutzer, was der Computer tut? Versteht der
Benutzer, was er tun muss?

Das \textbf{Herddesign}-Beispiel zeigt, wie die Anordnung der Knöpfe beim \emph{natural mappings}-Design
an die Fähigkeit/Eigenschaft des Menschen angepasst werden, nämlich die Anordnung
analog von den Schaltern auf die Platten anzuwenden.

Eine \textbf{Uhr} soll einfach/schnell und präzise/korrekt abgelesen werden können.
Vorteil der Analoguhr: kein Zahlenverständnis. Vorteil der Digitaluhr: einfaches
Ablesen. Negativ-Beispiel: Berlinuhr (es muss \emph{gerechnet} werden).
