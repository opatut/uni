\section{02 - Grundlagen der Informationsverarbeitung}

Die \textbf{Informatik} befasst sich mit Informationen. Die
\textbf{Kognitionswissenschaft} beruht auf der Informationsverarbeitung.
\textbf{Kognitive System} (oder ,,Agenten'') sind gemeinsames Forschungsthema.

\textbf{Daten} sind Artefakte, welche Inhalte speichern. \textbf{Eine Information}
lässt sich aus Daten interpretieren, wenn ein Zusammenhang (Vorwissen, Hintergrundwissen,
aktuelle Umgebung) gegeben sind. \textbf{Wissen} ergibt sich aus einer Anhäufung
relevanter Informationen und deren Anwendung.

\begin{tabular}{|p{20em}|p{15em}|}
\hline
Das System muss auf Fähigkeiten des Menschen angepasst sein
& Kenntnisse über Kognition, Perzeption, Motorik\\ \hline

Der Mensch muss verstehen, was das System tut
& Ein- und Ausgabeschnittstellen \\ \hline

Kommunikation zwischen Computer und Mensch muss ,,funktionieren''
& Prozesse der Interaktion\\ \hline
\end{tabular}

Menschen sind nur sehr eingeschränkt in der Lage, kognitive und perzeptive Fähigkeiten
durch Training zu verbessern (Beispiel: Blinde hören nur etwas besser als sehende Menschen).

Eingabeinformationen werden durch eine Operation (möglicherweise komplexe Operation bestehend
aus mehrerenen elementaren Operationen) in Ausgabeinformationen überführt, die wiederum zu
Verhalten/Aktionen führen.

Die interne Struktur ist nicht immer zu beobachten (Black Box), nur das Verhalten. Empirisch
(Experiment und Beobachtung) lässt sich in beschränktem Umfang auf die Operation schließen.
