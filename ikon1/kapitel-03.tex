\section{03 - Neurowissenschaftliche und Neuroinformatische Grundlagen}

\textbf{Sensor-Neurone} erkennen physikalische/chemische Signale, \textbf{Motor-Neurone}
steuern die Muskelkontraktion, \textbf{Interneurone} übertragen die Signale.

Interneurone haben 2 Funktionen: Integration der Eingangs-Information und Weiterleitung
an andere Neurone.

Hirnregionen haben spezielle Aufgaben, allerdings nicht ausschließlich. Eine Region kann
auch Neuronen enthalten, die andere Aufgaben ausführen, und gewisse Neuronengruppen können
auf ,,fremde'' Aufgaben übernehmen (\emph{Plastizität}). Diese nicht-exklusive Einteilung
heißt \textsc{Large Grain Feature}.

Neuronen empfangen Signale an den \emph{Dendriten}, integrieren diese am \emph{Zellkörper}, leiten
sie über die \emph{Axone} zu den \emph{Terminalen}, wo sie an \emph{Synapsen} an die Dentriten anderer
Neuronen weitergegeben werden. \emph{Exzitatorische} Synapsen erhöhen den Eingabe-Wert,
\emph{inhibitorische} verringern ihn. Wird ein Schwellwert (\emph{threshold}) erreicht, ,,feuert'' das Neuron.

In künstlichen \emph{neuronalen Netzen} wird dies simuliert. Eingabeverbindungen werden
mit den Konnektions-Gewichten multipliziert und zum inneren Produkt addiert (1. Phase). In einer 2. Phase
wird dieser Wert durch eine Funktion abgebildet auf einen Ausgangs-Wert. Diese Funktion
kann unter anderem linear, beschränkt linear, nichtlinear oder treppenförmig (Schwellwert) sein.

Neuronen im Gehirn sind zwar stark vernetzt, jedoch nicht nur regional, sonder weit
verteilt. Das ist wichtig für \emph{Vorwärts- und Rückwärtsprojektion}, um Informationen
aus verschiedenen Arealen zu kombinieren (z.B. Hintergrundwissen und visuelle Information).
