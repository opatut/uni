\section{VL 13: IT und Frauen}


\subsection{Warum dieses Thema wiederbeleben?}

\begin{itemize}
	\item Informatik gewinnt an Bedeutung
	\item Beherrschbarkeit der Komplexität
	\item IT-Fachkräftemangel wird zum Standortproblem
	\item Immer noch wenige Frauen in der Informatik tätig
\end{itemize}


\subsection{Beiersdorf}

\begin{itemize}
	\item Die meisten Kunden von Beiersdorf sind weiblich
	\item Frauenanteile der jeweiligen Stufen
\end{itemize}

\begin{tabularx}{0.6\textwidth}{|X|X|X|X|} \hline
	AT 3   & FK 3   & FK 2   & Fk 1  & Board \\\hline
	50.2\% & 22.2\% & 20.3\% & 7.3\% & 0\% \\\hline
\end{tabularx}

\begin{itemize}
	\item 15\% IT-Fachkräfte
	\item 6\% IT-Managerinnen
	\item 18\% Studienanfängerinnen im MINT-Bereich
\end{itemize}


\subsection{Die Probleme}

\begin{itemize}
	\item Trotzdem habe die Frauen keine besonderen Hindernisse zu überwinden, wenn sie in dem IT-Bereich aktiv sind
	\item »Frauenmangel« ist eher förderlich für die Frauen in der Informatik, da Frauen erwünscht sind und sie somit schneller aufsteigen können
	\item Einige Männer (Stichwort Uni-Lübeck) sind noch vorurteilsbelastet
\end{itemize}