\newpage
\section{VL 10: Kontexte verändern sich}
\subsection{Veränderung: Auf dem Weg zur Dienstleistungsgesellschaft}
\begin{multicols}{2}
\textbf{Arbeitsteilung nach F.W. Taylor}
\begin{itemize}
\item Trennung von Hand- und Kopfarbeit (Arbeitsleiter und normaler Arbeiter)
\item Arbeitsleiter optimiert Betriebsablauf [Arbeitszerlegung, Standardisierung und Formaliesierung]
\item Arbeiter denkt nicht darüber nach, führt nur aus
\item Resultat $!$Bummelsystem = weniger faulenzen = Mehr Kontrolle über Arbeiter \\
\\
\end{itemize}
\textbf{Arbeitsteilung nach Henry Ford}
\begin{itemize}
\item Fließbandarbeit ftw, da Arbeiter möglichst wenig außer ihrer eigentlichen Tätigkeit tun (wie etwa Werkzeug holen gehen)
\item einzelne Mitarbeiter sind egal, nur die Fließband-Organisation muss optimiert werden
\item Zerlegung der Arbeit in einzelne Schritte
\item Spezialisierung: Jeder kann 1-2 Handgriffe ganz toll + kein Zeitverlust durch Wechsel zwischen Aufgaben
\end{itemize}
\end{multicols}

\subsection{Veränderung in Dienstleistungen, Auswirkungen}
Wenn Arbeitsprozesse optimiert werden, besteht die Gefahr des Silo-Denkens ( Jeder Aufgabenbereich wird einzeln optimiert aber die Zusammenarbeit läuft immernoch so schlecht wie vorher. \\
Auswege:
\begin{itemize}
\item Aufbauorganisation: Flachere Hierarchien (Weniger Chefetagen + Normale Mitarbeiter haben mehr Entscheidungsmacht)
\item Ablauforganisation: Business Process Reengineering (Obvious)
\item Informationssysteme: Enterprise Ressource Planning (Geschäftsprozesse automatisieren)
\end{itemize}
Beispiel der Veränderung: Outsourcing durch Call-center
\begin{itemize}
\item ist nicht immer cool, wegen Sprachbarrieren zwischen Kunden und Leuten aus Indien, keine Verlässlichkeitsgarantie, keine Qualitätsgarantie für das Unternehmen..
\end{itemize}
	\subsubsection{Auswirkung auf Kunden : Der Arbeitende Kunde}
	\begin{itemize}
	\item Wandlung des Kunden vom Konsumenten zum unbezahlten Mitarbeiter
	\item Bsp.: Du nimmst deine Sachen selbst aus dem Regal bei Edeka, Bestellst deine Flug/Bahn-tickets selber, baust dein Regal von Ikea selbst auf (Ikea-Prinzip)
	\item Kunde als Wertquelle (Feedback, Bewertungen, Produktvorschläge) => Crowdsourcing
	\item
	"Von Crowdsourcing ist dann zu sprechen, wenn Unternehmen zur Herstellung oder Nutzung eines Produktes bis dahin intern erledigte Aufgaben in Form eines offenen Aufrufes über das Internet auslagern". Kleeman/Voß/Rieder
	\item
	"The act of taking a job traditionally performed by a designated agent (usually an employee) and outsourcing it to an undefined, generally large group of people in the form of an open call". Howe
	\end{itemize}
	\begin{multicols}{2}
	\subsubsection{Auswirkung auf Unternehmen}
	Open Innovation Modell: Open Source, Kooperationen mit anderen Entitäten als dem Unternehmen \\ 
	Interaktive Wertschöpfung = Open Innovation + Crowdsourcing \\ Freiwilliger Interaktionsprozess , Kundenintegration \\
	Closed Innovation Modell: Gegenteil halt
	\subsubsection{Auswirkung auf Mitarbeiter}
	\begin{itemize}
	\item Personalabbau
	\item Freie Zeiteinteilung
	\item Lose Bindung an das Unternehmen
	\item Wenig soziale Absicherung
	\end{itemize}
	=> Freelancer
	\end{multicols}