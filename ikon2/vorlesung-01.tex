\section{VL 01: Motivation}

In dieser Vorlesung geht es im Prinzip darum, dass der Kontext wichtig für Informatiker ist, da:
\begin{itemize}
	\item Informatik den Kontext verändert
	\item Informatik Teil des Kontextes ist
	\item Informatik neue Kontexte schafft
	\item Kontexte die Informatik verändern
\end{itemize}

Kontexte können sein:
\begin{itemize}
	\item Gesellschaft
	\item Organisationen
	\item Geschäftsmodelle / -prozesse
	\item Dienstleistungen
	\item Individuen
\end{itemize}

Informatiker müssen den Kontext \em dekontextualisieren\em, d.h. ihn \em verstehen, analysieren und modellieren \em können. Anders herum müssen sie auch Informatiksysteme \em rekontextualisieren \em, d.h. sie \em verändern, einführen und warten \em können.

{\large \textbf{Die Kontexte sind miteinander verzahnt, d.h. sie beeinflussen einander positiv und negativ. Die IT ist mit den Kontexten verzahnt, ihre Dynamik nimmt zu.}}
