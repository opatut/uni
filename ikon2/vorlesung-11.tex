\newpage
\section{VL 11: Kontexte sind verzahnt - GreenIT}

Harald Welzer, spiegel online: So kann es nicht weitergehen, ökologisch und
finanziell. Ressources werden zur Neige gehen.  Klimawandel ist Scheiße --
deshalb müssen wir weniger Strom verbrauchen.

Leider ist wirtschaftliche Entwicklung regional orientiert. ,,Sollen doch die anderen
machen!'' Hippies haben in den 80ern angefangen mit \emph{Recycling}. Das war mega
cool (Kreislaufwirtschaft)! Und in den 90ern haben auch Unternehmen angefangen, möglichst
Müll-effizient (ökologische Nachhaltigkeit) zu \emph{produzieren}.

Es wird immer mehr Energie verbraucht (wirklich? Wahnsinn!). Warum handeln Unternehmen
jetzt?

\begin{enumerate}
\item Treibhausgase und Ressourcen einsparen (\emph{ökologisch}, wie langweilig)
\item Geld sparen, weil Energie immer teurer wird (\emph{ökonomisch}, yay, Wirtschaft!)
\end{enumerate}

Seit 2008 wird von Unternehmen versucht, viel Strom in IT zu sparen, insbesondere
im Serverbetrieb. Einige sagen, ,,GreenIT'' sei nur ein Deckmantel, um neue (stromsparende) Hardware
zu verkaufen, aber neue Hardware herzustellen koste viel mehr Strom, als sie im
Endeffekt einspart. Man muss beide Seiten betrachten:

\begin{multicols}{2}
\subsection{GreenIT im engeren Sinne}

\begin{itemize}
\item Sparsame Hardware
\item Thin-Clients
\item effizientere Kühlsysteme
\item Energieoptionen, Herunterfahren!
\item ökologisches Design
\item Server-Virtualisierung
\end{itemize}

\subsection{GreenIT im weiteren Sinne}

\begin{itemize}
\item Herstellung von Hardware (Rohstoffe, Chemikalien, Energie)
\item Entsorgung von Hardware (Recyclingaufwand)
\item Energieaufwand für Betrieb (siehe oben)
\item Software-Hardware-Beziehung (braucht man wirklich neue Hardware?)
\end{itemize}
\end{multicols}

Und bitte, schickt nicht die alten Computer nach Afrika zu armen Kindern zum Recycling!
Das ist ungesund und schlecht für die Umwelt.

\subsection{Dematerialisierung und Reboundeffekt}

\textbf{Dematerialisierung:} Ein vorher physisch vorhandenes Objekt wird mit neuer Technologie weniger/kleiner oder
fällt weg. Beispiele: elektronische Dokumente, mp3-Download statt CD kaufen.
$\Rightarrow$ \textbf{Erhöhung der Ressourcenproduktivität}

\textbf{Reboundeffekt:} Weil Dinge Ressourcen sparen, werden sie günstiger, deshalb werden mehr genutzt, und
wieder mehr Ressourcen verbraucht. Beispiel: früher gabs ein paar große Mainframes, dann kamen
PCs, jetzt gibts davon Millionen.

Software kann solche Dinge modellieren und beim Ressourcen sparen/Klimaschutz helfen. Yay!
