\newpage
\section{VL 07: Kontext Individuum -- Technologieakzeptanz}

\begin{multicols}{2}
Wesentlich im Lebenszyklus eines Informationssystems ist die Einführung - hier entscheidet
sich der Erfolg und die Bedeutung. Damit Informationssysteme \textbf{Nutzen} stiften können,
müssen sie (korrekt) \textbf{genutzt} werden:

\begin{center}{\large,,Vor dem \emph{Nutzen} kommt die \emph{Nutzung}.''}\end{center}

Um Akzeptanz und Verständnis zu erreichen, ist ein geplanter \textbf{Einführungsprozess}
des Systems innerhalb der Organisation notwendig.

\begin{center}{\textbf{Einführung von Informationssystemen}:\\
Eine organisatorische Maßnahme zur Verbreitung (Installation) und\\
Aneignung (Verwendung) von Informationstechnik in einer Nutzergruppe.}\end{center}

Wie breitet sich eine Innovation aus? Rogers Prozess der \textbf{Diffusion:} Mitglieder eines
sozialen Systems kommunizieren über verschiedene Kanäle $\Rightarrow$ Es muss über die
Innovation gesprochen/geschrieben werden.
\\
\\

\begin{center}
\textbf{Schritte der Aneignung einer Innovation}\\
Wissen (\emph{aha, sowas gibts!})\\
Überzeugung (\emph{das klingt gut!})\\
Entscheidung (\emph{ich probier das aus})\\
Umsetzung (\emph{click here to download, next, agree, next, next, finish})\\
Bestätigung (\emph{is ja echt voll geil!})
\end{center}

Merkmale des Individuums beeinflusses sein Wissen, Merkmale der Innovation beeinflussen
seine Überzeugung. Der Kommunikationskanal muss auf die Zielgruppe angepasst werden.

\textbf{Innovationsfreudigkeit}\\
2.5\% Innovatoren\\
13.5\% Frühe Nutzer\\
34\% Frühe Mehrheit\\
34\% Späte Mehrheit\\
16\% Nachzügler

\textbf{Innovations-Merkmale nach Rogers:}\\
\emph{Wahrgenommener} Vorteil\\
Komplexität (\emph{vs. Einfachheit})\\
Kompatibilität (\emph{technisch, sozial, ...})\\
Beobachtbarkeit (\emph{Preview bei anderen})\\
Probierbarkeit (\emph{Demo-Version etc.})
\end{multicols}

Leistungserwartung + Aufwandserwartung + Sozialer Einfluss = Nutzungsintention

Nutzungsintention + Fördernde Bedingungen = Nutzungsverhalten

Ansonsten gibt's immer \emph{Moderatorgrößen}, die ,,halt einfach noch eine
Rolle spielen'' (so wie das Wetter den Genuss beim Eis-Essen beeinflusst). Typische
Faktoren: Alter, Geschlecht, ... .
