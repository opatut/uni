\section{VL 04: Kontext Organisation}

\textbf{Organisation:} Ordnung, die zielgerichtet arbeitsteilig Aufgaben und
Tätigkeiten regelt. -- Zweckorientiertes soziales Gebilde.

Eine Organisation muss \textbf{Koordinieren}, \textbf{Motivieren}, und eigentlich auch
Orientieren (Knowledge Sharing), aber darum geht's nicht in IKON.

Dazu gehört Arbeitsteilung -- \textbf{horizontale Arbeitsteilung} nach Projekten, Regionen, Objekten und Funktionen,
\textbf{vertikale Arbeitsteilung} heißt auch ,,Hierarchie''. Dabei ist es für den \emph{sinvollen} Einsatz von IT wichtig das zugleich auch dezentralisiert wird (Siehe Vorlesung 1).

\subsection{Koordination}

\begin{description}
    \item[Leitungsbeziehungen] Wer darf wen kommandieren?
    \item[Standardisierung] Generelle Regelungen
    \item[Delegation] Du machst das!
    \item[Partizipation] Wir bauen zusammen was tolles...
\end{description}

\subsection{Motivation}

\begin{description}
    \item[Extrinsisch] Belohnung / Bestrafung (finanziell oder reputationell) / Zielvereinbarung
    \item[Intrinsisch] Individuelles Bestreben, Interesse, ...
\end{description}

\begin{center}{\large Bestimmt IT unser Handeln \emph{oder} bestimmt unser Handeln die IT?}\\
{\footnotesize Antwort vom Troll-Prof: Beides! \includegraphics[width=0.3cm]{wichtige-anmerkung.png}}\end{center}

Weiterer Vorlesungsinhalt: Open Source ist awesome! Und funktioniert nur wegen IT (CVS, Mail, Bugtracker, Github!).

\subsection{Technochange}

\begin{center}{\large \emph{Zusammenspiel} zwischen IT und \emph{Organisations}veränderung.}

... aka neue Technik, die Geschäftsprozesse etc. beeinflusst.
\end{center}
Dabei ist es erforderlich das eine wirklich Zusammenarbeit von IT und Restrukturierung der Organisation stattfindet. Sonst ist es kein Technochange Projekt!1!!!
Es handelt sich dann entweder um ein IT- oder um ein Organisationsprojekt.