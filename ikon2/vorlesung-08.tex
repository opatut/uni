\section{VL 08: Kontext Markt}
IT Dienstleistungen und Cloud Computing

Der IT-Markt besteht aus Software, Hardware und IT-Dienstleistungen, wobei die Kategorie Software noch einmal in die Unterkategorien System-Infrastruktur, Werkzeuge und Anwendungs-Software unterteilt werden kann.

\begin{itemize}
\item IT-Leistungen werden zum Teil an andere Unternehmen weitergeleitet, falls das billiger ist, als es selbst zu machen (= Outsourcing). Das geht in jedem Bereich des IT-Marktes.
\item In jedem Unternehmen wird Software anders gehandhabt: Manche entwickeln alles selbst, andere setzen auf Standardsoftware und Dienstleistungen. Der Trend geht allerdings zur Standardsoftware mit Dienstleistungen hin
\item Trend hin zu Cloud-Diensten ( "Public Cloud" ): 
\begin{itemize}
\item "Kann ich nicht statt den Einzelteilen gleich das fertige Paket haben?"
\item Kapazität einfach bestellbar ohne Zusatzaufwand
\item Für die Anbieter: Optimalere Auslastung durch Nutzergruppen in verschiedenen Zeitzonen
\item Große Rechenzentren an Orten mit Kostenvorteilen (Antarktis -> Kühlung)
\item Bei großen Unternehmen kann man beide Seiten intern kombinieren: "Private Cloud"
\end{itemize}
\item Hypes verlaufen in einer $\sim$ -Kurve. Siehe dazu den "Gartner Emerging Technologies Hype Cycle".

\end{itemize}