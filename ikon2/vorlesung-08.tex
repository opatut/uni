\section{VL 08: Kontext Markt}
IT Dienstleistungen und Cloud Computing

\begin{itemize}
\item IT-Leistungen werden zum Teil an andere Unternehmen weitergeleitet, falls das billiger ist, als es selbst zu machen
\item Zum geringen Anteil Eigenentwicklung, wegen Ideen-schutz
\item Generell gilt: Standardsoftware > Individualisierte Software (von den Kosten her)
\item IT-Markt besteht aus:
\begin{itemize}
\item IT-Beratung
\item System-integration (Anpassung/Wartung von Software)
\item IT-Training
\end{itemize}
\item Trend hin zu Cloud-Diensten ( "Public Cloud" ): 
\begin{itemize}
\item ("Kann ich nich anstatt den Einzelteilen gleich das fertige Paket haben?")
\item Kapazität einfach bestellbar ohne Zusatzaufwand
\item Cloud-Service-Anbieter: Optimalere Auslastung durch Nutzergruppen in verschiedenen Zeitzonen z.B.
\item Cloud-Service-Anbieter: Große Rechenzentren an Orten mit Kostenvorteilen
\item Bei großen Unternehmen kann man beide Seiten intern kombinieren: "Private Cloud"
\end{itemize}
\item Hypes verlaufen in ner ~ -Kurve

\end{itemize}