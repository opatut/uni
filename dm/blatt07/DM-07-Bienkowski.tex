\newcommand{\authorinfo}{Paul Bienkowski, Jascha Andersen, Benedikt Bushart}
\newcommand{\titleinfo}{DM 07-B (HA) zum 07.11.2012}

% PREAMBLE ===============================================================

\documentclass[a4paper,10pt]{scrartcl}
\usepackage[german,ngerman]{babel}
\usepackage[utf8]{inputenc}
\usepackage[T1]{fontenc}
\usepackage{lmodern}
\usepackage{amssymb}
\usepackage{mathtools}
\usepackage{amsmath}
\usepackage{enumerate}
\usepackage{array}
\usepackage{listings}
\usepackage{fullpage}
\usepackage{breqn}
\usepackage{fancyhdr}
\usepackage{lastpage}
\usepackage{tikz}
\usetikzlibrary{decorations.markings}

\author{\authorinfo}
\title{\titleinfo}
\date{\today}

\pagestyle{fancy}
\fancyhf{}
\fancyhead[L]{\authorinfo}
\fancyhead[R]{\titleinfo}
\fancyfoot[C]{\thepage}
\renewcommand{\headrulewidth}{0.4pt}
\renewcommand{\footrulewidth}{0pt}
\renewcommand{\headheight}{12pt}
\renewcommand{\headsep}{12pt}

\begin{document}
\setcounter{secnumdepth}{0}
\maketitle

% DOCUMENT ===============================================================

\begin{enumerate}
    \item[\textbf{1.}]
        \begin{enumerate}
            \item[a)]
                Die Zahl $473$ ist in $\mathbb{Z}_{2413}$ invertierbar, da $473$ und $2413$ teilerfremd sind. Es gilt:

                $ggT(2413, 473) = 1$

                $\begin{array}{rcrclcl}
                    2413 &=& 473 &\cdot&  5 &+& 48 \\
                    473  &=&  48 &\cdot&  9 &+& 41 \\
                    48   &=&  41 &\cdot&  1 &+&  7 \\
                    41   &=&   7 &\cdot&  5 &+&  6 \\
                    7    &=&   6 &\cdot&  1 &+&  1 \\
                \end{array}$

                Durch Rückwärtseinsetzen lässt sich das Inverse ermitteln:

                $\begin{array}{lclcl}
                    1
                    &=& 7 - 6 \cdot 1\\
                    &=& 7 - (41 - 7 \cdot 5) &=& 6 \cdot 7 - 41 \\
                    &=& 6 \cdot (48 -  41 \cdot 1) - 41 &=& 6 \cdot 48 - 7 \cdot 41\\
                    &=& 6 \cdot 48 - 7 \cdot (473 - 48 \cdot 9) &=& 69 \cdot 48 - 7 \cdot 473\\
                    &=& 69 \cdot (2413 - 473 \cdot 5) - 7 \cdot 473
                        &=& \underbracket{-352 \cdot 473} + 69 \cdot 2413
                \end{array}$

                Es lässt sich ablesen, dass $-352 \cdot 473 \equiv 1$ (mod 2413) gilt. Dies lässt sich umformen zu

                $$-352 \cdot 473 \equiv 2061 \cdot 473 \equiv 1 \text{ (mod 2413)}$$

                Also ist 2061 das Multiplikative Inverse von $473$ in $\mathbb{Z}_{2413}$.

            \item[b)]
                Die Zahl $1672$ ist in $\mathbb{Z}_{2413}$ nicht invertierbar, da $19$ ein gemeinsamer Teiler ist.

                $ggT(2413, 1672) = 19$

                $\begin{array}{rcrclcl}
                    2413 &=& 1672 &\cdot&  1 &+& 741 \\
                    1672 &=&  741 &\cdot&  2 &+& 190 \\
                    741  &=&  190 &\cdot&  3 &+& 171 \\
                    190  &=&  171 &\cdot&  1 &+& 19  \\
                    171  &=&   19 &\cdot&  9 &+& 0
                \end{array}$

            \item[c)]
                Da $2412 \equiv -1$ (mod 2413) gilt, ist $2412$ sein eigenes Inverses in $\mathbb{Z}_{2413}$:

                $$2412 \cdot 2412 = (-1) \cdot (-1) = 1 \hspace{1cm}\text{(in }\mathbb{Z}_{2413}\text{)}$$
        \end{enumerate}

    \item[\textbf{2.}]
        Nach dem Satz von Fermat gilt $3^{18} = 1$ in $\mathbb{Z}_{19}$. Damit lässt sich ermitteln:

        $$3^{1000} = (3^{18})^{55} \cdot 3^{10} = 1^{55} \cdot 3^{10} = 3 \cdot (3^3)^3 = 3 \cdot 8^3 = 16 \hspace{1cm}\text{(in }\mathbb{Z}_{19}\text{)}$$

    \item[\textbf{3.}]
        \begin{enumerate}
            \item[a)]
                $\pi = (1,7,6)(2,10,8,5,11,13)(3,4)(9,12)$

            \item[b)]
                $\pi = (1,6)\circ(1,7)\circ(2,12)\circ(2,11)\circ(2,5)\circ(2,8)\circ(2,10)\circ(3,4)\circ(9,12)$

            \item[c)]
                sign $\pi = -1$ (ungerade)
        \end{enumerate}

    \item[\textbf{4.}]
        \begin{enumerate}
            \item[a)]
                Es gibt 3 Möglichkeiten, das erste Element des Tupels (mit Elementen aus $A$) zu belegen, dazu
                jeweils 5 Möglichkeiten für das zweite und 2 Möglichkeiten für das dritte Element. Da es auf die
                Reihenfolge der Elemente ankommt, gilt die Multiplikationsregel:

                $$3 \cdot 5 \cdot 2 = 30$$

            \item[b)]
                Eine ternäre Relation wird immer aus 3 Mengen gebildet. Da 3 Mengen zur Auswahl stehen, es auf
                die Reihenfolge der Mengen im kartesischen Produkt ankommt, und eine Menge auch mehrfach verwendet
                werden darf (z.B. $A \times A \times B$), gilt für die Anzahl der möglichen
                Relationen (nach dem Prinzip ,,ziehen mit Zurücklegen, geordnet''):

                $$3^3 = 27$$

                Hier sind auch Relationen einbezogen, in denen eine Menge mehrfach vorkommt, und somit mindestens eine
                der Mengen A, B und C nicht vorkommt. Man könnte demnach argumentieren, dies sei keine Relation über die
                drei Mengen. Um diese Möglichkeiten auszuschließen, muss das Prinzip
                ,,ziehen ohne Zurücklegen, geordnet'' gewählt werden. Dann gilt:

                $$3^{\underline{3}} = 3! = 3 \cdot 2 \cdot 1 = 6$$

        \end{enumerate}

\end{enumerate}

\end{document}
