\newcommand{\authorinfo}{Paul Bienkowski, Jascha Andersen, Benedikt Bushart}
\newcommand{\titleinfo}{DM 04-B (HA) zum 16.11.2012}

\newcommand{\FF}[2]{\Bigg\lfloor\displaystyle\frac{#1}{#2}\Bigg\rfloor}

% PREAMBLE ===============================================================

\documentclass[a4paper,10pt]{scrartcl}
\usepackage[german,ngerman]{babel}
\usepackage[utf8]{inputenc}
\usepackage[T1]{fontenc}
\usepackage{lmodern}
\usepackage{amssymb}
\usepackage{mathtools}
\usepackage{amsmath}
\usepackage{enumerate}
\usepackage{array}
\usepackage{listings}
\usepackage{fullpage}
\usepackage{breqn}
\usepackage{fancyhdr}
\usepackage{lastpage}

\author{\authorinfo}
\title{\titleinfo}
\date{\today}

\pagestyle{fancy}
\fancyhf{}
\fancyhead[L]{\authorinfo}
\fancyhead[R]{\titleinfo}
\fancyfoot[C]{\thepage}
\renewcommand{\headrulewidth}{0.4pt}
\renewcommand{\footrulewidth}{0pt}
\renewcommand{\headheight}{12pt}
\renewcommand{\headsep}{12pt}

\begin{document}
\setcounter{secnumdepth}{0}
\maketitle

% DOCUMENT ===============================================================

\begin{enumerate}
        % Aufgabe 1
    \item[\textbf{1.}]
        \begin{enumerate}
            \item[a)]
                Es gibt insgesamt $7 \cdot 7 \cdot 7 \cdot 7 \cdot 7 = 16807$ Abbildungen.

                Davon sind $7 \cdot 6 \cdot 5 \cdot 4 \cdot 3 = 2520$ injektiv.

                Es gibt $7 \cdot 7 \cdot 6 \cdot 5 \cdot 7 = 10290$ Abbildungen, für die $g(2), g(3)$ und $g(4)$
                verschiedene Elemente sind.

            \item[b)]
                Da die Reihenfolge der gewählten Zahlen irrelevant ist, gilt:

                $$\frac{49^{\underline{6}}}{6!} = \frac{49 \cdot 48 \cdot 47 \cdot 46 \cdot 45 \cdot 44}{6 \cdot 5 \cdot 4 \cdot 3 \cdot 2 \cdot 1} = 13983816 \approx 14 \text{ Mio. verschiedene Tipps.}$$

            \item[c)]
                Die Anzahl der n-teiligen Teilmengen lässt sich mit $\binom{1000}{n}$ bestimmen, damit ergibt sich für die
                Anzahl der gesuchten Teilmengen:

                $$\sum_{n=997}^{1000} \binom{1000}{n} = \binom{1000}{997} + \binom{1000}{998} + \binom{1000}{999} + \binom{1000}{1000}$$

                $$= 166\;167\;000 + 499\;500 + 1\;000 + 1 = 166\;667\;501$$
        \end{enumerate}

    \item[\textbf{2.}]
        \begin{enumerate}
            \item[a)]
                Der Koeffizient von $x^5y^{11}$ in $(x + y)^{16}$ lautet:

                $$\frac{16!}{5! \cdot 11!} = \binom{16}{5} = \binom{16}{11}$$

                Der Koeffizient von $x^3y^5z^2$ in $(x + y + z)^{10}$ lautet:

                $$\frac{10!}{2! \cdot 3! \cdot 5!} \text{ (keine Schreibweise als Binomialkoeffizient)}$$

            \item[b)]
                \textbf{CAPPUCCINO} - 10 Buchstaben, davon 3 $\times$ ,,C'' und 2 $\times$ ,,P''

                $$\frac{10!}{2! \cdot 3!} = 302\;400$$

                \textbf{MANGOLASSI} - 10 Buchstaben, davon 2 $\times$ ,,A'' und 2 $\times$ ,,S''

                $$\frac{10!}{2! \cdot 2!} = 907\;200$$

                \textbf{SELTERWASSER} - 12 Buchstaben, davon 3 $\times$ ,,S'', 3 $\times$ ,,E'', 2 $\times$ ,,R''

                $$\frac{12!}{2! \cdot 3! \cdot 3!} = 6\;652\;800$$

            \item[c)]
                Da ein Flaschentyp mehrfach vorkommen darf, und die Reihenfolge egal ist, gibt es $10^6$ Möglichkeiten
                (die Belegungen der einzelnen Kisten-Plätze sind unabhängig voneinander).
        \end{enumerate}
    \newpage

    \item[\textbf{3.}]
        \textbf{Induktionsanfang:} Es gilt $A(3):$

            $$\sum_{i=3}^{3} \binom{i}{i-3} = \binom{3}{3 - 3} = \binom{3}{0} = 1 = \binom{3+1}{4} = \binom{4}{4} = 1$$

        \textbf{Induktionsanfang:}
            Es wird angenommen, dass für $n \geq 3$ folgende Aussage gilt:

            \begin{equation}\label{3IA}\tag{IA}
                A(n): \sum_{i=3}^{n} \binom{i}{i-3} = \binom{n+1}{4}
            \end{equation}

            Es ist zu zeigen, dass dann auch $A(n+1)$ gilt:

            \begin{equation}\label{3IS1}
                A(n+1): \sum_{i=3}^{n + 1} \binom{i}{i-3} = \binom{n+2}{4}
            \end{equation}

            Dies lässt sich wie folgt beweisen:

            \begin{dmath*}
                \sum_{i=3}^{n + 1} \binom{i}{i-3}
                = \sum_{i=3}^{n} \binom{i}{i-3} + \binom{n + 1}{n + 1 - 3}
                \overset{\eqref{3IA}}{=} \binom{n + 1}{4} + \binom{n + 1}{n - 2}
                = \binom{n + 1}{4} + \binom{n + 1}{(n + 1) - (n - 2)}
                = \binom{n + 1}{4} + \binom{n + 1}{3}
                = \binom{(n + 2) - 1}{4} + \binom{(n + 2) - 1}{4 - 1}
                = \binom{n + 2}{4}
            \end{dmath*}

            Damit ist \eqref{3IS1} bewiesen. $\Box$

    \item[\textbf{4.}]
        \begin{enumerate}
            \item[a)]
                $$2000 - \FF{2000}{3} - \FF{2000}{5} - \FF{2000}{7} + \FF{2000}{3 \cdot 5} + \FF{2000}{3 \cdot 7} + \FF{2000}{5 \cdot 7}
                - \FF{2000}{3 \cdot 5 \cdot 7}$$

                $$= 2000 - 666 - 400 - 285 + 133 + 95 + 57 - 19 = 915$$

            \item[b)]
                $$\begin{array}{rcl}
                   1000 &-& \FF{1000}{3} - \FF{1000}{5} - \FF{1000}{7} - \FF{1000}{11}\\[1.8em]
                &+& \FF{1000}{3 \cdot 5} + \FF{1000}{3 \cdot 7} + \FF{1000}{3 \cdot 11}
                + \FF{1000}{5 \cdot 7} + \FF{1000}{5 \cdot 11} + \FF{1000}{7 \cdot 11}\\[1.8em]
                &-& \FF{1000}{3 \cdot 5 \cdot 7} - \FF{1000}{3 \cdot 5 \cdot 11}
                - \FF{1000}{3 \cdot 7 \cdot 11} - \FF{1000}{5 \cdot 7 \cdot 11}\\[1.8em]
                &+& \FF{1000}{3 \cdot 5 \cdot 7 \cdot 11} = \\[1.8em]
                1000 &-& 333 - 200 - 142 - 90 + 66 + 47 + 30 + 28 + 18 + 12 - 9 - 6 - 4 - 2 + 0 = 415
                \end{array}$$

        \end{enumerate}
\end{enumerate}

\end{document}
