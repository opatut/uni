\documentclass[a4paper]{scrartcl}
\usepackage[german,ngerman]{babel}
\usepackage[utf8]{inputenc}
\usepackage[T1]{fontenc}
\usepackage{lmodern}
\usepackage{amssymb}
\usepackage{amsmath}
\usepackage{enumerate}
\usepackage{scrpage2}\pagestyle{scrheadings}
\usepackage{tikz}
\usetikzlibrary{patterns}
\usetikzlibrary{arrows}
\newcommand{\midarrow}{\tikz \draw[-triangle 90] (0,0) -- +(.1,0);}

\author{Paul Bienkowski, Jascha Andersen}
\title{DM 01-B (HA) zum 26.10.2012}
\date{\today}
\begin{document}
\setcounter{secnumdepth}{0}
\maketitle

\begin{enumerate}
        % Aufgabe 1
    \item[\textbf{1.}]
        \begin{enumerate}
            \item[a)]
                \begin{enumerate}
                    \item[(i)]
                        Eine mögliche Funktion:
                        
                        \makebox[\linewidth]{
                            \centering
                            \begin{minipage}[c]{0.5\textwidth}
                            \begin{tikzpicture}[]
                                \node[label=left:$1$, fill, circle, inner sep=2pt] (A1) at (-2, 1) {};
                                \node[label=left:$2$, fill, circle, inner sep=2pt] (A2) at (-2, 0) {};
                                \node[label=left:$3$, fill, circle, inner sep=2pt] (A3) at (-2,-1) {};
                                \draw (-2.2, 0) ellipse (1 and 1.8) {};
                                \node[label=$A$] (A) at (-2, 2) {};

                                \node[label=right:$3$, fill, circle, inner sep=2pt] (B3) at (2, 0.5) {};
                                \node[label=right:$4$, fill, circle, inner sep=2pt] (B4) at (2,-0.5) {};
                                \draw (2.2, 0) ellipse (1 and 1.8) {};
                                \node[label=$B$] (B) at (2, 2) {};

                                \draw[->] (A1) to (B3);
                                \draw[->] (A2) to (B3);
                                \draw[->] (A3) to (B4);
                            \end{tikzpicture}
                            \end{minipage}
                            \begin{minipage}[c]{0.2\textwidth}
                                \begin{tabular}[t]{c|c}
                                    $x$ & $f(x)$\\
                                    \hline
                                    1 & 3\\
                                    2 & 3\\
                                    3 & 4
                                \end{tabular}
                            \end{minipage}
                        }
                        
                    \item[(ii)]
                        Bilden einer injektiven Funktion ist nicht möglich,
                        da $\left|A\right| > \left|B\right|$ ist.
                    \item[(iii)]
                        Eine bijektive Funktion ist nicht möglich, da keine injektive Funktion gebildet werden kann, siehe (ii).
                \end{enumerate}
            \item[b)]
                \begin{enumerate}
                    \item[(i)]
                        Nicht möglich, da in einer surjektiven Funktion jeder Wert aus B genau, einmal zugeordnet wäre ($\left|A\right| = \left|B\right|$), damit wäre die Funktion automatisch ebenfalls injektiv.
                    \item[(ii)]
                        Nicht möglich aus demselben Grund.
                    \item[(iii)]
                        Eine mögliche Funktion:
                        
                        \makebox[\linewidth]{
                            \centering
                            \begin{minipage}[c]{0.5\textwidth}
                            \begin{tikzpicture}[]
                                \node[label=left:$1$, fill, circle, inner sep=2pt] (A1) at (-2, 1) {};
                                \node[label=left:$2$, fill, circle, inner sep=2pt] (A2) at (-2, 0) {};
                                \node[label=left:$3$, fill, circle, inner sep=2pt] (A3) at (-2,-1) {};
                                \draw (-2.2, 0) ellipse (1 and 1.8) {};
                                \node[label=$A$] (A) at (-2, 2) {};

                                \node[label=right:$3$, fill, circle, inner sep=2pt] (B3) at (2, 1) {};
                                \node[label=right:$4$, fill, circle, inner sep=2pt] (B4) at (2, 0) {};
                                \node[label=right:$5$, fill, circle, inner sep=2pt] (B5) at (2,-1) {};
                                \draw (2.2, 0) ellipse (1 and 1.8) {};
                                \node[label=$B$] (B) at (2, 2) {};

                                \draw[->] (A1) to (B3);
                                \draw[->] (A2) to (B5);
                                \draw[->] (A3) to (B4);
                            \end{tikzpicture}
                            \end{minipage}
                            \begin{minipage}[c]{0.2\textwidth}
                                \begin{tabular}[t]{c|c}
                                    $x$ & $h(x)$\\
                                    \hline
                                    1 & 3\\
                                    2 & 5\\
                                    3 & 4
                                \end{tabular}
                            \end{minipage}
                        }
                        
                \end{enumerate}
        \item[c)]
            \begin{enumerate}
                    \item[(i)]
                        Nicht möglich, da für eine surjektiven Funktion jeder Wert in $B$ zugeordnet werden muss, es stehen jedoch 
                        nicht genügend Werte in $A$ zur Verfügung ($\left|A\right| < \left|B\right|$).
                    \item[(ii)]
                        Eine mögliche Funktion:
                        
                        \makebox[\linewidth]{
                            \centering
                            \begin{minipage}[c]{0.5\textwidth}
                            \begin{tikzpicture}[]
                                \node[label=left:$1$, fill, circle, inner sep=2pt] (A1) at (-2, 1) {};
                                \node[label=left:$2$, fill, circle, inner sep=2pt] (A2) at (-2, 0) {};
                                \node[label=left:$3$, fill, circle, inner sep=2pt] (A3) at (-2,-1) {};
                                \draw (-2.2, 0) ellipse (1 and 1.8) {};
                                \node[label=$A$] (A) at (-2, 2) {};

                                \node[label=right:$3$, fill, circle, inner sep=2pt] (B3) at (2, 0.9) {};
                                \node[label=right:$4$, fill, circle, inner sep=2pt] (B4) at (2, 0.3) {};
                                \node[label=right:$5$, fill, circle, inner sep=2pt] (B5) at (2,-0.3) {};
                                \node[label=right:$6$, fill, circle, inner sep=2pt] (B6) at (2,-0.9) {};
                                \draw (2.2, 0) ellipse (1 and 1.8) {};
                                \node[label=$B$] (B) at (2, 2) {};

                                \draw[->] (A1) to (B5);
                                \draw[->] (A2) to (B6);
                                \draw[->] (A3) to (B3);
                            \end{tikzpicture}
                            \end{minipage}
                            \begin{minipage}[c]{0.2\textwidth}
                                \begin{tabular}[t]{c|c}
                                    $x$ & $g(x)$\\
                                    \hline
                                    1 & 5\\
                                    2 & 6\\
                                    3 & 3
                                \end{tabular}
                            \end{minipage}
                        }
                        
                    \item[(iii)]
                        Eine bijektive Funktion ist nicht möglich, da keine surjektive Funktion gebildet werden kann, siehe (i).
                \end{enumerate}
        \end{enumerate}

        % Aufgabe 2
    \item[\textbf{2.}]
    
        % Aufgabe 3
    \item[\textbf{3.}]
        
        % Aufgabe 4
    \item[\textbf{4.}]
        \begin{enumerate}
            \item[a)]
                \begin{tabular}[t]{c|c||c|c|c|c|c}
                    $A$ & $B$ & $A \cap B$ & $\overline{A \cap B}$ & $\overline{A}$ & $\overline{B}$ & $\overline{A} \cup \overline{B}$ \\
                    \hline
                    0 & 0 & 0 & 1 & 1 & 1 & 1\\
                    0 & 1 & 0 & 1 & 1 & 0 & 1\\
                    1 & 0 & 0 & 1 & 0 & 1 & 1\\
                    1 & 1 & 1 & 0 & 0 & 0 & 0
                \end{tabular}\\

                Die Spalten für $\overline{A \cap B}$ und $\overline{A} \cup \overline{B}$ sind identisch, daher gilt für alle $x \in M$:
                $\overline{A \cap B} = \overline{A} \cup \overline{B}$


                \begin{tikzpicture}[fill=gray]
                    % first pattern
                    % filled
                    \fill[pattern color=gray,pattern=north west lines] (-6,-2) rectangle (-1,2);

                    \scope
                    \clip (-3,0) circle (1);
                    \clip (-4,0) circle (1);
                    \fill[fill=white] (-6,-2) rectangle (-1,2);
                    \fill[pattern color=gray,pattern=north east lines] (-6,-2) rectangle (-1,2);
                    \endscope
                    % outline
                    \draw (-4,0) circle (1) (-4,1)  node [above] {$A$}
                    (-3,0) circle (1) (-3,1)  node [above] {$B$}
                    (-6,-2) rectangle (-1,2) node [right] {$M$};
                    % labels
                    \node[draw, fill, circle, pattern=north east lines, pattern color=gray, label=right:$A \cap B$] (LABEL1) at (-4, -2.5) {};
                    \node[draw, circle, pattern=north west lines,pattern color=gray,label=right:$\overline{A \cap B}$] (LABEL2) at (-4, -3) {};

                    % second pattern
                    % filled
                    \fill[pattern=north east lines, pattern color=lightgray] (1,-2) rectangle (6,2);
                    \fill[pattern=north west lines, pattern color=lightgray] (1,-2) rectangle (6,2);
                    \fill[fill=white] (4,0) circle (1);
                    \fill[fill=white] (3,0) circle (1);

                    \fill[pattern=north east lines, pattern color=lightgray] (4,0) circle (1);
                    \fill[pattern=north west lines, pattern color=lightgray] (3,0) circle (1);

                    \scope
                    \clip (4,0) circle (1);
                    \fill[fill=white] (3,0) circle (1);
                    \endscope
                    % outline
                    \draw (3,0) circle (1) (3,1)  node [above] {$A$}
                    (4,0) circle (1) (4,1)  node [above] {$B$}
                    (1,-2) rectangle (6,2) node [right] {$M$};
                    % labels
                    \node[draw, fill, circle, pattern=north east lines, label=right:$\overline{A}$] (LABEL1) at (3, -2.5) {};
                    \node[draw, fill, circle, pattern=north west lines, label=right:$\overline{B}$] (LABEL2) at (3, -3) {};

                    \node[draw, fill, circle, pattern=north west lines, label=right:$oder$] (LABEL3) at (1.6, -3.5) {};
                    \node[draw, fill, circle, pattern=north east lines, label=right:$\overline{A} \cup \overline{B}$] (LABEL4) at (3, -3.5) {};

                \end{tikzpicture}

            \item[b)]
                \begin{align}
                    \nonumber \mathcal{P}(M) = \Big\{ & \emptyset, \{a\}, \{b\}, \{c\}, \{d\},\\
                    \nonumber & \{a,b\}, \{a,c\}, \{a,d\}, \{b,c\}, \{b,d\}, \{c,d\},\\
                              & \{a,b,c\}, \{a,b,d\}, \{a,c,d\}, \{b,c,d\}, \{a,b,c,d\} \Big\}
                \end{align}

            \item[c)]
                \begin{enumerate}
                    \item[(i)]
                        falsch, $a$ ist keine Menge, es gilt: $a \not\subseteq M$, daher $a \notin \mathcal{P}(M)$.
                    \item[(ii)]
                        falsch, $a$ ist keine Menge, daher kann es auch keine Teilmenge sein
                    \item[(iii)]
                        wahr, da $\{a\} \subseteq M$
                    \item[(iv)]
                        falsch, $a \notin \mathcal{P}(M)$, daher ist $\{a\}$ keine Teilmenge
                    \item[(v)]
                        falsch, $\{\{a\}\}$ ist keine Element der Potenzmenge
                    \item[(vi)]
                        wahr, da $\{a\} \in \mathcal{P}(M)$, siehe (iii).
                \end{enumerate}


        \end{enumerate}
\end{enumerate}

\end{document}
