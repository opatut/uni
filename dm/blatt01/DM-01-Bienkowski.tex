\documentclass[a4paper]{scrartcl}
\usepackage[german,ngerman]{babel}
\usepackage[utf8]{inputenc}
\usepackage[T1]{fontenc}
\usepackage{lmodern}
\usepackage{amssymb}
\usepackage{amsmath}
\usepackage{enumerate}
\usepackage{scrpage2}\pagestyle{scrheadings}
\usepackage{tikz}
\usetikzlibrary{patterns}
\usetikzlibrary{arrows}

\author{Paul Bienkowski, Jascha Andersen}
\title{DM 01-B (HA) zum 26.10.2012}
\date{\today}
\begin{document}
\setcounter{secnumdepth}{0}
\maketitle

\begin{enumerate}
        % Aufgabe 1
    \item[\textbf{1.}]
        \begin{enumerate}
            \item[a)]
                \begin{enumerate}
                    \item[(i)]
                        Eine mögliche Funktion:
                        
                        \makebox[\linewidth]{
                            \centering
                            \begin{minipage}[c]{0.5\textwidth}
                            \begin{tikzpicture}[]
                                \node[label=left:$1$, fill, circle, inner sep=2pt] (A1) at (-2, 1) {};
                                \node[label=left:$2$, fill, circle, inner sep=2pt] (A2) at (-2, 0) {};
                                \node[label=left:$3$, fill, circle, inner sep=2pt] (A3) at (-2,-1) {};
                                \draw (-2.2, 0) ellipse (1 and 1.8) {};
                                \node[label=$A$] (A) at (-2, 2) {};

                                \node[label=right:$3$, fill, circle, inner sep=2pt] (B3) at (2, 0.5) {};
                                \node[label=right:$4$, fill, circle, inner sep=2pt] (B4) at (2,-0.5) {};
                                \draw (2.2, 0) ellipse (1 and 1.8) {};
                                \node[label=$B$] (B) at (2, 2) {};

                                \draw[->] (A1) to (B3);
                                \draw[->] (A2) to (B3);
                                \draw[->] (A3) to (B4);
                            \end{tikzpicture}
                            \end{minipage}
                            \begin{minipage}[c]{0.2\textwidth}
                                \begin{tabular}[t]{c|c}
                                    $x$ & $f(x)$\\
                                    \hline
                                    1 & 3\\
                                    2 & 3\\
                                    3 & 4
                                \end{tabular}
                            \end{minipage}
                        }
                        
                    \item[(ii)]
                        Bilden einer injektiven Funktion ist nicht möglich,
                        denn es gilt:
                        
                        $$\left|A\right| > \left|B\right|.$$
                    \item[(iii)]
                        Eine bijektive Funktion ist nicht möglich, da keine injektive Funktion gebildet werden kann, siehe (ii).
                \end{enumerate}
            \item[b)]
                \begin{enumerate}
                    \item[(i)]
                        Nicht möglich, da in einer surjektiven Funktion jeder Wert aus B genau einmal zugeordnet wäre:
                        
                        $$\left|A\right| = \left|B\right|.$$
                        
                        Damit wäre die Funktion automatisch ebenfalls injektiv.
                    \item[(ii)]
                        Nicht möglich aus demselben Grund.
                        
                    \item[(iii)]
                        Eine mögliche Funktion:
                        
                        \makebox[\linewidth]{
                            \centering
                            \begin{minipage}[c]{0.5\textwidth}
                            \begin{tikzpicture}[]
                                \node[label=left:$1$, fill, circle, inner sep=2pt] (A1) at (-2, 1) {};
                                \node[label=left:$2$, fill, circle, inner sep=2pt] (A2) at (-2, 0) {};
                                \node[label=left:$3$, fill, circle, inner sep=2pt] (A3) at (-2,-1) {};
                                \draw (-2.2, 0) ellipse (1 and 1.8) {};
                                \node[label=$A$] (A) at (-2, 2) {};

                                \node[label=right:$3$, fill, circle, inner sep=2pt] (B3) at (2, 1) {};
                                \node[label=right:$4$, fill, circle, inner sep=2pt] (B4) at (2, 0) {};
                                \node[label=right:$5$, fill, circle, inner sep=2pt] (B5) at (2,-1) {};
                                \draw (2.2, 0) ellipse (1 and 1.8) {};
                                \node[label=$B$] (B) at (2, 2) {};

                                \draw[->] (A1) to (B3);
                                \draw[->] (A2) to (B5);
                                \draw[->] (A3) to (B4);
                            \end{tikzpicture}
                            \end{minipage}
                            \begin{minipage}[c]{0.2\textwidth}
                                \begin{tabular}[t]{c|c}
                                    $x$ & $h(x)$\\
                                    \hline
                                    1 & 3\\
                                    2 & 5\\
                                    3 & 4
                                \end{tabular}
                            \end{minipage}
                        }
                        
                \end{enumerate}
        \item[c)]
            \begin{enumerate}
                    \item[(i)]
                        Nicht möglich, da für eine surjektiven Funktion jeder Wert in $B$ zugeordnet werden muss, es stehen jedoch 
                        nicht genügend Werte in $A$ zur Verfügung:
                        
                        $$\left|A\right| < \left|B\right|.$$
                    \item[(ii)]
                        Eine mögliche Funktion:
                        
                        \makebox[\linewidth]{
                            \centering
                            \begin{minipage}[c]{0.5\textwidth}
                            \begin{tikzpicture}[]
                                \node[label=left:$1$, fill, circle, inner sep=2pt] (A1) at (-2, 1) {};
                                \node[label=left:$2$, fill, circle, inner sep=2pt] (A2) at (-2, 0) {};
                                \node[label=left:$3$, fill, circle, inner sep=2pt] (A3) at (-2,-1) {};
                                \draw (-2.2, 0) ellipse (1 and 1.8) {};
                                \node[label=$A$] (A) at (-2, 2) {};

                                \node[label=right:$3$, fill, circle, inner sep=2pt] (B3) at (2, 0.9) {};
                                \node[label=right:$4$, fill, circle, inner sep=2pt] (B4) at (2, 0.3) {};
                                \node[label=right:$5$, fill, circle, inner sep=2pt] (B5) at (2,-0.3) {};
                                \node[label=right:$6$, fill, circle, inner sep=2pt] (B6) at (2,-0.9) {};
                                \draw (2.2, 0) ellipse (1 and 1.8) {};
                                \node[label=$B$] (B) at (2, 2) {};

                                \draw[->] (A1) to (B5);
                                \draw[->] (A2) to (B6);
                                \draw[->] (A3) to (B3);
                            \end{tikzpicture}
                            \end{minipage}
                            \begin{minipage}[c]{0.2\textwidth}
                                \begin{tabular}[t]{c|c}
                                    $x$ & $g(x)$\\
                                    \hline
                                    1 & 5\\
                                    2 & 6\\
                                    3 & 3
                                \end{tabular}
                            \end{minipage}
                        }
                        
                    \item[(iii)]
                        Eine bijektive Funktion ist nicht möglich, da keine surjektive Funktion gebildet werden kann, siehe (i).
                \end{enumerate}
        \end{enumerate}

    \newpage
        % Aufgabe 2
    \item[\textbf{2.}]
        \begin{tabular}[t]{c||c|c|c}
              & injektiv & surjektiv & bijektiv\\
            \hline
            f & nein (i) & nein (ii) & nein\\
            g & ja (iii) & nein (iv) & nein\\
            h & ja (v)   & ja (vi)   & ja
        \end{tabular}
        
        \begin{enumerate}
            \item[(i)]
                \textbf{Behauptung:} f ist nicht injektiv.
                
                \textbf{Beweis:} Es sind $x_1, x_2 \in \mathbb{Z}$ anzugeben, für die $f(x_1) = f(x_2)$ gilt.
                
                \textbf{Annahme:} Für $x_1 = 3$ und $x_2 = -3$ ist dies der Fall.
                
                \textbf{Nachweis:}
                
                $$\begin{array}{rcl}
                    f(3) &=& f(-3)\\
                    3^2 - 5 &=& (-3)^2 - 5\\
                    9 - 5 &=& 9 - 5\\
                    4 &=& 4
                \end{array}$$
                
                Es ist gezeigt dass es Werte für $f(x)$ gibt, welche durch verschiedene $x$ zugeordnet werden, daher
                ist $f$ nicht injektiv. $\Box$

            \item[(ii)]
                \textbf{Behauptung:} f ist nicht surjektiv.
                
                \textbf{Beweis:} Es sei ein $y \in \mathbb{Z}$ anzugeben, für das gilt: es gibt kein $x \in \mathbb{Z}$ mit
                $f(x) = y$.
                
                \textbf{Annahme:} Für $y = -6$ ist dies der Fall.
                
                \textbf{Nachweis:}
                
                $$\begin{array}{rcl}
                    f(x) &=& -6\\
                    x^2 - 5 &=& -6\\
                    x^2 &=& -1
                \end{array}$$
                
                Es gibt kein $x \in \mathbb{Z}$, für das gilt $x^2 = -1$, daher ist $f$ nicht surjektiv. $\Box$
            
            \item[(iii)]
                \textbf{Behauptung:} g ist injektiv.
                
                \textbf{Beweis:} Wäre g nicht injektiv, würde für mindestens ein $x_1, x_2 \in \mathbb{Z}$ mit $x_1 \not= x_2$ gelten:
                
                $$\begin{array}{rcl}
                   g(x_1) &=& g(x_2)\\
                   5 \cdot x_1 - 3 &=& 5 \cdot x_2 - 3\\
                   5 \cdot x_1 &=& 5 \cdot x_2\\
                   x_1 &=& x_2
                \end{array}$$
                
                Die steht im Widerspruch zur Annahme $x_1 \not= x_2$, daher ist $g$ injektiv. $\Box$
                
            \item[(iv)]
                \textbf{Behauptung:} g ist nicht surjektiv.

                \textbf{Beweis:} Es sei ein $y \in \mathbb{Z}$ anzugeben, für das gilt: es gibt kein $x \in \mathbb{Z}$ mit
                $g(x) = y$.
                
                \textbf{Annahme:} Für $y = 0$ ist dies der Fall.
                
                \textbf{Nachweis:}
                
                $$\begin{array}{rcl}
                    g(x) &=& 0\\
                    5 \cdot x - 3 &=& 0\\
                    5 \cdot x &=& 3\\
                    x &=& \frac{3}{5}
                \end{array}$$
                
                Der errechnete Wert für $x$ liegt nicht im Definitionsbereich $\mathbb{Z}$, daher ist $g$ nicht surjektiv. $\Box$
            
            \item[(v)]
                \textbf{Behauptung:} h ist injektiv.
                
                \textbf{Beweis:} Wäre h nicht injektiv, würde für mindestens ein $x_1, x_2 \in \mathbb{Z}$ mit $x_1 \not= x_2$ gelten:
                
                $$\begin{array}{rcl}
                   h(x_1) &=& h(x_2)\\
                   x_1 + 5 &=& x_2 + 5\\
                   x_1 &=& x_2
                \end{array}$$
                
                Die steht im Widerspruch zur Annahme $x_1 \not= x_2$, daher ist $h$ injektiv. $\Box$
                
            \item[(vi)]
                \textbf{Behauptung:} h ist surjektiv.

                \textbf{Beweis:} Es gibt für jedes $y \in \mathbb{Z}$ ein $x \in \mathbb{Z}$, sodass gilt: $f(x) = y$. Dieses
                $x$ lässt sich wie folgt berechnen:
                
                $$\begin{array}{rcl}
                    y &=& x + 5\\
                    x &=& y - 5 
                \end{array}$$
                
                Da die Subtraktion im $\mathbb{Z}$ unbegrenzt ausführbar ist, lässt sich diese Berechnung auf jedes $y \in \mathbb{Z}$
                anwenden. Daher ist h surjektiv. $\Box$
        \end{enumerate}
    
    \newpage
        % Aufgabe 3
    \item[\textbf{3.}]
        \begin{enumerate}
            \item[a)]
                \textbf{Behauptung:} f ist nicht injektiv.

                \textbf{Beweis:} Es sind $(n_1, m_1), (n_2, m_2) \in \mathbb{Z} \times \mathbb{Z}$ anzugeben, für die $f(n_1, m_1) = f(n_2, m_2)$ gilt.

                \textbf{Annahme:} Für $(n_1, m_1) = (5, 2)$ und $(n_2, m_2) = (4, 1)$ ist dies der Fall.

                \textbf{Nachweis:}

                $$\begin{array}{rcl}
                    f(5, 2) &=& f(4, 1)\\
                    5 - 2 &=& 4 - 1\\
                    3 &=& 3
                \end{array}$$

                Es ist gezeigt dass es Werte für $f(n, m)$ gibt, welche durch verschiedene $(n, m)$ zugeordnet werden, daher
                ist $f$ nicht injektiv. $\Box$
                
                \textbf{Behauptung:} f ist surjektiv.

                \textbf{Beweis:} Jede ganze Zahl lässt sich als Differenz zweier anderer ganzen Zahlen ausdrücken. 
                Für jede $x, k \in \mathbb{Z}$ gilt:

                $$\begin{array}{rcl}
                    f(x + k, k) = x + k - k = x
                \end{array}$$

                Somit lässt sich jedes $x \in \mathbb{Z}$ durch $f$ abbilden, also
                ist $f$ surjektiv. $\Box$
                
            \item[b)]
                \textbf{Behauptung:} g ist injektiv.
                
                \textbf{Beweis:} Es gilt $(n_1, m_1), (n_2, m_2) \in \mathbb{Z} \times \mathbb{Z}$ und
                $(n_1, m_1) \not= (n_2, m_2)$. Wäre $g$ injektiv, wäre:
                
                $$\begin{array}{rcl}
                    g(n_1, m_1) &=& g(n_2, m_2)\\
                    (n_1 + m_1, n_1 - m_1) &=& (n_2 + m_2, n_2 - m_2)
                \end{array}$$
                $$\begin{array}{rclcrcl}
                    n_1 + m_1 &=& n_2 + m_2 &\wedge& n_1 - m_1 &=& n_2 - m_2\\
                    n_1 - n_2 &=& m_2 - m_1 &\wedge& n_1 - n_2 &=& m_1 - m_2
                \end{array}$$
                $$\begin{array}{rcl}
                    m_2 - m_1 &=& m_1 - m_2\\
                    2 \cdot m_2 &=& 2 \cdot m_1\\
                    m_2 &=& m_1\\
                    \\
                    n_1 - n_2 &=& m_2 - m_1\\
                    n_1 - n_2 &=& 0\\
                    n_1 &=& n_2
                \end{array}$$
                
                Dies steht im Widerspruch zu $(n_1, m_1) \not= (n_2, m_2)$, also ist $g$ nicht injektiv. $\Box$
                
                \textbf{Behauptung:} g ist nicht surjektiv.
                
                \textbf{Beweis:} Es ist ein $y \in \mathbb{Z} \times \mathbb{Z}$ anzugeben, für die gilt: es gibt 
                kein $(n, m) \in \mathbb{Z} \times \mathbb{Z}$ mit $g(n, m) = y$.
                
                \textbf{Annahme:} Für $y = (1, 0)$ ist dies der Fall.
                
                \textbf{Nachweis:}
                
                $$\begin{array}{rcl}
                    g(n, m) &=& (1, 0)\\
                    (n + m, n - m) &=& (1, 0)\\
                    n + m = 1 & \wedge & n - m = 0\\
                    n &=& m\\
                    n + n &=& 1\\
                    2 \cdot n &=& 1\\
                    n = m &=& \frac{1}{2}\\
                    \Rightarrow (n, m) &=& (\frac{1}{2}, \frac{1}{2})
                \end{array}$$
                
                Der errechnete Wert für $(n, m)$ liegt nicht im Definitionsbereich $\mathbb{Z} \times \mathbb{Z}$, 
                daher ist $g$ nicht surjektiv. $\Box$
            \item[c)]
                \textbf{Behauptung:} h ist injektiv.

                \textbf{Beweis:} Wäre h nicht injektiv, gäbe es $n_1, n_2 \in \mathbb{Z}$ mit $n_1 \not= n_2$, für die gilt:
                
                $$\begin{array}{rcl}
                    h(n_1) &=& h(n_2)\\
                    \big((n_1 + 1)^2, n_1^2 + 1\big) &=& \big((n_2 + 1)^2, n_2^2 + 1\big)
                \end{array}$$
                $$\begin{array}{rclcrcl}
                    (n_1 + 1)^2 &=& (n_2 + 1)^2 &\wedge& n_1^2 + 1 &=& n_2^2 + 1\\
                    n_1^2 + 2 \cdot n_1 + 1 = n_2^2 + 2 \cdot n_2 + 1 &\wedge& n_1^2 = n_2^2\\
                    2 \cdot n_1 + 1 &=& 2 \cdot n_2 + 1\\
                    n_1 &=& n_2
                \end{array}$$

                Dies steht im Widerspruch zu Annahme $n_1 \not= n_2$, also ist h injektiv. $\Box$
                
                \textbf{Behauptung:} h ist nicht surjektiv.
                
                \textbf{Beweis:} Es sei ein $y \in \mathbb{Z} \times \mathbb{Z}$ anzugeben, für das gilt: 
                es gibt kein $n \in \mathbb{Z}$ mit $h(n) = y$.
                
                \textbf{Annahme:} Für $y = (k, 4)$ mit beliebigem $k \in \mathbb{Z}$ ist dies der Fall.
                
                \textbf{Nachweis:}
                
                $$\begin{array}{rcl}
                    h(n) &=& (k, 4)\\
                    ((n + 1)^2, n^2 + 1) &=& (k, 4) \Rightarrow \\
                    n^2 + 1 &=& 4\\
                    n^2 &=& 3\\
                    n &=& \sqrt{3}
                \end{array}$$
                
                Der errechnete Wert für $n$ ist kein Element der Definitionsmenge ($\sqrt{3} \notin \mathbb{Z}$), also ist h nicht surjektiv. $\Box$
        \end{enumerate}
        
    % Aufgabe 4
    \item[\textbf{4.}]
        \begin{enumerate}
            \item[a)]
                \begin{tabular}[t]{c|c||c|c|c|c|c}
                    $A$ & $B$ & $A \cap B$ & $\overline{A \cap B}$ & $\overline{A}$ & $\overline{B}$ & $\overline{A} \cup \overline{B}$ \\
                    \hline
                    0 & 0 & 0 & 1 & 1 & 1 & 1\\
                    0 & 1 & 0 & 1 & 1 & 0 & 1\\
                    1 & 0 & 0 & 1 & 0 & 1 & 1\\
                    1 & 1 & 1 & 0 & 0 & 0 & 0
                \end{tabular}

                Die Spalten für $\overline{A \cap B}$ und $\overline{A} \cup \overline{B}$ sind identisch, daher gilt für alle $x \in M$:
                
                $$\overline{A \cap B} = \overline{A} \cup \overline{B}$$


                \begin{tikzpicture}[fill=gray]
                    % first pattern
                    % filled
                    \fill[pattern color=gray,pattern=north west lines] (-6,-2) rectangle (-1,2);

                    \scope
                    \clip (-3,0) circle (1);
                    \clip (-4,0) circle (1);
                    \fill[fill=white] (-6,-2) rectangle (-1,2);
                    \fill[pattern color=gray,pattern=north east lines] (-6,-2) rectangle (-1,2);
                    \endscope
                    % outline
                    \draw (-4,0) circle (1) (-4,1)  node [above] {$A$}
                    (-3,0) circle (1) (-3,1)  node [above] {$B$}
                    (-6,-2) rectangle (-1,2) node [right] {$M$};
                    % labels
                    \node[draw, fill, circle, pattern=north east lines, pattern color=gray, label=right:$A \cap B$] (LABEL1) at (-4, -2.5) {};
                    \node[draw, circle, pattern=north west lines,pattern color=gray,label=right:$\overline{A \cap B}$] (LABEL2) at (-4, -3) {};

                    % second pattern
                    % filled
                    \fill[pattern=north east lines, pattern color=lightgray] (1,-2) rectangle (6,2);
                    \fill[pattern=north west lines, pattern color=lightgray] (1,-2) rectangle (6,2);
                    \fill[fill=white] (4,0) circle (1);
                    \fill[fill=white] (3,0) circle (1);

                    \fill[pattern=north east lines, pattern color=lightgray] (4,0) circle (1);
                    \fill[pattern=north west lines, pattern color=lightgray] (3,0) circle (1);

                    \scope
                    \clip (4,0) circle (1);
                    \fill[fill=white] (3,0) circle (1);
                    \endscope
                    % outline
                    \draw (3,0) circle (1) (3,1)  node [above] {$A$}
                    (4,0) circle (1) (4,1)  node [above] {$B$}
                    (1,-2) rectangle (6,2) node [right] {$M$};
                    % labels
                    \node[draw, fill, circle, pattern=north east lines, label=right:$\overline{A}$] (LABEL1) at (3, -2.5) {};
                    \node[draw, fill, circle, pattern=north west lines, label=right:$\overline{B}$] (LABEL2) at (3, -3) {};

                    \node[draw, fill, circle, pattern=north west lines, label=right:$oder$] (LABEL3) at (1.6, -3.5) {};
                    \node[draw, fill, circle, pattern=north east lines, label=right:$\overline{A} \cup \overline{B}$] (LABEL4) at (3, -3.5) {};

                \end{tikzpicture}

            \item[b)]
                $$\begin{array}{rl}
                    \mathcal{P}(M) = \Big\{ & \emptyset, \{a\}, \{b\}, \{c\}, \{d\},\\
                        & \{a,b\}, \{a,c\}, \{a,d\}, \{b,c\}, \{b,d\}, \{c,d\},\\
                        & \{a,b,c\}, \{a,b,d\}, \{a,c,d\}, \{b,c,d\}, \{a,b,c,d\} \Big\}
                \end{array}$$

            \item[c)]
                \begin{enumerate}
                    \item[(i)]
                        falsch, $a$ ist keine Menge, es gilt: $a \not\subseteq M$, daher $a \notin \mathcal{P}(M)$.
                    \item[(ii)]
                        falsch, $a$ ist keine Menge, daher kann es auch keine Teilmenge sein
                    \item[(iii)]
                        wahr, da $\{a\} \subseteq M$
                    \item[(iv)]
                        falsch, $a \notin \mathcal{P}(M)$, daher ist $\{a\}$ keine Teilmenge
                    \item[(v)]
                        falsch, $\{\{a\}\}$ ist keine Element der Potenzmenge
                    \item[(vi)]
                        wahr, da $\{a\} \in \mathcal{P}(M)$, siehe (iii).
                \end{enumerate}


        \end{enumerate}
\end{enumerate}

\end{document}
