\newcommand{\authorinfo}{Paul Bienkowski, Jascha Andersen, Benedikt Bushart}
\newcommand{\titleinfo}{DM 05-B (HA) zum 23.11.2012}

% PREAMBLE ===============================================================

\documentclass[a4paper,10pt]{scrartcl}
\usepackage[german,ngerman]{babel}
\usepackage[utf8]{inputenc}
\usepackage[T1]{fontenc}
\usepackage{lmodern}
\usepackage{amssymb}
\usepackage{mathtools}
\usepackage{amsmath}
\usepackage{enumerate}
\usepackage{array}
\usepackage{listings}
\usepackage{fullpage}
\usepackage{breqn}
\usepackage{fancyhdr}
\usepackage{lastpage}
\usepackage{tikz}
\usetikzlibrary{decorations.markings}

\author{\authorinfo}
\title{\titleinfo}
\date{\today}

\pagestyle{fancy}
\fancyhf{}
\fancyhead[L]{\authorinfo}
\fancyhead[R]{\titleinfo}
\fancyfoot[C]{\thepage}
\renewcommand{\headrulewidth}{0.4pt}
\renewcommand{\footrulewidth}{0pt}
\renewcommand{\headheight}{12pt}
\renewcommand{\headsep}{12pt}

\begin{document}
\setcounter{secnumdepth}{0}
\maketitle

% DOCUMENT ===============================================================

\tikzstyle{n}=[fill,circle,radius=1cm,font=\huge,inner sep=0,minimum size=0.5em]
%\tikzstyle{->-}=[decoration={markings, mark=at position #1 with {\arrow{latex}}},postaction={decorate}]
\tikzstyle{->-}=[decoration={markings, mark=at position #1 with {
    \fill (1.2pt,0)--(-1.2pt,1.5pt)--(-1.2pt,-1.5pt)--cycle;
}},postaction={decorate}]
\tikzstyle{a}=[->-=0.5]
\tikzstyle{add}=[color=gray]
\tikzstyle{ref}=[out=#1+40, in=#1-40, distance=0.5cm]

\begin{enumerate}
    \item[\textbf{1.}]
        \begin{enumerate}
            \item[a)]
                Die vorgegebene Relation:

                \vspace{0.5cm}
                \begin{minipage}[c]{0.3\textwidth}
                    \centering
                    \begin{tikzpicture}
                        \node[n,label=below:a] (a) at (0, 0) {};
                        \node[n,label=below:b] (b) at (1, 0) {};
                        \node[n,label=below:c] (c) at (2, 0) {};
                        \node[n,label=below:d] (d) at (3, 0) {};
                        \node[n,label=above:e] (e) at (1, 1) {};
                        \node[n,label=above:f] (f) at (2, 1) {};

                        \draw[a] (a) -> (b);
                        \draw[a] (b) -> (c);
                        \draw[a] (c) -> (d);
                        \draw[a] (b) -> (e);
                        \draw[a] (e) -> (f);
                    \end{tikzpicture}
                \end{minipage}
                \begin{minipage}[c]{0.5\textwidth}
                    \centering
                    \begin{tabular}{c|cccccc}
                        & a & b & c & d & e & f\\
                        \hline
                        a & 0 & 1 & 0 & 0 & 0 & 0\\
                        b & 0 & 0 & 1 & 0 & 1 & 0\\
                        c & 0 & 0 & 0 & 1 & 0 & 0\\
                        d & 0 & 0 & 0 & 0 & 0 & 0\\
                        e & 0 & 0 & 0 & 0 & 0 & 1\\
                        f & 0 & 0 & 0 & 0 & 0 & 0\\
                    \end{tabular}
                \end{minipage}
                \vspace{0.5cm}

            \item[b) c)]
                Die Ordnungsrelation $R^+$ als gerichteter Graph und als Hasse-Diagramm:

                \vspace{0.5cm}
                \begin{minipage}[c]{0.3\textwidth}
                    \centering
                    \begin{tikzpicture}
                        \node[n,label=below:a] (a) at (0, 0) {};
                        \node[n,label=below:b] (b) at (1, 0) {};
                        \node[n,label=below:c] (c) at (2, 0) {};
                        \node[n,label=below:d] (d) at (3, 0) {};
                        \node[n,label=above:e] (e) at (1, 1) {};
                        \node[n,label=above:f] (f) at (2, 1) {};

                        \draw[a] (a) -> (b);
                        \draw[a] (b) -> (c);
                        \draw[a] (c) -> (d);
                        \draw[a] (b) -> (e);
                        \draw[a] (e) -> (f);

                        %\draw (a) [bend left=40] -> (b);
                        \path (a) edge[a, add, ref=180] (a);
                        \path (b) edge[a, add, ref=225] (b);
                        \path (c) edge[a, add, ref=315] (c);
                        \path (d) edge[a, add, ref=0] (d);
                        \path (e) edge[a, add, ref=180] (e);
                        \path (f) edge[a, add, ref=0] (f);

                        \path (a) edge[a, add] (e);
                        \path (a) edge[a, add, bend right=60] (c);
                        \path (a) edge[a, add, bend right=60] (d);
                        \path (b) edge[a, add, bend left=30] (d);
                        \path (b) edge[a, add] (f);
                        \path (a) edge[->-=0.7, add] (f);
                    \end{tikzpicture}
                \end{minipage}
                \begin{minipage}[c]{0.5\textwidth}
                    \centering
                    \begin{tikzpicture}
                        \node[n,label=right:a] (a) at (1, 0) {};
                        \node[n,label=right:b] (b) at (1, 1) {};
                        \node[n,label=right:c] (c) at (1, 2) {};
                        \node[n,label=right:d] (d) at (1, 3) {};
                        \node[n,label=left:e]  (e) at (0, 2) {};
                        \node[n,label=left:f]  (f) at (0, 3) {};

                        \draw (a) -- (b);
                        \draw (b) -- (c);
                        \draw (b) -- (e);
                        \draw (c) -- (d);
                        \draw (e) -- (f);
                    \end{tikzpicture}
                \end{minipage}
                \vspace{0.5cm}

            \item[d)]
                Da in der Äquivalenzrelation jedes Paar aus Elementen jeder Äquivalenzklasse auftritt, gilt:

                $$S = \bigcup_{k \in K} k^2$$

                wobei $K$ die Menge der Äquivalenzklassen darstellt. In diesem Fall gibt es nur eine Äquivalenzklasse
                $K = \{k_1\} = A$, daher gilt:

                $$S = A^2$$

                \vspace{0.2cm}
                \begin{center}
                    \begin{tikzpicture}
                        \node[n,label=left:a]  (a) at (0, 0) {};
                        \node[n,label=right:b] (b) at (2, 0) {};
                        \node[n,label=below right:c] (c) at (2.6, 1.6) {};
                        \node[n,label=above right:d] (d) at (1, 3) {};
                        \node[n,label=below left:e]  (e) at (-0.6, 1.6) {};

                        \path (a) edge[a, bend left=10     ] (b);
                        \path (b) edge[a, bend left=10, add] (a);

                        \path (b) edge[a, bend left=10     ] (c);
                        \path (c) edge[a, bend left=10, add] (b);

                        \path (c) edge[a, bend left=10     ] (d);
                        \path (d) edge[a, bend left=10, add] (c);

                        \path (b) edge[a, bend left=10     ] (e);
                        \path (e) edge[a, bend left=10, add] (b);

                        \path (a) edge[a, bend left=10, add] (c);
                        \path (c) edge[a, bend left=10, add] (a);

                        \path (a) edge[a, bend left=10, add] (d);
                        \path (d) edge[a, bend left=10, add] (a);

                        \path (a) edge[a, bend left=10, add] (e);
                        \path (e) edge[a, bend left=10, add] (a);

                        \path (b) edge[a, bend left=10, add] (d);
                        \path (d) edge[a, bend left=10, add] (b);

                        \path (c) edge[a, bend left=10, add] (e);
                        \path (e) edge[a, bend left=10, add] (c);

                        \path (d) edge[a, bend left=10, add] (e);
                        \path (e) edge[a, bend left=10, add] (d);

                        \path (a) edge[a, add, ref=234] (a);
                        \path (b) edge[a, add, ref=306] (b);
                        \path (c) edge[a, add, ref=18] (c);
                        \path (d) edge[a, add, ref=90] (d);
                        \path (e) edge[a, add, ref=162] (e);
                    \end{tikzpicture}
                \end{center}

        \end{enumerate}

    \item[\textbf{2.}]
        \begin{enumerate}
            \item[a)]
                Die vorgegebene Relation:

                \vspace{0.5cm}
                \begin{minipage}[c]{0.3\textwidth}
                    \centering
                    \begin{tikzpicture}
                        \node[n,label=below:a] (a) at (0, 0) {};
                        \node[n,label=below:b] (b) at (1.5, 0) {};
                        \node[n,label=right:c] (c) at (2.2, 1) {};
                        \node[n,label=above:d] (d) at (1.5, 2) {};
                        \node[n,label=above:e] (e) at (0, 2) {};
                        \node[n,label=left:f] (f) at (-0.7, 1) {};

                        \draw[a] (a) -> (b);
                        \draw[a] (b) -> (d);
                        \draw[a] (e) -> (f);
                    \end{tikzpicture}
                \end{minipage}
                \begin{minipage}[c]{0.5\textwidth}
                    \centering
                    \begin{tabular}{c|cccccc}
                        & a & b & c & d & e & f\\
                        \hline
                        a & 0 & 1 & 0 & 0 & 0 & 0\\
                        b & 0 & 0 & 0 & 1 & 0 & 0\\
                        c & 0 & 0 & 0 & 0 & 0 & 0\\
                        d & 0 & 0 & 0 & 0 & 0 & 0\\
                        e & 0 & 0 & 0 & 0 & 0 & 1\\
                        f & 0 & 0 & 0 & 0 & 0 & 0\\
                    \end{tabular}
                \end{minipage}
                \vspace{0.5cm}

            \item[b) c)]
                Die Ordnungsrelation $R^+$ als gerichteter Graph und als Hasse-Diagramm:

                \vspace{0.5cm}
                \begin{minipage}[c]{0.3\textwidth}
                    \centering
                    \begin{tikzpicture}
                        \node[n,label=below:a] (a) at (0, 0) {};
                        \node[n,label=below:b] (b) at (1.5, 0) {};
                        \node[n,label=right:c] (c) at (2.2, 1) {};
                        \node[n,label=above:d] (d) at (1.5, 2) {};
                        \node[n,label=above:e] (e) at (0, 2) {};
                        \node[n,label=left:f] (f) at (-0.7, 1) {};

                        \draw[a] (a) -> (b);
                        \draw[a] (b) -> (d);
                        \path (e) edge[a, bend left=10] (f);

                        \path (a) edge[a, add, ref=180] (a);
                        \path (b) edge[a, add, ref=0] (b);
                        \path (c) edge[a, add, ref=90] (c);
                        \path (d) edge[a, add, ref=0] (d);
                        \path (e) edge[a, add, ref=0] (e);
                        \path (f) edge[a, add, ref=270] (f);

                        \path (a) edge[a, add] (d);
                        \path (f) edge[a, add, bend left=10] (e);
                    \end{tikzpicture}
                \end{minipage}
                \begin{minipage}[c]{0.5\textwidth}
                    \centering
                    \begin{tikzpicture}
                        \node[n,label=left:a]  (a) at (0, 0) {};
                        \node[n,label=left:b]  (b) at (0, 1) {};
                        \node[n,label=left:d]  (d) at (0, 2) {};
                        \node[n,label=below:c] (c) at (1, 1) {};
                        \node[n,label=right:e] (e) at (2, 0.5) {};
                        \node[n,label=right:f] (f) at (2, 1.5) {};

                        \draw (a) -- (b);
                        \draw (b) -- (d);
                        \draw (e) -- (f);
                    \end{tikzpicture}
                \end{minipage}

            \item[d)]
                Da in diesem Fall die drei Äquivalenzklassen
                $\left\{\left\{a, b, d\right\}, \left\{c\right\}, \left\{e, f\right\}\right\}$ vorliegen, gilt:

                $$S = \left\{a, b, d\right\}^2 \cup \left\{c\right\}^2 \cup \left\{e, f\right\}^2$$

                \vspace{0.3cm}
                \begin{center}
                    \begin{tikzpicture}
                        \node[n,label=below:a] (a) at (0, 0) {};
                        \node[n,label=below:b] (b) at (1.5, 0) {};
                        \node[n,label=right:c] (c) at (2.2, 1) {};
                        \node[n,label=above:d] (d) at (1.5, 2) {};
                        \node[n,label=above:e] (e) at (0, 2) {};
                        \node[n,label=left:f] (f) at (-0.7, 1) {};

                        \path (a) edge[a, bend left=10     ] (b);
                        \path (b) edge[a, bend left=10, add] (a);

                        \path (b) edge[a, bend left=10     ] (d);
                        \path (d) edge[a, bend left=10, add] (b);

                        \path (a) edge[a, bend left=10, add] (d);
                        \path (d) edge[a, bend left=10, add] (a);

                        \path (e) edge[a, bend left=10     ] (f);
                        \path (f) edge[a, bend left=10, add] (e);

                        \path (a) edge[a, add, ref=180] (a);
                        \path (b) edge[a, add, ref=0] (b);
                        \path (c) edge[a, add, ref=90] (c);
                        \path (d) edge[a, add, ref=0] (d);
                        \path (e) edge[a, add, ref=0] (e);
                        \path (f) edge[a, add, ref=270] (f);
                    \end{tikzpicture}
                \end{center}
        \end{enumerate}

    \item[\textbf{3.}]
        \begin{enumerate}
            \item[a)]
                $R = \{(a,b), (b, a), (b, c), (c, a), (a, a), (b, b), (c, c), (d, d)\}$

                \vspace{0.3cm}
                \begin{center}
                    \begin{tikzpicture}
                        \node[n, label=below:a] (a) at (0, 0) {};
                        \node[n, label=below:b] (b) at (1, 0) {};
                        \node[n, label=below:c] (c) at (2, 0) {};
                        \node[n, label=below:d] (d) at (3, 0) {};

                        \path (a) edge[a, ref=90] (a);
                        \path (b) edge[a, ref=90] (b);
                        \path (c) edge[a, ref=90] (c);
                        \path (d) edge[a, ref=90] (d);

                        \path (a) edge[a, bend left=20] (b);
                        \path (b) edge[a, bend left=20] (a);
                        \path (b) edge[a, bend left=20] (c);
                        \path (c) edge[a, bend left=20] (b);

                        \path (a) edge[a, dashed, add, bend right=70] (c);
                    \end{tikzpicture}
                \end{center}

                Diese Relation ist reflexiv (jeder Knoten hat eine Kante zu sich selbst) und
                symmetrisch (zwei Knoten sind immer durch zwei Kanten in beide Richtungen verbunden).

                Sie ist jedoch nicht transitiv, da z.B. die gestrichelte Kante $(a, c)$ fehlt, und
                es einen ,,Weg'' von $a$ über $b$ nach $c$ gibt ($(a,b),(b,c) \in R$).


            \item[b)]
                $R = \{(a,b), (b, c), (a, c), (a, a), (b, b), (c, c), (d, d)\}$

                \vspace{0.3cm}
                \begin{center}
                    \begin{tikzpicture}
                        \node[n, label=below:a] (a) at (0, 0) {};
                        \node[n, label=below:b] (b) at (1, 0) {};
                        \node[n, label=below:c] (c) at (2, 0) {};
                        \node[n, label=below:d] (d) at (3, 0) {};

                        \path (a) edge[a, ref=90] (a);
                        \path (b) edge[a, ref=90] (b);
                        \path (c) edge[a, ref=90] (c);
                        \path (d) edge[a, ref=90] (d);

                        \path (a) edge[a] (b);
                        \path (b) edge[a, bend left=20] (c);
                        \path (a) edge[a, bend right=70] (c);

                        \path (c) edge[a, dashed, add, bend left=20] (b);
                    \end{tikzpicture}
                \end{center}

                Diese Relation ist transitiv (da sowohl $(a,b),(b,c) \in R$ als auch $(a,c) \in R$ gilt)
                und ebenfalls reflexiv (siehe oben).

                Sie ist jedoch nicht symmetrisch, da z.B. $(b, c) \in R$ gilt, aber die
                gestrichelte Kante $(c,b) \not\in R$.

            \item[c)]
                $R = \{(a,b), (b, a), (b, c), (c, b), (a, c), (c, a), (a, a), (b, b), (c, c)\}$

                \vspace{0.3cm}
                \begin{center}
                    \begin{tikzpicture}
                        \node[n, label=below:a] (a) at (0, 0) {};
                        \node[n, label=below:b] (b) at (1, 0) {};
                        \node[n, label=below:c] (c) at (2, 0) {};
                        \node[n, label=below:d] (d) at (3, 0) {};

                        \path (a) edge[a, ref=90] (a);
                        \path (b) edge[a, ref=90] (b);
                        \path (c) edge[a, ref=90] (c);

                        \path (a) edge[a, bend left=20] (b);
                        \path (b) edge[a, bend left=20] (a);
                        \path (b) edge[a, bend left=20] (c);
                        \path (c) edge[a, bend left=20] (b);
                        \path (a) edge[a, bend right=70] (c);
                        \path (c) edge[a, bend right=50] (a);

                        \path (d) edge[a, dashed, add, ref=90] (d);
                    \end{tikzpicture}
                \end{center}

                Diese Relation ist transitiv (wie (b)) und symmetrisch (wie (a)). Allerdings fehlt
                dem Knoten $d$ die Schleife ($(d, d) \not\in R$, gestrichelte Kante), somit ist sie nicht reflexiv.
        \end{enumerate}

    \item[\textbf{4.}]
        \begin{enumerate}
            \item[a)]
                $R = \{(1, 1), (1, 2), (1, 3), (1, 4), (1, 5), (1, 6), (2, 2), (2, 4), (2, 6), (3, 3), (3, 6), (4, 4), (5, 5), (6, 6)\}$

                \vspace{0.4cm}
                \begin{minipage}[c]{0.3\textwidth}
                    \begin{tikzpicture}
                        \node[n,label=below:1] (1) at (0, 0) {};
                        \node[n,label=below:2] (2) at (1.5, 0) {};
                        \node[n,label=below:3] (3) at (3, 0) {};
                        \node[n,label=above:4] (4) at (0, 1.5) {};
                        \node[n,label=above:5] (5) at (1.5, 1.5) {};
                        \node[n,label=above:6] (6) at (3, 1.5) {};

                        \path (1) edge[a, ref=180] (1);
                        \path (2) edge[a, ref=0] (2);
                        \path (3) edge[a, ref=0] (3);
                        \path (4) edge[a, ref=180] (4);
                        \path (5) edge[a, ref=0] (5);
                        \path (6) edge[a, ref=0] (6);

                        \path (1) edge[a] (2);
                        \path (1) edge[a, bend right=40] (3);
                        \path (1) edge[a] (4);
                        \path (1) edge[->-=0.7] (5);
                        \path (1) edge[a] (6);

                        \path (2) edge[->-=0.7] (4);
                        \path (2) edge[a] (6);

                        \path (3) edge[a] (6);
                    \end{tikzpicture}
                \end{minipage}
                \begin{minipage}[c]{0.5\textwidth}
                    \centering
                    \begin{tikzpicture}
                        \node[n,label=right:1] (1) at (0, 0) {};
                        \node[n,label=right:2] (2) at (0, 1) {};
                        \node[n,label=left:3]  (3) at (1, 1) {};
                        \node[n,label=right:4] (4) at (0, 2) {};
                        \node[n,label=right:5] (5) at (-1, 1) {};
                        \node[n,label=right:6] (6) at (1, 2) {};

                        \draw (1) -- (2);
                        \draw (1) -- (3);
                        \draw (1) -- (5);
                        \draw (2) -- (4);
                        \draw (2) -- (6);
                        \draw (3) -- (6);
                    \end{tikzpicture}
                \end{minipage}

            \item[b)]
                $M = \{1, 2\}$

                $R = \{(\emptyset, \emptyset), (\emptyset, \{1\}), (\emptyset, \{2\}), (\emptyset, M), (\{1\}, \{1\}), (\{1\}, M), (\{2\}, \{2\}), (\{2\}, M), (M, M)\}$

                \vspace{0.4cm}
                \begin{minipage}[c]{0.3\textwidth}
                    \begin{tikzpicture}
                        \node[n,label=right:$\emptyset$]  (0) at (0, 0) {};
                        \node[n,label=right:$\{1\}$]       (1) at (-1, 1) {};
                        \node[n,label=left:$\{2\}$]      (2) at (1, 1) {};
                        \node[n,label={right:$\{1, 2\}$}] (12) at (0, 2) {};

                        \path (0)  edge[a, ref=270] (0);
                        \path (1)  edge[a, ref=180] (1);
                        \path (2)  edge[a, ref=0]   (2);
                        \path (12) edge[a, ref=90]  (12);

                        \path (0) edge[a]  (1);
                        \path (0) edge[a]  (2);
                        \path (0) edge[a]  (12);
                        \path (1) edge[a]  (12);
                        \path (2) edge[a]  (12);
                    \end{tikzpicture}
                \end{minipage}
                \begin{minipage}[c]{0.5\textwidth}
                    \centering
                    \begin{tikzpicture}
                        \node[n,label=below:$\emptyset$]  (0) at (0, 0) {};
                        \node[n,label=left:$\{1\}$]       (1) at (-0.6, 1) {};
                        \node[n,label=right:$\{2\}$]      (2) at (0.6, 1) {};
                        \node[n,label={above:$\{1, 2\}$}] (12) at (0, 2) {};

                        \draw (0) -- (1);
                        \draw (0) -- (2);
                        \draw (1) -- (12);
                        \draw (2) -- (12);
                    \end{tikzpicture}
                \end{minipage}
                \vspace{0.2cm}

            \item[c)]
                $A = \mathcal{P}(M) = \left\{ \emptyset, \left\{ 1 \right\}, \left\{ 2 \right\}, \left\{ 3 \right\}, \left\{ 1, 2 \right\}, \left\{ 1, 3 \right\}, \left\{ 2, 3 \right\}, \left\{ 1, 2, 3 \right\} \right\}$

                \vspace{0.5cm}
                \begin{tikzpicture}
                    \node[n,label=below:$\emptyset$]     (0)   at ( 0, 0) {};
                    \node[n,label=left:$\{1\}$]          (1)   at (-1, 1) {};
                    \node[n,label=right:$\{2\}$]         (2)   at ( 0, 1) {};
                    \node[n,label=right:$\{3\}$]         (3)   at ( 1, 1) {};
                    \node[n,label={left:$\{1, 2\}$}]     (12)  at (-1, 2) {};
                    \node[n,label={above:$\{1, 3\}$}]    (13)  at ( 0, 2) {};
                    \node[n,label={right:$\{2, 3\}$}]    (23)  at ( 1, 2) {};
                    \node[n,label={above:$\{1, 2, 3\}$}] (123) at ( 0, 3) {};

                    \draw (0) -- (1);
                    \draw (0) -- (2);
                    \draw (0) -- (3);
                    \draw (1) -- (12);
                    \draw (1) -- (13);
                    \draw (2) -- (12);
                    \draw (2) -- (23);
                    \draw (3) -- (23);
                    \draw (3) -- (13);
                    \draw (12) -- (123);
                    \draw (23) -- (123);
                    \draw (13) -- (123);
                \end{tikzpicture}

        \end{enumerate}

\end{enumerate}

\end{document}
