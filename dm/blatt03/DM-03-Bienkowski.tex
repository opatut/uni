\documentclass[a4paper,10pt]{scrartcl}
\usepackage[german,ngerman]{babel}
\usepackage[utf8]{inputenc}
\usepackage[T1]{fontenc}
\usepackage{lmodern}
\usepackage{amssymb}
\usepackage{mathtools}
\usepackage{subfigure}
\usepackage{amsmath}
\usepackage{enumerate}
\usepackage{array}
\usepackage{breqn}
\usepackage{fullpage}
\usepackage{tikz}

\usetikzlibrary{patterns}

\author{Paul Bienkowski, Jascha Andersen}
\title{DM 03-B (HA) zum 09.11.2012}
\date{\today}
\begin{document}
\setcounter{secnumdepth}{0}
\maketitle

\begin{enumerate}
        % Aufgabe 1
    \item[\textbf{1.}]
        \begin{enumerate}
            \item[a)]
                \begin{enumerate}
                    \item[(i)]      $177 - 18 = 159, 5 \nmid 159 \rightarrow$ falsch
                    \item[(ii)]     $177 - (-18) = 195, 5 \mid 195 \rightarrow$ wahr
                    \item[(iii)]    $-89 - (-12) = -77, 6 \nmid -77 \rightarrow$ falsch
                    \item[(iv)]     $-123 - 33= -156, 13 \mid -156 \rightarrow$ wahr
                    \item[(v)]      $39 - (-1) = 40, 40 \mid 40 \rightarrow$ wahr
                    \item[(vi)]     $77 - 0 = 77, 11 \mid 77 \rightarrow$  wahr
                    \item[(vii)]    $2^{51}$ ist gerade, also ist $2^{51} - 51$ ungerade, somit gilt $2 \nmid 2^{51} -51 \rightarrow$ falsch
                \end{enumerate}

            \item[b)]
                $ggT(7293, 378) = ggT(378, 111) = ggT(111, 45) = ggT(45, 21) = ggT(21, 3) = 3$

                $$\begin{array}{rcrll}
                    7293 &=& 19 &{}\cdot 378 &{}+ 111\\
                    378  &=&  3 &{}\cdot 111 &{}+ 45\\
                    111  &=&  2 &{}\cdot  45 &{}+ 21\\
                    45   &=&  2 &{}\cdot  21 &{}+ 3\\
                    21   &=&  7 &{}\cdot   3 &{}+ 0.
                \end{array}$$

            \item[c)]
                $\begin{array}[t]{lcr}
                    \lceil\sqrt{7}\rceil    &=& 3\\
                    \lfloor\sqrt{7}\rfloor  &=& 2\\
                    \lceil7.1\rceil     &=& 8\\
                    \lfloor7.1\rfloor   &=& 7\\
                    \lceil-7.1\rceil    &=& -7\\
                    \lfloor-7.1\rfloor  &=& -8\\
                    \lceil-7\rceil      &=& -7\\
                    \lfloor-7\rfloor    &=& -7\\
                \end{array}$
        \end{enumerate}

    \newpage
    \item[\textbf{2.}]
        \begin{enumerate}
            \item[(2)]
                Es ist gegeben, dass $b_1 \mid a_1$ und $b_2 \mid a_2$. Für beliebige $c_1, c_2 \in \mathbb{Z}$
                gilt also $a_1 = b_1 \cdot c_1$ und $a_2 = b_2 \cdot c_2$. Es ist zu zeigen, dass
                $b_1 \cdot b_2 \mid a_1 \cdot a_2$ wahr ist.

                Dies lässt sich wie folgt darstellen:

                $$b_1 \cdot b_2 \mid (b_1 \cdot c_1) \cdot (b_2 \cdot c_2)$$
                $$b_1 \cdot b_2 \mid (b_1 \cdot b_2) \cdot (c_1 \cdot c_2)$$

                Dies ist wahr, da $(c_1 \cdot c_2) \in \mathbb{Z}$ ist. $\Box$

            \item[(3)]
                Es ist gegeben, dass $c \cdot b \mid c \cdot a$. Es ist zu zeigen, dass $b \mid a$ wahr ist.
                Für $d \in \mathbb{Z}$ gilt also:

                $$\begin{array}{rcl}
                    c \cdot a &=& (c \cdot b) \cdot d\\
                    \Rightarrow a &=& b \cdot d
                \end{array}$$

                Aus $a = b \cdot d$ und $d \in \mathbb{Z}$ folgt $a \mid b$. $\Box$

            \item[(4)]
                Es ist gegeben, dass $b \mid a_1$ und $b \mid a_2$. Es ist zu zeigen, dass $b \mid c_1 \cdot a_1
                + c_2 \cdot a_2$ für $c_1, c_2 \in \mathbb{Z}$ gilt. Es folgt für $d_1, d_2 \in \mathbb{Z}$:

                \begin{equation}\label{eq:24}\begin{array}{rcl}
                    a_1 &=& b \cdot d_1\\
                    a_2 &=& b \cdot d_2
                \end{array}\end{equation}

                Damit $b \mid c_1 \cdot a_1 + c_2 \cdot a_2$ gilt, muss für $e$ im folgenden $e \in \mathbb{Z}$ gelten:
                $$c_1 \cdot a_1 + c_2 \cdot a_2 = b \cdot e$$

                Aus \eqref{eq:24} folgt:

                $$\begin{array}{rcl}
                    c_1 \cdot (b \cdot d_1) + c_2 \cdot (b \cdot d_2) &=& b \cdot e\\
                    b \cdot (c_1 \cdot d_1 + c_2 \cdot d_2) &=& b \cdot e\\
                    c_1 \cdot d_1 + c_2 \cdot d_2 &=& e
                \end{array}$$

                Da $c_1, d_1, c_2, d_2 \in \mathbb{Z}$ sind, ist auch $e \in \mathbb{Z}$, somit ist die Aussage bewiesen. $\Box$
        \end{enumerate}

    \item[\textbf{3.}]
        \begin{enumerate}
            \item[a)]
                Die Aussage $3 \mid (n^3 + 2n)$ wird als $A(n)$ bezeichnet.

                \textbf{Induktionsanfang:} $A(0): 3 \mid (0^3 + 2 \cdot 0) \Leftrightarrow 3 \mid 0$ ist wahr.

                \textbf{Induktionsschritt:} Wir nehmen an, die Aussage $A(n)$ gelte für ein beliebiges
                $n \in \mathbb{N}$. Dann gilt:

                \begin{equation}\label{eq:3aIA}\tag{IA}
                    3 \mid (n^3 + 2n)
                \end{equation}

                Es ist zu zeigen, dass die Aussage $A(n + 1)$ ebenfalls gilt, also:

                \begin{equation}\label{eq:3aIS1}
                    3 \mid \left((n+1)^3 + 2(n+1)\right)
                \end{equation}

                Dies lässt sich wie folgt zeigen:

                \begin{equation}\begin{array}{rcl}\label{eq:3aIS2}
                    3 &\mid& (n+1)^3 + 2(n+1)\\
                    3 &\mid& n^3 + 3n^2 + 3n +1 + 2n + 2\\
                    3 &\mid& (n^3 + 2n) + (3n^2  + 3n + 3)\\
                    3 &\mid& \underbrace{(n^3 + 2n)}_{\eqref{eq:3aIS1}} + 3 (n^2  + n + 1)
                \end{array}\end{equation}

                Da $3 \mid n^3 + 2n$ laut \eqref{eq:3aIA} gilt, und $3 | 3(n^2 + n + 1)$ ebenfalls
                wahr ist, ist auch \eqref{eq:3aIS1} wahr. $\Box$

            \item[b)]
                Aus 4 L-Stücken lässt sich ein größeres L-Stück mit doppelter Kantenlänge zusammenlegen \subref{fig:4b1}.
                Dieses L-Stück sei nun als $L_2$ bezeichnet (Länge einer kurzen Kante beträgt 2 Einheiten), das
                Original-L-Stück ist demnach $L_1$.

                Um für $n = 1$ ein $2 \times 2$ Feld nach den Vorgaben zu belegen, benötigt man nur ein $L_1$,
                wie in \subref{fig:4b2} gezeigt.

                Um für $n = 2$ ein $4 \times 4$ Feld zu belegen, benötigt man ein $L_1$ für die obere rechte
                Ecke, es bleibt genau Platz für ein $L_2$. Dies ist in \subref{fig:4b3} für $n = 2$ dargestellt.

                Auf diese Weise lässt sich jedes $2^n \times 2^n$ - Schachbrett mit der Belegung des vorigen Schachbrettes
                $2^{n-1} \times 2^{n-1}$ in der oberen rechten Ecke, plus eines L-Stücks der Größe $L_n$ belegen, da die Kanten
                der noch zu füllende Fläche genau doppelt so lang sind, wie in der vorherigen
                Iteration. Dieser Nachweis funktioniert ähnlich einer vollständigen Induktion,
                da gezeigt ist, dass die Aussage für $n = 1$ gilt, und dass eine geltende
                Aussage auch die nächste Aussage $n + 1$ beweist (ein $L_n$ lässt sich mit 4
                $L_{n-1}$ darstellen).

                \begin{figure}[!hb]
                    \centering
                    \subfigure[]{
                        \label{fig:4b1}
                        \begin{tikzpicture}[scale=0.5]
                            %\draw[fill=gray,pattern=north west lines] (3, 3) rectangle (4, 4);
                            %\draw (0, 0) -- (0, 4) -- (4, 4) -- (4, 0) -- (0, 0);
                            \draw (0, 0) -- (0, 4) -- (2, 4) -- (2, 2) -- (4, 2) -- (4, 0) -- (0, 0);
                            \draw (2, 4) -- (2, 2) -- (4, 2);
                            \draw (2, 3) -- (1, 3) -- (1, 1) -- (3, 1) -- (3, 2);
                            \draw (0, 2) -- (1, 2);
                            \draw (2, 0) -- (2, 1);
                        \end{tikzpicture}
                    }\subfigure[]{
                        \label{fig:4b2}
                        \begin{tikzpicture}[scale=1]
                            \draw[fill=gray,pattern=north west lines] (1, 1) rectangle (2, 2);
                            \draw (0, 0) -- (0, 2) -- (2, 2) -- (2, 0) -- (0, 0);
                        \end{tikzpicture}
                    }\subfigure[]{
                        \label{fig:4b3}
                        \begin{tikzpicture}[scale=0.5]
                            \draw[fill=gray,pattern=north west lines] (3, 3) rectangle (4, 4);
                            \draw (0, 0) -- (0, 4) -- (4, 4) -- (4, 0) -- (0, 0);
                            \node[label=center:$L_{n-1}$] at (3, 2.5) {};

                            \draw[fill=lightgray,draw=none] (0, 0) rectangle (4, 2);
                            \draw[fill=lightgray,draw=none] (0, 0) rectangle (2, 4);
                            \draw (0, 0) -- (0, 4) -- (2, 4) -- (2, 2) -- (4, 2) -- (4, 0) -- (0, 0);
                            \draw[style=dashed] (2, 4) -- (2, 2) -- (4, 2);
                            \draw[style=dashed] (2, 3) -- (1, 3) -- (1, 1) -- (3, 1) -- (3, 2);
                            \draw[style=dashed] (0, 2) -- (1, 2);
                            \draw[style=dashed] (2, 0) -- (2, 1);
                            \node[label=center:$L_{n}$] at (1.65, 1.65) {};
                        \end{tikzpicture}
                    }
                \end{figure}
        \end{enumerate}

    \item[\textbf{4.}]
        \begin{enumerate}
            \item[a)]
                \textbf{Behauptung:} $g$ ist injektiv.

                \textbf{Beweis:} Wäre $g$ nicht injektiv, gäbe es $(x_1, y_1), (x_2, y_2) \in \mathbb{Q} \times \mathbb{Q}$
                mit $(x_1, y_1) \not= (x_2, y_2)$ für die gilt:

                $$\begin{array}{rcl}
                    g(x_1, y_1) &=& g(x_2, y_2)\\[1em]
                    (x_1 y_1^2, x_1 y_1^2 - 3 x_1, (x_1^2 - 2) y_1) &=& (x_2 y_2^2, x_2 y_2^2 - 3 x_2, (x_2^2 - 2) y_2)
                \end{array}$$

                Daraus ergeben sich folgende Gleichungen:

                \begin{equation}\label{eq:4a1}
                    x_1 y_1^2 = x_2 y_2^2
                \end{equation}

                \begin{equation}\label{eq:4a2}
                    x_1 y_1^2 - 3 x_1 = x_2 y_2^2 - 3 x_2
                \end{equation}

                \begin{equation}\label{eq:4a3}
                    (x_1^2 - 2) y_1 = (x_2^2 - 2) y_2
                \end{equation}

                Setzt man \eqref{eq:4a1} in \eqref{eq:4a2} ein, folgt:

                \begin{equation}\label{eq:4a4}\begin{array}{rcl}
                    -3 x_1 &=& -3 x_2\\
                    \Rightarrow x_1 &=& x_2
                \end{array}\end{equation}

                Setzt man \eqref{eq:4a4} in \eqref{eq:4a3} ein, ergibt sich:

                $$\begin{array}{rcl}
                    (x_1^2 - 2) y_1 &=& (x_1^2 - 2) y_2\\
                    \Rightarrow y_1 &=& y_2
                \end{array}$$

                Somit ergibt sich $x_1 = x_2 \wedge y_1 = y_2$, was ein Widerspruch zu $(x_1, y_1) \not= (x_2, y_2)$ ist.
                Damit ist $g$ injektiv. $\Box$

            \item[b)]
                \textbf{Behauptung:} $h$ ist nicht surjektiv.

                \textbf{Beweis:} Es ist ein $(z_1, z_2) \in \mathbb{Z} \times \mathbb{Z}$ anzugeben, das nicht als $h(z)$
                dargestellt werden kann.

                \textbf{Annahme:} Für $(1, 0)$ ist dies der Fall.

                \textbf{Nachweis:}
                    $$\begin{array}{rcl}
                        h(z) &=& (1, 0)\\
                        (z + 2, z - 1) &=& (1, 0)\\
                        \Rightarrow z + 2 = 1 &\wedge& z - 1 = 0\\
                        \Rightarrow z = -1 &\wedge& z = 1
                    \end{array}$$

                    Dies stellt einen Widerspruch dar, somit ist $(1, 0)$ nicht als $h(z)$ dargestellt werden, also ist
                    $h$ nicht surjektiv. $\Box$



        \end{enumerate}


\end{enumerate}
\end{document}
