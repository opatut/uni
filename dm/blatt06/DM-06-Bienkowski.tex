\newcommand{\authorinfo}{Paul Bienkowski, Jascha Andersen, Benedikt Bushart}
\newcommand{\titleinfo}{DM 06-B (HA) zum 30.11.2012}

% PREAMBLE ===============================================================

\documentclass[a4paper,10pt]{scrartcl}
\usepackage[german,ngerman]{babel}
\usepackage[utf8]{inputenc}
\usepackage[T1]{fontenc}
\usepackage{lmodern}
\usepackage{amssymb}
\usepackage{mathtools}
\usepackage{amsmath}
\usepackage{enumerate}
\usepackage{array}
\usepackage{listings}
\usepackage{fullpage}
\usepackage{breqn}
\usepackage{fancyhdr}
\usepackage{lastpage}
\usepackage{tikz}
\usetikzlibrary{decorations.markings}

\author{\authorinfo}
\title{\titleinfo}
\date{\today}

\pagestyle{fancy}
\fancyhf{}
\fancyhead[L]{\authorinfo}
\fancyhead[R]{\titleinfo}
\fancyfoot[C]{\thepage}
\renewcommand{\headrulewidth}{0.4pt}
\renewcommand{\footrulewidth}{0pt}
\renewcommand{\headheight}{12pt}
\renewcommand{\headsep}{12pt}

\begin{document}
\setcounter{secnumdepth}{0}
\maketitle

% DOCUMENT ===============================================================

\begin{enumerate}
    \item[\textbf{1.}]
        \begin{enumerate}
            \item[a)]
                \begin{tabular}[t]{lll}
                    $AB = \begin{pmatrix} 7 & 5 & -2 \\ 2 & 1 & -1 \\ 30 & 17 & 5 \\ 3 & 2 & 5 \end{pmatrix}$ &
                    $AD = \begin{pmatrix} 2 \\ 4 \\ 26 \\ -4 \end{pmatrix}$ &
                    $BB = \begin{pmatrix} 10 & 5 & 1 \\ 5 & 4 & -1 \\ 4 & 2 & 1 \end{pmatrix}$ \\[3em]
                    $CD = \begin{pmatrix} 12 \end{pmatrix}$ &
                    $DC = \begin{pmatrix} 2 & 4 & -4 \\ 3 & 6 & -6 \\ -2 & -4 & 4 \end{pmatrix}$
                \end{tabular}

            \item[b)]
                Das gesuchte Element $(AB)_{3;2}$ lässt sich wie folgt berechnen:

                $$(AB)_{3;2} = 1 \cdot (-2) + 2 \cdot 2 + 3 \cdot 3 + 4 \cdot 1 = 15$$

                Die gesuchte Spalte $(AB)_{i;4}$ lautet folgendermaßen:

                $$(AB)_{i;4} = \begin{pmatrix} 13 \\ 8 \\ 3 \\ 23 \end{pmatrix}$$

        \end{enumerate}

    \item[\textbf{2.}]
        \begin{enumerate}
            \item[a)]

                $B_1 + B_2 = \begin{pmatrix} 2 & 0 \\ 6 & 8 \end{pmatrix}$

                $A(B_1 + B_2) =
                    \begin{pmatrix} 5 & 7 \\ 9 & -1 \\ 8 & 2 \end{pmatrix} \cdot
                    \begin{pmatrix} 2 & 0 \\ 6 & 8 \end{pmatrix} =
                    \begin{pmatrix} 52 & 56 \\ 12 & -8 \\ 28 & 16 \end{pmatrix}
                    $

                \vspace{1em}

                $AB_1 =
                    \begin{pmatrix} 5 & 7 \\ 9 & -1 \\ 8 & 2 \end{pmatrix} \cdot
                    \begin{pmatrix} 1 & 2 \\ 3 & 6 \end{pmatrix} =
                    \begin{pmatrix} 26 & 52 \\ 6 & 12 \\ 14  & 28 \end{pmatrix}
                    $

                $AB_2 =
                    \begin{pmatrix} 5 & 7 \\ 9 & -1 \\ 8 & 2 \end{pmatrix} \cdot
                    \begin{pmatrix} 1 & -2 \\ 3 & 2 \end{pmatrix} =
                    \begin{pmatrix} 26 & 4 \\ 6 & -20 \\ 14  & -12 \end{pmatrix}
                    $

                $AB_1 + AB_2 =
                    \begin{pmatrix} 26 & 52 \\ 6 & 12 \\ 14  & 28 \end{pmatrix} +
                    \begin{pmatrix} 26 & 4 \\ 6 & -20 \\ 14  & -12 \end{pmatrix} =
                    \begin{pmatrix} 52 & 56 \\ 12 & -8 \\ 28 & 16 \end{pmatrix}
                    = A(B_1 + B_2) \;\;\Box
                    $

            \item[b)]
                $AB =
                    \begin{pmatrix}1 & 3 \\ 2 6 6 \end{pmatrix} \cdot
                    \begin{pmatrix}2 & -1 & 5 \\ 3 & 2 & 3 \end{pmatrix} =
                    \begin{pmatrix}11 & 5 & 17 \\ 22 & 10 & 34\end{pmatrix}
                \Leftrightarrow
                 (AB)^T =
                    \begin{pmatrix}11 & 22 \\ 5 & 10 \\ 17 & 34\end{pmatrix}
                    $

                $A^T = \begin{pmatrix} 1 & 2 \\ 3 & 6 \end{pmatrix}, B^T = \begin{pmatrix} 2 & 3 \\ -1 & 2 \\ 5 & 4 \end{pmatrix}
                    \Leftrightarrow
                    B^TA^T =
                    \begin{pmatrix} 2 & 3 \\ -1 & 2 \\ 5 & 4 \end{pmatrix} \cdot
                    \begin{pmatrix} 1 & 2 \\ 3 & 6 \end{pmatrix} =
                    \begin{pmatrix}11 & 22 \\ 5 & 10 \\ 17 & 34\end{pmatrix} =
                    (AB)^T \;\;\Box
                    $

            \item[c)]
                $A^TB^T$ ist unsinnig, da $B^T$ 3 Zeilen hat, $A^T$ jedoch nur 2 Spalten.
        \end{enumerate}

    \item[\textbf{3.}]

    \item[\textbf{4.}]
        \begin{enumerate}
            \item[a)]

            \item[b)]

        \end{enumerate}


\end{enumerate}

\end{document}
