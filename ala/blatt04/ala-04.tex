\newcommand{\authorinfo}{Paul Bienkowski, Hans Ole Hatzel}
\newcommand{\titleinfo}{ALA 04 (HA) zum 02.05.2013}

% PREAMBLE ===============================================================

\documentclass[a4paper,11pt]{scrartcl}
\usepackage[german,ngerman]{babel}
\usepackage[utf8]{inputenc}
\usepackage[T1]{fontenc}
\usepackage{lmodern}
\usepackage{amssymb}
\usepackage{amsmath}
\usepackage{enumerate}
\usepackage{fancyhdr}
\usepackage{pgfplots}
\usetikzlibrary{calc}

\author{\authorinfo}
\title{\titleinfo}
\date{\today}

\pagestyle{fancy}
\fancyhf{}
\fancyhead[L]{\authorinfo}
\fancyhead[R]{\titleinfo}
\fancyfoot[C]{\thepage}

\newcommand{\bra}[1]{\left(#1\right)}
\newcommand{\limnn}[2]{\lim\limits_{n \rightarrow #1}\bra{#2}}
\newcommand{\limn}[1]{\lim\limits_{n \rightarrow \infty}\bra{#1}}
\newcommand{\rowi}[1]{\sum_{i=#1}^{\infty}}
\newcommand{\row}{\rowi{0}}
\newcommand{\step}[1]{\textbf{#1}}

\begin{document}
\maketitle
\begin{enumerate}
    \item[\textbf{1.}]
    \begin{enumerate}
        \item[a)]
        \begin{enumerate}
            \item[(i)]  \( f'(x) = 35x^4 + 9x^2 + 1 \)
            \item[(ii)] \( f'(x) = 8(3x^7 - 4x^3 + x^2 - 3x + 1)^7 \cdot (21x^6 - 12x^2 + 2x -3) \)
            \item[(iii)]\( f'(x) = (3x^4 + 2x)(\frac{x}{\sqrt{x^2+1}}) + (12x^3 + 2) \sqrt{x^2 + 1} \)
            \item[(iv)] \( f'(x) = (x^3 + 1)(\frac{4x^3+6x}{x^4+3x^2+1}) + 3x \cdot \ln\bra{x^4+3x^2+1} \)
            \item[(v)]  \( f'(x) = e^{x^3 + x^2 +1} \bra{\frac{1}{2\sqrt{x}} + 3x^2\sqrt{x} + 2x\sqrt{x}} \)
            \item[(vi)] \( f'(x) = \frac{4x^3}{2\sqrt{x^4+1}} \cdot \ln(x) + \frac{1}{x} \sqrt{x^4 + 1} \)
        \end{enumerate}

        \item[b)]
            \( q(x) = \frac{5x^2+1}{x-3} \)

            \( q'(x) = \frac{10x(x - 3) - (5x^2 + 1)}{(x-3)^2} = \frac{5x^2 - 30x - 1}{(x-3)^2} \)

            \( q''(x) = \frac{(10x - 30)(x-3)^2 - (5x^2 - 30x - 1) \cdot 2(x-3)}{(x-3)^4}
            = \frac{10 (x-3)^2 - 10x^2 + 60x + 2}{(x-3)^3} \)

            \( q'''(x) = \frac{\bra{20(x-3)-20x+60}(x-3)^3 - \bra{10(x-3)^2-10x^2+60x+2} \cdot 3(x-3)^2}{(x-3)^6}
            = -\frac{21}{(x-3)^4} \)
    \end{enumerate}

    \item[\textbf{2.}]
        \[
            \lim\limits_{x \rightarrow 6}\bra{\frac{f(x) - f(6)}{x-6}} =
            \lim\limits_{x \rightarrow 6}\bra{\frac{|3 - \frac{1}{2}x| - |3 - \frac{1}{2} \cdot 6}{x-6}} =
            \lim\limits_{x \rightarrow 6}\bra{\frac{|3 - \frac{1}{2}x|}{x-6}} = \]
        \[
            \lim\limits_{x \rightarrow 6}\bra{\sqrt{\frac{(3 - \frac{1}{2}x)^2}{(x-6)^2}}} =
            \sqrt{\lim\limits_{x \rightarrow 6}\bra{\frac{9 - 3x + \frac{1}{4}x^2}{x^2 - 2x + 36}}} =
            \sqrt{\frac{1}{4}} = \frac{1}{2}
        \]

        \begin{center}\begin{tikzpicture}[>=stealth]
            \begin{axis}[
                ymin=0,ymax=4,
                xmin=-1,xmax=8,
                x=1cm, y=1cm,
                axis x line=middle,
                axis y line=middle,
                axis line style=->,
                xlabel={$x$},
                ylabel={$y$},
                ]
                \addplot[no marks, black, -] expression[domain=-1:6,samples=4]{3-(1/2)*x} node[pos=0.65,anchor=north]{};
                \addplot[no marks, black, -] expression[domain=6:9,samples=4]{-3+(1/2)*x} node[pos=0.65,anchor=north]{};
            \end{axis}
        \end{tikzpicture}\end{center}

    \item[\textbf{3.}]
    \begin{enumerate}
        \item[a)]
            \( f(x) = (x^4 + 1)^{x+2} = e^{\ln\bra{(x^4 + 1)^{x+2}}} = e^{\ln\bra{x^4 + 1} (x + 2)} \)

            \( f'(x) = e^{\ln\bra{x^4 + 1} (x + 2)} \cdot \bra{\ln\bra{x^4+1} + (x+2)\frac{4x^3}{x^4+1}} \)

            \( =  (x^4 + 1)^{x+2} \cdot \bra{\ln\bra{x^4+1} + \frac{4x^4 + 8x^3}{x^4+1}} \)

        \item[b)]
            \( f(x) = x^{\frac{1}{2}} = e^{\ln\bra{x^{\frac{1}{2}}}} = e^{\frac{1}{2} \cdot \ln x} \)

            \( f'(x) = e^{\frac{1}{2} \cdot \ln x} \cdot \frac{1}{2x} = x^{\frac{1}{2}} \cdot x^{-1} \cdot \frac{1}{2} =
            \frac{1}{2} \cdot x^{-\frac{1}{2}} = \frac{1}{2 \sqrt{x}} \)

            \( g(x) = \bra{\frac{1}{2}}^x = e^{\ln\bra{\bra{\frac{1}{2}}^x}} = e^{x \ln\frac{1}{2}} \)

            \( g'(x) = e^{x \ln\frac{1}{2}} \cdot \ln\frac{1}{2} = \bra{e^{\ln\frac{1}{2}}}^x \cdot \ln\frac{1}{2} =  \bra{\frac{1}{2}}^x \cdot \ln \frac{1}{2} \)

        \item[c)]
        \begin{enumerate}
            \item[(i)]
                \( g(x) = \bra{x^2 + 1}^{4x + 1} = e^{\ln\bra{(x^2+1)^{4x+1}}} = e^{(4x + 1) \ln\bra{x^2 + 1}} \)

                \( g'(x) = \bra{e^{\ln\bra{x^2 + 1}}}^{4x + 1} \cdot \bra{ 4 \ln\bra{x^2+1} + \bra{4x+1}\bra{\frac{2x}{x^2+1}}) } \)

                \( = \bra{x^2+1}^{4x+1} \bra{4\ln\bra{x^2+1} + \frac{8x^2+2x}{x^2+1} } \)

            \item[(ii)]
                \( h(x) = \bra{x-3}^{3x^4 + 5} = e^{\ln\bra{x-3} \cdot \bra{3x^4 + 5}} \)

                \( h'(x) = \bra{x-3}^{3x^4+5} \cdot \bra{ 12x^3 \cdot \ln\bra{x-3} + \frac{1}{x-3}\bra{3x^4 + 5} } \)

        \end{enumerate}

    \end{enumerate}

    \newpage
    \item[\textbf{4.}]
    \begin{enumerate}
        \item[a)]
            \( g(p) = 10^5 \bra{\frac{1}{p} - \frac{3}{p^2}} \)

            \( g'(p) = 10^5 \bra{-\frac{1}{p^2} + \frac{3 \cdot 2p}{p^4}} = 10^5 \bra{\frac{6}{p^3} - \frac{1}{p^2}} \)

            \( g''(p) = 10^5 \bra{-\frac{18 p^2}{p^6} + \frac{2p}{p^4}} = 10^5 \bra{\frac{2}{p^3} - \frac{18}{p^4}} \)

            Es ist das Maximum der Funktion $g(p)$ zu bestimmen. Am Maximum gilt $g'(p) = 0$ und $g''(p) < 0$.

            \[ g'(p) = 0
            \Leftrightarrow 0 = 10^5 \bra{\frac{6}{p^3} - \frac{1}{p^2}}
            \Leftrightarrow \frac{6}{p^3} = \frac{1}{p^2}
            \Leftrightarrow 6p^2 = p^3
            \Leftrightarrow p = 6 \]

            Der Wert $p=6$ liegt zunächst im zulässigen Intervall von $[3, 100]$.
            Um zu zeigen, dass es sich wirklich um ein Maximum handelt, wird gezeigt,
            dass $g''(p)$ für $p = 6$ kleiner als $0$ ist:

            \[ g''(6) = 10^5 \bra{\frac{2}{6^3} - \frac{18}{6^4}} = 10^5 \bra{\frac{1}{108} - \frac{1}{72}} = - \frac{12500}{27} < 0.
            \hspace{1em} \Box \]

        \item[b)]
        \begin{enumerate}
            \item[(i)]
                \step{Ableitungen:}

                \( f(x) = -2x^3-x+25 \)\\
                \( f'(x) = -6x^2-1 \)\\
                \( f''(x) = -12x \)

                \step{Stellen mit Steigung 0:}

                \[ 0 = -6x^2 - 1
                \Leftrightarrow 6x^2 = -1
                \Leftrightarrow x \not\in \mathbb{R} \]

                \step{Funktionswerte an Intervallgrenzen:}

                \( f(5) = - 2 \cdot 5^3 - 5 + 25 = -250-5+25 = -230  \)\\
                \( f(-5) = - 2 \cdot (-5)^3 + 5 + 25 = 250+5+25 = 280  \)

                Die Funktion hat keinen Punkt mit Steigung 0, ist also streng monoton.
                Die globalen Extremwerte liegen daher an den Grenzen des Definitionsintervalls,
                das globale Maximum bei $x = -5$, das globale Minimum bei $x = 5$.

            \newpage
            \item[(ii)]
                \step{Ableitungen:}

                \( g(x) = x^3-6x^2+3x+8 \)\\
                \( g'(x) = 3x^2-12x+3 \)\\
                \( g''(x) = 6x-12 \)

                \step{Stellen mit Steigung 0 bestimmen:}

                \(\begin{array}{rrcl}& 0 &=& 3x^2 - 12x + 3 \\
                \Leftrightarrow& 0 &=& x^2 - 4x + 1 \\
                \Leftrightarrow& 0 &=& (x - 2)^2 - 3 \\
                \Leftrightarrow& x - 2 &=& \pm \sqrt{3} \\
                \Leftrightarrow& x_1 &=& 2 - \sqrt{3} \approx 0.2679 \\
                               & x_2 &=& 2 + \sqrt{3} \approx 3.7321 \\
                \end{array}\)

                Nur die Stelle $x_1 = 2 - \sqrt{3}$ liegt im Definitionsbereich.

                \step{Art der berechneten Extremstelle:}

                \( g''(2 - \sqrt{3}) = 6 \bra{2 - \sqrt{3}} - 12 = 12 - 6\sqrt{3} - 12 = - 6 \sqrt{3} < 0 \)

                \step{Funktionswerte an Intervallgrenzen:}

                \( g(0) = 0^3 - 6 \cdot 0^2 + 3 \cdot 0 + 8 = 8 \)\\
                \( g(3) = 3^3 - 6 \cdot 3^2 + 3 \cdot 3 + 8 = 27 - 54 + 9 + 8 = -10 \)

                An der Stelle $x = 2 - \sqrt{3}$ liegt das globale Maximum der Funktion
                vor. Das globale Minimum liegt am rechten Rand des Defintionsintervalles,
                bei $x = 3$.

            \item[(iii)]
                \step{Ableitungen:}

                \( h(x) = e^{2x-3} - e^{x+2} \)\\
                \( h'(x) = 2e^{2x-3} - e^{x+2} \)\\
                \( h''(x) = 4e^{2x-3} - e^{x+2} \)

                \step{Stellen mit Steigung 0 bestimmen:}

                \(\begin{array}{rrcl}& 0 &=& 2e^{2x-3} - e^{x+2} \\
                \Leftrightarrow& 2e^{2x-3} &=& e^{x+2}\\
                \Leftrightarrow& \ln 2 + 2x - 3 &=& \bra{x+2} \\
                \Leftrightarrow& x &=& 5 - \ln 2 \approx 4.3069
                \end{array}\)

                \step{Art der berechneten Extremstelle:}

                \( h''(5 - \ln 2) \approx 548.317 > 0 \)

                \step{Funktionswerte an Intervallgrenzen:}

                \( h(0) = e^{2\cdot0 - 3} - e^{0 + 2} = e^{-3} - e^2 \approx -7.3393 \)\\
                \( h(5) = e^{2\cdot5 - 3} - e^{5 + 2} = 0 \)\\

                Die einzige Stelle mit Steigung 0 liegt an bei $x = 5 - \ln 2$, es handelt
                sich hierbei um das globale Minumum. Das globale Maximum der Funktion
                liegt an der rechten Grenze des Definitionsintervalls bei $x = 5$.

        \end{enumerate}
    \end{enumerate}
\end{enumerate}

\end{document}
