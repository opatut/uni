\newcommand{\authorinfo}{Paul Bienkowski, Hans Ole Hatzel}
\newcommand{\titleinfo}{ALA 10 (HA) zum 27.06.2013}

% PREAMBLE ===============================================================

\documentclass[a4paper,11pt,fleqn]{scrartcl}
\usepackage[german,ngerman]{babel}
\usepackage[utf8]{inputenc}
\usepackage[T1]{fontenc}
\usepackage{lmodern}
\usepackage{amssymb}
\usepackage{amsmath}
\usepackage{enumerate}
\usepackage{fancyhdr}
\usepackage{pgfplots}
\usepackage{multicol}
\usetikzlibrary{calc}
\usetikzlibrary{patterns}

\author{\authorinfo}
\title{\titleinfo}
\date{\today}

\pagestyle{fancy}
\fancyhf{}
\fancyhead[L]{\authorinfo}
\fancyhead[R]{\titleinfo}
\fancyfoot[C]{\thepage}

\newcommand{\bra}[1]{\left(#1\right)}
\newcommand{\limnn}[2]{\lim\limits_{n \rightarrow #1}\bra{#2}}
\newcommand{\limxn}[2]{\lim\limits_{x \rightarrow #1}\bra{#2}}
\newcommand{\limn}[1]{\lim\limits_{n \rightarrow \infty}\bra{#1}}
\newcommand{\limx}[1]{\lim\limits_{x \rightarrow \infty}\bra{#1}}
\newcommand{\limxx}[2]{\lim\limits_{x \rightarrow #1}\bra{#2}}
\newcommand{\limz}[1]{\lim\limits_{z \rightarrow \infty}\bra{#1}}
\newcommand{\rowi}[1]{\sum_{i=#1}^{\infty}}
\newcommand{\row}{\rowi{0}}
\newcommand{\step}[1]{\textbf{#1}}
\newcommand{\dX}[1]{\, \mathrm{d}#1}
\newcommand{\dx}[0]{\dX{x}}
\newcommand{\dt}[0]{\dX{t}}
\newcommand{\partx}[0]{\frac{\partial f}{\partial x}  (x,y)}
\newcommand{\party}[0]{\frac{\partial f}{\partial y}  (x,y)}
\newcommand{\partxp}[1]{\frac{\partial^#1 f}{\partial x^#1}  (x,y)}
\newcommand{\partyp}[1]{\frac{\partial^#1 f}{\partial y^#1}  (x,y)}
\newcommand{\partxy}[1]{\frac{\partial^#1 f}{\partial xy}  (x,y)}
\newcommand{\partyx}[1]{\frac{\partial^#1 f}{\partial yx}  (x,y)}
\begin{document}
\maketitle
\begin{enumerate}
    \item[\textbf{1.}]
        \begin{enumerate}
            \item[(a)]
                \[\partx = 4xy^2 - 3y + 4\]
                \[\party = 4x^2 y - 3x\]  
            \item[(b)]
                \[\partx = (-sin(x^2y) \cdot 2xy) \cdot e^{xy} + cos(x^2y) \cdot (e^{xy} \cdot y)\]
                \[\party = (-sin(x^2y) \cdot x^2) \cdot e^{xy} + cos(x^2y) \cdot (e^{xy} \cdot x)\]
            \item[(c)]
                \[\partx = \frac{cos(x) \cdot (x^2 + y^2) - (2x + y^2) \cdot (sin(x) + cos(x))}{(x^2 + y^2)^2}\]
                \[\party = \frac{-sin(y) \cdot (x^2 + y^2) - (x^2 + 2y) \cdot (sin(x) + cos(x))}{(x^2 + y^2)^2}\]
            \item[(d)]
                \[f(x,y) = (1 - x^2 - y^2)^{\frac{1}{2}}\]
                \[\partx = \frac{(1 - x^2 -y^2)^{-\frac{1}{2}}}{2} \cdot (1 - 2x)\]
                \[\party = \frac{(1 - x^2 -y^2)^{-\frac{1}{2}}}{2} \cdot (1 - 2y)\] 
        \end{enumerate}
    \item[\textbf{2.}]
        \[\partx = 4xy^3 + ye^{x^2 y} \cdot 2xy\]
        \[\party = 6x^2y^2 + e^{x^2 y} \cdot x^2\]
        \[\partxy{2} = 6xy^2 + ((e^{x^2 y} * x^2) + (2xy * 2x)) \]
        \[\partyx{2} = 6xy^2 + e^{x^2 y} * 2xy + 2x * x^2\]

    \item[\textbf{3.}]
        \begin{enumerate}
            \item[(i)]
            Ableitungen bestimmen:
                \begin{multicols}{2}
                \[\partx = 4x -2y -2\]
                \[\party = 2y -2x -4\]
                \[\partxp{2} = 4\]


                \[\partyp{2} = 2\]
                \[\partxy{2} = \partyx{2} = -2\]
                \end{multicols}
                Hessesche Matrix: $                
                \begin{pmatrix}
                     4 & -2 \\
                    -2 &  2
                \end{pmatrix}
                $

                $4 > 0$,
                $\Delta = 4 > 0$

                Es handelt sich somit um ein strenges lokales Minimum, denn $H_f$ ist positiv definiert.
            \item[(ii)]
            \begin{multicols}{2}
                \[\partx = 2x -3y -1\]
                \[\party = 4y -3x -1\]
                \[\partxp{2} = 2\]


                \[\partyp{2} = 4\]
                \[\partxy{2} = \partyx{2} = -3\]
            \end{multicols}
                Hessesche Matrix: $                
                \begin{pmatrix}
                     2 & -3 \\
                    -3 &  4
                \end{pmatrix}
                $

                $2 > 0$,
                $\Delta = -1 < 0$

                Es ist keien Aussage über das maximum möglich, denn $H_f$ ist indefinit.
            \item[(iii)]
            \begin{multicols}{2}
                \[\partx = 6x^2 - 12\]
                \[\party = 3y^2 - 27\]
                \[\partxp{2} = 12x\]


                \[\partyp{2} = 6y\]
                \[\partxy{2} = \partyx{2} = 0\]
            \end{multicols}
                kritische Stellen bestimmen:
                \begin{eqnarray*}
                    3y^2 - 27 &=& 0 \\
                    6x^2 - 12 &=& 0 
                \end{eqnarray*}

                \begin{eqnarray*}
                    y &=& \sqrt{\frac{27}{3}} = 3\\
                    x &=& \sqrt{2} 
                \end{eqnarray*}
                Hessesche Matrix: $               
                \begin{pmatrix}
                     12 \cdot \sqrt{2} & 0 \\
                    0 &  18
                \end{pmatrix}
                $

                $\approx 16,97 > 0$,
                $\Delta \approx 305,470 > 0$

                Es handelt sich somit um ein strenges lokales Minimum, denn $H_f$ ist positiv definiert.
        \end{enumerate}
    \item[\textbf{4.}]
        \begin{enumerate}
            \item[(a)]
                Sei $G(x,y)$ eine Funktion die den Gewinn darstellt. 
                Der Gewinn erechnet sich aus dem Abziehen der Kosten vom Erlös.
                $G(x,y) = 12x + 28y - C(x,y)$
                Von dieser Funktion ist nun ein Maximum zu suchen.
            \item[(b)]
                \[x = 320 - 2y\]
                Wir setzen die Zusatzbedingung in die Funktion aus a ein. \\
                $7(320 - 2y) + 22y - 0,01(320 - 2y)^2 - 0,02(320 - 2y)y - 0,16y^2 -120 =$
                $-0,16y^2 + 14,4y + 1096$

                Extrempunkt suchen: \\
                Es handelt sich um ein Polynom zweiten Grades, also besitzt es genau einen Extrempunkt. Durch ausprobieren kommt man leicht auf 45. \\
                Der Gewinn bei mit dieser Zusatzbedingung ist also bei 45 verkauften Einheiten des Gutes B maximal (also 230 Einheiten des Gutes A).
            \item[(c)]
                b: $-0,16(45)^2 + 14,4 \cdot 45 + 1096 = 1420$
        \end{enumerate}
\end{enumerate}

\end{document}
