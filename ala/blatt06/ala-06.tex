\newcommand{\authorinfo}{Paul Bienkowski, Hans Ole Hatzel}
\newcommand{\titleinfo}{ALA 06 (HA) zum 30.05.2013}

% PREAMBLE ===============================================================

\documentclass[a4paper,11pt,fleqn]{scrartcl}
\usepackage[german,ngerman]{babel}
\usepackage[utf8]{inputenc}
\usepackage[T1]{fontenc}
\usepackage{lmodern}
\usepackage{amssymb}
\usepackage{amsmath}
\usepackage{enumerate}
\usepackage{fancyhdr}
\usepackage{pgfplots}
\usepackage{multicol}
\usetikzlibrary{calc}
\usetikzlibrary{patterns}

\author{\authorinfo}
\title{\titleinfo}
\date{\today}

\pagestyle{fancy}
\fancyhf{}
\fancyhead[L]{\authorinfo}
\fancyhead[R]{\titleinfo}
\fancyfoot[C]{\thepage}

\newcommand{\bra}[1]{\left(#1\right)}
\newcommand{\limnn}[2]{\lim\limits_{n \rightarrow #1}\bra{#2}}
\newcommand{\limn}[1]{\lim\limits_{n \rightarrow \infty}\bra{#1}}
\newcommand{\limx}[1]{\lim\limits_{x \rightarrow \infty}\bra{#1}}
\newcommand{\rowi}[1]{\sum_{i=#1}^{\infty}}
\newcommand{\row}{\rowi{0}}
\newcommand{\step}[1]{\textbf{#1}}
\newcommand{\dx}[0]{\, \mathrm{d}x}

\begin{document}
\maketitle
\begin{enumerate}
    \item[\textbf{1.}]
        \begin{enumerate}
            \item[(a)]
                Wir berechnen $\lim\limits_{n \to \infty} O_n$:
                \begin{eqnarray*}
                O_n &=& \frac{b-a}{n}\sum_{i=1}^n f(x_i)=\frac{3}{n}\sum_{i=1}^n f\left(\frac{3i}{n}\right)=\frac{3}{n}\sum_{i=1}^n\left(\frac{3i}{n}\right)^3=\frac{3 \cdot 3^3}{n^4}\sum_{i=1}^n i^3 \\ \nonumber
                &=& \frac{81}{n^4} \cdot \frac{n^2\cdot\left(n+1\right)^2}{4}=\frac{81n^4+192n^3+81n^2}{4n^4}
                \end{eqnarray*}
                Also gilt $\lim\limits_{n\to \infty} O_n = \lim \limits_{n\to\infty}\left(\frac{81n^4+192n^3+81n^2}{4n^4}\right)= \lim \limits_{n\to\infty}\left(\frac{n^4}{n^4}\cdot\frac{81+\frac{192}{n}+\frac{81}{n^2}}{4}\right)=\frac{81}{4}$
            \item[(b)]
                    \[\int\limits_0^3 f(x) \dx = \int\limits_0^3\frac{1}{4}x^4 \, dx = \frac{1}{4}3^4 - \frac{1}{4}0^4= \frac{81}{4}\]
        \end{enumerate}
    \item[\textbf{2.}]
        \begin{enumerate}
            \item[(i)]
                $\int\limits_1^3 f(x) \dx = [\frac{1}{3}x^3 - \frac{1}{2}x^2 - 6x]_1^3= \left( \frac{1}{3}3^3 - \frac{1}{2}3^2 - 18 \right)- \left( \frac{1}{3} - \frac{1}{2} - 6\right) = - \frac{27}{2}$

            \item[(ii)]
                $\int\limits_1^3 f(x) \dx = [\sqrt[3]{x}]_1^3 = \sqrt[3]{3}-\sqrt[3]{1} \approx 0.442$

            \item[(iii)]

                $\int\limits_1^3 f(x) \dx =[\arctan(x)]_1^3=\arctan(3) - \arctan(1) \approx 26.565$
            \item[(iv)]
                $\int\limits_1^3 f(x) \dx = [x \cdot \ln(x) - x]_1^3 = \left( 3 \cdot \ln (3) -3 \right)-\left(1 \cdot \ln (1) - 1 \right) \approx 1.296$

            \item[(v)]
                $\int\limits_1^3 f(x) \dx = [-e^{-x}]_1^3 = -e^{-3} - \left( -e^{-1} \right) \approx 0.318$

            \begin{multicols}{2}
                \item[(i)]
                    \begin{tikzpicture}[baseline=(current bounding box.north)]
                        \begin{axis}[
                            ymin=-7,ymax=5,
                            xmin=-5,xmax=5,
                            x=0.3cm, y=0.3cm,
                            axis x line=middle,
                            axis y line=middle,
                            axis line style=->,
                            xlabel={$x$},
                            ylabel={$y$},
                            ]
                            \addplot[no marks, black, -] expression[domain=-10:10,samples=200]{x^2 - x - 6};
                            \addplot[pattern=north east lines, pattern color=red, no marks] expression[domain=1:3,samples=200]{x^2 - x - 6} \closedcycle;
                        \end{axis}
                    \end{tikzpicture}
                \item[(ii)]
                    \begin{tikzpicture}[baseline=(current bounding box.north)]
                        \begin{axis}[
                            ymin=-1,ymax=5,
                            xmin=-1,xmax=5,
                            x=0.5cm, y=0.5cm,
                            axis x line=middle,
                            axis y line=middle,
                            axis line style=->,
                            xlabel={$x$},
                            ylabel={$y$},
                            ]
                            \addplot[no marks, black, -] expression[domain=-10:10,samples=400]{x^(1/3)};
                            \addplot[pattern=north east lines, pattern color=red, no marks, -] expression[domain=1:3,samples=400]{x^(1/3)} \closedcycle;
                        \end{axis}
                    \end{tikzpicture}
                \item[(iii)]
                    \begin{tikzpicture}[baseline=(current bounding box.north)]
                        \begin{axis}[
                            ymin=0,ymax=2,
                            xmin=0,xmax=5,
                            x=1cm, y=1cm,
                            axis x line=middle,
                            axis y line=middle,
                            axis line style=->,
                            xlabel={$x$},
                            ylabel={$y$},
                            ]
                            \addplot[no marks, black, -] expression[domain=0:10,samples=400]{1/(1+x^2)};
                            \addplot[pattern=north east lines, pattern color=red, no marks, -] expression[domain=1:3,samples=400]{1/(1+x^2)} \closedcycle;
                        \end{axis}
                    \end{tikzpicture}
                \item[(iv)]
                    \begin{tikzpicture}[baseline=(current bounding box.north)]
                        \begin{axis}[
                            ymin=0,ymax=2,
                            xmin=0,xmax=5,
                            x=1cm, y=1cm,
                            axis x line=middle,
                            axis y line=middle,
                            axis line style=->,
                            xlabel={$x$},
                            ylabel={$y$},
                            ]
                            \addplot[no marks, black, -] expression[domain=0:10,samples=400]{ln(x)};
                            \addplot[pattern=north east lines, pattern color=red, no marks, -] expression[domain=1:3,samples=400]{ln(x)} \closedcycle;
                        \end{axis}
                    \end{tikzpicture}
                \item[(v)]
                    \begin{tikzpicture}[baseline=(current bounding box.north)]
                        \begin{axis}[
                            ymin=0,ymax=2,
                            xmin=0,xmax=5,
                            x=1cm, y=1cm,
                            axis x line=middle,
                            axis y line=middle,
                            axis line style=->,
                            xlabel={$x$},
                            ylabel={$y$},
                            ]
                            \addplot[no marks, black, -] expression[domain=0:5,samples=50]{e^(-x)};
                            \addplot[pattern=north east lines, pattern color=red, no marks, -] expression[domain=1:3,samples=50]{e^(-x)} \closedcycle;
                        \end{axis}
                    \end{tikzpicture}
            \end{multicols}
        \end{enumerate}
    \item[\textbf{3.}]
        \begin{enumerate}
            \item[(i)]
                $F(x) = \frac{1}{5}x^5 + \frac{1}{2}x^2 + 5x$

            \item[(ii)]
                $\int x^{-\frac{3}{2}} \dx = \frac{2}{5} x^{-\frac{5}{2}}$

            \item[(iii)]
                $f'(x)=\sin (3x) , g(x)=x \textnormal{. Dann ist } f(x)= -\frac{1}{3} \cos 3x \textnormal{ und } g'(x)=1$

            \begin{eqnarray*}
                \int x\cdot\sin 3x \, dx &=& -\frac{1}{3}\sin 3x \cdot x - \int - \frac{1}{3} \sin 3x \cdot 1 \, dx \nonumber  \\
                &=& -\frac{1}{3}\sin 3x \cdot x - \frac{1}{3} \int -\sin 3x \, dx \nonumber \\
                &=&-\frac{1}{3} x \cdot \sin 3x  -\frac{1}{9} \cdot \cos 3x \nonumber \nonumber
            \end{eqnarray*}
            Probe: $\left( -\frac{1}{3} x \cdot \sin 3x  -\frac{1}{9} \cdot \cos 3x \right)' = x\cdot\sin (3x)$

            \item[(iv)]
            \begin{eqnarray*}
                f'(x)=x^3 , g(x)&=&ln x \textnormal{. Dann ist} f(x)= \frac{1}{4}x^4 \textnormal{ und } g'(x)=\frac{1}{x} \nonumber \\
                \int x^3 \cdot ln x \, dx &=& \frac{1}{4}x^4 \cdot \frac{1}{x} - \int \frac{1}{4}x^4 \cdot \frac{1}{x} \, dx \nonumber  \\
                &=& \frac{1}{4}x^3 - \frac{1}{16} x^4 \nonumber
            \end{eqnarray*}
            Probe: $\left( \frac{1}{4}x^3 - \frac{1}{16} x^4 \right)' = x^3 \cdot ln(x)$

            \item[(v)]
            \begin{eqnarray*}
                \int x^2e^x\, \mathrm{d}x &=& x^2e^x - \int 2xe^x \dx \\
                &=& x^2e^x - 2 \int xe^x \dx \\
                &=& x^2e^x - 2 \left( xe^x - \int e^x \dx \right) \\
                &=& x^2e^x - 2xe^x + 2e^x \\
                &=& e^x(x^2 - 2x + 2)
            \end{eqnarray*}
            Probe: $(e^x(x^2 - 2x + 2))' = x^2e^x$
        \end{enumerate}

    \item[\textbf{4.}]
        \begin{enumerate}
            \item[(i)]
                Es sei $u(x) = 5 + 2x$ und $s(u) = \sqrt{u}$.

                \begin{eqnarray*}
                    \int \cos(\sqrt{2x+5}) dx
                    &=& \int \frac{1}{2} \cos(\sqrt{u}) du \\
                    &=& \int \frac{1}{2} \cos(s) \cdot \frac{1}{s'(u)} ds \\
                    &=& \int \sqrt{u} \cos(s) ds \\
                    &=& \int s \cos(s) ds \\
                    &=& s \sin(s) - \int \sin(s) ds \\
                    &=& s \sin(s) + \cos(s) \\
                    &=& \sqrt{5+2x} \sin(\sqrt{5+2x}) + \cos(\sqrt{5+2x})
                \end{eqnarray*}

                Probe:

                \begin{eqnarray*}
                    && \bra{\sqrt{5+2x} \sin(\sqrt{5+2x}) + \cos(\sqrt{5+2x})}' \\
                    &=& \frac{2}{2\sqrt{5+2x}} \sin(\sqrt{5+2x}) + \sqrt{5+2x} \cos(\sqrt{5+2x}) \frac{2}{2\sqrt{5+2x}} \\
                    && - \sin(\sqrt{5+2x}) \frac{2}{2\sqrt{5+2x}}\\
                    &=& \cos(\sqrt{5+2x}) \Box
                \end{eqnarray*}

            \newpage
            \item[(ii)]
                Es sei $u(x) = \sqrt[3]{x}$.

                \begin{eqnarray*}
                    \int \sin(\sqrt[3]{x}) dx
                    &=& 3 \int u^2 \sin(u) du \\
                    &=& 3 \bra{- u^2 \cos(u) - 2 \int -u\cos(u) du} \\
                    &=& - 3 u^2 \cos(u) - 6 \bra{ -u\sin(u) - \int -\sin(u) du } \\
                    &=& - 3 u^2 \cos(u) + 6 u\sin(u) + 6 \cos(u) \\
                    &=& (6 - 3 u^2) \cos(u) + 6u \sin(u) \\
                    &=& (6 - 3 \sqrt[3]{x}^2) \cos(\sqrt[3]{x}) + 6\sqrt[3]{x} \sin(\sqrt[3]{x})
                \end{eqnarray*}

            \item[(iii)]
                Es sei $u(x) = \sqrt{\frac{2}{7}x + 3}$.

                \begin{eqnarray*}
                    \int e^{\sqrt{\frac{2}{7}x + 3}} dx
                    &=& \int e^u \cdot \frac{1}{u'} du \\
                    &=& \int 7u e^u du \\
                    &=& 7u \cdot e^u - 7 \int e^u du \\
                    &=& 7u \cdot e^u - 7 \cdot e^u \\
                    &=& 7e^u (u - 1) \\
                    &=& 7 e^{\sqrt{\frac{2}{7}x + 3}} \bra{\sqrt{\frac{2}{7}x + 3} - 1}
                \end{eqnarray*}

            \item[(iv)]
        \end{enumerate}

    \item[\textbf{5.}]
        \begin{enumerate}
            \item[(i)]
            \begin{multicols}{2}
            Nullstellen der ersten Ableitung: $2,6$ \\
            $f''(2) = -12$ \\
            $f''(6) = 12$

            Grenzen überprüfen: \\
            $f(6) = 1$ \\
            $f(0) = 1$

            Extremwerte berechnen: \\
            $f(2) = 26$ \\
            $f(6) = 1$ \\
            \\
            \end{multicols}
            Bei $x = 6$ befindet sich somit ein lokales minimum und bei $x = 2$ ein lokales maximum.
            Die Tageshöchsttemperatur lässt sich am Maximum der Funktion im Intervall $[0,6]$ erkennen sie liegt also bei x = 2.
            Also beträgt die Tageshöchsttemperatur $26\,^{\circ}\mathrm{C}.$

            \item[(ii)]
            Die Tagestiefsttemperatur lässt sich am Minimum der Funktion im Intervall $[0,6]$ erkennen, aus (i) folg somit das sie zweimal gemessen wurde (bei $x = 6$ und $x = 0$).
            Die Tagestiefsttemperatur beträgt also: $1\,^{\circ}\mathrm{C}$
            \item[(iii)]
            \[ F(x) = \frac{1}{4}x^4 - \frac{12}{3}x^3 + 18x^2 + 1x\]
            \[ \int_0^6 x^3 - 12x^2 + 36x + 1 \, \mathrm{d}x = F(6) - F(0) = 114\] \\
            \[ \frac{114}{6} = 19 \]

            Die Durchschnittstemperatur beträgt also $19\,^{\circ}\mathrm{C}$.
        \end{enumerate}
\end{enumerate}

\end{document}
