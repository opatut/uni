\newcommand{\authorinfo}{Paul Bienkowski, Hans Ole Hatzel}
\newcommand{\titleinfo}{ALA 05 (HA) zum 15.05.2013}

% PREAMBLE ===============================================================

\documentclass[a4paper,11pt]{scrartcl}
\usepackage[german,ngerman]{babel}
\usepackage[utf8]{inputenc}
\usepackage[T1]{fontenc}
\usepackage{lmodern}
\usepackage{amssymb}
\usepackage{amsmath}
\usepackage{enumerate}
\usepackage{fancyhdr}
%\usepackage{pgfplots}
%\usetikzlibrary{calc}

\author{\authorinfo}
\title{\titleinfo}
\date{\today}

\pagestyle{fancy}
\fancyhf{}
\fancyhead[L]{\authorinfo}
\fancyhead[R]{\titleinfo}
\fancyfoot[C]{\thepage}

\newcommand{\bra}[1]{\left(#1\right)}
\newcommand{\limnn}[2]{\lim\limits_{n \rightarrow #1}\bra{#2}}
\newcommand{\limn}[1]{\lim\limits_{n \rightarrow \infty}\bra{#1}}
\newcommand{\rowi}[1]{\sum_{i=#1}^{\infty}}
\newcommand{\row}{\rowi{0}}
\newcommand{\step}[1]{\textbf{#1}}

\begin{document}
\maketitle
\begin{enumerate}
    \item[\textbf{1.}]
        \begin{enumerate}
            \item[(i)]
                \(f(x) = x^{-\frac{5}{4}} + x^\frac{7}{6}\) \\
                \(f'(x)= -\frac{1}{12}x^{-\frac{13}{12}}\)
            \item[(ii)]
                \(f'(x)= cos(x^2) \cdot 2x\)
            \item[(iii)]
                \(f'(x)= 2 \cdot sin(x) \cdot cos(x)\)
            \item[(iv)]
                \(f'(x)= cos(x)^2 + sin(x) \cdot - sin(x)\)
            \item[(v)]  
                \(f'(x)= \frac{1}{\sqrt{1-\sqrt{x^2}}} \cdot \frac{1}{2 \sqrt{x}} = \frac{1}{2 \sqrt{x-x^2}}\)
            \item[(vi)]
                \(f'(x)= (x^3 - 1)^{acrtan(x)} \cdot \frac{1}{1+x^2} \cdot ln(x^3 - 1) + arctan(x) \cdot \left( \frac{1}{x^3-1} \cdot 3x^2 \right) \)
        \end{enumerate}

    \item[\textbf{2.}]

            
    \item[\textbf{3.}]
        \( f(1) = -1 \) \\
        \( f(2) = 17 \) 

        Die werte von $f(x)$ haben im Intervall $[1,2]$ einen Vorzeichenwechsel. Da die Funktion (wie alle Polynome) stetig ist, muss es dementsprechend eine Nullstelle im Intervall geben.

        \[ x_1 = \frac{2 - 4 \cdot 2^3 - (10 \cdot 2) + 5}{24 - 10} = 1.5526316\]
        Entsprechend berechnen sich weitere \(x_n\) Werte.


        \begin{align*}
        x_0 &= 2 \\
        x_1 &= 1.5526316 \\
        x_2 &= 1.3177844 \\
        x_3 &= 1.2277567 \\
        x_4 &= 1.2122722 \\
        x_5 &= 1.2118115 \\
        x_6 &= 1.2118111 \\
        x_7 &= 1.2118111 \\
        \end{align*}

        Das Newtensche Näherungsverfahren gibt uns also einen ungefähren Wert von 1.2118111 für die Nullstelle zurück.

    \newpage
    \item[\textbf{4.}]
        Die Seitenlängen des Rechtecks seien $a$ und $b$ Seitenlängen des Rechtecks, und $F$ seine Fläche.
        Dabei haben die sich gegenüberliegenden Seiten die beide Teil des Seils sind beide die Länge $a$.
        Sowie L wie in der Aufgabe vorgegeben eine Konstante für die Länge der Leine.

        \(F = a \cdot b\) \\
        \(L = 2a + b\) \\
        \(b = L - 2a\) \\
        \(F = a \cdot (L - 2a) = aL - 2a^2\) \\
        Diese Funktion zeigt nun die Fläche in Abhängikeit der Seitenlänge $a$. Nun ist das Maximum zu suchen. \\
        \(f(a) = aL - 2a^2\) \\
        \(f'(a) = L - 4a\) \\
        Nullstelle der 1. Ableitung suchen: \\
        \(L - 4a = 0\) \\
        \(a = \frac{L}{4}\) \\
        Da es sich um eine quadratische Funktion handelt haben wir damit bereits das Maximum gefunden.
        Das Rechteck ist also maximal groß wenn wir $a = \frac{L}{4}$ und folglich $b = \frac{L}{2}$ wählen.
    \item[\textbf{5.}]
        \begin{enumerate}
            \item[a)]
            \item[b)]
        \end{enumerate}
    \end{enumerate}

\end{document}
