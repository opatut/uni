\newcommand{\authorinfo}{Paul Bienkowski, Hans Ole Hatzel}
\newcommand{\titleinfo}{ALA 04 (HA) zum 02.05.2013}

% PREAMBLE ===============================================================

\documentclass[a4paper,11pt]{scrartcl}
\usepackage[german,ngerman]{babel}
\usepackage[utf8]{inputenc}
\usepackage[T1]{fontenc}
\usepackage{lmodern}
\usepackage{amssymb}
\usepackage{amsmath}
\usepackage{enumerate}
\usepackage{fancyhdr}
\usepackage{pgfplots}
\usetikzlibrary{calc}

\author{\authorinfo}
\title{\titleinfo}
\date{\today}

\pagestyle{fancy}
\fancyhf{}
\fancyhead[L]{\authorinfo}
\fancyhead[R]{\titleinfo}
\fancyfoot[C]{\thepage}

\newcommand{\bra}[1]{\left(#1\right)}
\newcommand{\limnn}[2]{\lim\limits_{n \rightarrow #1}\bra{#2}}
\newcommand{\limn}[1]{\lim\limits_{n \rightarrow \infty}\bra{#1}}
\newcommand{\rowi}[1]{\sum_{i=#1}^{\infty}}
\newcommand{\row}{\rowi{0}}
\newcommand{\step}[1]{\textbf{#1}}

\begin{document}
\maketitle
\begin{enumerate}
    \item[\textbf{1.}]
        \begin{enumerate}
            \item[(i)] 
            \item[(ii)] 
            \item[(iii)]
            \item[(iv)] 
            \item[(v)]  
            \item[(vi)] 
        \end{enumerate}

    \item[\textbf{2.}]


    \item[\textbf{3.}]
        \[ f(1) = -1 \]
        \[ f(2) = 17 \]

        Die werte von $f(x)$ haben im Intervall $[1,2]$ einen Vorzeichenwechsel. Da die FUnktion (wie alle Polynome) stetig ist,
        muss es dementsprechend eine Nullstelle im Intervall geben.

        \[ x_0 = \frac{42^3 - 10 \cdot 2 + 5}{12 \cdot 2^2 - 10} = 1.5526316\]
        \[ x_1 = \frac{41.5526316^3 - 10 \cdot 1.5526316 + 5}{12 \cdot 1.5526316^2 - 10} = 1.3177844\]
        \[ x_2 = \frac{41.3177844^3 - 10 \cdot 1.3177844 + 5}{12 \cdot 1.3177844^2 - 10} = 1.2277567\]
        \[ x_3 = \frac{41.2277567^3 - 10 \cdot 1.2277567 + 5}{12 \cdot 1.2277567^2 - 10} = 1.2122722\]
        \[ x_4 = \frac{41.2122722^3 - 10 \cdot 1.2122722 + 5}{12 \cdot 1.2122722^2 - 10} = 1.2118115\]
        \[ x_5 = \frac{41.2118115^3 - 10 \cdot 1.2118115 + 5}{12 \cdot 1.2118115^2 - 10} = 1.2118111\]

        Das Newtensche Näherungsverfahren gibt uns also einen ungefähren Wert von 1.2118111 für die Nullstelle zurück.

    \newpage
    \item[\textbf{4.}]
    \item[\textbf{5.}]
        \begin{enumerate}
            \item[a)]
            \item[b)]
        \end{enumerate}
    \end{enumerate}

\end{document}
