\newcommand{\authorinfo}{Paul Bienkowski, Hans Ole Hatzel}
\newcommand{\titleinfo}{ALA 05 (HA) zum 16.05.2013}

% PREAMBLE ===============================================================

\documentclass[a4paper,11pt]{scrartcl}
\usepackage[german,ngerman]{babel}
\usepackage[utf8]{inputenc}
\usepackage[T1]{fontenc}
\usepackage{lmodern}
\usepackage{amssymb}
\usepackage{amsmath}
\usepackage{enumerate}
\usepackage{fancyhdr}
\usepackage{pgfplots}
\usepackage{multicol}
\usetikzlibrary{calc}

\author{\authorinfo}
\title{\titleinfo}
\date{\today}

\pagestyle{fancy}
\fancyhf{}
\fancyhead[L]{\authorinfo}
\fancyhead[R]{\titleinfo}
\fancyfoot[C]{\thepage}

\newcommand{\bra}[1]{\left(#1\right)}
\newcommand{\limnn}[2]{\lim\limits_{n \rightarrow #1}\bra{#2}}
\newcommand{\limn}[1]{\lim\limits_{n \rightarrow \infty}\bra{#1}}
\newcommand{\limx}[1]{\lim\limits_{x \rightarrow \infty}\bra{#1}}
\newcommand{\rowi}[1]{\sum_{i=#1}^{\infty}}
\newcommand{\row}{\rowi{0}}
\newcommand{\step}[1]{\textbf{#1}}

\begin{document}
\maketitle
\begin{enumerate}
    \item[\textbf{1.}]
        \begin{enumerate}
            \item[(i)]
                \(f(x) = x^{-\frac{5}{4}} \cdot x^\frac{7}{6}\) \\
                \(f'(x)= -\frac{1}{12}x^{-\frac{13}{12}}\)
            \item[(ii)]
                \(f'(x)= cos(x^2) \cdot 2x\)
            \item[(iii)]
                \(f'(x)= 2 \cdot sin(x) \cdot cos(x)\)
            \item[(iv)]
                \(f'(x)= cos(x)^2 + sin(x) \cdot - sin(x)\)
            \item[(v)]
                \(f'(x)= \frac{1}{\sqrt{1-\sqrt{x^2}}} \cdot \frac{1}{2 \sqrt{x}} = \frac{1}{2 \sqrt{x-x^2}}\)
            \item[(vi)]
                \(f'(x)= (x^3 - 1)^{arctan(x)} \cdot \frac{1}{1+x^2} \cdot ln(x^3 - 1) + arctan(x) \cdot \left( \frac{1}{x^3-1} \cdot 3x^2 \right) \)
        \end{enumerate}

    \item[\textbf{2.}]
        \begin{enumerate}
            \item[1.]
                $f$ ist definiert in $\mathbb{R}$, denn dadurch kann im Nenner nie
                $0$ stehen, da $x^2$ nie negativ wird.

            \item[2.]
                \(f(x) =  \frac{2x}{1+x^2}\) \\
                \(f'(x) =  -\frac{2(x^2-1)}{(1+x^2)^2}\) \\
                \(f''(x) =  \frac{4x \cdot (-1+x^2)}{(1+x^2)^3}\) \\
                Nullstellen von $f$: 0 \\
                Nullstellen von $f'$: 1, -1 \\
                Nullstellen von $f''$: 0, 1 und -1
            \item[3.]
                \( \limx{\frac{2x}{1+x^2}}=\limx{\frac{x \cdot 2}{x^2 \cdot (\frac{1}{x^2})}} =\limx{\frac{2}{x \cdot (\frac{1}{x^2})}} \) \\
                Folglich gilt für die gesuchten Grenzwerte: \\
                \(\lim\limits_{x \rightarrow \infty}f(x) = 0\) \\
                \(\lim\limits_{x \rightarrow -\infty}f(x) = 0\)
            \item[4.]
                Aus 3. folgt:
                \begin{multicols}{2}
                    \begin{itemize}
                        \item $f(x) > 0$ für alle $x \in (0,\infty)$
                        \item $f(x) < 0$ für alle $x \in (-\infty,0)$
                    \end{itemize}

                    \begin{itemize}
                        \item $f'(x) > 0$ für alle $x \in (-1,1)$
                        \item $f'(x) < 0$ für alle $x \in (1,\infty)$
                        \item $f'(x) < 0$ für alle $x \in (-\infty,-1)$
                    \end{itemize}

                    \begin{itemize}
                        \item $f''(x) < 0$ für alle $x \in (0,1)$
                        \item $f''(x) > 0$ für alle $x \in (-1,0)$
                        \item $f''(x) < 0$ für alle $x \in (-\infty,-1)$
                        \item $f''(x) > 0$ für alle $x \in (\infty,1)$
                    \end{itemize}
                \end{multicols}

                Diese Ergebnisse lassen sich nun folgendermaßen darstellen:

                \begin{center} \begin{tikzpicture}
                    \draw[-](0,2)--(10,2);
                    \draw[-](3.33,2.2)--(3.33,1.8);
                    \draw[-](6.66,2.2)--(6.66,1.8);
                    \node[above] at (3.33,2.35){-1};
                    \node[above] at (6.66,2.35){1};
                    \node[align=center, below] at (1.66,2){$f$ ist streng \\ monoton fallend};
                    \node[align=center, below] at (5,2){$f$ ist streng \\ monoton wachsend};
                    \node[align=center, below] at (8.33,2){$f$ ist streng \\ monoton fallend};

                    \draw[-](0,4)--(10,4);
                    \draw[-](5,4.2)--(5,3.8);
                    \node[below] at (5,3.75){0};
                    \node[align=right, below] at (2.5,4){$f$ ist negativ};
                    \node[align=right, below] at (7.5,4){$f$ ist positiv};

                    \draw[-](0,0)--(10,0);
                    \draw[-](2.5,0.2)--(2.5,-.2);
                    \draw[-](5,0.2)--(5,-.2);
                    \draw[-](7.5,0.2)--(7.5,-.2);
                    \node[above] at (2.5,.25){-1};
                    \node[above] at (5,.25){0};
                    \node[above] at (7.5,.25){1};
                    \node[align=center, below] at (1.25,0){$f$ ist \\ streng konvex};
                    \node[align=center, below] at (3.75,0){$f$ ist \\ streng konkav};
                    \node[align=center, below] at (6.25,0){$f$ ist \\ streng konvex};
                    \node[align=center, below] at (8.75,0){$f$ ist \\ streng konkav};
                \end{tikzpicture} \end{center}

            \item[5.]
                Aus 4. ergibt sich:

                \begin{itemize}
                    \item $f$ hat bei x = 1 ein Maximum
                    \item $f$ hat bei x = -1 ein Minimum
                    \item $f$ hat bei x = 0 einen Wendepunkt
                \end{itemize}

            \item[6.]
                Asymptote:
                $g(x) = ax + b$ für $x \to \infty$ :

                \begin{align*}
                        a&= &\limx{\frac{f(x)}{x}}&= &\limx{\frac{2x}{x+x^{3}}}   &=& 0 \\
                        b&= &\limx{f(x)-ax}       &= &\limx{\frac{2x}{1+x^{2}}-0} &=& 0
                \end{align*}
                Für $x \to \infty$ ist die Asymptote also $g(x)=0$.
                Enstsprechend kann man einfach zeigen, dass $f$ für \( x \to -\infty \) ebenso eine Asymptote \( g(x)=0 \) hat.

                Da $g(x)$ immer gleich 0 ist befindet sich der Snittpunkt mit $f(x)$ bei der bereits gefundenen Nullstelle der Funktion, also $x = 0$.

            \item[7.]
                \begin{tikzpicture}[baseline=(current bounding box.north)]
                    \begin{axis}[
                        ymin=-2,ymax=2,
                        xmin=-5,xmax=5,
                        x=1cm, y=1cm,
                        axis x line=middle,
                        axis y line=middle,
                        axis line style=->,
                        xlabel={$x$},
                        ylabel={$y$},
                        ]
                        \addplot[no marks, black, -] expression[domain=-10:10,samples=400]{(2*x)/(1+x^2)};
                    \end{axis}
                \end{tikzpicture}

                \def\arraystretch{1.5}
                \begin{tabular}{c || c | c | c | c | c | c | c}
                    \hline
                    $x$    & $0$ & $0.5$ & $1$ & $4$ & $-0.5$ & $-1$ & $-4$  \\ \hline
                    $f(x)$ & $0$ & $0.8$ & $1$ & $\frac{8}{17}$ & $-0.6$ & $-1$ & $-\frac{8}{17}$ \\ \hline
                \end{tabular}
        \end{enumerate}

    \item[\textbf{3.}]
        \( f(1) = -1 \)

        \( f(2) = 17 \)

        Die Werte von $f(x)$ haben im Intervall $[1,2]$ einen Vorzeichenwechsel. Da die Funktion (wie alle Polynome) stetig ist, muss es dementsprechend eine Nullstelle im Intervall geben.

        \[ x_1 = \frac{2 - 4 \cdot 2^3 - (10 \cdot 2) + 5}{24 - 10} = 1.5526316\]
        Entsprechend berechnen sich weitere \(x_n\) Werte.

        \begin{multicols}{2}
            \(\begin{array}{rl}
            x_0 &= 2 \\
            x_1 &= 1.5526316 \\
            x_2 &= 1.3177844 \\
            x_3 &= 1.2277567 \\
            \end{array}\)

            \(\begin{array}{rl}
            x_4 &= 1.2122722 \\
            x_5 &= 1.2118115 \\
            x_6 &= 1.2118111 \\
            x_7 &= 1.2118111 \\
            \end{array}\)
        \end{multicols}

        Das Newtensche Näherungsverfahren gibt uns also einen ungefähren Wert von 1.2118111 für die Nullstelle zurück.

    \item[\textbf{4.}]
        Die Seitenlängen des Rechtecks seien $a$ und $b$ Seitenlängen des Rechtecks, und $F$ seine Fläche.
        Dabei haben die sich gegenüberliegenden Seiten die beide Teil des Seils sind beide die Länge $a$.
        Sowie L wie in der Aufgabe vorgegeben eine Konstante für die Länge der Leine.

        \(F = a \cdot b\) \\
        \(L = 2a + b\) \\
        \(b = L - 2a\) \\
        \(F = a \cdot (L - 2a) = aL - 2a^2\) \\

        Diese Funktion zeigt nun die Fläche in Abhängikeit der Seitenlänge $a$. Nun ist das Maximum zu suchen. \\

        \(f(a) = aL - 2a^2\) \\
        \(f'(a) = L - 4a\) \\

        Nullstelle der 1. Ableitung suchen: \\

        \(L - 4a = 0\) \\
        \(a = \frac{L}{4}\) \\

        Da es sich um eine quadratische Funktion handelt haben wir damit bereits das Maximum gefunden.
        Das Rechteck ist also maximal groß wenn wir $a = \frac{L}{4}$ und folglich $b = \frac{L}{2}$ wählen.
    \item[\textbf{5.}]
        Die Einheit $cm$ wurde weggelassen.

        \( V = 1000 = \pi r^2 h \Leftrightarrow h = \frac{1000}{\pi r^2} \)

        \begin{enumerate}
            \item[a)]
                \( A(r) = 2 \pi r^2 + 2 \pi r h = 2 \pi r^2 + 2 \pi r \frac{1000}{\pi r^2} = 2 \pi r^2 + \frac{500}{r} \)

                \( A'(r) = 4 \pi r - \frac{500}{r^2} \)

                \( A''(r) = 4 \pi + \frac{1000}{r^3} \)

                Am Minimum für die Fläche $A$ gilt $A'(r) = 0$ und $A''(r) > 0$:

                \( A'(r) = 0 \Leftrightarrow 4 \pi r = \frac{500}{r^2} \Leftrightarrow r^3 = \frac{125}{\pi} \Leftrightarrow r = \sqrt[3]{\frac{125}{\pi}} \approx 3.4139 \)

                \( A''(r) = 4 \pi + \frac{1000}{\sqrt[3]{\frac{125}{\pi}}} > 0 \)

                Die Höhe muss dann so gewählt werden:

                \( h = \frac{1000}{2 \pi \sqrt[3]{\frac{125}{\pi}}} \approx 27.3114 \)

            \item[b)]
                \( A(r) = \pi r^2 + 2 \pi r h = \pi r^2 + 2 \pi r \frac{1000}{\pi r^2} = \pi r^2 + \frac{500}{r} \)

                \( A'(r) = 2 \pi r - \frac{500}{r^2} \)

                \( A''(r) = 2 \pi + \frac{1000}{r^3} \)

                Am Minimum für die Fläche $A$ gilt $A'(r) = 0$ und $A''(r) > 0$:

                \( A'(r) = 0 \Leftrightarrow 2 \pi r = \frac{500}{r^2} \Leftrightarrow r^3 = \frac{250}{\pi} \Leftrightarrow r = \sqrt[3]{\frac{250}{\pi}} \approx 4.3013 \)

                \( A''(r) = 2 \pi + \frac{1000}{\sqrt[3]{\frac{250}{\pi}}} > 0 \)

                Die Höhe muss dann so gewählt werden:

                \( h = \frac{1000}{\pi \sqrt[3]{\frac{125}{\pi}}} \approx 17.2051 \)
        \end{enumerate}
    \end{enumerate}

\end{document}
