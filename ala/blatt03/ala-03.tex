\newcommand{\authorinfo}{Paul Bienkowski, Hans Ole Hatzel}
\newcommand{\titleinfo}{ALA 02 (HA) zum 18.04.2013}

% PREAMBLE ===============================================================

\documentclass[a4paper,11pt]{scrartcl}
\usepackage[german,ngerman]{babel}
\usepackage[utf8]{inputenc}
\usepackage[T1]{fontenc}
\usepackage{lmodern}
\usepackage{amssymb}
\usepackage{amsmath}
\usepackage{enumerate}
\usepackage{fancyhdr}
\usepackage{pgfplots}

\author{\authorinfo}
\title{\titleinfo}
\date{\today}

\pagestyle{fancy}
\fancyhf{}
\fancyhead[L]{\authorinfo}
\fancyhead[R]{\titleinfo}
\fancyfoot[C]{\thepage}

\newcommand{\bra}[1]{\left(#1\right)}
\newcommand{\limn}[1]{\lim\limits_{n \rightarrow \infty}\bra{#1}}
\newcommand{\rowi}[1]{\sum_{i=#1}^{\infty}}
\newcommand{\row}{\rowi{0}}

\begin{document}
\maketitle
\begin{enumerate}
% Aufgabe 1
    \item[\textbf{1.}]
        \begin{enumerate}
            \item[(a)]
                \begin{tikzpicture}[>=stealth]
                    \begin{axis}[
                        ymin=0,ymax=9,
                        axis x line=middle,
                        axis y line=middle,
                        axis line style=->,
                        xlabel={$x$},
                        ylabel={$y$},
                        ]
                    \addplot[no marks, black, -] expression[domain=0:2,samples=100]{(3/2)*x+2} 
                                    node[pos=0.65,anchor=north]{};
                    \addplot[no marks, black, -] expression[domain=2:4,samples=100]{-x+5} 
                                    node[pos=0.65,anchor=north]{};
                    \addplot[no marks, black, -] expression[domain=4:6,samples=100]{(1/2)*x-1} 
                                    node[pos=0.65,anchor=north]{};
                    \addplot[no marks, black, -] expression[domain=6:8,samples=100]{x-3} 
                                    node[pos=0.65,anchor=north]{};
                    \addplot[no marks, mark repeat=20, black, -] expression[domain=8:10,samples=100]{2*x-11} 
                                    node[pos=0.65,anchor=north]{};
                    \end{axis}
                \end{tikzpicture}
                Unstetigkeitsstellen: 2,4,6,8
            \item[(b)]
                \begin{tikzpicture}[>=stealth]
                    \begin{axis}[
                        ymin=0,ymax=9,
                        axis x line=middle,
                        axis y line=middle,
                        axis line style=->,
                        xlabel={$x$},
                        ylabel={$y$},
                        ]
                         \addplot[no marks, black, -] expression[domain=0:10,samples=100]{x) - floor(x};
                    \end{axis}
                \end{tikzpicture}
                TODO: stetigkeit für $x \notin \mathbb{Z}$ nachweisen
        \end{enumerate}

    \item[\textbf{2.}]
        \begin{enumerate}
            \item[a]

        \end{enumerate}
        % Aufgabe 3
    \item[\textbf{3.}]
        \begin{enumerate}
            Zu zeigen ist das $g(f(x_n)) \rightarrow g(f(x_0))$ die Nacheinanderausführung also stetig sind.
            Eine Folge $x_n \rightarrow x_0$ wobei $x_n$ und $x_0$ jeweils im Definitionsbereich von $g \circ f$ liegen. Da $f$ in $x_0$ stetig ist folgt daraus das $f(x_n) \rightarrow f(x_0)$ g ist für $f(x_0)$ stetig also gilt: $$g(f(x_n)) \rightarrow g(f(x_0))$$
            Stetigkeit der Nacheinanderausführung ist somit bewiesen.
        \end{enumerate}
\end{enumerate}

\end{document}
