\newcommand{\authorinfo}{Paul Bienkowski, Hans Ole Hatzel}
\newcommand{\titleinfo}{ALA 02 (HA) zum 18.04.2013}

% PREAMBLE ===============================================================

\documentclass[a4paper,11pt]{scrartcl}
\usepackage[german,ngerman]{babel}
\usepackage[utf8]{inputenc}
\usepackage[T1]{fontenc}
\usepackage{lmodern}
\usepackage{amssymb}
\usepackage{amsmath}
\usepackage{enumerate}
\usepackage{fancyhdr}
\usepackage{pgfplots}
\usetikzlibrary{calc}

\author{\authorinfo}
\title{\titleinfo}
\date{\today}

\pagestyle{fancy}
\fancyhf{}
\fancyhead[L]{\authorinfo}
\fancyhead[R]{\titleinfo}
\fancyfoot[C]{\thepage}

\newcommand{\bra}[1]{\left(#1\right)}
\newcommand{\limnn}[2]{\lim\limits_{n \rightarrow #1}\bra{#2}}
\newcommand{\limn}[1]{\lim\limits_{n \rightarrow \infty}\bra{#1}}
\newcommand{\rowi}[1]{\sum_{i=#1}^{\infty}}
\newcommand{\row}{\rowi{0}}

\begin{document}
\maketitle
\begin{enumerate}
% Aufgabe 1
    \item[\textbf{1.}]
        \begin{enumerate}
            \item[(a)]
                Graph:

                \begin{tikzpicture}[>=stealth]
                    \begin{axis}[
                        ymin=0,ymax=9,
                        x=1em,
                        y=1em,
                        axis x line=middle,
                        axis y line=middle,
                        axis line style=->,
                        xlabel={$x$},
                        ylabel={$y$},
                        ]
                    \addplot[no marks, black, -] expression[domain=0:2,samples=100]{(3/2)*x+2}
                                    node[pos=0.65,anchor=north]{};
                    \addplot[no marks, black, -] expression[domain=2:4,samples=100]{-x+5}
                                    node[pos=0.65,anchor=north]{};
                    \addplot[no marks, black, -] expression[domain=4:6,samples=100]{(1/2)*x-1}
                                    node[pos=0.65,anchor=north]{};
                    \addplot[no marks, black, -] expression[domain=6:8,samples=100]{x-3}
                                    node[pos=0.65,anchor=north]{};
                    \addplot[no marks, mark repeat=20, black, -] expression[domain=8:10,samples=100]{2*x-11}
                                    node[pos=0.65,anchor=north]{};

                    \fill[black] (0em, 2em) circle (2pt);
                    \fill[black] (2em, 3em) circle (2pt);
                    \fill[black] (6em, 3em) circle (2pt);
                    \end{axis}
                \end{tikzpicture}

                unstetig bei $x = 2$ und $x = 6$
            \item[(b)]
                Graph:

                \begin{tikzpicture}[>=stealth]
                    \begin{axis}[
                        ymin=0,ymax=5,
                        axis x line=middle,
                        axis y line=middle,
                        axis line style=->,
                        x=1em,
                        y=1em,
                        xlabel={$x$},
                        ylabel={$y$}
                        ]

                        \foreach \n in {0,...,9}{
                            \addplot[no marks, black, -] expression[domain=\n:\n+1,samples=100]{x - \n};
                        }

                        \fill[black] (0em, 0em) circle (2pt);
                        \fill[black] (1em, 0em) circle (2pt);
                        \fill[black] (2em, 0em) circle (2pt);
                        \fill[black] (3em, 0em) circle (2pt);
                        \fill[black] (4em, 0em) circle (2pt);
                        \fill[black] (5em, 0em) circle (2pt);
                        \fill[black] (6em, 0em) circle (2pt);
                        \fill[black] (7em, 0em) circle (2pt);
                        \fill[black] (8em, 0em) circle (2pt);
                        \fill[black] (9em, 0em) circle (2pt);
                    \end{axis}
                \end{tikzpicture}

                Für $x     \in \mathbb{N}$ gilt:
                $$\lim\limits_{x \rightarrow n}\bra{g(x)} = n - \lfloor n \rfloor = n - (n - 1) = 1$$

                Für $x     \in \mathbb{N}$ gilt:
                $$g(x) = x - \lfloor x \rfloor = x - x = 0$$
        \end{enumerate}

    \newpage
    \item[\textbf{2.}]
        \begin{enumerate}
            \item[a)]

                $\limn{\frac{\sqrt{3n^2-2n+5} - \sqrt{n}}{\sqrt{n^2 - n + 1} + 4n}}
                = \limn{\frac{\sqrt{3-\frac{2}{n}+\frac{5}{n^2}} - \sqrt{\frac{1}{n}}}{\sqrt{1 - \frac{1}{n} + \frac{1}{n^2}} + 4}}
                = \frac{\sqrt{3}}{\sqrt{1} + 4} \stackrel{*}{=} \frac{\sqrt{3}}{5}$

                Im mit * markierten Schritt wird die Stetigkeit der Wurzelfunktion vorausgesetzt.

            \item[b)]

                $\limn{\cos\bra{\frac{\sqrt{10n^2-n}-n}{2n + 3}}}
                \stackrel{*}{=} \cos\bra{\limn{\frac{\sqrt{10 - \frac{1}{n}} - 1}{2 - \frac{3}{n}}}}
                \stackrel{**}{=} \cos\bra{\frac{\sqrt{10} - 1}{2}}
                \approx 0.4703$

                Im mit ** markierten Schritt wird die Stetigkeit der Wurzelfunktion vorausgesetzt.
                Im mit * markierten Schritt wird die Stetigkeit der Cosinusfunktion vorausgesetzt.

        \end{enumerate}

    \item[\textbf{3.}]
        Zu zeigen ist, dass $g(f(x_n)) \rightarrow g(f(x_0))$, die Nacheinanderausführung also stetig ist.
        Eine Folge $x_n \rightarrow x_0$ wobei $x_n$ und $x_0$ jeweils im Definitionsbereich von $g \circ f$ liegen.
        Da $f$ in $x_0$ stetig ist folgt daraus das $f(x_n) \rightarrow f(x_0)$ g ist für $f(x_0)$ stetig
        also gilt:

        $$g(f(x_n)) \rightarrow g(f(x_0))$$

        Stetigkeit der Nacheinanderausführung ist somit bewiesen.

    \item[\textbf{4.}]

        $$\lim\limits_{x \rightarrow 0}\bra{g(x)} =
        \lim\limits_{x \rightarrow 0}\bra{x^2 \cdot \sin\bra{\frac{1}{x}}} = 0$$

        Der erste Teil ($x^2$) konvergiert gegen 0, der zweite Teil ($\sin\bra{\frac{1}{x}}$)
        gegen $\sin(\infty)$, was zwar nicht definiert ist, jedoch im Intervall
        zwischen -1 und 1 liegen muss (Sinusfunktion). Daher konvergiert die ganze Funktion
        für $x \rightarrow 0$ gegen 0. Da für $x = 0$ der Funktionswert ebenfalls
        $0$ ist, ist die Funktion an dieser Stelle stetig.

        $$\lim\limits_{x \rightarrow 0}\bra{h(x)} =
        \lim\limits_{x \rightarrow 0}\bra{\sin\bra{\frac{1}{x}}} = \sin(\infty)$$

        Die Funktion alterniert immer schneller, je kleiner $|x|$ wird, da $\frac{1}{x}$
        immer schneller wächst, und die Sinusfunktion eine konstant lange Periode
        hat. Also ist die Funktion nicht stetig an der Stelle $x = 0$.

\end{enumerate}

\end{document}
