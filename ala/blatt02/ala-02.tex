\newcommand{\authorinfo}{Paul Bienkowski, Hans Ole Hatzel}
\newcommand{\titleinfo}{ALA 02 (HA) zum 18.04.2013}

% PREAMBLE ===============================================================

\documentclass[a4paper,11pt]{scrartcl}
\usepackage[german,ngerman]{babel}
\usepackage[utf8]{inputenc}
\usepackage[T1]{fontenc}
\usepackage{lmodern}
\usepackage{amssymb}
\usepackage{amsmath}
\usepackage{enumerate}
\usepackage{fancyhdr}

\author{\authorinfo}
\title{\titleinfo}
\date{\today}

\pagestyle{fancy}
\fancyhf{}
\fancyhead[L]{\authorinfo}
\fancyhead[R]{\titleinfo}
\fancyfoot[C]{\thepage}

\newcommand{\bra}[1]{\left(#1\right)}
\newcommand{\limn}[1]{\lim\limits_{n \rightarrow \infty}\bra{#1}}
\newcommand{\rowi}[1]{\sum_{i=#1}^{\infty}}
\newcommand{\row}{\rowi{0}}

\begin{document}
\maketitle
\begin{enumerate}
% Aufgabe 1
    \item[\textbf{1.}]
        \begin{enumerate}
            \item[(i)]
                $\limn{\frac{-3n^4+2n^2+n+1}{-7n^4+25}} =
                \limn{\frac{-3+\frac{2}{n^2}+\frac{1}{n^3}+\frac{1}{n^4}}{-7+\frac{25}{n^4}}} =
                \limn{\frac{-3}{-7}} =
                \frac{3}{7}$
            \item[(ii)]
                $\limn{\frac{-3n^4+2n^2+n+1}{-7n^5+25}} =
                \limn{\frac{-3+\frac{2}{n^2}+\frac{1}{n^3}+\frac{1}{n^4}}{-7n+\frac{25}{n^4}}} =
                \limn{\frac{3}{n \cdot 7}} = 0$
            \item[(iii)]
                $\limn{\frac{-3n^5+2n^2+n+1}{-7n^4+25}} =
                \limn{\frac{-3n+\frac{2}{n^2}+\frac{1}{n^3}+\frac{1}{n^4}}{-7+\frac{25}{n^4}}} =
                \limn{\frac{n \cdot 3}{7}}=\infty$
            \item[(iv)]
                $\limn{\frac{6n^3+2n-3}{9n^2+2}-\frac{2n^3+5n^2+7}{3n^2+3}}$

                $= \limn{\frac{18n^5+6n^3-9n^2+18n^3+6n-9-18n^5-45n^4-63n^2-4n^3-10n^2-14}{(9n^2+2)\cdot(3n^2+3)}}$

                $= \limn{\frac{-45n^4+20n^3-82n^2+6n-23}{27n^4+33n^2+6}} =
                \limn{\frac{-45+\frac{20}{n}-\frac{82}{n^2}+\frac{6}{n^3}-\frac{23}{n^4}}{27+\frac{33}{n^2}+\frac{6}{n^4}}} =
                -\frac{45}{27} = -\frac{5}{3}$

            \item[(v)]
                $\limn{\frac{\sqrt{9n^4+n^2+1}-2n^2+3}{\sqrt{2n^2+1}\cdot \sqrt{2n^2+n+1}}}
                = \limn{\frac{\sqrt{9n^4+n^2+1}-2n^2+3}{\sqrt{4n^4+2n^3+4n^2+n+1}}}$

                $= \limn{\frac{\sqrt{9+\frac{1}{n^2}+\frac{1}{n^4}}-2+\frac{3}{n^2}}{\sqrt{4+\frac{2}{n}+\frac{4}{n^2}+\frac{1}{n}+\frac{1}{n^4}}}}
                =\frac{1}{2}$

        \end{enumerate}

    \item[\textbf{2.}]
        \begin{enumerate}
            \item[a)]
            \begin{tabular}[t]{|r|rl|rl|rl|}
            \hline
              & (i) $a_n$ & $s_n$ & (ii) $a_n$ & $s_n$ & (iii) $a_n$ & $s_n$ \\ \hline
            0 & $1$              & $1$                & $1$              & $1$               & $1$              & $1$ \\[0.5em]
            1 & $\frac{2}{5}$    & $\frac{7}{5}$      & $\frac{5}{2}$    & $\frac{7}{2}$     & $-\frac{2}{5}$   & $\frac{3}{5}$ \\[0.5em]
            2 & $\frac{4}{25}$   & $\frac{39}{25}$    & $\frac{25}{4}$   & $\frac{39}{4}$    & $\frac{4}{25}$   & $\frac{19}{25}$ \\[0.5em]
            3 & $\frac{8}{125}$  & $\frac{203}{125}$  & $\frac{125}{8}$  & $\frac{203}{8}$   & $-\frac{8}{125}$ & $\frac{87}{125}$ \\[0.5em]
            4 & $\frac{16}{625}$ & $\frac{1031}{625}$ & $\frac{625}{16}$ & $\frac{1031}{16}$ & $\frac{16}{625}$ & $\frac{451}{625}$ \\[0.5em]
            \hline
            \end{tabular}

            \begin{enumerate}
                \item[(i)]
                    Geometrische Reihe mit $q = \frac{2}{5} \Rightarrow |q| < 1$:
                    $$\row\bra{\frac{2}{5}}^i = \frac{1}{1 - \frac{2}{5}} = \frac{5}{3}$$

                \item[(ii)]
                    Geometrische Reihe mit $q = \frac{5}{2} \Rightarrow |q| > 1$,
                    gemäß der Regel zur geometrischen Reihe divergiert diese also.

                \item[(iii)]
                    Geometrische Reihe mit $q = -\frac{2}{5} \Rightarrow |q| < 1$:
                    $$\row\bra{-\frac{2}{5}}^i = \frac{1}{1 + \frac{2}{5}} = \frac{5}{7}$$


            \end{enumerate}
            \item[b)]
                \begin{enumerate}
                    \item[(i)]
                        Es handelt sich auch hier um die geometrische Reihe mit $q = x$,
                        und $|q| = |-\frac{3}{10}| < 1$, daher konvergiert sie gegen
                        $$\frac{1}{1 + \frac{3}{10}} = \frac{10}{13}$$

                    \item[(ii)]
                        Die Formel ist nach q aufzulösen:
                        $$\frac{1}{1-q}=\frac{5}{8} \Leftrightarrow q=-\frac{3}{5}$$
                        Man wähle $x=-\frac{3}{5}$, dann konvergiert die Reihe gegen $\frac{5}{8}$
                \end{enumerate}

        \end{enumerate}
        % Aufgabe 3
    \item[\textbf{3.}]
        \begin{enumerate}
            \item[(i)]
                Es handelt sich um eine geometrische Reihe die gegen $\frac{9}{2}$
                konvergiert (da $|q| < 1$).
                $\displaystyle
                \sum_{i=0}^{\infty}\left(\frac{7}{9}\right)^i
                = \frac{1}{1-\frac{7}{9}}= \frac{9}{2}
                $

            \item[(ii)]
                Es handelt sich um eine geometrische Reihe die gegen $-\frac{7}{16}$
                konvergiert (da $|q| < 1$).

                $\displaystyle
                \rowi{1}\bra{-\frac{7}{9}}^i
                = \row\bra{-\frac{7}{9}}^i - 1
                = \frac{1}{1 + \frac{7}{9}} - 1
                = -\frac{7}{16}
                $

            \item[(iii)]
                $\displaystyle
                \rowi{2}(-1)^i \cdot \bra{\frac{7}{9}}^{i+1}
                = \frac{7}{9} \cdot \rowi{2}\bra{-\frac{7}{16}}^i
                % = \frac{7}{9} \cdot \bra{ \row\bra{-\frac{7}{16}}^i - \bra{-\frac{7}{16}}^0 - \bra{-\frac{7}{16}}^1 }
                = \frac{7}{9} \cdot \bra{ \frac{1}{1 + \frac{7}{16}} - 1 + \frac{7}{16}}
                = -\frac{343}{1296}
                $

            \item[(iv)]
                $\displaystyle
                \rowi{2}\frac{1}{(i + 1)i}
                = \rowi{2}\frac{1}{i} \cdot \frac{1}{i - 1}
                = \rowi{2}\frac{1}{i} \cdot \rowi{2}\frac{1}{i - 1}
                = \rowi{2}\frac{1}{i} \cdot \rowi{3}\frac{1}{i}
                = \infty \cdot \infty = \infty
                $

        \end{enumerate}

    % Aufgabe 4
    \item[\textbf{4.}]
        \begin{enumerate}
            \item[(i)]
                $\displaystyle
                \limn{1 + \frac{1}{n}}^{n+3}
                = \limn{1 + \frac{1}{n}}^n \cdot \limn{1 + \frac{1}{n}}^3
                = e \cdot 1 = e
                $

            \item[(ii)]
                $\displaystyle
                \limn{1 + \frac{1}{n}}^{3n}
                = \bra{\limn{1 + \frac{1}{n}}^n}^3
                = e^3
                $

            \item[(iii)]
                $\displaystyle
                \limn{1 + \frac{1}{n}}^3
                = \limn{1 + \frac{1}{n}} \cdot \limn{1 + \frac{1}{n}} \cdot \limn{1 + \frac{1}{n}}
                = 1 \cdot 1 \cdot 1
                = 1
                $

            \item[(iv)]
                $\displaystyle
                \limn{1 + \frac{1}{3n}}^{3n}
                = \limn{1 + \frac{1}{3n}}^3 \cdot \limn{1 + \frac{1}{3n}}^n
                = 1 \cdot e = e
                $
        \end{enumerate}
\end{enumerate}

\end{document}
