\newcommand{\authorinfo}{Paul Bienkowski, Hans Ole Hatzel}
\newcommand{\titleinfo}{ALA 11 (HA) zum 4.07.2013}

% PREAMBLE ===============================================================

\documentclass[a4paper,11pt,fleqn]{scrartcl}
\usepackage[german,ngerman]{babel}
\usepackage[utf8]{inputenc}
\usepackage[T1]{fontenc}
\usepackage{lmodern}
\usepackage{amssymb}
\usepackage{amsmath}
\usepackage{enumerate}
\usepackage{fancyhdr}
\usepackage{pgfplots}
\usepackage{multicol}
\usepackage{tikz}
\usetikzlibrary{calc}
\usetikzlibrary{patterns}

\author{\authorinfo}
\title{\titleinfo}
\date{\today}

\pagestyle{fancy}
\fancyhf{}
\fancyhead[L]{\authorinfo}
\fancyhead[R]{\titleinfo}
\fancyfoot[C]{\thepage}

\newcommand{\bra}[1]{\left(#1\right)}
\newcommand{\limnn}[2]{\lim\limits_{n \rightarrow #1}\bra{#2}}
\newcommand{\limxn}[2]{\lim\limits_{x \rightarrow #1}\bra{#2}}
\newcommand{\limn}[1]{\lim\limits_{n \rightarrow \infty}\bra{#1}}
\newcommand{\limx}[1]{\lim\limits_{x \rightarrow \infty}\bra{#1}}
\newcommand{\limxx}[2]{\lim\limits_{x \rightarrow #1}\bra{#2}}
\newcommand{\limz}[1]{\lim\limits_{z \rightarrow \infty}\bra{#1}}
\newcommand{\rowi}[1]{\sum_{i=#1}^{\infty}}
\newcommand{\row}{\rowi{0}}
\newcommand{\step}[1]{\textbf{#1}}
\newcommand{\dX}[1]{\, \mathrm{d}#1}
\newcommand{\dx}[0]{\dX{x}}
\newcommand{\dt}[0]{\dX{t}}
\newcommand{\partx}[0]{\frac{\partial f}{\partial x}  (x,y)}
\newcommand{\party}[0]{\frac{\partial f}{\partial y}  (x,y)}
\newcommand{\partxp}[1]{\frac{\partial^#1 f}{\partial x^#1}  (x,y)}
\newcommand{\partyp}[1]{\frac{\partial^#1 f}{\partial y^#1}  (x,y)}
\newcommand{\partxy}[1]{\frac{\partial^#1 f}{\partial xy}  (x,y)}
\newcommand{\partyx}[1]{\frac{\partial^#1 f}{\partial yx}  (x,y)}
\begin{document}
\maketitle
\begin{enumerate}
    \item[\textbf{2.}]
        \item[a)]
            Spaltenzahl A ungleich Zeilenzahl C, die Multiplikation AC ist also nicht m"oglich.

            \[AB = 
            \begin{pmatrix}
                -1 \\
                4i-1
            \end{pmatrix}
            \]

            \[BC = 
            \begin{pmatrix}
                1 & -1 \\
                -i + 1 & i-1
            \end{pmatrix}
            \]

            \[CB = 
                i
            \]
        \item[b)]
            \[\overline{z} = \frac{3+2i}{4-3i} = \frac{12-6}{16+9} + i \frac{8+9}{16+9}\]
            \[z = \frac{12-6}{16+9} + i \cdot -\frac{8+9}{16+9}\]
            \[a = \frac{6}{25} \\
            b = - \frac{17}{25}\]
        \item[c)]
        Gaußsche Zahlenebene:\\
        \begin{tikzpicture}
            \begin{scope}[thick,font=\scriptsize]
            \draw [->] (-3,0) -- (3,0) node [above left]  {$\R$};
            \draw [->] (0,-3) -- (0,3) node [below right] {$\I$};

            \draw (1,-3pt) -- (1,3pt)   node [above] {$1$};
            \draw (-1,-3pt) -- (-1,3pt) node [above] {$-1$};
            \draw (-3pt,1) -- (3pt,1)   node [right] {$i$};
            \draw (-3pt,-1) -- (3pt,-1) node [right] {$-i$};

            \draw (2,-3pt) -- (2,3pt)   node [above] {$2$};
            \draw (-2,-3pt) -- (-2,3pt) node [above] {$-2$};
            \draw (-3pt,2) -- (3pt,2)   node [right] {$i$};
            \draw (-3pt,-2) -- (3pt,-2) node [right] {$-2i$};

            \end{scope}
            \draw [->, green] (0,0) -- (-1,-1) node [below] {$z_1$};
            \draw [->, orange] (0,0) -- (-1,1) node [above] {$z_4$};
            \draw [->, red] (0,0) -- (0,-2) node [left] {$z_3$};
            \draw [->, blue] (0,0) -- (1,1) node [left] {$z_2$};
            %\draw [->] (0,0) -- (1,1) node [below right] {$z_3$};
        \end{tikzpicture} 

        \item[d)]
            Bei $M_2$ handelt es sich um die Menge der Punkte die einen Kreis um den Punkt i+1 bilden der den Radius 1 hat.
    \item[\textbf{3.}]
        \begin{enumerate}
            \item[a)]
                Abelitungen der Nebenbedingungen in die Matrix einsetzen:
                \[ 
                \begin{pmatrix}
                    0.2 & 1 \\
                \end{pmatrix}
                \]
                Matrix des Ranges 1. Die Regularitätsbedingung ist erfüllt.

                Es lässt sich also einfach eine Lagrange Funktion aufstellen. \\
                $L(x,y,\lambda) = -0.2x^2 - xy - 2.5y^2 + 48x + 235y - 88 \lambda(0.2x+y-40)$ \\
                Daraus ergibt sich das folgende Gleichungssystem:
                \begin{eqnarray*}
                    -0.4x - y+48 + 0.2 \lambda &=& 0\\
                    -x - 5y + 235 + \lambda &=& 0\\
                    0.2x\lambda + \lambda y -40\lambda &=& 0 
                \end{eqnarray*} 
                Daraus ergeben sich durch l"osen zwei Punkte.
                \[x = 5  \\ 
                y = 39 \\
                \lambda = -35\]
                \[x = 5  \\ 
                y = 46 \\
                \lambda = 0\]     
                Einsetzen ergibt das nur der erste Punkt auch die Nebenbedingung erfüllt $(5,39)$ ist somit uns vermutet maximum, dies gilt es nun in b zu best"atigen.
            \item[b)]
                \[H_f(5, 39) =
                \begin{pmatrix}
                    0.2 & 0.1 \\
                    0.1 & -5
                \end{pmatrix} \]
                \[
                \Delta_1 = -0.4 < 0 \\
                \Delta_2 = 1 > 0
                \]
                Die Matrix ist negativ definit, der gefundene Punkt ist also ein strenges lokales Maximum.
        \end{enumerate}
    \item[\textbf{4.}]
        \begin{enumerate}
            \item[(a)]
                \begin{eqnarray*}
                    0.2x+y &=& 40\\
                    0.2x &=& 40-y\\
                    x &=& 200 - 5y 
                \end{eqnarray*}

                Einsetzen und umformen:
                $$f(y) = 27.5y^2 - 605y + 1512$$
                $$f'(y) = 55y - 605$$


                \begin{eqnarray*}
                    55y - 605 &=& 0\\
                    55y &=& 605\\
                    y &=& 11
                \end{eqnarray*}
                Nullstelle von $f'$: $11$ \\
                Der Extrempunkt befindet sich also bei $y = 11$ und $0.2x = 29$ also $x = 145$.
                Leider stimmt das nicht mit den anderen Ergebnissen überein, obwohl es das natürlich sollte.
        \end{enumerate}
\end{enumerate}

\end{document}
