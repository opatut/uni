\documentclass [a4paper,11pt]{article}
\usepackage[german,ngerman]{babel}
\usepackage[utf8]{inputenc}
\usepackage[T1]{fontenc}
\usepackage{lmodern}
\usepackage{amssymb}
\usepackage{mathtools}
\usepackage{amsmath}
\usepackage{enumerate}
\usepackage{breqn}
\usepackage{fancyhdr}
\pagestyle{fancyplain}
\lhead{Hans Ole Hatzel (6416555), Paul Bienkowski}
\rhead{7.11.12}
\title{\textbf{ALA, Blatt 1}}
\date{\today}
\begin{document}
\maketitle
	\begin {enumerate}
		%Aufgabe 1
		\item [1.]
			$$\frac{2}{x+5} \geq 3$$

			Für $x > -2$:
				\[ \frac{2}{x+5} \geq 3 \leftrightarrow 2 \geq 3 \cdot (x+5) \]
				\[ \leftrightarrow 2 \geq 3x + 15 \]
				\[ \leftrightarrow -13 \geq 3x \]
				\[ \leftrightarrow \frac{-13}{3} \geq x \]
				Intervallschreibweise: $[-\frac{13}{3},\infty)$
			Für $x < -2$:
				\[ \frac{2}{x+5} \geq 3 \leftrightarrow 2 \geq 3 \cdot (x+5) \]
				\[ \leftrightarrow 2 \leq 3x + 15 \]
				\[ \leftrightarrow -13 \leq 3x \]
				\[ \leftrightarrow \frac{-13}{3} \leq x \]
				Intervallschreibweise: $[-\frac{13}{3},-\infty)$

			Insgesamt erfüllen also folgende $x \in \mathbb{R} \backslash \{-5\}$ die Ungleichung:
			$$[-\frac{13}{3},-\infty) \cup [-\frac{13}{3},\infty)$$
		%Aufgabe 3
		\item[3.]
			\begin{enumerate}
				\item[a)]
					\[|a_n - a| = \Big| \frac{2n-1}{n+3}-2 \Big| =\Big| \frac{2n-1}{n+3}-\frac{2 \cdot (n+3)}{n+3}\Big|= \Big| \frac{2n-1}{n+3}-\frac{2n+6}{n+3} \Big| = \Big|-\frac{7}{n+3} \Big| =\frac{7}{n+3}\]
				\item[b)]
					Es sei $\mathcal{E} > 0$ folgilich ergibt sich aus a:
					\[|a_{n}-a| < \mathcal{E} \leftrightarrow \frac{7}{n+3} < \mathcal{E}\]
					Durch Umformen ergibt sich daraus:
					\[n+3> \frac{7}{\mathcal{E}} \leftrightarrow n> \frac{7}{\mathcal{E}}-3\]
					Entsprechend kann man ein $N$ wählen sodass $|a_{n}-a| < \mathcal{E}$ für $n \geq N$ wie in der Definition von Konvergenz gefordert.
				\item[c)]
					Hier muss die oben berechnete Gleichung benutzt werden. 
					Zum Beispiel so: $n> \frac{7}{\frac{1}{10}}-3$ daraus folgt $n > 67$ folglich muss das kleinstmögliche $N = 68$ sein (da $n$ \em{größer als} $67$).

					\begin{tabular}{c|c}
						$\mathcal{E}$ & $N$ \\ \hline
						$\frac{1}{10}$ & $68$ \\ [1.5ex]
						$\frac{1}{100}$ & $698$ \\ [1.5ex]
						$\frac{1}{100000}$ & $699998$ \\ [1.5ex]

					\end{tabular}
			\end{enumerate}
	\end {enumerate}
      

\end{document}