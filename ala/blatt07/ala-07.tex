\newcommand{\authorinfo}{Paul Bienkowski, Hans Ole Hatzel}
\newcommand{\titleinfo}{ALA 07 (HA) zum 06.06.2013}

% PREAMBLE ===============================================================

\documentclass[a4paper,11pt,fleqn]{scrartcl}
\usepackage[german,ngerman]{babel}
\usepackage[utf8]{inputenc}
\usepackage[T1]{fontenc}
\usepackage{lmodern}
\usepackage{amssymb}
\usepackage{amsmath}
\usepackage{enumerate}
\usepackage{fancyhdr}
\usepackage{pgfplots}
\usepackage{multicol}
\usetikzlibrary{calc}
\usetikzlibrary{patterns}

\author{\authorinfo}
\title{\titleinfo}
\date{\today}

\pagestyle{fancy}
\fancyhf{}
\fancyhead[L]{\authorinfo}
\fancyhead[R]{\titleinfo}
\fancyfoot[C]{\thepage}

\newcommand{\bra}[1]{\left(#1\right)}
\newcommand{\limnn}[2]{\lim\limits_{n \rightarrow #1}\bra{#2}}
\newcommand{\limn}[1]{\lim\limits_{n \rightarrow \infty}\bra{#1}}
\newcommand{\limx}[1]{\lim\limits_{x \rightarrow \infty}\bra{#1}}
\newcommand{\rowi}[1]{\sum_{i=#1}^{\infty}}
\newcommand{\row}{\rowi{0}}
\newcommand{\step}[1]{\textbf{#1}}
\newcommand{\dX}[1]{\, \mathrm{d}#1}
\newcommand{\dx}[0]{\dX{x}}
\newcommand{\dt}[0]{\dX{t}}

\begin{document}
\maketitle
\begin{enumerate}
    \item[\textbf{1.}]
        \begin{enumerate}
            \item[(i)]
                \[ \frac{x+1}{x^2-x-6} = \frac{A}{x+2} + \frac{B}{x-3} = \frac{(A+B)x + (2B - 3A)}{(x+2)(x-3)} \]
                \begin{multicols}{3}
                    \[\begin{array}{lrcl}
                        \Rightarrow & A + B &=& 1\\
                        & 2B - 3A &=& 1
                    \end{array}\]

                    \[\begin{array}{lrcl}
                        \Rightarrow & A &=& 1 - B\\
                        & 4 &=& 5B
                    \end{array}\]

                    \[\begin{array}{lrcl}
                        \Rightarrow & A &=& 1 / 5\\
                        & B &=& 4 / 5
                    \end{array}\]
                \end{multicols}

                \[ \int \frac{x+1}{x^2-x-6} \dx = \int \frac{1}{5(x+2)} + \frac{4}{5(x-3)} \dx = \frac{1}{5} \ln |x+2| + \frac{4}{5} \ln |x-3| \]

                Probe:

                \( \frac{1}{5} \ln |x+2| + \frac{4}{5} \ln |x-3| = \frac{1}{5} \cdot \frac{1}{x+2} + \frac{4}{5} \cdot \frac{1}{x-3} = \frac{(x-3)+4(x+2)}{5(x+2)(x-3)} = \frac{x + 1}{(x+2)(x-3)} \;\;\Box \)

            \item[(ii)]
            \item[(iii)]
        \end{enumerate}
    \item[\textbf{2.}]
        \begin{enumerate}
            \item[(a)]
            \item[(b)]
            \item[(c)]
        \end{enumerate}
    \item[\textbf{3.}]
        \begin{enumerate}
            \item[(i)]
            \item[(ii)]
            \item[(iii)]
        \end{enumerate}
    \item[\textbf{4.}]
        \begin{enumerate}
            \item[(a)]
            \item[(b)]
            \item[(c)]
            \item[(d)]
            \item[(e)]
        \end{enumerate}
    \item[\textbf{5.}]
        \begin{enumerate}
            \item[(a)]
            \item[(b)]
            \item[(c)]
                Der Nenner lässt sich in Faktoren zerlegen: \(x^2-x-6 = (x+2)\cdot(x-3)\)
                \[ \frac{3x+2}{x^2-x-6} = \frac{A}{x+2} + \frac{B}{x-3} = \frac{A(x-3)+B(x+2)}{x^2-x-6} =\]
                \[ \frac{(A+B)x -3A + 2B}{x^2-x-6}\]
                Für A und B ergeben sich somit: $A=\frac{11}{5}$, $B=\frac{4}{5}$ \\
                Man erhält eine Partialbruchzerlegung: \( \frac{3x+2}{x^2-x-6} = \frac{11}{5(x+2)} + \frac{4}{5(x-3)} \)
                Nun kann man ganz einfach gemäß der Regeln für integrieren. \\
                \[ \int \frac{3x+2}{x^2-x-6} \dx = \int \frac{11}{5x+10} + \int \frac{4}{5x-15} \dx = \frac{11}{5} \cdot ln|x+10| + \frac{4}{5} \cdot ln|x-3| \]

            \item[(d)]
                Der Nenner lässt sich in Faktoren zerlegen: \(x^2+8x+16 = (x+4)^2 \) \\
                $\frac{x+1}{x^2+8x+16} = \frac{A}{x+4} + \frac{B}{x+4} = \frac{A(x+4)+B(x+4)}{x^2+8x+16}$ \\
                Für A und B ergeben sich somit: $A + B= 1$ und $4A+4B = 1$ also zB: $A=B=\frac{1}{2}$ \\
                Man erhält eine Partialbruchzerlegung: $\frac{1}{2x+8} + \frac{1}{2x+8}$
                Nun kann man ganz einfach gemäß der Regeln für Integrieren. \\
                \[ \int \frac{x+1}{x^2+8x+16} \dx = \int \frac{1}{2x+8} + \int \frac{1}{2x+8} \dx = \frac{1}{2} \cdot ln|x+8| + \frac{1}{2} \cdot ln|x+8| = ln|x+8|  \]
        \end{enumerate}
\end{enumerate}

\end{document}
