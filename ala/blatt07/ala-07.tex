\newcommand{\authorinfo}{Paul Bienkowski, Hans Ole Hatzel}
\newcommand{\titleinfo}{ALA 07 (HA) zum 06.06.2013}

% PREAMBLE ===============================================================

\documentclass[a4paper,11pt,fleqn]{scrartcl}
\usepackage[german,ngerman]{babel}
\usepackage[utf8]{inputenc}
\usepackage[T1]{fontenc}
\usepackage{lmodern}
\usepackage{amssymb}
\usepackage{amsmath}
\usepackage{enumerate}
\usepackage{fancyhdr}
\usepackage{pgfplots}
\usepackage{multicol}
\usetikzlibrary{calc}
\usetikzlibrary{patterns}

\author{\authorinfo}
\title{\titleinfo}
\date{\today}

\pagestyle{fancy}
\fancyhf{}
\fancyhead[L]{\authorinfo}
\fancyhead[R]{\titleinfo}
\fancyfoot[C]{\thepage}

\newcommand{\bra}[1]{\left(#1\right)}
\newcommand{\limnn}[2]{\lim\limits_{n \rightarrow #1}\bra{#2}}
\newcommand{\limn}[1]{\lim\limits_{n \rightarrow \infty}\bra{#1}}
\newcommand{\limx}[1]{\lim\limits_{x \rightarrow \infty}\bra{#1}}
\newcommand{\limz}[1]{\lim\limits_{z \rightarrow \infty}\bra{#1}}
\newcommand{\rowi}[1]{\sum_{i=#1}^{\infty}}
\newcommand{\row}{\rowi{0}}
\newcommand{\step}[1]{\textbf{#1}}
\newcommand{\dX}[1]{\, \mathrm{d}#1}
\newcommand{\dx}[0]{\dX{x}}
\newcommand{\dt}[0]{\dX{t}}

\begin{document}
\maketitle
\begin{enumerate}
    \item[\textbf{1.}]
        \begin{enumerate}
            \item[(i)]
                \[ \frac{x+1}{x^2-x-6} = \frac{A}{x+2} + \frac{B}{x-3} = \frac{(A+B)x + (2B - 3A)}{(x+2)(x-3)} \]
                \begin{multicols}{3}
                    \[\begin{array}{lrcl}
                        \Rightarrow & A + B &=& 1\\
                        & 2B - 3A &=& 1
                    \end{array}\]

                    \[\begin{array}{lrcl}
                        \Rightarrow & A &=& 1 - B\\
                        & 4 &=& 5B
                    \end{array}\]

                    \[\begin{array}{lrcl}
                        \Rightarrow & A &=& 1 / 5\\
                        & B &=& 4 / 5
                    \end{array}\]
                \end{multicols}

                \[ \int \frac{x+1}{x^2-x-6} \dx = \int \frac{1}{5(x+2)} + \frac{4}{5(x-3)} \dx = \frac{1}{5} \ln |x+2| + \frac{4}{5} \ln |x-3| \]

                Probe:

                \( \frac{1}{5} \ln |x+2| + \frac{4}{5} \ln |x-3| = \frac{1}{5} \cdot \frac{1}{x+2} + \frac{4}{5} \cdot \frac{1}{x-3} = \frac{(x-3)+4(x+2)}{5(x+2)(x-3)} = \frac{x + 1}{(x+2)(x-3)} \;\;\Box \)

            \item[(ii)]
                TODO

            \item[(iii)]
                TODO
        \end{enumerate}
    \item[\textbf{2.}]
        \begin{enumerate}
            \item[(a)]
                Skizze:

                \begin{tikzpicture}
                    \begin{axis}[
                        ymin=0,ymax=1.2,
                        xmin=0,xmax=12.7,
                        x=1cm, y=3cm,
                        axis x line=middle,
                        axis y line=middle,
                        axis line style=->,
                        xlabel={$x$},
                        ylabel={$y$},
                        ]
                        \addplot[very thick, no marks, red,   -] expression[domain=0:10,samples=100]{e^-x}; \addlegendentry{f(x)}
                        \addplot[very thick, no marks, black, -] expression[domain=0:10,samples=100]{1/(1+x)}; \addlegendentry{g(x)}
                        \addplot[very thick, no marks, blue,  -] expression[domain=0:10,samples=100]{1/(1+x^2)}; \addlegendentry{h(x)}
                        \addplot[very thick, no marks, red,   -, densely dotted] expression[domain=10:12,samples=100]{e^-x};
                        \addplot[very thick, no marks, black, -, densely dotted] expression[domain=10:12,samples=100]{1/(1+x)};
                        \addplot[very thick, no marks, blue,  -, densely dotted] expression[domain=10:12,samples=100]{1/(1+x^2)};
                    \end{axis}

                    \node[fill=blue, inner sep=2pt, circle, label=right:{\small Wendepunkt $h(x)$}] at (0.5773,2.25) {};
                \end{tikzpicture}

                Nur \(h(x)\) hat einen Wendepunkt:

                \( h(x) = \frac{1}{1+x^2} \)

                \( h'(x) = -\frac{2x}{(1+x^2)^2} \)

                \( h''(x) = \frac{-2(1+x^2)^2 + 8x^2(1+x^2)}{(1+x^2)^4} = \frac{6x^2-2}{(1+x^2)^4} \)

                Wendestelle bei \( 6x^2 - 2 = 0 \Leftrightarrow x = \pm \sqrt{\frac{1}{3}} \).

            \item[(b)]

                \[ \limz{\int_0^z e^{-x} \dx} = \limz{\left[ -e^{-x} \right]_0^z}  =
                    \limz{e^0 - e^{-z}} = 1 \]

                \[ \limz{\int_0^z \frac{1}{1+x} \dx} = \limz{\left[ \ln (x+1) \right]_0^z}  =
                    \limz{\ln(z+1)} - \ln(1) = \infty \]

                \[ \limz{\int_0^z \frac{1}{1+x^2} \dx} = \limz{\left[ \arctan x \right]_0^z}  =
                    \limz{\arctan(z)} - \arctan(0) = \frac{\pi}{2} \]

            \item[(c)]
                Skizze:

                \begin{tikzpicture}
                    \begin{axis}[
                        ymin=0,ymax=4,
                        xmin=-1.3,xmax=1.3,
                        x=2.5cm, y=1cm,
                        axis x line=middle,
                        axis y line=middle,
                        axis line style=->,
                        xlabel={$x$},
                        ylabel={$y$},
                        ]
                        \addplot[very thick, no marks, black, -] expression[domain=-1:1,samples=100]{1/sqrt(1-x^2)};
                    \end{axis}
                \end{tikzpicture}

                Da die Funktion an der y-Achse spiegelsymmetrisch ist, gilt f"ur die Fl"ache:

                \[ A = 2 \cdot \int_0^1 \frac{1}{\sqrt{1-x^2}} \dx = 2 \left[ \arcsin(x) \right]_0^1 = 2 \bra{\arcsin(1)-\arcsin(0)} = 2(\frac{\pi}{2} - 0) = \pi \]

        \end{enumerate}
    \item[\textbf{3.}]
        \begin{enumerate}
            \item[(i)]
                \( \frac{1}{8} \bra{ \sin(0) + 2\sin(0.25) + 2\sin(0.5)  + 2\sin(0.75)  + \sin(1) } \approx 0.45720099376 \)

            \item[(ii)]
                \( \frac{1}{10} \bra{ \sin(0) + 2\sin(0.2) + \cdots + 2\sin(0.8) + \sin(1) } \approx 0.458164346 \)

            \item[(iii)]
                \( \frac{1}{20} \bra{ \sin(0) + 2\sin(0.1) + \cdots + 2\sin(0.9) + \sin(1) } \approx 0.4593145489 \)
        \end{enumerate}
    \item[\textbf{4.}]
        \begin{enumerate}
            \item[(a)]
                \( f( 1) =  10 \cdot e^{-\frac{ 2}{5}} \approx 6.7032 \)\\
                \( f( 2) =  20 \cdot e^{-\frac{ 4}{5}} \approx 8.9866 \)\\
                \( f( 6) =  60 \cdot e^{-\frac{12}{5}} \approx 5.4431 \)\\
                \( f(12) = 120 \cdot e^{-\frac{24}{5}} \approx 0.9876 \)\\

            \item[(b)]
                \( f'(x) = 10x \cdot e^{-\frac{2}{5}x} \cdot \bra{-\frac{2}{5}} + 10 e^{-\frac{2}{5}x} = 10e^{-\frac{2}{5}x} \bra{1-\frac{2}{5}} = 0 \Leftrightarrow x = \frac{5}{2} \)

                \( f(\frac{5}{2}) =  25 e \approx 9.1970 \)

                Die maximale Konzentration von etwa $9.1970 \frac{mg}{l}$ wird nach zweieinhalb Stunden erreicht.

            \item[(c)]
                \begin{eqnarray*}
                    \int e^{-\frac{2}{5}x} \cdot 10x \dx &=& -\frac{5}{2} e^{-\frac{2}{5}x} \cdot 10x - \int -\frac{5}{2} e^{-\frac{2}{5}x} \cdot 10 \dx \\
                    &=& -25x \cdot e^{-\frac{2}{5}x} + 25 \int e^{-\frac{2}{5}x} \dx \\
                    &=& -25x \cdot e^{-\frac{2}{5}x} + 25 \cdot \bra{-\frac{5}{2}} \cdot e^{-\frac{2}{5}x} \\
                    &=& -25x \cdot e^{-\frac{2}{5}x} \bra{x+\frac{5}{2}}
                \end{eqnarray*}

                \[ \frac{1}{6} \left[ -25x \cdot e^{-\frac{2}{5}x} \bra{x+\frac{5}{2}} \right]_0^6 \approx 7.2037 \]

            \item[(d)]
                \[ \frac{1}{6} \left[ -25x \cdot e^{-\frac{2}{5}x} \bra{x+\frac{5}{2}} \right]_6^{12} \approx 2.7157 \]

            \item[(e)]
                Skizze:

                \begin{tikzpicture}[scale=0.9]
                    \begin{axis}[
                        ymin=0,ymax=10,
                        xmin=0,xmax=14.5,
                        x=1cm, y=0.5cm,
                        axis x line=middle,
                        axis y line=middle,
                        axis line style=->,
                        xlabel={$x$},
                        ylabel={$y$},
                        ]
                        \addplot[very thick, no marks, black, -] expression[domain=0:12,samples=100]{10*x*e^(-0.4*x)};
                        \addplot[very thick, no marks, black, -, dotted] expression[domain=12:14,samples=20]{10*x*e^(-0.4*x)};
                    \end{axis}

                    \node[fill, inner sep=2pt, circle, label=above right:{\small Stärkster Abbau/Wendepunkt}] at (5,3.3834) {};
                \end{tikzpicture}

                \( f'(x) = 10 e^{-\frac{2}{5}x} \bra{1 - \frac{2}{5}x} \)

                \( f''(x) = e^{-\frac{2}{5}x} \bra{-8 + \frac{8}{5}x} = 0 \Rightarrow -8 + \frac{8}{5}x = 0 \Rightarrow x = 5 \)


        \end{enumerate}
    \item[\textbf{5.}]
        \begin{enumerate}
            \item[(a)]
                \[ h(x) = \bra{x^2+1}^{\cos x} = e^{\ln \bra{x^2+1}^{\cos x}} = e^{\cos x \cdot \ln \bra{x^2+1}}) \]

                \[ h'(x) = \bra{x^2+1}^{\cos x} \cdot \bra{ -\sin x \cdot \ln \bra{x^2+1} + \cos x \cdot \frac{2x}{x^2+1} } \]

            \item[(b)]
                Es sei

                \[ t = \sqrt{\frac{x}{4} + 3} \Leftrightarrow x = 4t^2-12 \Rightarrow \frac{dt}{dx} = 8t  \]

                Dann gilt

                \begin{eqnarray*}
                    \int \sin\bra{\sqrt{\frac{x}{4} + 3}} \dx &=& \int \sin t \cdot 8t \dt\\
                    &=& - \cos t \cdot 8t - \int -8 cos t \dt \\
                    &=& -8t \cdot \cos t + 8 \sin t\\
                    &=& -8 \sqrt{\frac{x}{4} + 3} \cdot \cos \bra{\sqrt{\frac{x}{4} + 3}} + 8 \sin \bra{\sqrt{\frac{x}{4} + 3}}\\
                \end{eqnarray*}



            \item[(c)]
                TODO

            \item[(d)]
                TODO
        \end{enumerate}
\end{enumerate}

\end{document}
