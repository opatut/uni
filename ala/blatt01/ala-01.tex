\newcommand{\authorinfo}{Paul Bienkowski, Hans Ole Hatzel}
\newcommand{\titleinfo}{ALA 01 (HA) zum 11.04.2013}

% PREAMBLE ===============================================================

\documentclass [a4paper,11pt]{article}
\usepackage[german,ngerman]{babel}
\usepackage[utf8]{inputenc}
\usepackage[T1]{fontenc}
\usepackage{lmodern}
\usepackage{amssymb}
\usepackage{mathtools}
\usepackage{amsmath}
\usepackage{enumerate}
\usepackage{breqn}
\usepackage{fancyhdr}
\usepackage{multicol}

\author{\authorinfo}
\title{\titleinfo}
\date{\today}

\pagestyle{fancy}
\fancyhf{}
\fancyhead[L]{\authorinfo}
\fancyhead[R]{\titleinfo}
\fancyfoot[C]{\thepage}

\begin{document}
\maketitle
	\begin {enumerate}
		% Aufgabe 1
		\item[\textbf{1.}]
			$$\begin{array}{lrcl}
							& \frac{2}{x+5} &\geq& 3 \\
			\Leftrightarrow & 2 			&\geq& 3 \cdot (x + 5) \\
			\Leftrightarrow & 2 			&\geq& 3x + 15 \\
			\Leftrightarrow & -13 			&\geq& 3x \\
			\Leftrightarrow & -\frac{13}{3} &\geq& x
			\end{array}$$

			$$L = \Big(-\infty,-\frac{13}{3}\Big]$$

		% Aufgabe 2
		\item[\textbf{2.}]
			\begin{multicols}{2}
				\textbf{Fall 1}: $x \geq \frac{4}{3}$

				$$\begin{array}{lrcl}
								& 3x - 4 &\geq& 2 \\
				\Leftrightarrow & 3x &\geq& 6 \\
				\Leftrightarrow & x &\geq& 2
				\end{array}$$

				\textbf{Fall 2}: $x < \frac{4}{3}$

				$$\begin{array}{lrcl}
								& - 3x + 4 &\geq& 2 \\
				\Leftrightarrow & - 3x &\geq& -2 \\
				\Leftrightarrow & - 3x &\geq& -2 \\
				\Leftrightarrow & x &\leq& \frac{2}{3}
				\end{array}$$
			\end{multicols}

			$$L=\Big(-\infty, \frac{2}{3}\Big] \cup \Big[2, \infty\Big)$$

		% Aufgabe 3
		\item[\textbf{3.}]
			\begin{enumerate}
				\item[a)]
					\[|a_n - a| = \Big| \frac{2n-1}{n+3}-2 \Big| =\Big| \frac{2n-1}{n+3}-\frac{2 \cdot (n+3)}{n+3}\Big|= \Big| \frac{2n-1}{n+3}-\frac{2n+6}{n+3} \Big|\]
					\[ = \Big|-\frac{7}{n+3} \Big| =\frac{7}{n+3}\]

				\item[b)]
					Es sei $\epsilon > 0$ folglich ergibt sich aus a:
					\[|a_{n}-a| < \epsilon \Leftrightarrow \frac{7}{n+3} < \epsilon \Leftrightarrow n> \frac{7}{\epsilon}-3\]
					Entsprechend kann man ein $N$ wählen sodass $|a_{n}-a| < \epsilon$ für $n \geq N$ wie in der Definition von Konvergenz gefordert: $N > \frac{7}{\epsilon} - 3$.

				\item[c)]
					Hier muss die oben berechnete Unleichung benutzt werden.
					Beispiel: $N > \frac{7}{\epsilon} - 3 \Rightarrow n > 67$, folglich muss das kleinstmögliche $N = 68$ sein (da $N$ \em{größer als} $67$ sein soll).

					\begin{tabular}{c|c}
						$\epsilon$ & $N$ \\ \hline
						$\frac{1}{10}$ & $68$ \\ [1.5ex]
						$\frac{1}{100}$ & $698$ \\ [1.5ex]
						$\frac{1}{100000}$ & $699998$ \\ [1.5ex]

					\end{tabular}
			\end{enumerate}

		% Aufgabe 4
		\item[\textbf{4.}]

			\textbf{Beschränktheit:}

			Induktionsannahme (IA): $0 \leq a_n < \frac{1}{2}$

			Induktionsanfang: $0 \leq \frac{2}{5} < \frac{1}{2}$ gilt.

			Induktionsschritt: Es ist zu zeigen:

			$$0 \leq a_{n+1} \leq \frac{1}{2}$$

			$$0 \leq a_n^2 + \frac{1}{4} \leq \frac{1}{2}$$

			Der erste Teil der Ungleichung gilt immer, da $a_n^2$ nie negativ (und damit
			auch nicht kleiner als $\frac{1}{4}$) sein kann.

			Der zweite Teil der Ungleichung lautet:

			$$a_n^2 + \frac{1}{4} < \frac{1}{2}
			\Leftrightarrow a_n^2 < \frac{1}{4}
			\Leftrightarrow a_n < \frac{1}{2}$$

			Dies gilt laut der Induktionsannahme, damit ist die Beschränktheit von
			$a$ gezeigt.

			\textbf{Monotonie:}

			Es ist zu zeigen, dass gilt:

			$$a_n^2 + \frac{1}{4} \geq a_n$$

			Angenommen, die Folge sei nicht monoton, so müsse demnach für mindestens
			ein $a_n$ gelten:

			$$\begin{array}{lrcl}
							& a_n^2 + \frac{1}{4} &<& a_n \\[2pt]
			\Leftrightarrow & a_n^2 - a_n + \frac{1}{4} &<& 0 \\[2pt]
			\Leftrightarrow & (a_n - \frac{1}{2})^2 &<& 0
			\end{array}$$

			Dies ist nicht möglich, da ein Quadrat niemals negativ ist. Also existiert
			kein $a_n$, welches kleiner ist, als sein Nachfolger $a_{n+1}$, somit
			ist die Folge Monoton. $\Box$

	\end {enumerate}


\end{document}
