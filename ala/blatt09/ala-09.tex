\newcommand{\authorinfo}{Paul Bienkowski, Hans Ole Hatzel}
\newcommand{\titleinfo}{ALA 09 (HA) zum 20.06.2013}

% PREAMBLE ===============================================================

\documentclass[a4paper,11pt,fleqn]{scrartcl}
\usepackage[german,ngerman]{babel}
\usepackage[utf8]{inputenc}
\usepackage[T1]{fontenc}
\usepackage{lmodern}
\usepackage{amssymb}
\usepackage{amsmath}
\usepackage{enumerate}
\usepackage{fancyhdr}
\usepackage{pgfplots}
\usepackage{multicol}
\usetikzlibrary{calc}
\usetikzlibrary{patterns}

\author{\authorinfo}
\title{\titleinfo}
\date{\today}

\pagestyle{fancy}
\fancyhf{}
\fancyhead[L]{\authorinfo}
\fancyhead[R]{\titleinfo}
\fancyfoot[C]{\thepage}

\newcommand{\bra}[1]{\left(#1\right)}
\newcommand{\limnn}[2]{\lim\limits_{n \rightarrow #1}\bra{#2}}
\newcommand{\limxn}[2]{\lim\limits_{x \rightarrow #1}\bra{#2}}
\newcommand{\limn}[1]{\lim\limits_{n \rightarrow \infty}\bra{#1}}
\newcommand{\limx}[1]{\lim\limits_{x \rightarrow \infty}\bra{#1}}
\newcommand{\limxx}[2]{\lim\limits_{x \rightarrow #1}\bra{#2}}
\newcommand{\limz}[1]{\lim\limits_{z \rightarrow \infty}\bra{#1}}
\newcommand{\rowi}[1]{\sum_{i=#1}^{\infty}}
\newcommand{\row}{\rowi{0}}
\newcommand{\step}[1]{\textbf{#1}}
\newcommand{\dX}[1]{\, \mathrm{d}#1}
\newcommand{\dx}[0]{\dX{x}}
\newcommand{\dt}[0]{\dX{t}}

\begin{document}
\maketitle
\begin{enumerate}
    \item[\textbf{1.}]
        \begin{enumerate}
            \item[(a)]

                \[\begin{array}{lllllllll}
                    T_8(x) &= T_9(x) &= 1 &- \frac{x^2}{2!} &+ \frac{x^4}{4!} &- \frac{x^6}{6!} &+ \frac{x^8}{8!}\\[1em]
                    T_{10}(x) &= T_{11}(x) &= 1 &- \frac{x^2}{2!} &+ \frac{x^4}{4!} &- \frac{x^6}{6!} &+ \frac{x^8}{8!} &- \frac{x^{10}}{10!}\\[1em]
                    T_{12}(x) &= T_{13}(x) &= 1 &- \frac{x^2}{2!} &+ \frac{x^4}{4!} &- \frac{x^6}{6!} &+ \frac{x^8}{8!} &- \frac{x^{10}}{10!} &+ \frac{x^{12}}{12!}
                \end{array}\]

                \[\begin{array}{llll}
                    T_9(1) &= \frac{1}{1} - \frac{1}{2!} + \frac{1}{4!} - \frac{1}{6!} + \frac{1}{8!} &= \frac{4357}{8064} &\approx 0.540302579365\\[1em]
                    T_{11}(1) &= \frac{1}{1} - \frac{1}{2!} + \frac{1}{4!} - \frac{1}{6!} + \frac{1}{8!} - \frac{1}{10!} &= \frac{1960649}{3628800} &\approx 0.540302303791\\[1em]
                    T_{13}(1) &= \hspace{1em}\cdots &= \hspace{1em}\cdots &\approx 0.540302305879
                \end{array}\]

            \item[(b)]
                Taylorpolynome für $f(x)$ und $g(x)$ an $x_0 = 0$:
                \begin{multicols}{2}
                    \(\begin{array}{ll}
                        f(x)   &= \sqrt{1+x} \\[0.6em]
                        T_0(x) &= 1 \\[0.3em]
                        T_1(x) &= 1 + \frac{x}{2} \\[0.3em]
                        T_2(x) &= 1 + \frac{x}{2} - \frac{x^2}{4 \cdot 2!} \\[0.3em]
                        T_3(x) &= 1 + \frac{x}{2} - \frac{x^2}{4 \cdot 2!} + \frac{x^3}{2 \cdot 3!} \\[0.3em]
                        T_4(x) &= 1 + \frac{x}{2} - \frac{x^2}{4 \cdot 2!} + \frac{x^3}{2 \cdot 3!} - \frac{5x^4}{4 \cdot 4!}
                    \end{array}\)

                    \(\begin{array}{ll}
                        g(x)   &= \frac{1}{\sqrt[3]{1+x}} \\[0.6em]
                        T_0(x) &= 1 \\[0.3em]
                        T_1(x) &= 1 - \frac{x}{3} \\[0.3em]
                        T_2(x) &= 1 - \frac{x}{3} + \frac{4x^2}{9 \cdot 2!} \\[0.3em]
                        T_3(x) &= 1 - \frac{x}{3} + \frac{4x^2}{9 \cdot 2!} - \frac{28x^3}{27 \cdot 3!} \\[0.3em]
                        T_4(x) &= 1 - \frac{x}{3} + \frac{4x^2}{9 \cdot 2!} - \frac{28x^3}{27 \cdot 3!} + \frac{280x^4}{81 \cdot 4!}
                    \end{array}\)
                \end{multicols}

            \item[(c)]
                Taylorpolynom für $f(x) = e^x \cdot \sin x$ an $x_0 = 0$:
                \[ T_5(x) = x + x^2 + \frac{x^3}{3} - \frac{x^4}{6} - \frac{x^5}{30} \]

        \end{enumerate}
    \item[\textbf{2.}]
        \begin{enumerate}
            \item[(i)]
                \[ \limxx{1}{\frac{x^3 - 3x^2 + x + 2}{x^2 - 5x + 6}} = \frac{1}{2}\]
            \item[(ii)]
                Es handelt sich um den Typ $\frac{0}{0}$ entsprechend lässt sich der Nenner vom Zähler getrennt ableiten.
                Man erhält so die Funktion: \[ \limxx{2}{\frac{3x^2 - 6x + 1}{2x - 5}} = -1\]
            \item[(iii)]
                \[ \limxx{0}{(1+3x)^\frac{1}{2x}} \limxx{0}{e^{ln(1+3x)^\frac{1}{2x}}} = e^{\limxx{0}{\frac{ln(1+3x)}{2x}}} \]
                Dabei handelt es sich um den Typ $\frac{0}{0}$, Ableiten von Zähler und Nenner:
                \[ \limxx{0}{\frac{ \frac{1}{1+3x} \cdot 3 }{2}} = e^{\frac{3}{2}} \approx 4,482 \]
            \item[(iv)]
                \[ \limxx{0}{ \frac{1}{e^x - 1} - \frac{1}{sin(x)}} = \limxx{0}{\frac{sin(x) - e^x - 1}{(sin(x)) \cdot (e^x-1)}} \]
                Da der Nenner gegen Null strebt während der Zähler gegen 1 strebt, strebt der gesamte Wert trivialerweise gegen $- \infty$
        \end{enumerate}
    \item[\textbf{3.}]
        \begin{enumerate}
            \item[(a)]
                \( f(x) = 3^x \)\\
                \( f'(x) = 3^x \ln 3 \)\\
                \( f'(2) = 9 \ln 3 \approx 9.887510598 \)\\
                \( x_0 = 2 - \frac{9}{9 \ln 3} \approx 1.08761 \)

                Der Schnittpunkt befindet sich an $S(0, x_0)$.

            \item[(b)]
                \[\begin{array}{ll}
                    T_0(x) &= \sqrt[7]{2} \\[0.5em]
                    T_1(x) &= \sqrt[7]{2} + \frac{x}{7 \sqrt[7]{2^6}} \\[0.5em]
                    T_2(x) &= \sqrt[7]{2} + \frac{x}{7 \sqrt[7]{2^6}} - \frac{3x^2}{49 \sqrt[7]{2^{13}} }
                \end{array}\]

            \item[(c)]
                Für Stetigkeit ist zu zeigen, dass $\limxn{0}{\cos\bra{\frac{1}{x}}} = 0$
                ist. Hierfür eignet sich die Folge $(a_n) = (\frac{1}{n})$, also:
                \[ \limn{\cos\bra{\frac{1}{(a_n)}}} =
                \limn{\cos\bra{\frac{1}{\frac{1}{n}}}} =
                \limn{\cos\bra{n}} \]
                Es existiert kein Grenzwert, also konvergiert $\cos\bra{\frac{1}{x}}$
                für $x \rightarrow 0$ \textbf{nicht} gegen $0$, somit ist die
                Funktion nicht stetig.

            \item[(d)]
                Bei $\mathcal{B}$ handelt es sich um alle beschränkten Folgen, da
                es in jedem Fall ein (beliebiges) Limit für den Abstand geben
                muss. Allerdings sind nicht alle dieser Folgen konvergent, sie
                können z.B. auch innerhalb der Schranken oszillieren (z.B.
                $f(x) = \sin x$, $\varepsilon = 2$, $a = 0$).

        \end{enumerate}
    \item[\textbf{4.}]
        \begin{enumerate}
            \item[(a)]
                \[ \limx{ \frac{a^x}{x^n} }\] Dabei handelt es sich um den Typ $\frac{\infty}{\infty}$ (Sowohl die Folge im Nenner als auch die im Zähler divergieren.)
                Also können Nenner und Zähler getrennt abgeleitet werden. 
                \[\limx{\frac{ln(a) \cdot a^x}{n \cdot x^{n-1}}}\] Dabei handelt es sich offensichtlich wieder um den Typ $\frac{\infty}{\infty}$ das gilt solange im Nenner der Exponent von $x$ großer 0 bleibt somit darf man die Ableitungen bis zur n-te Ableitung Bilden:

                \[\limx{\frac{ln(a)^n \cdot a^x}{n! \cdot x^{n-n}}} = \limx{\frac{ln(a)^n \cdot a^x}{n!}} = \frac{\limx{ln(a)^n \cdot a^x}}{n!} = \infty\]
                Hier wird offensichtilich, dass im Nenner eine Konstante steht während der Zähler weiterhin divergiert. Der Grenzwert beträgt $\infty$. Also wächst $f(x)$ schneller.
            \item[(b)]
                Hierbei handelt es sich wieder um den Typen $\frac{\infty}{\infty}$:

                \( \limx{ \frac{x^r}{\ln^k x} }\)

                Also können Nenner und Zähler getrennt abgeleitet werden.

                \( \limx{\frac{x^r}{\ln^k x}} = \limx{\frac{r \cdot x^{r-1}}{k\bra{\ln x}^{k-1} \cdot \frac{1}{x}}} =
                \frac{r}{k} \limx{ \frac{x^r}{\bra{\ln x}^{k-1}} } =
                \frac{r}{k} \limx{ \bra{\ln x} \cdot \frac{x^r}{\ln^k x} } \)

                Da der erste Teil im Limes ($\ln x$) gegen $\infty$ konvergiert,
                und der Rest identisch mit dem Ausgansbruch ist (und somit beliebig
                oft auf die gleiche Weise umgeformt werden könnte), konvertiert
                der gesamte Bruch gegen $\infty$. Also wächst $g(x)$ schneller als
                $h(x)$.

            \item[(c)]
                \begin{enumerate}
                \item[(i)]
                Die Umformung ließe sich genau wie in a ausführen, somit steht dann im Nenner lediglich eine andere Konstante. Der Beweis ist in ähnlicher Form gültig.
                \end{enumerate}
        \end{enumerate}
\end{enumerate}

\end{document}
