\newcommand{\authorinfo}{Paul Bienkowski, Hans Ole Hatzel}
\newcommand{\titleinfo}{ALA 09 (HA) zum 20.06.2013}

% PREAMBLE ===============================================================

\documentclass[a4paper,11pt,fleqn]{scrartcl}
\usepackage[german,ngerman]{babel}
\usepackage[utf8]{inputenc}
\usepackage[T1]{fontenc}
\usepackage{lmodern}
\usepackage{amssymb}
\usepackage{amsmath}
\usepackage{enumerate}
\usepackage{fancyhdr}
\usepackage{pgfplots}
\usepackage{multicol}
\usetikzlibrary{calc}
\usetikzlibrary{patterns}

\author{\authorinfo}
\title{\titleinfo}
\date{\today}

\pagestyle{fancy}
\fancyhf{}
\fancyhead[L]{\authorinfo}
\fancyhead[R]{\titleinfo}
\fancyfoot[C]{\thepage}

\newcommand{\bra}[1]{\left(#1\right)}
\newcommand{\limnn}[2]{\lim\limits_{n \rightarrow #1}\bra{#2}}
\newcommand{\limn}[1]{\lim\limits_{n \rightarrow \infty}\bra{#1}}
\newcommand{\limx}[1]{\lim\limits_{x \rightarrow \infty}\bra{#1}}
\newcommand{\limz}[1]{\lim\limits_{z \rightarrow \infty}\bra{#1}}
\newcommand{\rowi}[1]{\sum_{i=#1}^{\infty}}
\newcommand{\row}{\rowi{0}}
\newcommand{\step}[1]{\textbf{#1}}
\newcommand{\dX}[1]{\, \mathrm{d}#1}
\newcommand{\dx}[0]{\dX{x}}
\newcommand{\dt}[0]{\dX{t}}

\begin{document}
\maketitle
\begin{enumerate}
    \item[\textbf{1.}]
        \begin{enumerate}
            \item[(a)]

                \[\begin{array}{lllllllll}
                    T_8(x) &= T_9(x) &= 1 &- \frac{x^2}{2!} &+ \frac{x^4}{4!} &- \frac{x^6}{6!} &+ \frac{x^8}{8!}\\[1em]
                    T_{10}(x) &= T_{11}(x) &= 1 &- \frac{x^2}{2!} &+ \frac{x^4}{4!} &- \frac{x^6}{6!} &+ \frac{x^8}{8!} &- \frac{x^{10}}{10!}\\[1em]
                    T_{12}(x) &= T_{13}(x) &= 1 &- \frac{x^2}{2!} &+ \frac{x^4}{4!} &- \frac{x^6}{6!} &+ \frac{x^8}{8!} &- \frac{x^{10}}{10!} &+ \frac{x^{12}}{12!}
                \end{array}\]

                \[\begin{array}{llll}
                    T_9(1) &= \frac{1}{1} - \frac{1}{2!} + \frac{1}{4!} - \frac{1}{6!} + \frac{1}{8!} &= \frac{4357}{8064} &\approx 0.540302579365\\[1em]
                    T_{11}(1) &= \frac{1}{1} - \frac{1}{2!} + \frac{1}{4!} - \frac{1}{6!} + \frac{1}{8!} - \frac{1}{10!} &= \frac{1960649}{3628800} &\approx 0.540302303791\\[1em]
                    T_{13}(1) &= \hspace{1em}\cdots &= \hspace{1em}\cdots &\approx 0.540302305879
                \end{array}\]

            \item[(b)]
                Taylorpolynome für $f(x)$ und $g(x)$ an $x_0 = 0$:
                \begin{multicols}{2}
                    \(\begin{array}{ll}
                        f(x)   &= \sqrt{1+x} \\[0.6em]
                        T_0(x) &= 1 \\[0.3em]
                        T_1(x) &= 1 + \frac{x}{2} \\[0.3em]
                        T_2(x) &= 1 + \frac{x}{2} - \frac{x^2}{4 \cdot 2!} \\[0.3em]
                        T_3(x) &= 1 + \frac{x}{2} - \frac{x^2}{4 \cdot 2!} + \frac{x^3}{2 \cdot 3!} \\[0.3em]
                        T_4(x) &= 1 + \frac{x}{2} - \frac{x^2}{4 \cdot 2!} + \frac{x^3}{2 \cdot 3!} - \frac{5x^4}{4 \cdot 4!}
                    \end{array}\)

                    \(\begin{array}{ll}
                        g(x)   &= \frac{1}{\sqrt[3]{1+x}} \\[0.6em]
                        T_0(x) &= 1 \\[0.3em]
                        T_1(x) &= 1 - \frac{x}{3} \\[0.3em]
                        T_2(x) &= 1 - \frac{x}{3} + \frac{4x^2}{9 \cdot 2!} \\[0.3em]
                        T_3(x) &= 1 - \frac{x}{3} + \frac{4x^2}{9 \cdot 2!} - \frac{28x^3}{27 \cdot 3!} \\[0.3em]
                        T_4(x) &= 1 - \frac{x}{3} + \frac{4x^2}{9 \cdot 2!} - \frac{28x^3}{27 \cdot 3!} + \frac{280x^4}{81 \cdot 4!}
                    \end{array}\)
                \end{multicols}

            \item[(c)]
                Taylorpolynom für $f(x) = e^x \cdot \sin x$ an $x_0 = 0$:
                \[ T_5(x) = x + x^2 + \frac{x^3}{3} - \frac{x^4}{6} - \frac{x^5}{30} \]

        \end{enumerate}
    \item[\textbf{2.}]
        \begin{enumerate}
            \item[(i)]
            \item[(ii)]
            \item[(iii)]
            \item[(iv)]
        \end{enumerate}
    \item[\textbf{3.}]
        \begin{enumerate}
            \item[(a)]
            \item[(b)]
            \item[(c)]
            \item[(d)]
        \end{enumerate}
    \item[\textbf{4.}]
        \begin{enumerate}
            \item[(a)]
            \item[(b)]
            \item[(c)]
        \end{enumerate}
\end{enumerate}

\end{document}
