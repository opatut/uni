\documentclass[compress,t]{beamer}

\newif\iffinal
\finalfalse

\usetheme{Custom}

\usepackage[utf8]{inputenc}
\usepackage[ngerman]{babel}
\usepackage{floatflt}
\usepackage{graphicx}
\usepackage{tikz}
\usepackage{xcolor}
\usepackage{listings}
\usepackage{mdframed}

\usetikzlibrary{shapes,arrows,calc,decorations.markings}

\newcommand{\qq}[1]{\symbol{34}#1\symbol{34}}
\newcommand{\bash}[1]{
    \begin{mdframed}[backgroundcolor=black!80,
            innerbottommargin=1pt,
            innertopmargin=3pt,
            innerleftmargin=3pt,
            innerrightmargin=3pt,
            linewidth=0]
        \textcolor{white}{\texttt{\small #1}}
    \end{mdframed}
}

\title{Lustre}
\author{Paul Bienkowski \\[0.2em] \scriptsize 2bienkow@informatik.uni-hamburg.de}
\institute{Proseminar ``Ein-/Ausgabe - Stand der Wissenschaft''}
\date{2013-06-10}
\setbeamerfont{note page}{size=\small}

\tikzset{
    onslide/.code args={<#1>#2}{% http://tex.stackexchange.com/a/6155/16595
        \only<#1>{\pgfkeysalso{#2}}
    }
}

\tikzset{
    o/.style={
        decoration={
            markings,
            mark={
                at position 0
                with {
                    \fill circle [xshift=1pt,radius=1pt];
                }
            }
        },
        postaction=decorate,
        -latex,
        thick
    }
}

%\AtBeginSection[]{
\newcommand{\sectiontoc}{
    \begin{frame}{\textbf{\insertsectionhead}}
        \tableofcontents[current]
    \end{frame}

    \addtocounter{framenumber}{-1}% If you don't want them to affect the slide number
}


\newcommand{\OutlineColumnNumbers}{2}

\begin{document}

%%%%%%%%%%%%%%%%%%%%%%%%%%%%%%%%%%%%%%%%%%%%%%%%%%%%%%%%%%%%%%%%%%%%%%%%%%%%%%%%

\iffinal\else
\begin{frame}{TODO}
    \begin{itemize}
        \item Caching?
    \end{itemize}
\end{frame}
\fi

\begin{frame}
    \titlepage
\end{frame}

\begin{frame}{\textbf{Outline}}
    \tableofcontents
\end{frame}

%%%%%%%%%%%%%%%%%%%%%%%%%%%%%%%%%%%%%%%%%%%%%%%%%%%%%%%%%%%%%%%%%%%%%%%%%%%%%%%%

\section{Introduction}

\begin{frame}{What is Lustre}
    parallel, scaling, for clusters, based within linux kernel...
\end{frame}

%%%%%%%%%%%%%%%%%%%%%%%%%%%%%%%%%%%%%%%%%%%%%%%%%%%%%%%%%%%%%%%%%%%%%%%%%%%%%%%%

\section{The Project}
\sectiontoc

\subsection{Goals and Priorities}
\begin{frame}{Goals}
    2007: performance \textgreater features \textgreater stability

    ``it\'s a science project''

    2010: stability \textgreater performance \textgreater features

    used in high-performance production environments
\end{frame}

\subsection{History}
\begin{frame}{History}
    ?
\end{frame}

\subsection{Involved Companies}
\begin{frame}{Involved Companies}
    Sun, Oracle, Cray, ... (what do they do, why do they do it?)
\end{frame}
\begin{frame}{Supercomputers}
    Titan \& Co. use it!
\end{frame}

%%%%%%%%%%%%%%%%%%%%%%%%%%%%%%%%%%%%%%%%%%%%%%%%%%%%%%%%%%%%%%%%%%%%%%%%%%%%%%%%

\section{Lustre Architecture}
\sectiontoc

\subsection{Network Architecture}
\begin{frame}{Network Structure}
    What data is stored where? (graph)
\end{frame}
\begin{frame}{Metadata Server}
    Where are they? How to access?
\end{frame}
\begin{frame}{Object Storage Server}
    Contain OSTs (Object storage targets)
\end{frame}
\begin{frame}{Network Capabilities}
    How is data transfered?

    Protocol stack (TCP, ...)

    Different network types (ethernet, infiniband, ...)
\end{frame}
\begin{frame}{Failover}
    Failover mechanism and typical setups (graphs)

    Why are is a failover mechanism cool? Live-Upgrades!
\end{frame}

\subsection{Data Storage and Access}
\begin{frame}{Excursion: INodes}
    ... because MDS do something similar (metadata records)

    (graph)

    compare this on next slide
\end{frame}
\begin{frame}{Metadata}
    how metadata is stored in the MDS

    what metadata is stored?

    how metadata is fetched from the MDS
\end{frame}
\begin{frame}{Striping}
    [repeat] what is striping (RAID 0)

    why do they use it in lustre -- speed advantage
\end{frame}

\subsection{Software Architecture}
\begin{frame}{Software Architecture}
    what software is running where?
\end{frame}
\begin{frame}{Interversion Compatibility}
    Sun ``guarantees'' [citation needed] compatibbility between
    minor versions

    $\rightarrow$ on-line upgrade-ability using failover systems
\end{frame}
\begin{frame}{ldiskfs - Customized ext3}
    why we need a customized filesystem to work ON TOP of
\end{frame}
\begin{frame}{Kernel patching (serverside)}
    just tell them the kernel needs to be patched (2.6.*) and what that means
\end{frame}
\begin{frame}{Patchfree Client}
    How can clients access the data? (lustre-fs, liblustre, NFS)

    (kernel-independent)

    even NFS, that works everywhere!
\end{frame}
\begin{frame}{Limitations}
    \textbf{Server}

    very platform dependent

    needs compatible kernel

    \textbf{Client}

    all linux kernel \textgreater2.6 supported

    NFS for Windows, MacOS

    even FUSE support on the way
\end{frame}

%%%%%%%%%%%%%%%%%%%%%%%%%%%%%%%%%%%%%%%%%%%%%%%%%%%%%%%%%%%%%%%%%%%%%%%%%%%%%%%%

\section{Performance}
\sectiontoc

\subsection{Throughput Examples}
\begin{frame}{Throughput Examples}
    Yes, the speeds add up!

    There are systems with 5000 OSS.

    Up to 160 OSS / file.

    16 OST/OSS is quite normal.

    1 TiB/OST
\end{frame}

\subsection{Scalability}
\begin{frame}{Scalability}
    Just multiply. Works as long as your network supports it (e.g. InfiniBand \textgreater WiFi ... )
\end{frame}

\subsection{???}
\begin{frame}{???}
    Maybe I can find some more data samples -- where is the interesting stuff?
\end{frame}

%%%%%%%%%%%%%%%%%%%%%%%%%%%%%%%%%%%%%%%%%%%%%%%%%%%%%%%%%%%%%%%%%%%%%%%%%%%%%%%%

\section{Conclusion}

\begin{frame}{\textbf{Conclusion}}
    ...
\end{frame}

%%%%%%%%%%%%%%%%%%%%%%%%%%%%%%%%%%%%%%%%%%%%%%%%%%%%%%%%%%%%%%%%%%%%%%%%%%%%%%%%

\section{References}

\begin{frame}{\textbf{References}}
    ...
\end{frame}

%%%%%%%%%%%%%%%%%%%%%%%%%%%%%%%%%%%%%%%%%%%%%%%%%%%%%%%%%%%%%%%%%%%%%%%%%%%%%%%%

\end{document}
