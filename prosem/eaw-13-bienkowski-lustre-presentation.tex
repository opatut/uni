\documentclass[compress,t,xcolor=dvipsnames]{beamer}

\newif\iffinal
\finalfalse
\finaltrue

\usetheme{Custom}

\usepackage[utf8]{inputenc}
\usepackage[ngerman]{babel}
\usepackage{floatflt}
\usepackage{graphicx}
\usepackage{tikz}
%\usepackage[usenames]{xcolor}
\usepackage{listings}
\usepackage{mdframed}

\usetikzlibrary{shapes,arrows,calc,decorations.markings,shadows.blur}

\newcommand{\qq}[1]{\symbol{34}#1\symbol{34}}
\newcommand{\bash}[1]{
    \begin{mdframed}[backgroundcolor=black!80,
            innerbottommargin=1pt,
            innertopmargin=3pt,
            innerleftmargin=3pt,
            innerrightmargin=3pt,
            linewidth=0]
        \textcolor{white}{\texttt{\small #1}}
    \end{mdframed}
}

\title{Lustre}
\author{Paul Bienkowski \\[0.2em] \scriptsize 2bienkow@informatik.uni-hamburg.de}
\institute{Proseminar ``Ein-/Ausgabe - Stand der Wissenschaft''}
\date{2013-06-10}
\setbeamerfont{note page}{size=\small}

\tikzset{
    onslide/.code args={<#1>#2}{% http://tex.stackexchange.com/a/6155/16595
        \only<#1>{\pgfkeysalso{#2}}
    }
}

\tikzset{
    o/.style={
        decoration={
            markings,
            mark={
                at position 0
                with {
                    \fill circle [xshift=1pt,radius=1pt];
                }
            }
        },
        postaction=decorate,
        -latex,
        thick
    }
}

%\AtBeginSection[]{
\newcommand{\sectiontoc}{
    \begin{frame}{\textbf{\insertsectionhead}}
        \tableofcontents[current]
    \end{frame}

    \addtocounter{framenumber}{-1}% If you don't want them to affect the slide number
}


\newcommand{\OutlineColumnNumbers}{2}

\begin{document}

%%%%%%%%%%%%%%%%%%%%%%%%%%%%%%%%%%%%%%%%%%%%%%%%%%%%%%%%%%%%%%%%%%%%%%%%%%%%%%%%

\iffinal\else
\begin{frame}{TODO}
    \begin{itemize}
        \item Caching?
    \end{itemize}
\end{frame}
\fi

\begin{frame}
    \titlepage
\end{frame}

\begin{frame}{\textbf{Outline}}
    \tableofcontents
\end{frame}

%%%%%%%%%%%%%%%%%%%%%%%%%%%%%%%%%%%%%%%%%%%%%%%%%%%%%%%%%%%%%%%%%%%%%%%%%%%%%%%%

\section{Introduction}

\begin{frame}{What is Lustre}
    parallel, scaling, for clusters, based within linux kernel...
\end{frame}

%%%%%%%%%%%%%%%%%%%%%%%%%%%%%%%%%%%%%%%%%%%%%%%%%%%%%%%%%%%%%%%%%%%%%%%%%%%%%%%%

\section{The Project}
\sectiontoc

\subsection{Goals and Priorities}
\begin{frame}{Goals}
    2007: performance \textgreater features \textgreater stability

    ``it\'s a science project''

    2010: stability \textgreater performance \textgreater features

    used in high-performance production environments
\end{frame}

\subsection{History}
\begin{frame}{History}
    \begin{itemize}
        \item started as a research project in 1999 by Peter Braam
        \item Braam founs \textbf{Cluster File Systems}
        \item 1.0 released in 2003
        \item \textbf{Sun Microsystems} aquires Cluster File Systems in 2007
        \item \textbf{Oracle Corporation} aquires Sun Mircrosystems in 2010
        \item Oracle ceases Lustre development, many new Organizations continue
            development, including \textbf{Xyratec},  \textbf{Whamcloud}, and
            more
        \item in 2012, \textbf{Intel} aquires Whamcloud
        \item in 2013, Xyratec purchases the original Lustre trademark from Oracle
    \end{itemize}
\end{frame}

\subsection{Who is involved?}
\begin{frame}{Who is involved?}
    \begin{description}
        \item[Oracle] \emph{no development}, only pre-1.8 support
        \item[Intel] funding, preparing for \emph{exascale computing}
        \item[Cray] funding, development (Titan Supercomputer)
        \item[Xyratex] hardware bundling
        \item[OpenSFS] (Open Scalable File Systems) ``keeping Lustre open''
        \item[EOFS] (EUROPEAN Open File Systems) (community collaboration)
        \item[FOSS Community] many joined one of the above to help development
            (e.g. Braam works for Xyratex now)
        \item[DDN, Dell, NetApp, Terascala, Xyratex]\hfill \\
            storage hardware bundled with Lustre
    \end{description}
\end{frame}
\begin{frame}{Supercomputers}
    Titan \& Co. use it!
\end{frame}

%%%%%%%%%%%%%%%%%%%%%%%%%%%%%%%%%%%%%%%%%%%%%%%%%%%%%%%%%%%%%%%%%%%%%%%%%%%%%%%%

\section{Lustre Architecture}
\sectiontoc

\subsection{Network Architecture}
\begin{frame}<1-5>[label=network]{Network Structure}
\iffinal
    \tikzset{
        area/.style=    { draw=black!30!white, dashed, very thick, rounded corners=3pt },
        highlight/.style = { fill=yellow!50!white, rounded corners=3pt },
        node/.style =   { circle, inner sep=0pt, minimum size=15pt },
        client/.style = { node, fill=RoyalBlue },
        mds/.style =    { node, fill=YellowGreen },
        mdt/.style =    { node, fill=Dandelion },
        oss/.style =    { node, fill=OliveGreen },
        ost/.style =    { node, fill=BrickRed },
        link/.style =   { line width=1.5pt, line cap=rect },
        olink/.style =  { link, draw=Red },
        mlink/.style =  { link, draw=Dandelion!50!white },
        net/.style =    { link, line width=2pt },
        net1/.style =   { net, draw=black!20!white },
        net2/.style =   { net, draw=black!50!white },
        failover/.style={
            %decoration={markings,mark=at position 1 with {\arrow{triangle 60}},scale=0.5},
            %postaction=decorate,
            line width=1.0pt,
            -stealth,
        },
        failover'/.style={
            failover,
            stealth-stealth
        },
    }

    \makebox[\textwidth][c] {
        \begin{tikzpicture}[y=-0.9cm,x=0.9cm]

            \uncover<1-> {
                \draw[area]
                    (-5.5, -0.5) rectangle (2.5, 0.5)
                    (-5.5, 1) rectangle (-3.4, 4.0)
                    (-5.5, 4.5) rectangle (4.5, 6.5)
                    ;

                \node[anchor=west] (clients)  at (-5.5, 0.0) {\scriptsize CLIENTS};
                \node[anchor=west,align=left] (objects)  at (-5.5, 5.5) {\scriptsize{}OBJECT\\[-0.5em]\scriptsize{}STORAGE};
                \node[anchor=west] (metadata) at (-5.5, 1.3) {\scriptsize METADATA};
            }

            \uncover<5-> {
                \node[anchor=south] (mds)     at (-4.0, 2.1) {\tiny MDS};
            }
            \uncover<6-> {
                \node[anchor=east] (oss)      at (-3.5, 5.0) {\tiny OSS};
            }
            \uncover<7-> {
                \node[anchor=south] (mdt)     at (-5.0, 2.1) {\tiny MDT};
                \node[anchor=east] (ost)      at (-3.5, 6.0) {\tiny OST};
            }

            \uncover<4-> {
                \node[align=left] (net) at (-1.3, 3.3) {\tiny different network types\\[-0.5em]\tiny (ethernet, InfiniBand)};
            }

            \uncover<5> {
                \fill[highlight] (-4.4, 2.1) rectangle (-3.6, 3.9);
            }
            \uncover<6> {
                \fill[highlight] (-2.9, 4.6) rectangle (4.4, 5.4);
            }

            \uncover<3-> {
                \draw[net1]
                    ( 0.0, 1) -- (-3.0, 1) -- (-3.0, 4) -- (3.9, 4)

                    (-2.0, 0) -- (-2.0, 1) %client1
                    (-1.0, 0) -- (-1.0, 1) %client2
                    ( 0.0, 0) -- ( 0.0, 1) %client3
                    %( 0.9, 0) -- ( 0.9, 1) %client4
                    %( 1.9, 0) -- ( 1.9, 1) %client5

                    (-4.0, 2.6) -- (-3.0, 2.6) %mds1
                    (-4.0, 3.6) -- (-3.0, 3.6) %mds2

                    (-2.6, 5) -- (-2.6, 4) %oss1
                    (-1.1, 5) -- (-1.1, 4) %oss2
                    (-0.1, 5) -- (-0.1, 4) %oss3
                    ( 1.9, 5) -- ( 1.9, 4) %oss4
                    ( 2.9, 5) -- ( 2.9, 4) %oss5
                    ( 3.9, 5) -- ( 3.9, 4) %oss6
                    ;
            }

            \uncover<4-> {
                \draw[net2]
                    (2.0, 1.2) -- (-2.8, 1.2) -- (-2.8, 3.8) -- (4.1, 3.8)

                    ( 1.0, 0) -- (1.0, 1.2) %client4
                    ( 2.0, 0) -- (2.0, 1.2) %client5

                    (-4.0, 2.4) -- (-2.8, 2.4) %mds1
                    (-4.0, 3.4) -- (-2.8, 3.4) %mds2

                    (-2.4, 5) -- (-2.4, 3.8) %oss1
                    (-0.9, 5) -- (-0.9, 3.8) %oss2
                    ( 0.1, 5) -- ( 0.1, 3.8) %oss3
                    ( 2.1, 5) -- ( 2.1, 3.8) %oss4
                    ( 3.1, 5) -- ( 3.1, 3.8) %oss5
                    ( 4.1, 5) -- ( 4.1, 3.8) %oss6
                    ;
            }

            \uncover<7-> {
                \path[mlink]
                    (-4, 2.5) -- (-5, 3) %mds1
                    (-4, 3.5) -- (-5, 3) %mds2
                    ;

                \path[olink]
                    (-2.5, 5) -- (-3, 6) %oss1
                    (-2.5, 5) -- (-2, 6) %oss1

                    (-1, 5) -- (-1, 6) %oss2
                    (-1, 5) -- (0, 6) %oss2
                    (0, 5) -- (-1, 6) %oss3
                    (0, 5) -- (0, 6) %oss3

                    (2, 5) -- (3, 6) %oss4
                    (3, 5) -- (3, 6) %oss5
                    (4, 5) -- (3, 6) %oss6
                    ;
            }

            \uncover<2-> {
                \node[client] (client1) at (-2, 0) {\includegraphics[width=10pt]{gfx/client.png}};
                \node[client] (client2) at (-1, 0) {\includegraphics[width=10pt]{gfx/client.png}};
                \node[client] (client3) at ( 0, 0) {\includegraphics[width=10pt]{gfx/client.png}};
                \uncover<4-> {
                    \node[client] (client4) at ( 1, 0) {\includegraphics[width=10pt]{gfx/client.png}};
                    \node[client] (client5) at ( 2, 0) {\includegraphics[width=10pt]{gfx/client.png}};
                }

                \node[mds] (mds1) at (-4, 2.5) {\includegraphics[width=9pt]{gfx/server.png}};
                \node[mds] (mds2) at (-4, 3.5) {\includegraphics[width=9pt]{gfx/server.png}};

                \node[oss] (oss1) at (-2.5, 5) {\includegraphics[width=9pt]{gfx/server.png}};
                \node[oss] (oss2) at (-1.0, 5) {\includegraphics[width=9pt]{gfx/server.png}};
                \node[oss] (oss3) at ( 0.0, 5) {\includegraphics[width=9pt]{gfx/server.png}};
                \node[oss] (oss4) at ( 2.0, 5) {\includegraphics[width=9pt]{gfx/server.png}};
                \node[oss] (oss5) at ( 3.0, 5) {\includegraphics[width=9pt]{gfx/server.png}};
                \node[oss] (oss6) at ( 4.0, 5) {\includegraphics[width=9pt]{gfx/server.png}};
            }

            \uncover<7-> {
                \node[mdt] (mdt1) at (-5, 3.0) {\includegraphics[width=9pt]{gfx/database.png}};

                \node[ost] (ost1) at (-3, 6) {\includegraphics[width=9pt]{gfx/database.png}};
                \node[ost] (ost2) at (-2, 6) {\includegraphics[width=9pt]{gfx/database.png}};
                \node[ost] (ost3) at (-1, 6) {\includegraphics[width=9pt]{gfx/database.png}};
                \node[ost] (ost4) at ( 0, 6) {\includegraphics[width=9pt]{gfx/database.png}};
                \node[ost] (ost5) at ( 3, 6) {\includegraphics[width=9pt]{gfx/database.png}};
            }

            \uncover<8-> {
                \draw[failover]  (mds1) to[bend left=30] (mds2);
                \draw[failover]  (oss4) to[bend left=30] (oss5);
                \draw[failover]  (oss5) to[bend left=30] (oss6);
                \draw[failover]  (oss6) to[bend left=30] (oss4);
                \draw[failover'] (oss2) to[bend left=30] (oss3);
            }
        \end{tikzpicture}
    }
\else
    (graph here, finaltrue to show)
    \uncover<1-> {1}
    \uncover<2-> {2}
    \uncover<3-> {3}
\fi
\end{frame}

\begin{frame}{Metadata Server (\textbf{MDS})}
    \begin{itemize}
        \item store file information (metadata)
        \item accessed by clients to access files
        \item \emph{manage} data storage
        \item at least one required
        \item up to $\sim$100 possible (failovers)
    \end{itemize}
\end{frame}

\againframe<5-6>{network}

\begin{frame}{Object Storage Server (\textbf{OSS})}
    \begin{itemize}
        \item store file content (objects)
        \item accessed by clients directly
        \item at least one required
        \item $>10000$ OSS are used in large scale computers
        \item multiple targets per server
        \item multiple servers per target
    \end{itemize}
\end{frame}

\againframe<6-7>{network}

\begin{frame}{Targets}
    \begin{itemize}
        \item two types
            \begin{itemize}
                \item object storage target (OST)
                \item metadata target (MDT)
            \end{itemize}
        \item can be any block device
            \begin{itemize}
                \item normal hard disk / flash drive / SSD
                \item advanced storage arrays
            \end{itemize}
        \item will be formatted for lustre
    \end{itemize}
\end{frame}

\begin{frame}{Failover}
    \begin{itemize}
        \item if one server failes, another one takes over
        \item backup server needs access to targets
        \item enabled on-line software upgrades (one-by-one)
    \end{itemize}
\end{frame}

\againframe<7-8>{network}

\begin{frame}{Network Capabilities}
    How is data transfered?

    Protocol stack (TCP, ...)

    Different network types (ethernet, infiniband, ...)
\end{frame}

\subsection{Data Storage and Access}
\begin{frame}{Excursion: INodes}
    ... because MDS do something similar (metadata records)

    (graph)

    compare this on next slide
\end{frame}
\begin{frame}{Metadata}
    how metadata is stored in the MDS

    what metadata is stored?

    how metadata is fetched from the MDS
\end{frame}
\begin{frame}{Striping}
    [repeat] what is striping (RAID 0)

    why do they use it in lustre -- speed advantage
\end{frame}

\subsection{Software Architecture}
\begin{frame}{Software Architecture}
    what software is running where?
\end{frame}
\begin{frame}{Interversion Compatibility}
    Sun ``guarantees'' [citation needed] compatibbility between
    minor versions

    $\rightarrow$ on-line upgrade-ability using failover systems
\end{frame}
\begin{frame}{ldiskfs - Customized ext3}
    why we need a customized filesystem to work ON TOP of
\end{frame}
\begin{frame}{Kernel patching (serverside)}
    just tell them the kernel needs to be patched (2.6.*) and what that means
\end{frame}
\begin{frame}{Patchfree Client}
    How can clients access the data? (lustre-fs, liblustre, NFS)

    (kernel-independent)

    even NFS, that works everywhere!
\end{frame}
\begin{frame}{Limitations}
    \textbf{Server}

    very platform dependent

    needs compatible kernel

    \textbf{Client}

    all linux kernel \textgreater2.6 supported

    NFS for Windows, MacOS

    even FUSE support on the way
\end{frame}

%%%%%%%%%%%%%%%%%%%%%%%%%%%%%%%%%%%%%%%%%%%%%%%%%%%%%%%%%%%%%%%%%%%%%%%%%%%%%%%%

\section{Performance}
\sectiontoc

\subsection{Throughput Examples}
\begin{frame}{Throughput Examples}
    Yes, the speeds add up!

    There are systems with 5000 OSS.

    Up to 160 OSS / file.

    16 OST/OSS is quite normal.

    1 TiB/OST
\end{frame}

\subsection{Scalability}
\begin{frame}{Scalability}
    Just multiply. Works as long as your network supports it (e.g. InfiniBand \textgreater WiFi ... )
\end{frame}

\subsection{???}
\begin{frame}{???}
    Maybe I can find some more data samples -- where is the interesting stuff?
\end{frame}

%%%%%%%%%%%%%%%%%%%%%%%%%%%%%%%%%%%%%%%%%%%%%%%%%%%%%%%%%%%%%%%%%%%%%%%%%%%%%%%%

\section{Conclusion}

\begin{frame}{\textbf{Conclusion}}
    ...
\end{frame}

%%%%%%%%%%%%%%%%%%%%%%%%%%%%%%%%%%%%%%%%%%%%%%%%%%%%%%%%%%%%%%%%%%%%%%%%%%%%%%%%

\section{References}

\begin{frame}{\textbf{References}}
    ...
\end{frame}

%%%%%%%%%%%%%%%%%%%%%%%%%%%%%%%%%%%%%%%%%%%%%%%%%%%%%%%%%%%%%%%%%%%%%%%%%%%%%%%%

\end{document}
