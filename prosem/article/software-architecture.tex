\section{Software Architecture}

\subsection{Server Software}

The Lustre servers have a variety of requirements and limitations to consider
before installation.

The object and metadata storage targets are formatted using \texttt{mkfs.lustre},
which is using the ldiskfs filesystem. This filesystem is based on ext4, and
requires some kernel patches only available for a limited number of Enterprise
Linux 2.6 kernels, for example Red Hat.

This makes the server software very platform dependent, a normal Linux user
cannot simply run a Lustre storage cluster ``at home''. This is not that much of
a problem in HPC environments, though, since the kernel that runs on the storage
servers is independent of the actual computing environment, allowing the clients
to run the kernels required/desired for the computations.

Starting with Lustre 2.4, ZFS support has finally been implemented. This is a
new and promising feature, that may not yet perform as well as the old ldiskfs
versions, but supports a wider range of systems, more data integrity features
\footnote{See \url{http://zfsonlinux.org/lustre.html} for detailled description
on ZFS features} and is actively being developed and improved.

\subsection{Client Software}

For Lustre clients, there is a broad variety of solutions available to access
the data. The basic implementation is a ``patchfree'' kernel module, available
for any Linux 2.6 kernel. This module lets the user mount the remote file system
using a native implementation, yielding the best performance results.

Other implementations include a userspace library (liblustre), that can be used
to develop custom applications that directly interact with the storage system,
a userspace filesystem driver (FUSE), which is important for OSX support, and
NFS access for MS Windows support and legacy versions, as well as any operating
system not supporting any of the above solutions.
Thus, the Lustre client is available cross-platform, with best performances
using the native filesystem module on Linux 2.6 or later.

\subsection{Interversion Compatibility}

The Lustre project usually supports some level of version \textbf{interoperability}
\cite{interoperability}. This means for example that clients of version 1.8 can
connect to servers running version 2.0 and vice versa.
This feature is mainly used to be able to upgrade a Lustre storage system without
taking it off-line, using the failover system, and upgrading the versions
sequentially. During this process, some devices still run the old version while
others run the new, in which case interoparability comes in handy.

