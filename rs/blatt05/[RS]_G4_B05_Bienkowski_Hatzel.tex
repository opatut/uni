\newcommand{\authorinfo}{Paul Bienkowski, Hans Ole Hatzel}
\newcommand{\titleinfo}{RS 05 (HA) zum 19.11.2012}

% PREAMBLE ===============================================================

\documentclass[a4paper,10pt]{scrartcl}
\usepackage[german,ngerman]{babel}
\usepackage[utf8]{inputenc}
\usepackage[T1]{fontenc}
\usepackage{lmodern}
\usepackage{amssymb}
\usepackage{mathtools}
\usepackage{amsmath}
\usepackage{enumerate}
\usepackage{array}
\usepackage{listings}
\usepackage{fullpage}
\usepackage{tikz}
\usepackage{pgffor}
\usepackage{fancyhdr}
\usepackage{lastpage}

\author{\authorinfo}
\title{\titleinfo}
\date{\today}

\pagestyle{fancy}
\fancyhf{}
\fancyhead[L]{\authorinfo}
\fancyhead[R]{\titleinfo}
\fancyfoot[C]{\thepage}
\renewcommand{\headrulewidth}{0.4pt}
\renewcommand{\footrulewidth}{0pt}
\renewcommand{\headheight}{12pt}
\renewcommand{\headsep}{12pt}

\begin{document}
\setcounter{secnumdepth}{0}
\maketitle

% DOCUMENT ===============================================================

\definecolor{awesomegray}{gray}{0.95}

\begin{enumerate}
    \item[\textbf{1.}]
        Die einzelnen Werte sollen nach folgendem Schema übersetzt werden:

            \begin{tikzpicture}[x=1em,y=-1em]
                \tikzstyle{every node}=[font=\small]

                \foreach \n in {0,...,11}{
                    \fill[fill=awesomegray] (\n*2,0) rectangle (\n*2+1, 1);
                    \fill[fill=awesomegray] (\n*2,2) rectangle (\n*2+1, 3);
                }

                \draw (0,0) rectangle (6,1);
                \draw (6,0) rectangle (12,1);
                \draw (12,0) rectangle (18,1);
                \draw (18,0) rectangle (24,1);

                \draw (0,2) rectangle (8,3);
                \draw (8,2) rectangle (16,3);
                \draw (16,2) rectangle (24,3);

                \node[label=center:a1] at (3,0.5) {};
                \node[label=center:a2] at (9,0.5) {};
                \node[label=center:a3] at (15,0.5) {};
                \node[label=center:a4] at (21,0.5) {};

                \node[label=center:b1] at (4,2.5) {};
                \node[label=center:b2] at (12,2.5) {};
                \node[label=center:b3] at (20,2.5) {};
            \end{tikzpicture}

            Um dies zu erreichen, werden die Eingabebits verschoben und mit ODER kombiniert. Um nur
            die ersten 8 bits zu verwenden, wird der Modulo 256 gebildet.

            \begin{lstlisting}[language=java]
int b1 = (a1 << 2 | a2 >>> 4) % 256;
int b2 = (a2 << 4 | a3 >>> 2) % 256;
int b3 = (a3 << 6 | a4)       % 256;
            \end{lstlisting}

    \item[\textbf{2.}]
        \begin{enumerate}
            \item[a)]
                Der Code hat $\frac{360^\circ}{15^\circ} = 24$ Codewörter.

            \item[b)]
                \textbf{Schritt 1} (2 Codewörter): \begin{verbatim}
0 1
                \end{verbatim}

                \textbf{Schritt 2} (4 Codewörter): \begin{verbatim}
00 01
11 10
                \end{verbatim}

                \textbf{Schritt 3} (8 Codewörter): \begin{verbatim}
 000  001 011 [010]
[110] 111 101  100
                \end{verbatim}

                \textbf{Schritt 4} (6 Codewörter): Die Wörter in [Klammern] werden weggelassen, um aus $6$ Codewörtern innerhalb von 2 Schritten $6 \cdot 2^2 = 24$
                Codewörter zu erhalten.

                \begin{verbatim}
000 001 011
111 101 100
                \end{verbatim}

                \textbf{Schritt 5} (12 Codewörter): \begin{verbatim}
0000 0001 0011 0111 0101 0100
1100 1101 1111 1011 1001 1100
                \end{verbatim}

                \textbf{Schritt 6} (24 Codewörter): \begin{verbatim}
00000 00001 00011 00111 00101 00100 01100 01101 01111 01011 01001 01100
11100 11001 11011 11111 11101 11100 10100 10101 10111 10011 10001 10000
                \end{verbatim}

                \newpage

                Die folgende Winkelcodierscheibe stellt diesen Code dar:
                \begin{figure}[!h]
                    \centering
                    \begin{tikzpicture}[x=1em,y=1em]
                        \foreach \n in {0,...,23}{
                            \draw[color=gray] (\n*15:1.5) -- (\n*15:6.5);
                        }

                        \draw[line width=1em]
                            (0:2) arc (0:180:2);

                        \draw[line width=1em]
                            (90:3) arc (90:270:3);

                        \draw[line width=1em]
                            (-45:4) arc (-45:45:4);
                        \draw[line width=1em]
                            (135:4) arc (135:225:4);

                        \draw[line width=1em]
                            (-30:5) arc (-30:30:5);
                        \draw[line width=1em]
                            (60:5) arc (60:120:5);
                        \draw[line width=1em]
                            (150:5) arc (150:210:5);
                        \draw[line width=1em]
                            (240:5) arc (240:300:5);

                        \draw[line width=1em]
                            (-15:6) arc (-15:15:6);
                        \draw[line width=1em]
                            (75:6) arc (75:105:6);
                        \draw[line width=1em]
                            (165:6) arc (165:195:6);
                        \draw[line width=1em]
                            (255:6) arc (255:285:6);

                        \draw (0,0) circle[radius=1.5];
                        \draw (0,0) circle[radius=2.5];
                        \draw (0,0) circle[radius=3.5];
                        \draw (0,0) circle[radius=4.5];
                        \draw (0,0) circle[radius=5.5];
                        \draw (0,0) circle[radius=6.5];
                    \end{tikzpicture}
                \end{figure}

        \end{enumerate}

    \item[\textbf{3.}]
        \begin{enumerate}
            \item[a)]
                Der entstehende Huffman-Baum kann so dargestellt werden (Wahrscheinlichkeits-Angaben
                werden zur Übersicht in Prozent notiert):

                \begin{figure}[!h]
                    \centering
                    \begin{tikzpicture}[x=1.5em,y=-1.5em,every node=radius\=2]
                        \tikzstyle{n}=[draw,circle,radius=1cm,font=\small,inner sep=0,minimum size=1.6em]

                        \node[n] (R) at (-1,0) {100};

                        \node[n] (R0) at (-3,3) {57};

                        \node[n] (R00) at (-4,6) {27};
                        \node[n,label=below:d] (R01) at (-2,6) {30};
                        \node[n] (R1) at (3,6) {43};

                        \node[n] (R000) at (-5,9) {15};
                        \node[n,label=below:k] (R001) at (-3,9) {12};
                        \node[n] (R10) at (1,9) {23};
                        \node[n] (R11) at (5,9) {20};

                        \node[n,label=below:l] (R0000) at (-6,12) {6};
                        \node[n] (R0001) at (-4,12) {9};
                        \node[n,label=below:a] (R100) at (0,12) {12};
                        \node[n] (R101) at (2,12) {11};
                        \node[n,label=below:g] (R110) at (4,12) {10};
                        \node[n,label=below:j] (R111) at (6,12) {10};

                        \node[n] (R00010) at (-3,15) {4};
                        \node[n,label=below:c] (R00011) at (-5,15) {5};
                        \node[n] (R1010) at (1,15) {6};
                        \node[n,label=below:f] (R1011) at (3,15) {5};

                        \node[n,label=below:e] (R000100) at (-4,18) {2};
                        \node[n,label=below:h] (R000101) at (-2,18) {2};
                        \node[n,label=below:b] (R10100) at (0,18) {3};
                        \node[n,label=below:i] (R10101) at (2,18) {3};


                        %\node[n,label=below:h] (h) at (2,0) {30};

                        \draw
                            (R) -- (R0)
                            (R) -- (R1)

                            (R0) -- (R00)
                            (R0) -- (R01)
                            (R1) -- (R10)
                            (R1) -- (R11)

                            (R00) -- (R000)
                            (R00) -- (R001)
                            (R10) -- (R100)
                            (R10) -- (R101)
                            (R11) -- (R110)
                            (R11) -- (R111)

                            (R000) -- (R0000)
                            (R000) -- (R0001)
                            (R101) -- (R1010)
                            (R101) -- (R1011)

                            (R0001) -- (R00010)
                            (R0001) -- (R00011)
                            (R1010) -- (R10100)
                            (R1010) -- (R10101)

                            (R00010) -- (R000100)
                            (R00010) -- (R000101)
                            ;

                    \end{tikzpicture}
                \end{figure}

                Legt man den reweils linken Zweig als 0 und den jeweils rechten Zweig
                als 1 fest, ergibt sich folgende Codierung:

                \begin{tabular}{rl|rl|rl}
                    a & 100    & b & 10100  & c & 00010 \\
                    d & 01     & e & 000110 & f & 1011  \\
                    g & 110    & h & 000111 & i & 10101 \\
                    j & 111    & k & 001    & l & 0000
                \end{tabular}

            \item[b)]
            $$\begin{array}{crl}
            H &=& {}
                - 2 \cdot 0.02 \cdot \log_2(0.02) - 2 \cdot 0.3 \cdot \log_2(0.3)
                - 2 \cdot 0.05 \cdot \log_2(0.05) - 0.06 \cdot \log_2(0.06)\\
            &&{}- 2 \cdot 0.10 \cdot \log_2(0.10) - 2 \cdot 0.12 \cdot \log_2(0.12)
                - 0.30 \cdot \log_2(0.30)\\
            &\approx& 3.12
            \end{array}$$

        \end{enumerate}

    \item[\textbf{4.}]
        \begin{enumerate}
            \item[a)]
                Eine unsystematischer Beispiel-Code:

                \begin{tabular}{c|c|c|c|c|c|c|c|c|c}
                    0 & 1 & 2 & 3 & 4 & 5 & 6 & 7 & 8 & 9 \\
                    \hline
                    0110 & 0010 & 0000 & 0100 & 0001 & 1000 & 1100 & 1001 & 0011 & 1111
                \end{tabular}

                $H_0 = 10 \cdot 0.1 \cdot \log_2(2^4) = \log_2(16) = 4$ Bit

                $H = - 10 \cdot 0.1 \cdot \log_2(0.1) = -\log_2(0.1) \approx 3.322$ Bit

                $R = 4$ Bit${}- 3.322$ Bit $= 0.678$ Bit

            \item[b)]
                Es werden mindestens 7 bits (128 mögliche Codewörter) benötigt. Ein möglicher
                Code wäre die binäre Repräsentation, in der das $(7-n)$-te bit den Wert $2^n$
                erhält.

                $$0000000, 0000001, 0000010, 0000011, 0000100, ..., 1100011$$

                $H_0 = 100 \cdot \frac{1}{100} \cdot \log_2(2^7) = 7 $ Bit

                $H = - 100 \cdot \frac{1}{100} \cdot \log_2(\frac{1}{100}) \approx 6.6438526$ Bit
                %meinst du man sollte das lieber als Summe schreiben?

                $R = H_0 - H \approx 0,35615$ Bit

                Redundanz pro Dezimalziffer:

                $\frac{R}{2}=0,178075$
                %nicht sicher ob richtig sollte doch =R/10 sein. Rundungsungenauigkeiten?


            \item[c)]
                \textbf{Zahlen 000...999:}

                    $H_0 = 1000 \cdot \frac{1}{1000} \cdot \log_2(2^{10}) = 10$

                    Informationsgehalt pro Dezimalstelle: $\frac{10}{3} \approx 3.33$

                \textbf{Zahlen 0000...9999:}

                    $H_0 = 10000 \cdot \frac{1}{10000} \cdot \log_2(2^{14}) = 14$

                    Informationsgehalt pro Dezimalstelle: $\frac{14}{4} = 3.5$

                    Da es sich um einen Blockcode (im Binärsystem) handelt lassen sich immer nur Worte bilden, die einen
                    Wertebereich ganzer Zweierpotenzen darstellen.
                    Die Zweierpotenzen im höheren Zahelenbereich liegen weiter auseinander.
                    In diesem Beispiel sind die dichtesten Zweierpotenzen: $2^{10} = 1024$ und $2^{14} = 16384$.
                    Im zweiten Fall gibt es also wesentlich mehr mögliche Worte als darzustellende Werte, daher ist die
                    Redundanz hoch bzw. der Informationsgehalt geringer. Es wären über 1.6 mal mehr Zahlen darstellbar.

            \item[d)]
            	Codierung:

                \begin{tabular}{c|c|c|c|c|c|c|c|c|c}
                    0 & 1 & 2 & 3 & 4 & 5 & 6 & 7 & 8 & 9 \\
                    \hline
                    111 & 110 & 101 & 100 & 0111 & 0110 & 0101 & 0100 & 001 & 000
                \end{tabular}

                $$H_0 = 6 \cdot \frac{1}{10} \cdot \log_2(2^3) + 4 \cdot \frac{1}{10} \cdot \log_2(2^4) = 3.4 \text{ Bit}$$

                $$H = - 10 \cdot \frac{1}{10} \cdot \log_2(\frac{1}{10}) \approx 3.322 \text{ Bit}$$

                $$R = H_0 - H \approx 0.078 \text{ Bit}$$

                Eine effektivere Codierung als eine solche Huffman-Codierung ist nicht möglich, solange das Binärsystem
                als Alphabet feststeht und nur die Worte 0...9 kodiert werden sollen. Mit anderem Alphabet oder durch
                Zusammenfassen der Worte/Ziffern zu Zahlen (siehe (c)) könnten noch geringere Redundanzen erreicht werden.

        \end{enumerate}
\end{enumerate}

\end{document}
