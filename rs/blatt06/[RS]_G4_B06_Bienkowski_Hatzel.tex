\newcommand{\authorinfo}{Paul Bienkowski, Hans Ole Hatzel}
\newcommand{\titleinfo}{RS 06 (HA) zum 30.11.2012}

% PREAMBLE ===============================================================

\documentclass[a4paper,10pt]{scrartcl}
\usepackage[german,ngerman]{babel}
\usepackage[utf8]{inputenc}
\usepackage[T1]{fontenc}
\usepackage{lmodern}
\usepackage{amssymb}
\usepackage{mathtools}
\usepackage{amsmath}
\usepackage{enumerate}
\usepackage{array}
\usepackage{listings}
\usepackage{fullpage}
\usepackage{tikz}
\usepackage{pgffor}
\usepackage{fancyhdr}
\usepackage{lastpage}

\author{\authorinfo}
\title{\titleinfo}
\date{\today}

\pagestyle{fancy}
\fancyhf{}
\fancyhead[L]{\authorinfo}
\fancyhead[R]{\titleinfo}
\fancyfoot[C]{\thepage}
\renewcommand{\headrulewidth}{0.4pt}
\renewcommand{\footrulewidth}{0pt}
\renewcommand{\headheight}{12pt}
\renewcommand{\headsep}{12pt}

\begin{document}
\setcounter{secnumdepth}{0}
\maketitle

% allow huuuuge matrix
\setcounter{MaxMatrixCols}{31}

% DOCUMENT ===============================================================


\begin{enumerate}
    \item[\textbf{1.}]
        Damit ein zyklischer Code entsteht, muss jede Bitänderung, die zum nächsten Codewort
        führt, irgendwann wieder umgekehrt werden. Da in einem einschrittigen Code nicht zwei
        Änderungen gleichzeitig durchgeführt werden dürfen, folgt daraus, dass für jedes Bit
        immer zwei Codewörter zu einem Paar zusammengehören. Damit ist klar, dass nur eine
        gerade Anzahl Codewörter möglich ist.

    \item[\textbf{2.}]
    \begin{enumerate}
        \item[a)]
            Der Minimalabstand beträgt 4 Bit, denn sobald sich ein Bit ändert, ändern
            sich damit auch die Paritätsbits der jeweiligen Spalte und Zeile, sowie das
            Paritätsbit der Spalten-Paritätsbits.

        \item[b)]
            \textbf{Einbitfehler} können stets korrigiert werden, da durch die jeweiligen
            Zeilen- und Spaltenparitätsbits die Position in der Matrix eindeutig bestimmt
            werden kann. An dieser Position muss dann ein Fehler vorliegen, das entsprechende
            Bit muss nur invertiert werden.

            \textbf{Zweibitfehler} werden mit Sicherheit erkannt, können jedoch nicht
            korrigiert werden. Es sind zwei Fälle zu unterscheiden. Solche, in denen die
            fehlerhaften Bits neben- oder übereinander liegen, und solche, in denen sie in
            unterschiedlichen Zeilen \emph{und} Spalten liegen. Im ersten Fall wird zwar
            entweder die Spalte oder Zeile beider Bits erkannt, allerdings ist in der anderen
            Dimension das Paritätsbit zweifach invertiert worden, sodass diese Dimension nicht
            erkannt wird. Damit ist nicht eindeutig, wo der Fehler auftrat. Im zweiten Fall
            stehen dem Korrekturalgorithmus zwei Spalten und zwei Zeilen zur Verfügung, an
            deren 4 Kreuzungen insgesamt zwei Fehler auftraten. Die Fehler müssen immer an
            gegenüberliegenden Kreuzungen liegen, allerdings gibt es somit immer noch 2
            Möglichkeiten, den Zweibitfehler zu lokalisieren.

            \textbf{Dreibitfehler} werden stets als Fehler erkannt, jedoch nicht zwingend
            als Dreibitfehler. Die einzige Anordnung, die korrekt erkannt wird, und korrigiert
            werden kann, beinhaltet die drei Fehler entweder in gleicher Zeile oder gleicher Spalte.
            Konstellationen, bei denen jeweils zwei Fehler in der gleichen Zeile und Spalte
            liegen, erscheinen in den Prüfsummen wie Einbitfehler an der korrekten ,,Ecke'', die
            ,,Korrektur'' dessen bewirkt einen Vierbitfehler. Andere Konstellationen haben wie
            Zweibitfehler mehrere mögliche Lösungen, sodass sie nicht sicher korrigiert werden können.

        \item[c)]
            \textbf{Vierbitfehler}, bei denen jeweils zwei Fehler in gleicher Zeile und Spalte stehen,
            und somit ein Rechteck aufspannen, werden nicht erkannt, da eine Spalte bzw. Zeile
            nur als fehlerhaft erkannt wird, wenn sie eine ungerade Anzahl Fehler enthält. In
            einem solchen Vierbitfehler werden somit keine fehlerhaften Spalten oder Zeilen erkannt.

            Bei Fehlern in den folgenden Bits wird beispielsweise kein Fehler erkannt:
            $d_{0,0}, d_{0,1}, d_{3,0}, d_{3,1}$.

        \item[d)]
            Die zwei das Rechteck aufspannende Ecken müssen in unterschiedlicher Zeile und Spalte
            platziert werden (Paritätsbits eingeschlossen). Da es 4 Möglichkeiten gibt, erste und
            zweite Ecke für das gleiche Rechteck zu definieren, muss durch 4 geteilt werden. Es
            ergibt sich:

            $$n_4 = \frac{9 \cdot 9 \cdot 8 \cdot 8}{4} = 1296$$

            Die Anzahl der Möglichkeiten, in einer $9 \times 9$ Matrix 4 verschiedene Fehler zu
            platzieren, ist

            $$N_4 = 81 \cdot 80 \cdot 79 \cdot 78 = 39\;929\;760$$

            Damit ist der Anteil der nicht erkennbaren Vierbitfehler

            $$\frac{n_4}{N_4} \approx 0.00325 \%$$

    \end{enumerate}
    \item[\textbf{3.}]
    \begin{enumerate}
        \item[a)]
            Das fehlerhafte Codewort lautet: 0 0 1 1 0 0 0.

            $x_a = 0 \oplus 1 \oplus 0 \oplus 0 = 1$\\
            $x_b = 0 \oplus 1 \oplus 0 \oplus 0 = 1$\\
            $x_c = 1 \oplus 0 \oplus 0 \oplus 0 = 1$

            Also ist in der Prüftabelle die Codewortstelle zu suchen, die in allen drei Prüfgruppen liegt, also $c_7/d_4$.

        \item[b)]
            Für die n-te Zeile der \textbf{Generatormatrix G} liest man im Schema ab, welche Datenbits den Wert von
            $c_n$ beeinflussen. Repräsentiert das Codebit $c_n$ ein Datenbit $d_m$, so wird in der Matrixzeile nur das
            m-te Bit gesetzt. Repräsentiert das Codebit ein Prüfbit, so wird in der Matrixzeile jedes Bit gesetzt, welches
            zur Berechnung des Prüfbits erforderlich ist. Die Generatormatix hat also für jedes Codebit eine Zeile, für
            jedes Datenbit eine Spalte.

            Die \textbf{Prüfmatrix H} bildet man anhand der farbigen Punkte (Prüfgruppen). Dabei werden die Zeilen in
            umgekehrter Reihenfolge (von unten nach oben) in der Matrix angegeben, ein Punkt bedeutet eine 1,
            kein Punkt eine 0.

            $$H = \begin{pmatrix}
            1 & 0 & 1 & 0 & 1 & 0 & \cdots \\ %& 1 & 0 & 1 & 0 & 1 & 0 & 1 & 0 & 1 & 0 & 1 & 0 & 1 & 0 & 1 & 0 & 1 & 0 & 1\\
            0 & 1 & 1 & 0 & 0 & 1 & \cdots \\ %& 1 & 0 & 0 & 1 & 1 & 0 & 0 & 1 & 1 & 0 & 0 & 1 & 1 & 0 & 0 & 1 & 1 & 0 & 0\\
            0 & 0 & 0 & 1 & 1 & 1 & \cdots \\ %& 1 & 0 & 0 & 0 & 0 & 1 & 1 & 1 & 1 & 0 & 0 & 0 & 0 & 1 & 1 & 1 & 1 & 0 & 0\\
            0 & 0 & 0 & 0 & 0 & 0 & \cdots \\ %& 0 & 1 & 1 & 1 & 1 & 1 & 1 & 1 & 1 & 0 & 0 & 0 & 0 & 0 & 0 & 0 & 0 & 1 & 1\\
            0 & 0 & 0 & 0 & 0 & 0 & \cdots \\ %& 0 & 0 & 0 & 0 & 0 & 0 & 0 & 0 & 0 & 1 & 1 & 1 & 1 & 1 & 1 & 1 & 1 & 1 & 1\\
            \end{pmatrix}$$
    \end{enumerate}
\end{enumerate}

\end{document}
