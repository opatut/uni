\newcommand{\authorinfo}{Paul Bienkowski, Hans Ole Hatzel}
\newcommand{\titleinfo}{RS 11 (HA) zum 18.01.2013}

% PREAMBLE ===============================================================

\documentclass[a4paper,10pt]{scrartcl}
\usepackage[german,ngerman]{babel}
\usepackage[utf8]{inputenc}
\usepackage[T1]{fontenc}
\usepackage{lmodern}
\usepackage{amssymb}
\usepackage{mathtools}
\usepackage{amsmath}
\usepackage{enumerate}
\usepackage{pgffor}
\usepackage{array}
\usepackage{listings}
\usepackage{fullpage}
\usepackage[usenames,dvipsnames]{xcolor}
\usepackage{tikz}
\usepackage{circuitikz}
\usepackage{fancyhdr}
\usepackage{lastpage}
\usepackage{multicol}\usepackage{multicol}

\usetikzlibrary{calc,decorations.markings}

\author{\authorinfo}
\title{\titleinfo}
\date{\today}

\pagestyle{fancy}
\fancyhf{}
\fancyhead[L]{\authorinfo}
\fancyhead[R]{\titleinfo}
\fancyfoot[C]{\thepage}
\renewcommand{\headrulewidth}{0.4pt}
\renewcommand{\footrulewidth}{0pt}
\renewcommand{\headheight}{12pt}
\renewcommand{\headsep}{12pt}

\newcommand*{\oline}[1]{\overline{\vphantom{A}#1}}

\begin{document}
\setcounter{secnumdepth}{0}
\maketitle

% DOCUMENT ===============================================================

\begin{enumerate}
    \item[\textbf{1.}]
        \begin{multicols}{5}
        \begin{enumerate}
            \item[a)] 30
            \item[b)] 40
            \item[c)] 50
            \item[d)] 40
            \item[e)] 40
        \end{enumerate}
        \end{multicols}
    \item[\textbf{2.}]
        Es gibt 3 Klassen von Befehlen, siehe Aufgabe, die jeweils voneinander
        unterscheidbar sein müssen.

        \begin{tikzpicture}[scale=0.4]
            \node[] (class1) at (-3,  0.4)  {\vphantom{Ag}Klasse 1:};
            \node[] (class2) at (-3, -1.6)  {\vphantom{Ag}Klasse 2:};
            \node[] (class3) at (-3, -3.6)  {\vphantom{Ag}Klasse 3:};


            \definecolor{awesomegray}{gray}{0.9}
            \foreach \n in {0,...,15}{
                \fill[fill=awesomegray] (\n*2,0) rectangle (\n*2+1, 1);
                \fill[fill=awesomegray] (\n*2,-2) rectangle (\n*2+1, -1);
                \fill[fill=awesomegray] (\n*2,-4) rectangle (\n*2+1, -3);
            }

            \draw
                (0, 0) rectangle (32, 1)
                (3, 0) -- (3, 1)
                (8, 0) -- (8, 1)

                (0, -2) rectangle (32, -1)
                (3, -2) -- (3, -1)
                (10, -2) -- (10, -1)
                (15, -2) -- (15, -1)
                (20, -2) -- (20, -1)

                (0, -4) rectangle (32, -3)
                (3, -4) -- (3, -3)
                (10, -4) -- (10, -3)
                (15, -4) -- (15, -3);

            \node[font=\small] (1opc1)   at (1.5, 0.4)  {\vphantom{Ag}opc1};
            \node[font=\small] (1reg)    at (5.5, 0.4)  {\vphantom{Ag}reg};
            \node[font=\small] (1addr24) at (20, 0.4)   {\vphantom{Ag}addr24};

            \node[font=\small] (2opc1)   at (1.5, -1.6)  {\vphantom{Ag}111};
            \node[font=\small] (2opc2)   at (6.5, -1.6)  {\vphantom{Ag}opc2};
            \node[font=\small] (2reg1)   at (12.5, -1.6) {\vphantom{Ag}regA};
            \node[font=\small] (2reg2)   at (17.5, -1.6) {\vphantom{Ag}regB};
            \node[font=\small] (2addr13) at (26, -1.6)   {\vphantom{Ag}addr13};

            \node[font=\small] (3opc1)   at (1.5, -3.6)  {\vphantom{Ag}111};
            \node[font=\small] (3opc2)   at (6.5, -3.6)  {\vphantom{Ag}1111111};
            \node[font=\small] (3opc3)   at (12.5, -3.6) {\vphantom{Ag}opc3};
            \node[font=\small] (3unused) at (23.5, -3.6) {\vphantom{Ag}unused};
        \end{tikzpicture}

        Um alle benötigten 32 bit für Klasse 1 (3 bit Opcode + 5 bit Register + 24 bit Adresse)
        unterbringen zu können, muss für den Opcode ein fester Platz vorgesehen sein. Um die
        anderen Klassen unterscheiden zu können, werden hier die ersten 3 Bits auf 1 gesetzt,
        da dies der einzige fehlende Opcode in der ersten Klasse ist. Ähnliches gilt
        für die Unterscheidung der Klassen 2 und 3, jedoch könnte hier jeder andere
        freie Opcode aus Klasse 2 verwendet werden. Der Übersichtlichkeit halber
        wurde ebenfalls 1111111 gewählt.

        Da für Klasse 2 zuerst 3 bit zur Unterscheidung von Klasse 1 nötig sind,
        7 bit für den Opcode ($2^7 = 128 \leq 100$) und $2 \times 5$ bit für
        die Register verwendet werden, bleiben $32 \text{ bit} - (3 + 7 + 2 \cdot 5) \text{ bit}
        = 12 \text{ bit}$ für das Adressoffset.


    \item[\textbf{3.}]
        \begin{enumerate}
            \item[a)] \texttt{0000 1011 1001}.
            \item[b)] Nicht möglich, da der Abstand vom ersten 1-bit bis zum letzten 1-bit im Ergebnis 9 Stellen ist, und dies nicht im \emph{imm8} dargestellt werden kann.
            \item[c)] Nicht möglich, da eine Rotation um 29 Stellen erfordert wäre.
            \item[d)] \texttt{1110 0110 0011}.
            \item[e)] \texttt{0010 0000 1001} \emph{oder} \texttt{0011 0010 0100} \emph{oder} \texttt{0100 1001 0000}.
        \end{enumerate}
    \item[\textbf{4.}]
        \begin{enumerate}
            \item[a)]
                \textbf{0-Adress-Maschine:}

                \begin{verbatim}Befehl         Stack (TOS ist links)

PUSH E          E
PUSH D          D ; E
MUL             D * E
PUSH F          F ; D * E
ADD             F + D * E
PUSH C          C ; F + D * E
PUSH B          B ; C ; F + D * E
MUL             B * C ; F + D * E
PUSH A          A ; B * C ; F + D * E
SUB             A - B * C ; F + D * E
DIV             (A - B * C) / (F + D * E)
POP R           R = (A - B * C) / (F + D * E)
\end{verbatim}

                \textbf{1-Adress-Maschine:}

                \begin{verbatim}Befehl         Akkumulator/Bedeutung

LOAD D          Akku = D
MUL E           Akku = D * E
ADD F           Akku = D * E + F
STORE G         Akku = G = D * E + F
LOAD B          Akku = B
MUL C           Akku = B * C
STORE H         MEM[H] = B * C
LOAD A          Akku = A
SUB H           Akku = A - H = A - B * C
DIV G           Akku = (A - B * C) / G = (A - B * C) / (D * E + F)
STORE R         MEM[R] = (A - B * C) / (D * E + F)
\end{verbatim}

                \textbf{2-Adress-Maschine:}

                \begin{verbatim}Befehl         Bedeutung

MUL D, E        MEM[D] = MEM[D] * MEM[E] = D * E
ADD D, F        MEM[D] = MEM[D] + MEM[F] = D * E + F
MUL B, C        MEM[B] = MEM[B] * MEM[C]
SUB A, B        MEM[A] = MEM[A] - MEM[B] = A - B * C
DIV A, D        MEM[A] = MEM[A] / MEM[D] = (A - B * C) / (D * E + F)
MOV R, A        MEM[R] = MEM[A] = (A - B * C) / (D * E + F)
\end{verbatim}

                \textbf{3-Adress-Maschine:}

                \begin{verbatim}Befehl          Bedeutung

LOAD Y, D       Y = MEM[D]
LOAD Z, E       Z = MEM[E]
MUL Z, Y, Z     Z = Y * Z = D * E
LOAD Y, F       Y = MEM[F]
ADD Z, Z, Y     Z = Z + Y = D * E + F
LOAD X, D       X = MEM[B]
LOAD Y, E       Y = MEM[C]
MUL Y, X, Y     Y = X * Y = B * C
LOAD X, A       X = MEM[A]
SUB X, X, Y     X = X - Y = A - B * C
DIV X, X, Z     X = X / Z = (A - B * C) / (D * E + F)
STORE R, X      MEM[R] = X
\end{verbatim}

            \item[b)]
                \begin{tabular}[t]{l|r|r|r|r}
                Maschine & Opcodes & Registernummern & Speicheradressen & Größe \\
                \hline
                \textbf{0-Adress} & 12 $\times$ 8 bit & 0  $\times$ 4 bit & 7  $\times$ 16 bit & 208 bit \\
                \textbf{1-Adress} & 11 $\times$ 8 bit & 0  $\times$ 4 bit & 11 $\times$ 16 bit & 264 bit \\
                \textbf{2-Adress} & 6  $\times$ 8 bit & 0  $\times$ 4 bit & 12 $\times$ 16 bit & 240 bit \\
                \textbf{3-Adress} & 12 $\times$ 8 bit & 22 $\times$ 4 bit & 7  $\times$ 16 bit & 296 bit \\
              \end{tabular}

        \end{enumerate}
\end{enumerate}
\end{document}
