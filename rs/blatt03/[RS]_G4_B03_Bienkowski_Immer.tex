\documentclass[a4paper,10pt]{scrartcl}
\usepackage[german,ngerman]{babel}
\usepackage[utf8]{inputenc}
\usepackage[T1]{fontenc}
\usepackage{lmodern}
\usepackage{amssymb}
\usepackage{mathtools}
\usepackage{amsmath}
\usepackage{enumerate}
\usepackage{array}
\usepackage{breqn}
\usepackage{fullpage}

\author{Paul Bienkowski, Vincent Immer}
\title{RS 03 (HA) zum 09.11.2012}
\date{\today}
\begin{document}
\setcounter{secnumdepth}{0}
\maketitle

% Stuff for manual multiplication/division/addition/...
\newlength{\digitwidth}
\newcommand\wmrule[2]{\noalign{\moveright#1\digitwidth\vbox{\hrule width#2\digitwidth\vspace{1pt}}}}
\newenvironment{wmcalc}
  {\vspace{0.6em}\begin{center}\settowidth\digitwidth{0}\setlength{\tabcolsep}{\digitwidth}\def~{\hspace{\digitwidth}}\begin{tabular}{r@{}}}
  {\end{tabular}\end{center}\vspace{0.6em}}

\begin{enumerate}
    \item[\textbf{1.}]
        \begin{enumerate}
            \item[a)]
                $1385 - 532 \Rightarrow 1385 + K_{10}(532) \Rightarrow 1385 + 9468 \Rightarrow [1]0853 \Rightarrow 0853 \Rightarrow 853$
                
                \begin{wmcalc}
                    ~~~1385\\
                    ~+~9368\\
                    \wmrule{1}{8}
                    \wmrule{1}{8}
                    ~1~$\underbracket[1pt]{0853}$
                \end{wmcalc}              
                
            \item[b)]
                $372 - 687 \Rightarrow 372 + K_{10}(687) \Rightarrow 0372 + 9313 \Rightarrow 9685 \overset{(\star)}{\Rightarrow} -K_{10}(9685) = -315$
                
                \begin{wmcalc}
                    ~~~0372\\
                    ~+~9313\\
                    \wmrule{1}{8}
                    \wmrule{1}{8}
                    ~~~9685
                \end{wmcalc}              
                
                $(\star)$: Da eine negative Zahl erwartet wird, muss aus dem Ergebnis der Addition $(9685)$ das 10er-Komplement gebildet und als negative Zahl
                interpretiert werden $(-315)$.
                
            \item[c)]
                $1385 = 101\;0110\;1001_2$ und $532 = 10\;0001\;0100_2$.
                
                \begin{dmath*}
                    K_2(10\;0001\;0100_2) 
                    = K_1(10\;0001\;0100_2) + 0001_2 
                    = 1101\;1110\;1011_2 + 0001_2 
                    = 1101\;1110\;1100_2
                \end{dmath*}
                
                \begin{wmcalc}
                    ~~0101~0110~1001\\
                    +~1101~1110~1100\\
                    \wmrule{1}{17}
                    1~1111~11~1~~~~~~\\
                    \wmrule{1}{17}
                    \wmrule{1}{17}
                    1~$\underbracket[1pt]{0011~0101~0101}$
                \end{wmcalc}
                
                $11\;0101\;0101_2 = 853$
            
            \item[d)]
                $372 = 1\;0111\;0100_2$ und $687 = 10\;1010\;1111_2$.
                
                \begin{dmath*}
                    K_2(10\;1010\;1111_2) 
                    = K_1(10\;1010\;1111_2) + 0001_2 
                    = 1101\;0101\;0000_2 + 0001_2 
                    = 1101\;0101\;0001_2
                \end{dmath*}
                
                \begin{wmcalc}
                    ~~0001~0111~0100\\
                    +~1101~0101~0001\\
                    \wmrule{1}{17}
                    ~~~1~~111~~~~~~~\\
                    \wmrule{1}{17}
                    \wmrule{1}{17}
                    ~~1110~1100~0101
                \end{wmcalc}

                Das Ergebnis muss wie in (b) als negative Zahl interpretiert werden, daher ist das 2er-Komplement zu bilden:
                
                \begin{dmath*}
                    K_2(1110\;1100\;0101_2)
                    = K_1(1110\;1100\;0101_2) + 0001_2
                    = 0001\;0011\;1010_2 + 0001_2
                    = 0001\;0011\;1011_2
                    = 315_{10}
                \end{dmath*}
                
                Das Ergebnis ist demnach $-315$.
        \end{enumerate}
    
    \item[\textbf{2.}]
        \begin{enumerate}
            \item[a)]
                $(6.9242 \mid 4)_{10}$
                
            \item[b)]
                $(-1.100101 \mid -010)_{2}$
                
            \item[c)]
                $(-2.\text{D}4\text{A} \mid B)_{16}$
                
        \end{enumerate}

    \item[\textbf{3.}]
        \begin{enumerate}
            \item[a)]
                $0 \mid 1000\;0101 \mid 0110\;1100\;0000\;0000\;0000\;000$
            
            \item[b)]
                $1 \mid 1000\;0110 \mid 0101\;0001\;0100\;0000\;0000\;000$
    
        \end{enumerate}
        
    \item[\textbf{4.}]
        \begin{enumerate}
            \item[a)]
                \begin{dmath*}
                    y 
                    = 8.626 \cdot 10^5 + 9.9442 \cdot 10^7
                    = 0.08626 \cdot 10^7 + 9.9442 \cdot 10^7
                    = (0.08626 + 9.9442) \cdot 10^7
                    = 10.03046 \cdot 10^7
                    = 1.003046 \cdot 10^8
                    \approx 1.0030 \cdot 10^8
                \end{dmath*}
            
            \item[b)]
                \begin{dmath*}
                    y 
                    = 8.626 \cdot 10^5 + 9.9442 \cdot 10^7
                    \approx 0.0863 \cdot 10^7 + 9.9442 \cdot 10^7
                    = (0.0863 + 9.9442) \cdot 10^7
                    = 10.030\textbf{5} \cdot 10^7
                    \approx  1.0031 \cdot 10^8
                \end{dmath*}

            \item[c)]
                Beim 2. Verfahren wird ein gerundetes Ergebnis erneut gerundet. Da die letzte entfernte Stelle zu
                einer 5 aufgerundet wurde, wird auch im 2. Rundungsschritt aufgerundet, was das Ergebnis verfälscht.
                Besser ist es daher, nach dem ersten Verfahren erst am Ende der Berechnungen zu runden.
            
        \end{enumerate}

    \item[\textbf{4.}]
        \begin{dmath*}
            y 
            = 5.6538 \cdot 10^7 \times 3.1415 \cdot 10^4
            = (5.6538 \times 3.1415) \cdot 10^{7 + 4}
            = 17.7614127 \cdot 10^{11}
            = 1.77614127 \cdot 10^{12}  \text{\hspace{2cm}(normalisiert)}
            \approx 1.7761 \cdot 10^{12}  \text{\hspace{2.7cm}(gerundet)}
        \end{dmath*}
    
    
\end{enumerate}

\end{document}
