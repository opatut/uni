\newcommand{\authorinfo}{Paul Bienkowski, Hans Ole Hatzel}
\newcommand{\titleinfo}{RS 09 (HA) zum 21.12.2012}

% PREAMBLE ===============================================================

\documentclass[a4paper,10pt]{scrartcl}
\usepackage[german,ngerman]{babel}
\usepackage[utf8]{inputenc}
\usepackage[T1]{fontenc}
\usepackage{lmodern}
\usepackage{amssymb}
\usepackage{mathtools}
\usepackage{amsmath}
\usepackage{enumerate}
\usepackage{array}
\usepackage{listings}
\usepackage{fullpage}
\usepackage[usenames,dvipsnames]{xcolor}
\usepackage{tikz}
\usepackage{circuitikz}
\usepackage{fancyhdr}
\usepackage{lastpage}

\usetikzlibrary{calc,decorations.markings}

\author{\authorinfo}
\title{\titleinfo}
\date{\today}

\pagestyle{fancy}
\fancyhf{}
\fancyhead[L]{\authorinfo}
\fancyhead[R]{\titleinfo}
\fancyfoot[C]{\thepage}
\renewcommand{\headrulewidth}{0.4pt}
\renewcommand{\footrulewidth}{0pt}
\renewcommand{\headheight}{12pt}
\renewcommand{\headsep}{12pt}

\newcommand*{\oline}[1]{\overline{\vphantom{A}#1}}

\begin{document}
\setcounter{secnumdepth}{0}
\maketitle

% DOCUMENT ===============================================================

\definecolor{lightblue}{RGB}{120, 140, 255}
\definecolor{pink}{RGB}{255, 0, 255}
\tikzstyle{iogrid}=[step=1,draw=lightblue]
\tikzstyle{ioaxis}=[draw=gray,thin]
\tikzstyle{ioline}=[draw=black,very thick]
\tikzstyle{ioline2}=[draw=Orange,very thick]
\tikzstyle{ioundf}=[fill=Cyan,draw=none,opacity=0.5]

\tikzstyle{kvloop}=[draw, opacity=1, line width=0.4mm, rounded corners=2mm]

\begin{enumerate}
    \item[\textbf{1.}]
    Flussdiagramm:

    \begin{tikzpicture}[x=0.5cm,y=0.5cm]
        \foreach \y/\l in {0/C,-2/D,-4/Q (D-Latch),-6/Q (D-FF)} {
            \draw[iogrid] (0, \y) grid (18,{1+\y});
            \draw[ioaxis] (0, \y) -- (0,{1.4 + \y})
                (-0.4, \y) -- (18.4, \y)
                (-0.4, {1 + \y}) -> (0, {1 + \y});

            \node[label=left:\tiny0] at (0, \y) {};
            \node[label=left:\tiny1] at (0, {\y + 1}) {};

            \node[label=left:\small\l] at (-1, {\y + 0.5}) {};
        }

        % C input
        \draw[ioline] (0, 0) -- (2, 0) -- (2, 1) -- (3, 1) -- (3, 0) -- (5, 0) -- (5, 1)
            -- (6, 1) -- (6, 0) -- (8, 0) -- (8, 1) -- (10, 1) -- (10, 0) -- (12, 0) -- (12, 1)
            -- (15, 1) -- (15, 0) -- (16, 0) -- (16, 1) -- (18, 1);

        % D input
        \draw[ioline] (0, -2) -- (1, -2) -- (1, -1) -- (4, -1) -- (4, -2) -- (7, -2) -- (7, -1)
            -- (9, -1) -- (9, -2) -- (11, -2) -- (11, -1) -- (12, -1) -- (12, -2) -- (14, -2)
            -- (14, -1) -- (17, -1) -- (17, -2) -- (18, -2);

        % D-Latch
        \draw[ioline2] (0, -4) -- (3, -4) -- (3, -3) -- (6, -3) -- (6, -4) -- (9, -4)
            -- (9, -3) -- (10, -3) -- (10, -4) -- (17, -4) -- (17, -3) -- (18, -3);

        % D-Flipflip
        \draw[ioline2] (0, -6) -- (3, -6) -- (3, -5) -- (6, -5) -- (6, -6) -- (9, -6)
            -- (9, -5) -- (13, -5) (17,-5) -- (18, -5);
        \draw[ioundf] (13, -5) rectangle (17, -6);

    \end{tikzpicture}
    \vspace{1em}

    \item[\textbf{2.}]
        \begin{enumerate}
            \item[a)]
                \begin{minipage}[t]{0.4\textwidth}
                Flipflop mit Multiplexer:\\

                    \begin{tabular}{|c|c|c||c|}
                        \hline
                        D & E & CLK & $Q^+$ \\ \hline
                        * & * & 0   & $Q$ \\
                        * & * & 1   & $Q$ \\
                        * & 0 & $\uparrow$ & $Q$ \\
                        * & 1 & $\uparrow$ & D \\
                        \hline
                    \end{tabular}

                \end{minipage}
                \begin{minipage}[t]{0.4\textwidth}
                Flipflop mit Taktausblendung:\\

                    \begin{tabular}{|c|c|c||c|}
                        \hline
                        D & E & CLK & $Q^+$ \\ \hline
                        * & * & 0   & $Q$ \\
                        * & 0 & *   & $Q$ \\
                        * & 1 & $\uparrow$ & D \\
                        * & $\uparrow$ & 1 & D \\
                        \hline
                    \end{tabular}
                \end{minipage}
                \vspace{1em}

            \item[b)]
                Solche Schaltungen werden in einem synchronen System wie etwa einer CPU als Buffer eingesetzt.
            \item[c)]
                Schaltung 2 speichert auch bei Vorderflanke auf dem Enable-Eingang (E) falls der Clock-Eingang (C) aktiv ist.Das umgeht die Synchronisation über den Clock-Eingang während einer Taktphase.

                Vorteil von Schalltung 2 ist, das weniger Bauelemente (AND-Gatter statt Multiplexer) benötigt werden.
                Außerdem bietet die zweite Schaltung ein einfacheres Zeitverhalten, da das Ausgabesignal (Q) nicht als Eingang für den Multiplexer verwendet wird.

                Schaltung 1 hat zudem eine höhere Vorlaufzeit da der Multiplexer das Datensignal verzögert. Im Gegensatz dazu hat Schaltung eine höhere Haltezeit, da hier der Takt durch das AND-Gatter minimal verzögert wird.
        \end{enumerate}

    \newpage
    \item[\textbf{3.}]
        \begin{enumerate}
            \item[a)]
            Zustandsdiagramm des Ampel-Automaten:

            \tikzstyle{n}=[circle,inner sep=0,minimum size=1.7em,draw]
            \tikzstyle{->-}=[decoration={markings, mark=at position #1 with {
                \fill (1.2pt,0)--(-1.2pt,1.5pt)--(-1.2pt,-1.5pt)--cycle;
            }},postaction={decorate}]
            \tikzstyle{a}=[->-=0.5]
            \tikzstyle{add}=[color=gray]
            \tikzstyle{ref}=[out=#1-40, in=#1+40, distance=2.5em]

            \hspace{4.15cm}\begin{tikzpicture}[x=1.75em,y=1.75em]
                \node[n] (Z0) at (0, 3) {$Z_0$};
                \node[n] (Z1) at (2, 2) {$Z_1$};
                \node[n] (Z2) at (3, 0) {$Z_2$};
                \node[n] (Z3) at (2,-2) {$Z_3$};
                \node[n] (Z4) at (0,-3) {$Z_4$};
                \node[n] (Z5) at (-2,-2) {$Z_5$};
                \node[n] (Z6) at (-3,0) {$Z_6$};
                \node[n] (Z7) at (-2, 2) {$Z_7$};

                \draw[a] (Z0) -> (Z1);
                \draw[a] (Z1) -> (Z2);
                \draw[a] (Z2) -> node[label={left:\tiny{i=1}}] {} (Z3);

                \node[label={right:\tiny{i=0}}] at ($(Z2)+(0.1,-0.9)$) {};
                \path (Z2) edge[a, ref=0] (Z2);

                \draw[a] (Z3) -> (Z4);
                \draw[a] (Z4) -> (Z5);
                \draw[a] (Z5) -> (Z6);
                \draw[a] (Z6) -> (Z7);
                \draw[a] (Z7) -> (Z0);
            \end{tikzpicture}

            \item[b)]
            Zustandstabelle:

            \begin{tabular}[t]{|c|ccc||ccc|ccc|ccc|}
                \hline
                $i$ & $z_2$ & $z_1$ & $z_0$ & $z_2^+$ & $z_1^+$ & $z_0^+$ & $rt_H$ & $ge_H$ & $gr_H$ & $rt_N$ & $ge_N$ & $gr_N$\\
                \hline
                * &     0 & 0 & 0 &     0 & 0 & 1 &     1 & 0 & 0 &     1 & 0 & 0 \\
                * &     0 & 0 & 1 &     0 & 1 & 0 &     1 & 1 & 0 &     1 & 0 & 0 \\
                0 &     0 & 1 & 0 &     0 & 1 & 0 &     0 & 0 & 1 &     1 & 0 & 0 \\
                1 &     0 & 1 & 0 &     0 & 1 & 1 &     0 & 0 & 1 &     1 & 0 & 0 \\
                * &     0 & 1 & 1 &     1 & 0 & 0 &     0 & 1 & 0 &     1 & 0 & 0 \\
                * &     1 & 0 & 0 &     1 & 0 & 1 &     1 & 0 & 0 &     1 & 0 & 0 \\
                * &     1 & 0 & 1 &     1 & 1 & 0 &     1 & 0 & 0 &     1 & 1 & 0 \\
                * &     1 & 1 & 0 &     1 & 1 & 1 &     1 & 0 & 0 &     0 & 0 & 1 \\
                * &     1 & 1 & 1 &     0 & 0 & 0 &     1 & 0 & 0 &     0 & 1 & 0 \\
                \hline
            \end{tabular}
            \vspace{1em}

            \item[c)]

                KV-Diagramme für die Folgezustands-Codierung:

                \begin{minipage}[t]{0.3\textwidth}
                    \centering
                    \begin{tikzpicture}[x=7mm, y=7mm, font=\sffamily\small, label distance=-1.5mm]
                        \draw[step=1] (0, 0) grid (4,4);

                        \draw (1, 4.3) -- (3, 4.3) node[midway,label=above:$z_0$] (z0) {};
                        \draw (2, -0.3) -- (4, -0.3) node[midway,label=below:$z_1$] (z1) {};
                        \draw (-0.3, 1) -- (-0.3, 3) node[midway,label=left:$z_2$] (z2) {};
                        \draw (4.3, 0) -- (4.3, 2) node[midway,label=right:$i$] (i) {};

                        \node[] at (0.5, 0.5) {0};
                        \node[] at (1.5, 0.5) {0};
                        \node[] at (2.5, 0.5) {1};
                        \node[] at (3.5, 0.5) {0};
                        \node[] at (0.5, 1.5) {1};
                        \node[] at (1.5, 1.5) {1};
                        \node[] at (2.5, 1.5) {0};
                        \node[] at (3.5, 1.5) {1};
                        \node[] at (0.5, 2.5) {1};
                        \node[] at (1.5, 2.5) {1};
                        \node[] at (2.5, 2.5) {0};
                        \node[] at (3.5, 2.5) {1};
                        \node[] at (0.5, 3.5) {0};
                        \node[] at (1.5, 3.5) {0};
                        \node[] at (2.5, 3.5) {1};
                        \node[] at (3.5, 3.5) {0};

                        \draw[kvloop, draw=Red] (0.2, 1.2) rectangle (1.8, 2.8);

                        \draw[kvloop, draw=LimeGreen]
                            (0, 1.1) -- (0.9, 1.1) -- (0.9, 2.9) -- (0, 2.9)
                            (4, 1.1) -- (3.1, 1.1) -- (3.1, 2.9) -- (4, 2.9);

                        \draw[kvloop, draw=Cyan]
                            (2.1, 0) -- (2.1, 0.9) -- (2.9, 0.9) -- (2.9, 0)
                            (2.1, 4) -- (2.1, 3.1) -- (2.9, 3.1) -- (2.9, 4);
                    \end{tikzpicture}

                    $z_2^+ = z_2\oline{z_1} \vee z_2\oline{z_0} \vee \oline{z_2}z_1z_0$
                \end{minipage}
                \begin{minipage}[t]{0.3\textwidth}
                    \centering
                    \begin{tikzpicture}[x=7mm, y=7mm, font=\sffamily\small, label distance=-1.5mm]
                        \draw[step=1] (0, 0) grid (4,4);

                        \draw (1, 4.3) -- (3, 4.3) node[midway,label=above:$z_0$] (z0) {};
                        \draw (2, -0.3) -- (4, -0.3) node[midway,label=below:$z_1$] (z1) {};
                        \draw (-0.3, 1) -- (-0.3, 3) node[midway,label=left:$z_2$] (z2) {};
                        \draw (4.3, 0) -- (4.3, 2) node[midway,label=right:$i$] (i) {};

                        \node[] at (0.5, 0.5) {0};
                        \node[] at (1.5, 0.5) {1};
                        \node[] at (2.5, 0.5) {0};
                        \node[] at (3.5, 0.5) {1};
                        \node[] at (0.5, 1.5) {0};
                        \node[] at (1.5, 1.5) {1};
                        \node[] at (2.5, 1.5) {0};
                        \node[] at (3.5, 1.5) {1};
                        \node[] at (0.5, 2.5) {0};
                        \node[] at (1.5, 2.5) {1};
                        \node[] at (2.5, 2.5) {0};
                        \node[] at (3.5, 2.5) {1};
                        \node[] at (0.5, 3.5) {0};
                        \node[] at (1.5, 3.5) {1};
                        \node[] at (2.5, 3.5) {0};
                        \node[] at (3.5, 3.5) {1};

                        \draw[kvloop, draw=Dandelion] (1.2, 0.1) rectangle (1.8, 3.9);
                        \draw[kvloop, draw=LimeGreen] (3.2, 0.1) rectangle (3.8, 3.9);
                    \end{tikzpicture}

                    $z_1^+ = \oline{z_1}z_0 \vee z_1\oline{z_0}$
                \end{minipage}
                \begin{minipage}[t]{0.3\textwidth}
                    \centering
                    \begin{tikzpicture}[x=7mm, y=7mm, font=\sffamily\small, label distance=-1.5mm]
                        \draw[step=1] (0, 0) grid (4,4);

                        \draw (1, 4.3) -- (3, 4.3) node[midway,label=above:$z_0$] (z0) {};
                        \draw (2, -0.3) -- (4, -0.3) node[midway,label=below:$z_1$] (z1) {};
                        \draw (-0.3, 1) -- (-0.3, 3) node[midway,label=left:$z_2$] (z2) {};
                        \draw (4.3, 0) -- (4.3, 2) node[midway,label=right:$i$] (i) {};

                        \node[] at (0.5, 0.5) {1};
                        \node[] at (1.5, 0.5) {0};
                        \node[] at (2.5, 0.5) {0};
                        \node[] at (3.5, 0.5) {1};
                        \node[] at (0.5, 1.5) {1};
                        \node[] at (1.5, 1.5) {0};
                        \node[] at (2.5, 1.5) {0};
                        \node[] at (3.5, 1.5) {1};
                        \node[] at (0.5, 2.5) {1};
                        \node[] at (1.5, 2.5) {0};
                        \node[] at (2.5, 2.5) {0};
                        \node[] at (3.5, 2.5) {1};
                        \node[] at (0.5, 3.5) {1};
                        \node[] at (1.5, 3.5) {0};
                        \node[] at (2.5, 3.5) {0};
                        \node[] at (3.5, 3.5) {0};

                        \draw[kvloop, draw=Cyan] (0.2, 0.1) rectangle (0.8, 3.9);
                        \draw[kvloop, draw=Red] (3.1, 0.1) rectangle (3.9, 1.9);
                        \draw[kvloop, draw=LimeGreen] (3.2, 1.1) rectangle (3.8, 2.9);
                    \end{tikzpicture}

                    $z_0^+ = \oline{z_1}\;\oline{z_0} \vee z_1\oline{z_0}i \vee z_2z_1\oline{z_0}$
                \end{minipage}

                \vspace{1em}

                Die Ausgangsfunktionen für die Ampellichter kann man auch ohne Hilfe
                von KV-Diagrammen aus der Tabelle ablesen:

                $rt_H = z_2 \vee \oline{z_1}$ \\[0.5em]
                $ge_H = z_2 \vee \oline{z_0}$ \\[0.5em]
                $gr_H = \oline{z_2}z_1\oline{z_0}$ \\[0.5em]
                $rt_N = \oline{z_2} \vee \oline{z_1}$ \\[0.5em]
                $ge_N = \oline{z_2} \vee \oline{z_0}$ \\[0.5em]
                $gr_N = z_2z_1\oline{z_0}$\\[0.5em]

        \end{enumerate}
\end{enumerate}

\end{document}
