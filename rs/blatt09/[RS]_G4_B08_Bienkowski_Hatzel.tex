\newcommand{\authorinfo}{Paul Bienkowski, Hans Ole Hatzel}
\newcommand{\titleinfo}{RS 09 (HA) zum 21.12.2012}

% PREAMBLE ===============================================================

\documentclass[a4paper,10pt]{scrartcl}
\usepackage[german,ngerman]{babel}
\usepackage[utf8]{inputenc}
\usepackage[T1]{fontenc}
\usepackage{lmodern}
\usepackage{amssymb}
\usepackage{mathtools}
\usepackage{amsmath}
\usepackage{enumerate}
\usepackage{array}
\usepackage{listings}
\usepackage{fullpage}
\usepackage[usenames,dvipsnames]{xcolor}
\usepackage{tikz}
\usepackage{circuitikz}
\usepackage{fancyhdr}
\usepackage{lastpage}

\usetikzlibrary{calc}

\author{\authorinfo}
\title{\titleinfo}
\date{\today}

\pagestyle{fancy}
\fancyhf{}
\fancyhead[L]{\authorinfo}
\fancyhead[R]{\titleinfo}
\fancyfoot[C]{\thepage}
\renewcommand{\headrulewidth}{0.4pt}
\renewcommand{\footrulewidth}{0pt}
\renewcommand{\headheight}{12pt}
\renewcommand{\headsep}{12pt}

\begin{document}
\setcounter{secnumdepth}{0}
\maketitle

% DOCUMENT ===============================================================

\definecolor{lightblue}{RGB}{120, 140, 255}
\definecolor{pink}{RGB}{255, 0, 255}
\tikzstyle{iogrid}=[step=1,draw=lightblue]
\tikzstyle{ioaxis}=[draw=gray,thin]
\tikzstyle{ioline}=[draw=black,very thick]
\tikzstyle{ioline2}=[draw=Orange,very thick]
\tikzstyle{ioundf}=[fill=Cyan,draw=none,opacity=0.5]

\begin{enumerate}
    \item[\textbf{1.}]
    Flussdiagramm:

    \begin{tikzpicture}[x=0.5cm,y=0.5cm]
        \foreach \y/\l in {0/C,-2/D,-4/Q (D-Latch),-6/Q (D-FF)} {
            \draw[iogrid] (0, \y) grid (18,{1+\y});
            \draw[ioaxis] (0, \y) -- (0,{1.4 + \y})
                (-0.4, \y) -- (18.4, \y)
                (-0.4, {1 + \y}) -> (0, {1 + \y});

            \node[label=left:\tiny0] at (0, \y) {};
            \node[label=left:\tiny1] at (0, {\y + 1}) {};

            \node[label=left:\small\l] at (-1, {\y + 0.5}) {};
        }

        % C input
        \draw[ioline] (0, 0) -- (2, 0) -- (2, 1) -- (3, 1) -- (3, 0) -- (5, 0) -- (5, 1)
            -- (6, 1) -- (6, 0) -- (8, 0) -- (8, 1) -- (10, 1) -- (10, 0) -- (12, 0) -- (12, 1)
            -- (15, 1) -- (15, 0) -- (16, 0) -- (16, 1) -- (18, 1);

        % D input
        \draw[ioline] (0, -2) -- (1, -2) -- (1, -1) -- (4, -1) -- (4, -2) -- (7, -2) -- (7, -1)
            -- (9, -1) -- (9, -2) -- (11, -2) -- (11, -1) -- (12, -1) -- (12, -2) -- (14, -2)
            -- (14, -1) -- (17, -1) -- (17, -2) -- (18, -2);

        % D-Latch
        \draw[ioline2] (0, -4) -- (3, -4) -- (3, -3) -- (6, -3) -- (6, -4) -- (9, -4)
            -- (9, -3) -- (10, -3) -- (10, -4) -- (17, -4) -- (17, -3) -- (18, -3);

        % D-Flipflip
        \draw[ioline2] (0, -6) -- (3, -6) -- (3, -5) -- (6, -5) -- (6, -6) -- (9, -6)
            -- (9, -5) -- (13, -5) (17,-5) -- (18, -5);
        \draw[ioundf] (13, -5) rectangle (17, -6);

    \end{tikzpicture}

    \item[\textbf{2.}]
        \begin{enumerate}
            \item[a)]
                \begin{minipage}[t]{0.4\textwidth}
                Flipflop mit Multiplexer:\\ 

                    \begin{tabular}{|c|c|c||c|}
                        \hline
                        D & E & CLK & $Q^+$ \\ \hline
                        * & * & 0   & $Q$ \\
                        * & * & 1   & $Q$ \\
                        * & 0 & $\uparrow$ & $Q$ \\
                        * & 1 & $\uparrow$ & D \\
                        \hline
                    \end{tabular}

                \end{minipage}
                \begin{minipage}[t]{0.4\textwidth}
                Flipflop mit Taktausblendung:\\

                    \begin{tabular}{|c|c|c||c|}
                        \hline
                        D & E & CLK & $Q^+$ \\ \hline
                        * & * & 0   & $Q$ \\
                        * & 0 & *   & $Q$ \\
                        * & 1 & $\uparrow$ & D \\
                        * & $\uparrow$ & 1 & D \\
                        \hline
                    \end{tabular}
                \end{minipage}

            \item[b)]
                Solche Schaltungen werden in einem synchronen System wie etwa einer CPU als Buffer eingesetzt.
            \item[c)]
                Schaltung 2 speichert auch bei Vorderflanke auf dem Enable-Eingang (E) falls der Clock-Eingang (C) aktiv ist.Das umgeht die Synchronisation über den Clock-Eingang während einer Taktphase.

                Vorteil von Schalltung 2 ist, das weniger Bauelemente (AND-Gatter statt Multiplexer) benötigt werden.
                Außerdem bietet die zweite Schaltung ein einfacheres Zeitverhalten, da das Ausgabesignal (Q) nicht als Eingang für den Multiplexer verwendet wird.

                Schaltung 1 hat zudem eine höhere Vorlaufzeit da der Multiplexer das Datensignal verzögert. Im Gegensatz dazu hat Schaltung eine höhere Haltezeit, da hier der Takt durch das AND-Gatter minimal verzögert wird.
        \end{enumerate}

    \item[\textbf{3.}]
        \begin{enumerate}
            \item[a)]
            \item[b)]
            \item[c)]
        \end{enumerate}
\end{enumerate}

\end{document}
