\newcommand{\authorinfo}{Paul Bienkowski, Hans Ole Hatzel}
\newcommand{\titleinfo}{RS 06 (HA) zum 30.11.2012}

% PREAMBLE ===============================================================

\documentclass[a4paper,10pt]{scrartcl}
\usepackage[german,ngerman]{babel}
\usepackage[utf8]{inputenc}
\usepackage[T1]{fontenc}
\usepackage{lmodern}
\usepackage{amssymb}
\usepackage{mathtools}
\usepackage{amsmath}
\usepackage{enumerate}
\usepackage{array}
\usepackage{listings}
\usepackage{fullpage}
\usepackage[usenames,dvipsnames]{xcolor}
\usepackage{tikz}
\usepackage{pgffor}
\usepackage{fancyhdr}
\usepackage{lastpage}

\usetikzlibrary{arrows,shapes.gates.logic.US,shapes.gates.logic.IEC,calc,decorations.markings}

\author{\authorinfo}
\title{\titleinfo}
\date{\today}

\pagestyle{fancy}
\fancyhf{}
\fancyhead[L]{\authorinfo}
\fancyhead[R]{\titleinfo}
\fancyfoot[C]{\thepage}
\renewcommand{\headrulewidth}{0.4pt}
\renewcommand{\footrulewidth}{0pt}
\renewcommand{\headheight}{12pt}
\renewcommand{\headsep}{12pt}

\begin{document}
\setcounter{secnumdepth}{0}
\maketitle

% allow huuuuge matrix
\setcounter{MaxMatrixCols}{31}

% DOCUMENT ===============================================================

\tikzstyle{->-}=[decoration={markings, mark=at position #1 with {
    \fill (1.2pt,0)--(-1.2pt,1.5pt)--(-1.2pt,-1.5pt)--cycle;
}},postaction={decorate}]
\tikzstyle{->}=[decoration={markings, mark=at position #1 with {
    \fill (0,0)--(-2.4pt,1.5pt)--(-2.4pt,-1.5pt)--cycle;
}},postaction={decorate}]
\tikzstyle{kvloop}=[draw, opacity=1, line width=0.4mm, rounded corners=2mm]
\tikzstyle{final}=[draw=black,minimum width=6mm,minimum height=6mm,inner sep=0]
\tikzstyle{binode}=[final, circle, minimum size=6mm]
\tikzstyle{bi1branch}=[->=1,draw=black]
\tikzstyle{bi0branch}=[bi1branch,dashed]
\tikzstyle{branch}=[fill,shape=circle,minimum size=3pt,inner sep=0pt]

\newcommand*{\oline}[1]{\overline{\vphantom{A}#1}}

\begin{enumerate}
    \item[\textbf{1.}]
    \begin{enumerate}
        \item[a)]
            Die Funktion liegt bereits als KNF vor:

            $$\begin{array}{rclr}
                f(x) &=& (x_3 \vee \oline{x_2}) \wedge (x_2 \vee \oline{x_1}) & \text{(KNF)}\\
                &=& x_2 x_3 \vee \oline{x_1}\;\oline{x_2} & \text{(DNF)}\\
                &=& 1 \oplus x_1 \oplus x_2 \oplus x_1x_2 \oplus x_2x_3 & \text{(Reed-Muller-Form)}\\
            \end{array}$$

        \item[b)]
            Die Funktion liegt bereits nahezu als Reed-Muller-Form vor:

            $$\begin{array}{rclr}
                g(x) &=& x_3 \oplus x_1 & \text{(Reed-Muller-Form)}\\
                &=& \oline{x_3}x_1 \vee x_3\oline{x_1} & \text{(DNF)}\\
                &=& (x_3 \vee \oline{x_1}) \wedge (x_1 \vee \oline{x_3}) & \text{(KNF)}\\
            \end{array}$$

    \end{enumerate}

    \item[\textbf{2.}]
            Es sei $a \barwedge b$ die Schreibweise für (a NAND b).

    \begin{enumerate}
        \item[a)]
            Da $a \wedge a = a$ gilt, ist $a \barwedge a = \oline{a}$, also lässt sich die Negation von a durch NAND-Kombination
            von a mit sich selbst bilden. Wahrheitstafel:

            \hspace{1cm}\begin{tabular}[t]{c|c||c|c}
                $a$ & $a \wedge a$ & $a \barwedge a$ & $\oline{a}$ \\
                \hline
                0 & 0 & 1 & 1 \\
                1 & 1 & 0 & 0
            \end{tabular}

            Um AND zu erreichen, kann das Ergebnis von NAND einfach negiert werden (siehe oben). Dann gilt
            $a \wedge b = \oline{a \barwedge b} = (a \barwedge b) \barwedge (a \barwedge b)$. Wahrheitstafel:

            \hspace{1cm}\begin{tabular}[t]{c|c||c||c|c}
                $a$ & $b$ & $a \wedge b$ & $a \barwedge b$ & $(a \barwedge b) \barwedge (a \barwedge b)$  \\
                \hline
                0 & 0 & 0 & 1 & 0 \\
                0 & 1 & 0 & 1 & 0 \\
                1 & 0 & 0 & 1 & 0 \\
                1 & 1 & 1 & 0 & 1
            \end{tabular}

            Nach de Morgan gilt $\oline{a \vee b} = \oline{a} \wedge \oline{b}$. Dies
            lässt sich umformen zu $a \vee b = \oline{a} \barwedge \oline{b}$. Die Negation von $a$ und $b$
            kann wie oben mit NAND dargestellt werden: $a \vee b = (a \barwedge a) \barwedge (b \barwedge b)$.

            \hspace{1cm}\begin{tabular}[t]{c|c||c||c|c|c}
                $a$ & $b$ & $a \vee b$ & $a \barwedge a = \oline{a}$ & $b \barwedge b = \oline{b}$ & $(a \barwedge a) \barwedge (b \barwedge b)$  \\
                \hline
                0 & 0 & 0 & 1 & 1 & 0 \\
                0 & 1 & 1 & 1 & 0 & 1 \\
                1 & 0 & 1 & 0 & 1 & 1 \\
                1 & 1 & 1 & 0 & 0 & 1
            \end{tabular}

        \item[b)]
            $$\begin{array}{rcl}
                f(x_3, x_2, x_1)
                &=& (\oline{x_3}(\oline{x_2} \vee x_1)) \vee (x_1 ( \oline{x_2} \vee x_1 ) )\\
                &=& (\oline{x_2} \vee x_1) \wedge (\oline{x_3} \vee x_1)\\
                &=& x_1 \wedge (\oline{x_2} \vee \oline{x_3})\\
                &=& x_1 \wedge (x_2 \barwedge x_3)\\
                &=& \big(x_1 \barwedge (x_2 \barwedge x_3)\big) \barwedge \big(x_1 \barwedge (x_2 \barwedge x_3)\big)
            \end{array}$$

    \end{enumerate}

    \newpage
    \item[\textbf{3.}]
    \begin{enumerate}
    \item[a)]
        \begin{minipage}[t]{0.4\textwidth}
        Funktionstabelle für A und B:

            \begin{tabular}[t]{c||c|c|c|c||c|c}
                $x$ & $x_4$ & $x_3$ & $x_2$ & $x_1$ & $A$ & $B$\\
                \hline
                0 & 0 & 0 & 0 & 0 & 1 & 1 \\
                1 & 0 & 0 & 0 & 1 & 0 & 1 \\
                2 & 0 & 0 & 1 & 0 & 1 & 1 \\
                3 & 0 & 0 & 1 & 1 & 1 & 1 \\
                4 & 0 & 1 & 0 & 0 & 0 & 1 \\
                5 & 0 & 1 & 0 & 1 & 1 & 0 \\
                6 & 0 & 1 & 1 & 0 & 1 & 0 \\
                7 & 0 & 1 & 1 & 1 & 1 & 1 \\
                8 & 1 & 0 & 0 & 0 & 1 & 1 \\
                9 & 1 & 0 & 0 & 1 & 1 & 1
            \end{tabular}

            \vspace{2em}
            $A(x) = \textcolor{Red}{x_3} \vee \textcolor{LimeGreen}{x_1} \vee \textcolor{Cyan}{x_2x_0} \vee \textcolor{Dandelion}{\oline{x_2}\;\oline{x_0}}$\\
            $B(x) = \textcolor{Dandelion}{\oline{x_2}} \vee \textcolor{Cyan}{\oline{x_1}\;\oline{x_0}} \vee \textcolor{LimeGreen}{x_1x_0}$
        \end{minipage}
        \hspace{1cm}
        \begin{minipage}[t]{0.4\textwidth}
        \item[b)]
            Karnaugh-Veitch-Diagramme:

                \begin{tikzpicture}[x=7mm, y=7mm, font=\sffamily\small, label distance=-1.5mm]
                    \draw[step=1] (0, 0) grid (4,4);

                    \draw (1, 4.3) -- (3, 4.3) node[midway,label=above:x0] (x0) {};
                    \draw (2, -0.3) -- (4, -0.3) node[midway,label=below:x1] (x1) {};
                    \draw (-0.3, 1) -- (-0.3, 3) node[midway,label=left:x2] (x2) {};
                    \draw (4.3, 0) -- (4.3, 2) node[midway,label=right:x3] (x3) {};

                    \node[] at (0.5, 0.5) {1};
                    \node[] at (1.5, 0.5) {1};
                    \node[] at (2.5, 0.5) {X};
                    \node[] at (3.5, 0.5) {X};
                    \node[] at (0.5, 1.5) {X};
                    \node[] at (1.5, 1.5) {X};
                    \node[] at (2.5, 1.5) {X};
                    \node[] at (3.5, 1.5) {X};
                    \node[] at (0.5, 2.5) {0};
                    \node[] at (1.5, 2.5) {1};
                    \node[] at (2.5, 2.5) {1};
                    \node[] at (3.5, 2.5) {1};
                    \node[] at (0.5, 3.5) {1};
                    \node[] at (1.5, 3.5) {0};
                    \node[] at (2.5, 3.5) {1};
                    \node[] at (3.5, 3.5) {1};

                    \draw[kvloop, draw=Red] (0.1, 0.1) rectangle (3.9, 1.9);
                    \draw[kvloop, draw=LimeGreen] (2.2, 0.2) rectangle (3.8, 3.8);0.6
                    \draw[kvloop, draw=Cyan] (1.1, 1.1) rectangle (2.9, 2.9);

                    \draw[kvloop, draw=Dandelion]
                        (0, 3.1) -- (0.9, 3.1) -- (0.9, 4)
                        (0, 0.9) -- (0.9, 0.9) -- (0.9, 0)
                        (3.1, 0) -- (3.1, 0.9) -- (4, 0.9)
                        (3.1, 4) -- (3.1, 3.1) -- (4, 3.1);
                \end{tikzpicture}

                \begin{tikzpicture}[x=7mm, y=7mm, font=\sffamily\small, label distance=-1.5mm]
                    \draw[step=1] (0, 0) grid (4,4);

                    \draw (1, 4.3) -- (3, 4.3) node[midway,label=above:x0] (x0) {};
                    \draw (2, -0.3) -- (4, -0.3) node[midway,label=below:x1] (x1) {};
                    \draw (-0.3, 1) -- (-0.3, 3) node[midway,label=left:x2] (x2) {};
                    \draw (4.3, 0) -- (4.3, 2) node[midway,label=right:x3] (x3) {};

                    \node[] at (0.5, 0.5) {1};
                    \node[] at (1.5, 0.5) {1};
                    \node[] at (2.5, 0.5) {X};
                    \node[] at (3.5, 0.5) {X};
                    \node[] at (0.5, 1.5) {X};
                    \node[] at (1.5, 1.5) {X};
                    \node[] at (2.5, 1.5) {X};
                    \node[] at (3.5, 1.5) {X};
                    \node[] at (0.5, 2.5) {1};
                    \node[] at (1.5, 2.5) {0};
                    \node[] at (2.5, 2.5) {1};
                    \node[] at (3.5, 2.5) {0};
                    \node[] at (0.5, 3.5) {1};
                    \node[] at (1.5, 3.5) {1};
                    \node[] at (2.5, 3.5) {1};
                    \node[] at (3.5, 3.5) {1};

                    \draw[kvloop, draw=Dandelion]
                        (0.1, 0) -- (0.1, 0.9) -- (3.9, 0.9) -- (3.9, 0)
                        (0.1, 4) -- (0.1, 3.1) -- (3.9, 3.1) -- (3.9, 4);
                    \draw[kvloop, draw=LimeGreen] (2.2, 0.1) rectangle (2.8, 3.9);
                    \draw[kvloop, draw=Cyan] (0.2, 0.1) rectangle (0.8, 3.9);
                \end{tikzpicture}
            \end{minipage}
        \end{enumerate}

    \item[\textbf{4.}]
        \begin{enumerate}
            \item[a)]
            \begin{minipage}[t]{0.4\textwidth}
                Funktionstabelle:

                \begin{tabular}{c||c|c|c|c||c}
                    $x$ & $x_3$ & $x_2$ & $x_1$ & $x_0$ & $y$\\
                    \hline
                     0 & 0 & 0 & 0 & 0 & 0\\
                     1 & 0 & 0 & 0 & 1 & 0\\
                     2 & 0 & 0 & 1 & 0 & 0\\
                     3 & 0 & 0 & 1 & 1 & 0\\
                     4 & 0 & 1 & 0 & 0 & 0\\
                     5 & 0 & 1 & 0 & 1 & 1\\
                     6 & 0 & 1 & 1 & 0 & 0\\
                     7 & 0 & 1 & 1 & 1 & 1\\
                     8 & 1 & 0 & 0 & 0 & 0\\
                     9 & 1 & 0 & 0 & 1 & 0\\
                    10 & 1 & 0 & 1 & 0 & 0\\
                    11 & 1 & 0 & 1 & 1 & 0\\
                    12 & 1 & 1 & 0 & 0 & 1\\
                    13 & 1 & 1 & 0 & 1 & 1\\
                    14 & 1 & 1 & 1 & 0 & 1\\
                    15 & 1 & 1 & 1 & 1 & 1
                \end{tabular}
                \vspace{1em}
            \end{minipage}
            \hspace{1cm}
            \begin{minipage}[t]{0.4\textwidth}
            \item[b)]Karnaugh-Veitch-Diagramm:
                \vspace{1em}

                \begin{tikzpicture}[x=7mm, y=7mm, font=\sffamily\small, label distance=-1.5mm]
                    \draw[step=1] (0, 0) grid (4,4);

                    \draw (1, 4.3) -- (3, 4.3) node[midway,label=above:x0] (x0) {};
                    \draw (2, -0.3) -- (4, -0.3) node[midway,label=below:x1] (x1) {};
                    \draw (-0.3, 1) -- (-0.3, 3) node[midway,label=left:x2] (x2) {};
                    \draw (4.3, 0) -- (4.3, 2) node[midway,label=right:x3] (x3) {};

                    \node[] at (0.5, 0.5) {0};
                    \node[] at (1.5, 0.5) {0};
                    \node[] at (2.5, 0.5) {0};
                    \node[] at (3.5, 0.5) {0};
                    \node[] at (0.5, 1.5) {1};
                    \node[] at (1.5, 1.5) {1};
                    \node[] at (2.5, 1.5) {1};
                    \node[] at (3.5, 1.5) {1};
                    \node[] at (0.5, 2.5) {0};
                    \node[] at (1.5, 2.5) {1};
                    \node[] at (2.5, 2.5) {1};
                    \node[] at (3.5, 2.5) {0};
                    \node[] at (0.5, 3.5) {0};
                    \node[] at (1.5, 3.5) {0};
                    \node[] at (2.5, 3.5) {0};
                    \node[] at (3.5, 3.5) {0};

                    \draw[kvloop, draw=Dandelion] (0.1, 1.2) rectangle (3.9, 1.8);
                    \draw[kvloop, draw=Cyan] (1.1, 1.1) rectangle (2.9, 2.9);
                \end{tikzpicture}
            \item[c)]
                $y = \textcolor{Dandelion}{x_3x_2} \vee \textcolor{Cyan}{x_2x_0}$
                \vspace{1em}

            \item[d)]
                Schaltnetz (US-Symbole):
                \vspace{1em}

                \begin{tikzpicture}[label distance=2mm,font=\sffamily]

                    \node (x3) at (-1.5,0) {$x_3$};
                    \node (x2) at (-1,0) {$x_2$};
                    \node (x1) at (-0.5,0) {$x_1$};
                    \node (x0) at (0,0) {$x_0$};

                    \node[and gate US, draw, logic gate inputs=nn] at ($(x0)+(1,-0.8)$) (And1) {\&};
                    \node[and gate US, draw, logic gate inputs=nn] at ($(And1)+(0,-1)$) (And2) {\&};
                    \node[or gate US, draw, logic gate inputs=nn] at ($(And1)+(1.5,-0.5)$) (Or) {\small{$\geq$1}};

                    \node (y) at ($(Or)+(1,0)$) {$y$};

                    \draw (x0) |- ($(x0)+(0,-2.5)$);
                    \draw (x1) |- ($(x1)+(0,-2.5)$);
                    \draw (x2) |- ($(x2)+(0,-2.5)$);
                    \draw (x3) |- ($(x3)+(0,-2.5)$);

                    \node[branch] at ($(x0 |- And1.input 1)$) (p1) {};
                    \node[branch] at ($(x2 |- And1.input 2)$) (p2) {};
                    \node[branch] at ($(x2 |- And2.input 1)$) (p3) {};
                    \node[branch] at ($(x3 |- And2.input 2)$) (p4) {};

                    \draw (p1) |- (And1.input 1);
                    \draw (p2) |- (And1.input 2);
                    \draw (p3) |- (And2.input 1);
                    \draw (p4) |- (And2.input 2);

                    \draw (And1.output) -- ([xshift=0.3cm]And1.output) |- (Or.input 1);
                    \draw (And2.output) -- ([xshift=0.3cm]And2.output) |- (Or.input 2);
                    \draw (Or.output) -- (y);
%                     \draw (Or1.output) -- (And2.input 1);
%                     \draw (Nor1.output) -- ([xshift=0.5cm]Nor1.output) |- (And2.input 2);
%                     \draw (And2.output) -- ([xshift=0.5cm]And2.output) node[above] {$f_1$};

                \end{tikzpicture}
            \end{minipage}
            \vspace{2em}

            \item[e)]Binäres Entscheidungsdiagramm (ROBDD):

                \begin{tikzpicture}[x=5em,y=3em]
                    \node[binode] (n0)   at (0, 0)           {$x_0$};
                    \node[binode] (n2_1) at ($(n0)+(+1, 1)$) {$x_2$};
                    \node[binode] (n2_2) at ($(n0)+(+1,-1)$) {$x_2$};
                    \node[binode] (n3)   at ($(n0)+( 2, 0)$) {$x_3$};

                    \node[final]  (f1)   at ($(n2_1)+( 2, 0)$) {$1$};
                    \node[final]  (f0)   at ($(n2_2)+( 2, 0)$) {$0$};

                    \draw[bi1branch] (n0) to (n2_1);
                    \draw[bi0branch] (n0) to (n2_2);
                    \draw[bi1branch] (n2_1) to (f1);
                    \draw[bi0branch] (n2_1) to (n3);
                    \draw[bi1branch] (n2_2) to (n3);
                    \draw[bi0branch] (n2_2) to (f0);
                    \draw[bi1branch] (n3) to (f1);
                    \draw[bi0branch] (n3) to (f0);
                \end{tikzpicture}

                Die Schaltvariable $x_1$ wurde weggelassen, da sie für den Wert der Schaltfunktion ohne Bedeutung ist. Durchgezogene Linien
                haben den Wert 1, gestrichelte den Wert 0.

        \end{enumerate}
\end{enumerate}

\end{document}
