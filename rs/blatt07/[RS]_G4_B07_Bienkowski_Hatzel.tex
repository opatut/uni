\newcommand{\authorinfo}{Paul Bienkowski, Hans Ole Hatzel}
\newcommand{\titleinfo}{RS 06 (HA) zum 30.11.2012}

% PREAMBLE ===============================================================

\documentclass[a4paper,10pt]{scrartcl}
\usepackage[german,ngerman]{babel}
\usepackage[utf8]{inputenc}
\usepackage[T1]{fontenc}
\usepackage{lmodern}
\usepackage{amssymb}
\usepackage{mathtools}
\usepackage{amsmath}
\usepackage{enumerate}
\usepackage{array}
\usepackage{listings}
\usepackage{fullpage}
\usepackage[usenames,dvipsnames]{xcolor}
\usepackage{tikz}
\usepackage{pgffor}
\usepackage{fancyhdr}
\usepackage{lastpage}

\author{\authorinfo}
\title{\titleinfo}
\date{\today}

\pagestyle{fancy}
\fancyhf{}
\fancyhead[L]{\authorinfo}
\fancyhead[R]{\titleinfo}
\fancyfoot[C]{\thepage}
\renewcommand{\headrulewidth}{0.4pt}
\renewcommand{\footrulewidth}{0pt}
\renewcommand{\headheight}{12pt}
\renewcommand{\headsep}{12pt}

\begin{document}
\setcounter{secnumdepth}{0}
\maketitle

% allow huuuuge matrix
\setcounter{MaxMatrixCols}{31}

% DOCUMENT ===============================================================

\tikzstyle{kvloop}=[draw, opacity=1, line width=0.4mm, rounded corners=2mm]

\begin{enumerate}
    \item[\textbf{1.}]
    \begin{enumerate}
        \item[a)]
            Die Funktion liegt bereits als KNF vor:

            $$\begin{array}{rclr}
                f(x) &=& (x_3 \vee \overline{x_2}) \wedge (x_2 \vee \overline{x_1}) & \text{(KNF)}\\
                &=& x_2 x_3 \vee \overline{x_1}\;\overline{x_2} & \text{(DNF)}\\
                &=& 1 \oplus x_1 \oplus x_2 \oplus x_1x_2 \oplus x_2x_3 & \text{(Reed-Muller-Form)}\\
            \end{array}$$

        \item[b)]
            Die Funktion liegt bereits nahezu als Reed-Muller-Form vor:

            $$\begin{array}{rclr}
                g(x) &=& x_3 \oplus x_1 & \text{(Reed-Muller-Form)}\\
                &=& \overline{x_3}x_1 \vee x_3\overline{x_1} & \text{(DNF)}\\
                &=& (x_3 \vee \overline{x_1}) \wedge (x_1 \vee \overline{x_3}) & \text{(KNF)}\\
            \end{array}$$

    \end{enumerate}

    \item[\textbf{2.}]
    \begin{enumerate}
        \item[a)]
            \begin{itemize}
                \item
                    Da AND(a, a) immer a ergibt, muss NAND die Negation von a sein:

                    {\centering
                    \textbf{NOT}(a) = NAND(a, a)

                    \begin{tabular}[t]{c||c|c}
                        a & NAND(a, a) & NOT(a) \\
                        \hline
                        0 & 1 & 1 \\
                        1 & 0 & 0
                    \end{tabular}\\
                    }

                \item
                    Um AND zu erreichen, kann ebenso das Inverse der Ausgabe von NAND verwendet werden:

                    {\centering
                    \textbf{AND}(a, b) = NOT(NAND(a, b)) = NAND(NAND(a, b), NAND(a, b))

                    \begin{tabular}[t]{c|c||c|c|c}
                        a & b & AND(a, b) & NAND(a, b) & NOT(NAND(a, b)) \\
                        \hline
                        0 & 0 & 0 & 1 & 0 \\
                        0 & 1 & 0 & 1 & 0 \\
                        1 & 0 & 0 & 1 & 0 \\
                        1 & 1 & 1 & 0 & 1
                    \end{tabular}\\
                    }

                \item
                    Nach de Morgan gilt:

                    {
                    \centering
                    NOT(\textbf{OR}(a, b)) = AND(NOT(A), NOT(B))\\
                    $\Downarrow$\\
                    \textbf{OR}(A, B) = NAND(NOT(A), NOT(B))\\
                    $\Downarrow$\\
                    \textbf{OR}(A, B) = NAND(NAND(A, A), NAND(B, B))\\
                    }
            \end{itemize}

        \item[b)]
            Es sei $a \barwedge b$ die Schreibweise für NAND(a, b).

            $$\begin{array}{rcl}
                f(x_3, x_2, x_1)
                &=& (\overline{x_3}(\overline{x_2} \vee x_1)) \vee (x_1 ( \overline{x_2} \vee x_1 ) )\\
                &=& (\overline{x_2} \vee x_1) \wedge (\overline{x_3} \vee x_1)\\
                &=& x_1 \wedge (\overline{x_2} \vee \overline{x_3})\\
                &=& x_1 \wedge (x_2 \barwedge x_3)\\
                &=& \big(x_1 \barwedge (x_2 \barwedge x_3)\big) \barwedge \big(x_1 \barwedge (x_2 \barwedge x_3)\big)
            \end{array}$$

    \end{enumerate}

    \newpage
    \item[\textbf{3.}]
    \begin{enumerate}
        \item[a)]
            Funktionstabelle für A und B:

            \begin{center}
                \begin{tabular}[t]{c||c|c|c|c||c|c}
                    $x$ & $x_4$ & $x_3$ & $x_2$ & $x_1$ & $A$ & $B$\\
                    \hline
                    0 & 0 & 0 & 0 & 0 & 1 & 1 \\
                    1 & 0 & 0 & 0 & 1 & 0 & 1 \\
                    2 & 0 & 0 & 1 & 0 & 1 & 1 \\
                    3 & 0 & 0 & 1 & 1 & 1 & 1 \\
                    4 & 0 & 1 & 0 & 0 & 0 & 1 \\
                    5 & 0 & 1 & 0 & 1 & 1 & 0 \\
                    6 & 0 & 1 & 1 & 0 & 1 & 0 \\
                    7 & 0 & 1 & 1 & 1 & 1 & 1 \\
                    8 & 1 & 0 & 0 & 0 & 1 & 1 \\
                    9 & 1 & 0 & 0 & 1 & 1 & 1
                \end{tabular}
            \end{center}

            Karnaugh-Veitch-Diagramme:

            \begin{minipage}[c]{0.4\textwidth}
                \centering
                \begin{tikzpicture}[x=7mm, y=7mm, font=\sffamily\small, label distance=-1.5mm]
                    \draw[step=1] (0, 0) grid (4,4);

                    \draw (1, 4.3) -- (3, 4.3) node[midway,label=above:x0] (x0) {};
                    \draw (2, -0.3) -- (4, -0.3) node[midway,label=below:x1] (x1) {};
                    \draw (-0.3, 1) -- (-0.3, 3) node[midway,label=left:x2] (x2) {};
                    \draw (4.3, 0) -- (4.3, 2) node[midway,label=right:x3] (x3) {};

                    \node[] at (0.5, 0.5) {1};
                    \node[] at (1.5, 0.5) {1};
                    \node[] at (2.5, 0.5) {X};
                    \node[] at (3.5, 0.5) {X};
                    \node[] at (0.5, 1.5) {X};
                    \node[] at (1.5, 1.5) {X};
                    \node[] at (2.5, 1.5) {X};
                    \node[] at (3.5, 1.5) {X};
                    \node[] at (0.5, 2.5) {0};
                    \node[] at (1.5, 2.5) {1};
                    \node[] at (2.5, 2.5) {1};
                    \node[] at (3.5, 2.5) {1};
                    \node[] at (0.5, 3.5) {1};
                    \node[] at (1.5, 3.5) {0};
                    \node[] at (2.5, 3.5) {1};
                    \node[] at (3.5, 3.5) {1};

                    \draw[kvloop, draw=Red] (0.1, 0.1) rectangle (3.9, 1.9);
                    \draw[kvloop, draw=LimeGreen] (2.2, 0.2) rectangle (3.8, 3.8);0.6
                    \draw[kvloop, draw=Cyan] (1.1, 1.1) rectangle (2.9, 2.9);

                    \draw[kvloop, draw=Dandelion]
                        (0, 3.1) -- (0.9, 3.1) -- (0.9, 4)
                        (0, 0.9) -- (0.9, 0.9) -- (0.9, 0)
                        (3.1, 0) -- (3.1, 0.9) -- (4, 0.9)
                        (3.1, 4) -- (3.1, 3.1) -- (4, 3.1);
                \end{tikzpicture}
                $$A(x) = \textcolor{Red}{x_3} \vee \textcolor{LimeGreen}{x_1} \vee \textcolor{Cyan}{x_2x_0} \vee \textcolor{Dandelion}{\overline{x_2}\;\overline{x_0}}$$
            \end{minipage}
            \hspace{1cm}
            \begin{minipage}[c]{0.4\textwidth}
                \centering
                \begin{tikzpicture}[x=7mm, y=7mm, font=\sffamily\small, label distance=-1.5mm]
                    \draw[step=1] (0, 0) grid (4,4);

                    \draw (1, 4.3) -- (3, 4.3) node[midway,label=above:x0] (x0) {};
                    \draw (2, -0.3) -- (4, -0.3) node[midway,label=below:x1] (x1) {};
                    \draw (-0.3, 1) -- (-0.3, 3) node[midway,label=left:x2] (x2) {};
                    \draw (4.3, 0) -- (4.3, 2) node[midway,label=right:x3] (x3) {};

                    \node[] at (0.5, 0.5) {1};
                    \node[] at (1.5, 0.5) {1};
                    \node[] at (2.5, 0.5) {X};
                    \node[] at (3.5, 0.5) {X};
                    \node[] at (0.5, 1.5) {X};
                    \node[] at (1.5, 1.5) {X};
                    \node[] at (2.5, 1.5) {X};
                    \node[] at (3.5, 1.5) {X};
                    \node[] at (0.5, 2.5) {1};
                    \node[] at (1.5, 2.5) {0};
                    \node[] at (2.5, 2.5) {1};
                    \node[] at (3.5, 2.5) {0};
                    \node[] at (0.5, 3.5) {1};
                    \node[] at (1.5, 3.5) {1};
                    \node[] at (2.5, 3.5) {1};
                    \node[] at (3.5, 3.5) {1};

                    \draw[kvloop, draw=Dandelion]
                        (0.1, 0) -- (0.1, 0.9) -- (3.9, 0.9) -- (3.9, 0)
                        (0.1, 4) -- (0.1, 3.1) -- (3.9, 3.1) -- (3.9, 4);
                    \draw[kvloop, draw=LimeGreen] (2.2, 0.1) rectangle (2.8, 3.9);
                    \draw[kvloop, draw=Cyan] (0.2, 0.1) rectangle (0.8, 3.9);
                \end{tikzpicture}
                $$B(x) = \textcolor{Dandelion}{\overline{x_2}} \vee \textcolor{Cyan}{\overline{x_1}\;\overline{x_0}} \vee \textcolor{LimeGreen}{x_1x_0}$$
            \end{minipage}

    \end{enumerate}

\end{enumerate}

\end{document}
