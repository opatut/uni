\documentclass[a4paper,10pt]{scrartcl}
\usepackage[german,ngerman]{babel}
\usepackage[utf8]{inputenc}
\usepackage[T1]{fontenc}
\usepackage{lmodern}
\usepackage{amssymb}
\usepackage{mathtools}
\usepackage{amsmath}
\usepackage{enumerate}
\usepackage{array}
\usepackage{breqn}
\usepackage{fullpage}

\author{Paul Bienkowski}
\title{RS 02 (HA) zum 02.11.2012}
\date{\today}
\begin{document}
\setcounter{secnumdepth}{0}
\maketitle

\begin{enumerate}
    \item[\textbf{1.}]
        \begin{enumerate}
            \item[a)]
                Das Register kann $2^{64}$ Zustände annehmen, also gilt:

                $$\frac{2^{64}}{3.2 \cdot 10^9} s \approx 5.7 \cdot 10^9 s \approx 182.67 a$$

                Folglich läuft das Register nach etwa 182.67 Jahren zum ersten Mal über.

            \item[b)]
                Moderne PCs haben meist mehrere CPU-Kerne, die asynchron laufen. Daher müsste es für 2 Kerne auch 2 verschiedene Register
                geben, die dann verschiedene Werte enthalten würden.

                Außerdem können moderne CPUs dynamisch nach Bedarf (ondemand) ihren Takt ändern (over-/underclocking), sodass die Anzahl der
                vergangenen Takte nicht als Zeitmaß ausreichen.
        \end{enumerate}

    \item[\textbf{2.}]
        \begin{enumerate}
            \item[a)]
                $53_{10} = 110101_2 = 65_8 = 35_{16}$

            \item[b)]
                $2012_{10} = 11111011100_2 = 3734_8 = \text{7DC}_{16}$

            \item[c)]
                $5.5625_{10} = 101.1001_2 = 5.44_8 = 5.9_{16}$

            \item[d)]
                $375,375_{10} = 101110111.011_2 = 567.3_8 = 177.6_{16}$
        \end{enumerate}

    \item[\textbf{3.}]
        \begin{enumerate}
            \item[a)]
                $1110,1001_2 = 14.5625_{10}$

            \item[b)]
                $10101.10011_2 = 21.59375_{10}$
        \end{enumerate}

    \item[\textbf{4.}]
        $25487_{10} = 110001110001111_2 = 61617_8 = 638\text{F}_{16}$

        $15190_{10} = 11101101010110_2 = 35526_8 = 3\text{B}56_{16}$

        \begin{verbatim}
  110 0011 1000 1111
+  11 1011 0101 0110
--------------------
 11    11    11 11
====================
 1001 1110 1110 0101
        \end{verbatim}

        $1001111011100101_2 = 117345_8 = 9\text{EE}5_{16} = 40677_{10}$

        Zur Überprüfung wird im Dezimalsystem gezeigt: $25487 + 15190 = 40677$.

    \newpage
    \item[\textbf{5.}]
        \begin{verbatim}
  10010011 * 1110001
 --------------------
  111001
   000000
    000000
     111001
      000000
       000000
        111001
         111001
 ----------------
 111111111
 ================
 10000010111011
        \end{verbatim}

        Zur Überprüfung:
        $$\begin{array}{rcl}
            10010011_2 &=& 147_{10}\\
            111001_2 &=& 57_{10}\\
            10000010111011_2 &=& 8376_{10}\\
            147_{10} \cdot 57_{10} &=& 8376_{10}
        \end{array}$$

    \item[\textbf{6.}]
        \begin{enumerate}
            \item[a)]
                $K_{10}(4.582)_{10} = 10^2 - 4.582 = 95.4180$

            \item[b)]
                $K_{9}(0.1274)_{10} = 10^2 - 10^{-4} - 0.1274 = 99.8725$

            \item[c)]
                $K_2(1.011)_2 = 2_{10}^2 - 1.375_{10} = 2.625_{10} = 10.101_2$

            \item[d)]
                $\begin{array}[t]{rclllcl}
                    K_1(100.01)_2 &=& 2_{10}^4 &{}- 2_{10}^{-3} &{}- 4.25_{10} &=& 11.625_{10} \\
                    &=& 10000.000_2 &{}- 0.001_2 &{}- 100.01_2 &=& 1011.101_2
                \end{array}$
        \end{enumerate}

    \item[\textbf{7.}]
        \bgroup
        \def\arraystretch{1.5}
        \begin{tabular}[t]{|l|c|c|c|c|}
            \hline
            & a) $0000\;1001_2$ & b) $0110\;0101_2$ & c) $1000\;0001_2$ & d) $1111\;1011_2$\\
            \hline
            1. &  9  & 101 & 129 & 251 \\
            2. &  9  & 101 &  -1 &-123 \\
            3. &-118 & -26 &   2 & 124 \\
            4. &  9  & 101 &-126 &  -4 \\
            5. &  9  & 101 &-127 &  -5 \\
            \hline
        \end{tabular}
        \egroup

\end{enumerate}

\end{document}
