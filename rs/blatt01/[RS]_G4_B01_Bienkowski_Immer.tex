\documentclass[10pt,a4paper]{article}

\usepackage{amstext}
\usepackage[utf8]{inputenc}
\usepackage[ngerman]{babel}
\usepackage{lmodern}
    
\author{Paul Bienkowski, Vincent Immer}
\title{RS - Blatt 01 (HA) zum 26.10.2012}
\date{\today}
\begin{document}
\setcounter{secnumdepth}{0}
\maketitle

\begin{enumerate}
        % Aufgabe 1
    \item[\textbf{1.1.}]
        \begin{enumerate}
            \item[a)]
                Ein Interpreter übersetzt Befehle einer höheren Sprache zu Laufzeit in 
                Abfolgen von Befehlen niedrigerer Sprachen und führt diese aus.
            \item[b)]
                Ein Compiler übersetzt Befehle wie ein Interpreter, nur erstellt der 
                Compiler ein Programm vor der Ausführung.
            \item[c)]
                Eine Virtuelle Maschine (VM) simuliert eine Maschine, sodass Befehle einer
                höheren Sprache L1 auf die gleiche Art ausgeführt werden können wie Hardware-Befehle,
                obwohl sie durch die Virtuelle Maschine erst weiter interpretiert werden.
                
                Dies ermöglicht z.B. die Virtualisierung ganzer Betriebssysteme auf fremden Host-Systemen.
        \end{enumerate}
    \item[\textbf{1.2.}]
        $A(x)$ sei die Anzahl der Befehle auf $L_0$, die für ein Kommando auf $L_x$ nötig sind.
        
        $$\begin{array}{rcl}
            A(x) &=& n^x\\
            A(3) &=& n^3\\
            A(2) &=& n^2\\
            A(1) &=& n\\
            A(0) &=& 1
        \end{array}$$
        
        $T(x)$ sei die Dauer eines Befehls auf $L_x$.
    
        $$\begin{array}{rcl}
            T(x) &=& k \cdot n^x\\
            T(3) &=& k \cdot n^3\\
            T(2) &=& k \cdot n^2\\
            T(1) &=& k \cdot n\\
            T(0) &=& k
        \end{array}$$
    
    \item[\textbf{1.3.}]
        Ein Vorteil des von-Neumann-Konzepts ist es, dass Programme dynamisch erstellt und verändert werden 
        können. Da Programme selbst Daten sind, ist es möglich, mit Compilern auf der Maschine selbst Programme
        zu erstellen und auszuführen. Somit ist die Programmierung der Maschine flexibel, da Programme garantiert
        wie die Daten änderbar sind.
        
        Allerdings können ohne weitere Sicherheitsmaßnahmen Programme andere Programme manipulieren, und so zum Beispiel
        die Funktionalität abändern. Dies kann ein Sicherheitsrisiko sein, wenn z.B. Schadsoftware die laufenden Programme
        verändern kann. Deshalb wird ein System benötigt, was den Speicherzugriff von Programmen einschränkt.
        
    \item[\textbf{1.4.}]
        \begin{enumerate}
            \item[a)]
                $$\begin{array}{rcl}
                    y &=& a \cdot x^6 + b \cdot x^5 + c \cdot x^4 + d \cdot x^3 + e \cdot x^2 + f \cdot x + g\\
                      &=& a \cdot x \cdot x \cdot x \cdot x \cdot x \cdot x + b \cdot x \cdot x \cdot x \cdot x \cdot x + c \cdot x \cdot x \cdot x \cdot x + \\
                       && d \cdot x \cdot x \cdot x + e \cdot x \cdot x + f \cdot x + g
                \end{array}$$
                
                Es werden $6 + 5 + 4 + 3 + 2 + 1 = 21$ Multiplikationen und $6$ Additionen benötigt. Dies ergibt
                eine Dauer von $21 * 6 \text{ns} + 6 * 1 \text{ns} = 132 \text{ns}$.
                
                Nach dem Horner-Schema ergibt sich eine Umformung zu
                
                $$\begin{array}{rcl}
                    y &=& g + x * (f + x * (e + x * (d + x * (c + x * (b + x * a)))))
                \end{array}$$
                
                Dies entspricht je 6 Multiplikationen und Additionen, also einer Laufzeit von $6 \cdot 6 \text{ns} + 6 \cdot 1 \text{ns} = 42 \text{ns}$.
            \item[b)]
                $$\begin{array}{ll}
                    a &= x + 1\\
                    b &= a \cdot a\\
                    c &= b \cdot b\\
                    d &= c \cdot c\\
                    e &= d \cdot d\\
                    y &= (x + 1)^{23} = a * b * c * e\\
                    &= a * a^2 * {a^2}^2 * {{{a^2}^2}^2}^2 \\
                    &= a * a^2 * a^4 * a^{16} = a^{23}
                \end{array}$$
                
                Für die Werte der Variablen $a$, $b$, $c$, $d$ und $e$ werden insgesamt 1 Addition und 4 Multiplikationen 
                benötigt, die Berechnung des Endergebnisses erfordert dann 3 weitere Multiplikationen. Daraus ergibt
                sich eine Gesamtzeit von
                
                $$(4 + 3) \cdot 6 \text{ns} + 1 \cdot 1 \text{ns} = 43 \text{ns}$$

        \end{enumerate}

        
    \item[\textbf{1.5.}]
        \begin{enumerate}
            \item[a)]
                Aufgrund von Schaltjahren hat ein Jahr im Durchschnitt $365.242$ Tage. Daraus ergibt sich:
                
                $$5 \frac{\text{MB}}{sec} \cdot (60 \cdot 60 \cdot 24 \cdot 365.242) \frac{sec}{a} \cdot 80 a = 12622763520 \text{ MB}$$
                
            \item[b)]
                Eine Datenmenge von $12622763520$ MB entspricht $11480.337$ TiB.
                
                $T$ sei die gesuchte Anzahl von Jahren, $lg$ steht für den dekadischen Logarithmus. Es ergibt sich:
                
                $$T = \frac{\lg{\frac{11480.337 \text{TiB}}{2 \text{TiB}}}}{\lg{1.45}} = \approx 23.3 $$
                
                Nach etwa $23.3$ Jahren passen also die gesamten Aufzeichnungen eines Lebens auf eine Festplatte.
            
            \item[c)]
                Eine Datenmenge von $12622763520$ MB entspricht $11755865$ GiB.
            
                $$T = \lg{\frac{11755865 \text{GiB}}{32 \text{GiB}}} / \lg{1.52} \approx 30.6$$
                
                Nach etwa $30.6$ Jahren passen also die gesamten Aufzeichnungen eines Lebens auf eine SD-Karte.
            \item[d)]
                $N$ sei die Anzahl benötigter Magnetbänder der Kapazität $140$ MB.
                
                $$N = \frac{12622763520 \text{MB}}{140 \text{MB}} = 90162597$$
                
                Man würde also etwas über $90$ Millionen Magnetbänder benötigen.
        \end{enumerate}

\end{enumerate}
\end{document}
