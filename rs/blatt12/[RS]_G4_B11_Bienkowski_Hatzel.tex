\newcommand{\authorinfo}{Paul Bienkowski, Hans Ole Hatzel}
\newcommand{\titleinfo}{RS 12 (HA) zum 25.01.2013}

% PREAMBLE ===============================================================

\documentclass[a4paper,10pt]{scrartcl}
\usepackage[german,ngerman]{babel}
\usepackage[utf8]{inputenc}
\usepackage[T1]{fontenc}
\usepackage{lmodern}
\usepackage{amssymb}
\usepackage{mathtools}
\usepackage{amsmath}
\usepackage{enumerate}
\usepackage{pgffor}
\usepackage{array}
\usepackage{listings}
\usepackage{fullpage}
\usepackage[usenames,dvipsnames]{xcolor}
\usepackage{tikz}
\usepackage{circuitikz}
\usepackage{fancyhdr}
\usepackage{lastpage}
\usepackage{multicol}\usepackage{multicol}

\usetikzlibrary{calc,decorations.markings}

\author{\authorinfo}
\title{\titleinfo}
\date{\today}

\pagestyle{fancy}
\fancyhf{}
\fancyhead[L]{\authorinfo}
\fancyhead[R]{\titleinfo}
\fancyfoot[C]{\thepage}
\renewcommand{\headrulewidth}{0.4pt}
\renewcommand{\footrulewidth}{0pt}
\renewcommand{\headheight}{12pt}
\renewcommand{\headsep}{12pt}

\newcommand*{\oline}[1]{\overline{\vphantom{A}#1}}

\renewcommand{\familydefault}{\sfdefault}

\begin{document}
\setcounter{secnumdepth}{0}
\maketitle

% DOCUMENT ===============================================================

\begin{enumerate}
    \item[\textbf{1.}]
        \begin{multicols}{5}
        \begin{enumerate}
            \item[a)]
            \item[b)]
            \item[c)]
            \item[d)]
            \item[e)]
            \item[f)]
        \end{enumerate}
        \end{multicols}

    \item[\textbf{2.}]
        Da ein beliebiges Bit per XOR mit sich selbst kombiniert immer 0 ergibt,
        lässt sich auch einfach schreiben:
        \begin{verbatim}xor %eax, %eax\end{verbatim}

    \item[\textbf{3.}]
        Um den Program-Counter zu erhalten, wird ein Hilfssprung als Funktionsaufruf
        ausgeführt. Der \texttt{call}-Befehl schreibt dabei den Programmzähler auf
        den Stack und springt zum Label (\texttt{jmp}). Danach kann man den Zähler
        manuell vom Stack holen:

        \begin{verbatim}
    call helper_label:
helper_label:
    pop %eax\end{verbatim}

    \item[\textbf{4.}]
    \item[\textbf{5.}]
        \begin{enumerate}
            \item[a)]
            \item[b)]
            \item[c)]
            \item[d)]
        \end{enumerate}
\end{enumerate}
\end{document}
