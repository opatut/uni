\newcommand{\authorinfo}{Paul Bienkowski, Hans Ole Hatzel}
\newcommand{\titleinfo}{RS 12 (HA) zum 25.01.2013}

% PREAMBLE ===============================================================

\documentclass[a4paper,10pt]{scrartcl}
\usepackage[german,ngerman]{babel}
\usepackage[utf8]{inputenc}
\usepackage[T1]{fontenc}
\usepackage{lmodern}
\usepackage{amssymb}
\usepackage{mathtools}
\usepackage{amsmath}
\usepackage{enumerate}
\usepackage{pgffor}
\usepackage{array}
\usepackage{listings}
\usepackage{fullpage}
\usepackage[usenames,dvipsnames]{xcolor}
\usepackage{tikz}
\usepackage{circuitikz}
\usepackage{fancyhdr}
\usepackage{lastpage}
\usepackage{multicol}\usepackage{multicol}

\usetikzlibrary{calc,decorations.markings}

\author{\authorinfo}
\title{\titleinfo}
\date{\today}

\pagestyle{fancy}
\fancyhf{}
\fancyhead[L]{\authorinfo}
\fancyhead[R]{\titleinfo}
\fancyfoot[C]{\thepage}
\renewcommand{\headrulewidth}{0.4pt}
\renewcommand{\footrulewidth}{0pt}
\renewcommand{\headheight}{12pt}
\renewcommand{\headsep}{12pt}

\newcommand*{\oline}[1]{\overline{\vphantom{A}#1}}

\renewcommand{\familydefault}{\sfdefault}

\begin{document}
\setcounter{secnumdepth}{0}
\maketitle

% DOCUMENT ===============================================================

\begin{enumerate}
    \item[\textbf{1.}]
        \begin{enumerate}
            \item[a)]
                \texttt{MEM[0x0000 0100] = 0x0000 CB02}
            \item[b)]
                \texttt{MEM[0x0000 0104] = 0x0000 00BF}
            \item[c)]
                \texttt{MEM[0x0000 0108] = 0x0024 6000}
            \item[d)]
                \texttt{MEM[0x0000 010C] = 0x0000 0043}
            \item[e)]
                \texttt{\%ecx = 0x0000 000B}
            \item[f)]
                \texttt{\%eax = 0x0000 00FC}
        \end{enumerate}

    \item[\textbf{2.}]
        Da ein beliebiges Bit per XOR mit sich selbst kombiniert immer 0 ergibt,
        lässt sich auch einfach schreiben:
        \begin{verbatim}xor %eax, %eax\end{verbatim}

    \item[\textbf{3.}]
        Um den Program-Counter zu erhalten, wird ein Hilfssprung als Funktionsaufruf
        ausgeführt. Der \texttt{call}-Befehl schreibt dabei den Programmzähler auf
        den Stack und springt zum Label (\texttt{jmp}). Danach kann man den Zähler
        manuell vom Stack holen:

        \begin{verbatim}
    call helper_label:
helper_label:
    pop %eax\end{verbatim}

    \item[\textbf{4.}]
                \begin{verbatim}
f2c:
    pushl   %ebp                ; standard-setup
    movl    %esp, %ebp

    movl    8(%ebp), %eax       ; eax = f (first and only argument)
    subl    $0x20, %eax         ; eax = (f - 32)
    imull   $0x8E, %eax         ; eax = (f - 32) * 142
    sarl    $8, %eax            ; eax = (f - 32) * 142 / 256

    movl    %ebp, %esp          ; standard-return
    popl    %ebp
    ret
\end{verbatim}

    \newpage
    \item[\textbf{5.}]
        \begin{enumerate}
            \item[a)]

                \begin{verbatim}
myst:
    pushl   %ebx                ; save ebx on stack
    subl    $24, %esp           ; esp -= 24 (6 words)
    movl    32(%esp), %ebx      ; ebx = first parameter
    movl    36(%esp), %edx      ; edx = second parameter
    movl    $1, %eax            ; eax = 1
    testl   %edx, %edx          ; ZF = (edx == 0)
    je      .L2                 ; jump to L2 if edx is 0
    subl    $1, %edx            ; edx--
    movl    %edx,  4(%esp)      ; second paramter = edx
    movl    %ebx, (%esp)        ; first parameter = ebx
    call    myst                ; recursive call: myst(edx, ebx)
    imull   %ebx, %eax          ; eax *= ebx
.L2:
    addl    $24, %esp           ; reset stack pointer
    popl    %ebx                ; reset ebx from stack
    ret                         ; return to caller
\end{verbatim}

                Dies ergibt folgenden C-Code:

                \begin{verbatim}
int myst(int a, int b) {
    int c = 1;
    if(b == 0) {
        return 1;
    }
    b--;
    c = myst(a, b) * a;
    return c;
}
\end{verbatim}

                Eine noch kompaktere Schreibweise wäre:

                \begin{verbatim}
int myst(int a, int b) {
    if(b == 0) return 1;
    return myst(a, b - 1) * a;
}
\end{verbatim}

            \item[b)]
                Es ist klar zu erkennen, dass das Unterprogramm/die Funktion die Potenz $a^b$ berechnet.

            \begin{multicols}{2}
            \item[c)]
                \begin{tabular}[t]{|c|}
                \hline
                \texttt{b = 2} \\ \hline
                \texttt{a = 5} \\ \hline
                \texttt{\%eip main} \\ \hline
                \texttt{\%ebx main} \\ \hline
                \texttt{b = 1} \\ \hline
                \texttt{a = 5} \\ \hline
                \texttt{\%eax myst-1} \\ \hline
                \texttt{\%eip myst-1} \\ \hline
                \texttt{\%ebx myst-1} \\ \hline
                \texttt{b = 0} \\ \hline
                \texttt{a = 5} \\ \hline
                \texttt{\%eax myst-2} \\ \hline
                \texttt{\%eip myst-3} \\ \hline
                \texttt{\%ebx myst-3} \\ \hline
                \texttt{0} \\ \hline
                \texttt{0} \\ \hline
                \texttt{0} \\ \hline
                \end{tabular}


            \item[d)]
                Die Abbruchbedingung wird nicht sofort erreicht. Bei 16 bit Wortbreite werden
                $2^{16} - 3 = 65533$ Rekursionsaufrufe (nach \emph{integer underflow}) gestartet.
                Bei jedem Aufruf wird der Stack jeweils um 5 Worte (10 Byte) größer, damit
                wird ein Stack von 655.33 KB benötigt. Ist der Stack dafür zu klein, wird ein
                \emph{stack overflow} vermutlich vom Kernel erkannt und das Programm
                abbrechen. Ansonsten wird das Ergebnis von $3^{65533}$ berechnet (etwa $1,5 \cdot 10^{31267}$),
                welches selbst mehrmals überläuft und somit $3^{65533} \text{ mod } 2^{16} = 55827$
                ergibt.
            \end{multicols}

        \end{enumerate}
\end{enumerate}
\end{document}
